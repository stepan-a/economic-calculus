\synctex=1

\documentclass[10pt,notheorems]{beamer}

\usepackage{etex}
\usepackage{fourier-orns}
\usepackage{ccicons}
\usepackage{amssymb}
\usepackage{amstext}
\usepackage{amsbsy}
\usepackage{amsopn}
\usepackage{amscd}
\usepackage{amsxtra}
\usepackage{amsthm}
\usepackage{float}
\usepackage{color, colortbl}
\usepackage{mathrsfs}
\usepackage{bm}
\usepackage{lastpage}
\usepackage[nice]{nicefrac}
\usepackage{setspace}
\usepackage{ragged2e}
\usepackage{listings}
\usepackage{polynom}
\usepackage{algorithms/algorithm}
\usepackage{algorithms/algorithmic}
\usepackage[frenchb]{babel}
\usepackage{tikz,pgfplots}
\pgfplotsset{compat=newest}
\usetikzlibrary{patterns, arrows, decorations.pathreplacing, decorations.markings, calc}
\pgfplotsset{plot coordinates/math parser=false}
\newlength\figureheight
\newlength\figurewidth
\usepackage[utf8x]{inputenc}
\usepackage{cancel}
\usepackage{tikz-qtree}
\usepackage{dcolumn}
\usepackage{adjustbox}
\usepackage{environ}
\usepackage[cal=boondox]{mathalfa}
\usepackage{manfnt}
\usepackage{hyperref}
\hypersetup{
  colorlinks=true,
  linkcolor=blue,
  filecolor=black,
  urlcolor=black,
}
\usepackage{venndiagram}
\usepackage{minted}

% Git hash
\usepackage{xstring}
\usepackage{catchfile}
\immediate\write18{git rev-parse HEAD > git.hash}
\CatchFileDef{\HEAD}{git.hash}{\endlinechar=-1}
\newcommand{\gitrevision}{\StrLeft{\HEAD}{7}}

\newcommand{\trace}{\mathrm{tr}}
\newcommand{\vect}{\mathrm{vec}}
\newcommand{\tracarg}[1]{\mathrm{tr}\left\{#1\right\}}
\newcommand{\vectarg}[1]{\mathrm{vec}\left(#1\right)}
\newcommand{\vecth}[1]{\mathrm{vech}\left(#1\right)}
\newcommand{\iid}[2]{\mathrm{iid}\left(#1,#2\right)}
\newcommand{\normal}[2]{\mathcal N\left(#1,#2\right)}
\newcommand{\dynare}{\href{http://www.dynare.org}{\color{blue}Dynare}}
\newcommand{\sample}{\mathcal Y_T}
\newcommand{\samplet}[1]{\mathcal Y_{#1}}
\newcommand{\slidetitle}[1]{\fancyhead[L]{\textsc{#1}}}

\newcommand{\R}{{\mathbb R}}
\newcommand{\C}{{\mathbb C}}
\newcommand{\N}{{\mathbb N}}
\newcommand{\Z}{{\mathbb Z}}
\newcommand{\binomial}[2]{\begin{pmatrix} #1 \\ #2 \end{pmatrix}}
\newcommand{\bigO}[1]{\mathcal O \left(#1\right)}
\newcommand{\red}{\color{red}}
\newcommand{\blue}{\color{blue}}

\renewcommand{\qedsymbol}{C.Q.F.D.}

\newcolumntype{d}{D{.}{.}{-1}}
\definecolor{gray}{gray}{0.9}
\newcolumntype{g}{>{\columncolor{gray}}c}

\setbeamertemplate{theorems}[numbered]

\theoremstyle{plain}
\newtheorem{theorem}{Théorème}

\theoremstyle{definition} % insert bellow all blocks you want in normal text
\newtheorem{definition}{Définition}
\newtheorem{properties}{Propriétés}
\newtheorem{lemma}{Lemme}
\newtheorem{property}[properties]{Propriété}
\newtheorem{example}{Exemple}
\newtheorem*{idea}{Éléments de preuve} % no numbered block



\setbeamertemplate{footline}{
  {\hfill\vspace*{1pt}\href{http://creativecommons.org/licenses/by-sa/3.0/legalcode}{\ccbysa}\hspace{.1cm}
    \raisebox{-.075cm}{\href{https://git.adjemian.eu/stepan/economic-calculus}{\includegraphics[scale=.1]{../img/gitlab.png}}}\enspace
    \href{https://git.adjemian.eu/stepan/economic-calculus/-/blob/\HEAD/cours/chapitre-2.tex}{\gitrevision}\enspace\today
  }\hspace{1cm}}

\setbeamertemplate{navigation symbols}{}
\setbeamertemplate{blocks}[rounded][shadow=true]
\setbeamertemplate{caption}[numbered]

\NewEnviron{notes}{\justifying\tiny\begin{spacing}{1.0}\BODY\vfill\pagebreak\end{spacing}}

\newenvironment{exercise}[1]
{\bgroup \small\begin{block}{Ex. #1}}
  {\end{block}\egroup}

\newenvironment{defn}[1]
{\bgroup \small\begin{block}{Définition. #1}}
  {\end{block}\egroup}

\newenvironment{exemple}[1]
{\bgroup \small\begin{block}{Exemple. #1}}
  {\end{block}\egroup}

\begin{document}

\title{Calcul Économique\\\small{II. Fonctions}}
\author[S. Adjemian]{Stéphane Adjemian}
\institute{\texttt{stephane.adjemian@univ-lemans.fr}} \date{Septembre 2020}

\begin{frame}
  \titlepage{}
\end{frame}

\begin{frame}
  \frametitle{Plan}
  \tableofcontents
\end{frame}


\section{Représentation graphique des fonctions}

\begin{frame}
  \frametitle{Réprésenter graphiquement une fonction, I}
  \hypertarget{slide_plan_cartesien_1}{}

  \begin{itemize}

  \item Une fonction est un ensemble de paires ordonnées, construites
    à partir du \textbf{produit cartésien} de deux ensembles, tel que chaque
    élément de l'ensemble de départ est associé à un et un seul
    élément de l'ensemble d'arrivée.\newline

  \item Dans le chapitre précédent nous avons donné comme exemple de fonction~:
    \[
      B = \{(x,y) | x\in\mathbb N \land y = 2x-1\}
    \]

  \item Pour réprésenter les fonctions on utilise un \textbf{plan cartésien}.\newline

  \item Les éléments de l'ensemble de départ sont représentés sur une ligne horizontale (l'axe des abscisses).\newline

  \item Les éléments de l'ensemble d'arrivée sont représentés sur une ligne verticale (l'axe des ordonnées).\newline

  \item Chaque paire représente les coordonnées d'un point dans le plan.

  \end{itemize}

\end{frame}


\begin{frame}
  \frametitle{Réprésenter graphiquement une fonction, II}
  \framesubtitle{L'ensemble $B$ dans le plan cartésien}
  \hypertarget{slide_plan_cartesien_2}{}

  \begin{center}
    \begin{tikzpicture}[scale=1.3]
      \begin{axis}[
        title={},
        width=10cm,height=7cm,
        xtick = {1, 2, 3, 4},
        xticklabels={1, 2, 3, 4},
        ytick = {7, 5, 3, 1, -1},
        yticklabels={7, 5, 3, 1, -1},
        enlargelimits=false,
        axis lines = middle,
        axis line style={thin,<->},
        xmin=-1,
        xmax=5,
        ymax=10,
        ymin=-3,
        ]
        \addplot[mark=*] coordinates {(0,-1)};
        \addplot[mark=*] coordinates {(1,1)};
        \addplot[mark=*] coordinates {(2,3)};
        \addplot[mark=*] coordinates {(3,5)};
        \addplot[mark=*] coordinates {(4,7)};
        \node[right] at (-.45, 9.1){{\small{$y$}}};
        \node[right] at (4.6, -.6){{\small{$x$}}};
      \end{axis}
    \end{tikzpicture}
  \end{center}

\end{frame}


\section{Les droites}


\begin{frame}
  \frametitle{Les droites, I}
  \hypertarget{slide_droites_1}{}

  \bigskip

  La fonction la plus simple que nous puissions considérer est la
  droite. Celle-ci est caractérisée par l'équation~:
  \[
    y = a x + b
  \]
  où $a$ et $b$ sont des paramètres réels, il s'agit d'une fonction de $\mathbb R$ dans $\mathbb R$.\newline

  \bigskip

  \begin{columns}
    \begin{column}{0.5\textwidth}
      {\small\begin{itemize}
        \item[--] Le paramètre $a$ est la pente de la droite.
        \item[--] Si $f(x) = ax+b$, alors~:
          \[
            a = \frac{f(x_0+h)-f(x_0)}{h}
          \]
          pour tout $x_0\in\mathbb R$ et $h\in\mathbb R^{\star}$
        \item[--] Le paramètre $b$ est l'ordonnée à l'origine.
        \item[--] Si $f(x) = ax+b$, alors~:
          \[
            f(0) = b
          \]
        \end{itemize}}
    \end{column}
    \begin{column}{0.5\textwidth}
      \begin{tikzpicture}[scale=.7]
        \begin{axis}[
          title={Ceci est une droite},
          xlabel=$x$,
          ylabel=$y$,
          xticklabels={,,},
          yticklabels={,,},
          enlargelimits=true,
          grid style={dashed, gray!60},
          axis x line = bottom,
          axis y line = left,
          axis line style={thin},
          ymax = 12,
          ymin = -1,
          xmax = 10,
          xmin = -1,
          axis lines = middle,
          small,
          clip=false,
          ]
          \addplot[
          draw=black,
          thick,
          smooth,
          samples=2,
          domain=-1:10,
          ]
          {x+2};
          \addplot[mark=*] coordinates {(0,2)};
          \node[left] at (0, 2){{\small{$b$}}};
          \addplot[dashed] coordinates {
            (4,0)
            (4,6)
            (0,6)};
          \node[left] at (0, 6){{\small{$f(x_0)$}}};
          \node[below] at (4, 0){{\small{$x_0$}}};
          \addplot[dashed] coordinates {
            (6,0)
            (6,8)
            (0,8)};
          \node[left] at (0, 8){{\small{$f(x_0+h)$}}};
          \node[below] at (6, 0){{\small{$x_0+h$}}};
        \end{axis}
      \end{tikzpicture}
    \end{column}
  \end{columns}

\end{frame}


\begin{frame}
  \frametitle{Les droites, II}
  \hypertarget{slide_droites_2}{}

  \bigskip

  \begin{itemize}

  \item On peut faire beaucoup de choses avec des droites. Vous verrez plus tard que l'on utilise souvent des droites pour approximer des fonctions plus générales (et complexes).\newline

  \item Ce que vous devez savoir faire~:\newline

    \begin{enumerate}

    \item Tracer une droite dans un plan,\newline

    \item Retrouver l'équation d'une droite à partir d'un tracé,\newline

    \item Trouver l'intersection d'une droite et de l'axe des abscisses,\newline

    \item Calculer le point d'intersection de deux droites.\newline

    \end{enumerate}

  \end{itemize}

\end{frame}


\begin{frame}
  \frametitle{Les droites, III}
  \framesubtitle{Tracer une droite}
  \hypertarget{slide_droites_3}{}

  \bigskip

  Il suffit de se donner deux points sur la droite, de représenter ces deux points dans le plan, puis de les relier (avec une rêgle).\newline

  \bigskip

  \begin{columns}
    \begin{column}{0.5\textwidth}
      {\small\begin{itemize}
        \item[--] Soit la droite $y = ax + b \equiv f(x)$.\newline

        \item[--] On sait qu'elle passe par $(0,b)$ ($b$ est l'ordonnée à l'origine).\newline

        \item[--] On se donne un autre point, par exemple $(1,a+b)$ appartient aussi à la droite.\newline
        \end{itemize}}
    \end{column}
    \begin{column}{0.5\textwidth}
      \begin{tikzpicture}[scale=.7]
        \begin{axis}[
          xlabel=$x$,
          ylabel=$y$,
          xticklabels={,,},
          yticklabels={,,},
          enlargelimits=true,
          grid style={dashed, gray!60},
          axis x line = bottom,
          axis y line = left,
          axis line style={thin},
          ymax = 12,
          ymin = -1,
          xmax = 10,
          xmin = -1,
          axis lines = middle,
          small,
          clip=false,
          ]
          \only<1->{
            \addplot[mark=*] coordinates {(0,2)};
            \node[left] at (0, 2){{\small{$b$}}};
            \node[below left] at (0, 0){{\small{0}}};
            \node[left] at (0, 8){{\small{$a+b$}}};
            \node[below] at (6,0) {{\small{1}}};
          }
          \only<2->{
            \addplot[mark=*] coordinates {(6,8)};
            \addplot[dashed] coordinates {
              (6,0)
              (6,8)
              (0,8)};
          }
          \only<3->{
            \addplot[
            draw=black,
            thick,
            smooth,
            samples=2,
            domain=-1:10,
            ]
            {x+2};
          }
        \end{axis}
      \end{tikzpicture}
    \end{column}
  \end{columns}

\end{frame}


\begin{frame}
  \frametitle{Les droites, IV}
  \framesubtitle{Déterminer l'équation d'une droite (a)}
  \hypertarget{slide_droites_4_1}{}

  \bigskip

  \begin{itemize}

  \item Supposons qu'une droite passe par les points $(x_0,y_0)$ et $(x_1,y_1)$.\newline

  \item Déterminons l'équation de la droite qui passe par ces deux
    points, c'est-à-dire déterminons la pente ($a$) et l'ordonnée à
    l'origine ($b$) de la droite qui passe par ces deux points.\newline

  \item Les paramètres $a$ et $b$ sont tels que~:
    \[
      \begin{cases}
        y_0 &= a x_0 + b\\
        y_1 &= a x_1 + b
      \end{cases}
    \]
    Nous avons deux incconues et deux équations.\newline

  \item La seconde équation peut s'écrire de façon équivalente~:
    \[
      b = y_1 - a x_1
    \]
  \end{itemize}

\end{frame}


\begin{frame}
  \frametitle{Les droites, IV}
  \framesubtitle{Déterminer l'équation d'une doite (b)}
  \hypertarget{slide_droites_4_2}{}

  \bigskip

  \begin{itemize}

  \item En substituant cette expression de $b$ dans la première équation~:
    \[
      y_0 = a x_0 +y_1 - a x_1
    \]

  \item D'où finalement~:
    \[
      a = \frac{y_0-y_1}{x_0-x_1}\quad\text{(pente)}
    \]

  \item Et donc~:
    \[
      b = y_1 - \frac{y_0-y_1}{x_0-x_1} x_1\quad\text{(ordonnée à l'origine)}
    \]

  \item Notons que tout cela ne fonctionne que si $x_0\neq x_1$ (voir l'expression de la pente)\ldots Les deux points doivent être différents~! Autrement il n'est pas possible d'identifier la droite.

  \end{itemize}

\end{frame}


\begin{frame}
  \frametitle{Les droites, IV}
  \framesubtitle{Déterminer l'équation d'une doite (c)}
  \hypertarget{slide_droites_4_3}{}

  \begin{Center}
    \begin{tikzpicture}[scale=1.3]
      \begin{axis}[
        xticklabels={,,},
        yticklabels={,,},
        enlargelimits=true,
        grid style={dashed, gray!60},
        axis x line = bottom,
        axis y line = left,
        axis line style={thin},
        ymax = 12,
        ymin = -1,
        xmax = 10,
        xmin = -1,
        axis lines = middle,
        small,
        clip=false,
        ]
        \addplot[
        draw=black,
        thick,
        smooth,
        samples=2,
        domain=-1:10,
        ]
        {x+2};
        \node[left] at (0, 2){{\small{$y_1 - \frac{y_0-y_1}{x_0-x_1} x_1$}}};
        \addplot[dashed] coordinates {
          (3,0)
          (3,5)
          (0,5)};
        \addplot[mark=*] coordinates {(3,5)};
        \node[left] at (0, 5){{\small{$y_0$}}};
        \node[below] at (3, 0){{\small{$x_0$}}};
        \addplot[dashed] coordinates {
          (6,0)
          (6,8)
          (0,8)};
        \addplot[mark=*] coordinates {(6,8)};
        \node[left] at (0, 8){{\small{$y_1$}}};
        \node[below] at (6, 0){{\small{$x_1$}}};
      \end{axis}
    \end{tikzpicture}
  \end{Center}

\end{frame}


\begin{frame}
  \frametitle{Les droites, V}
  \framesubtitle{Intersection d'une droite avec l'axe des abscisses}
  \hypertarget{slide_droites_5}{}

  \bigskip

  \begin{itemize}

  \item On cherche la valeur de $x$ telle que $ax+b = 0$.\newline

  \item Il s'agit donc de résoudre une simple équation linéaire~: $x = -\nicefrac{b}{a}$.\newline

  \end{itemize}

  \bigskip

  \begin{center}
    \begin{tikzpicture}[scale=1.3]
      \begin{axis}[
        xticklabels={,,},
        yticklabels={,,},
        enlargelimits=true,
        grid style={dashed, gray!60},
        axis x line = bottom,
        axis y line = left,
        axis line style={thin},
        ymax = 8,
        ymin = -2,
        xmax = 5,
        xmin = -5,
        axis lines = middle,
        small,
        clip=false,
        ]
        \addplot[
        draw=black,
        thick,
        smooth,
        samples=2,
        domain=-4:4,
        ]
        {x+2};
        \node[left] at (0, 2){{\small{$b$}}};
        \node[below] at (-2, 0){{\small{$-\nicefrac{b}{a}$}}};
      \end{axis}
    \end{tikzpicture}

  \end{center}

\end{frame}


\begin{frame}
  \frametitle{Les droites, VI}
  \framesubtitle{Intersection de deux droites (a)}
  \hypertarget{slide_droites_6_1}{}

  \bigskip

  \begin{itemize}

  \item Soient deux droites distinctes~:
    \begin{eqnarray*}
      y &= a_1 x + b_1\\
      y &= a_2 x + b_2
    \end{eqnarray*}
    où $a_1\neq a_2$ (autrement les droites sont parallèles et n'admettent donc pas d'intersection).\newline

  \item On cherche le point d'intersection $(x^{\star},y^{\star})$, celui-ci doit vérifier~:
    \[
      a_1 x^{\star} + b_1 = a_2 x^{\star} + b_2
    \]
    soit de façon équivalente~:
    \[
      x^{\star} = \frac{b_2-b_1}{a_1-a_2}
    \]

  \end{itemize}

\end{frame}


\begin{frame}
  \frametitle{Les droites, VI}
  \framesubtitle{Intersection de deux droites (b)}
  \hypertarget{slide_droites_6_2}{}

  \bigskip

  \begin{itemize}

  \item Et on déduit~:
    \[
      y^{\star} = a_1 x^{\star} + b_1
    \]

  \end{itemize}

  \begin{center}
    \begin{tikzpicture}[scale=1.3]
      \begin{axis}[
        xticklabels={,,},
        yticklabels={,,},
        enlargelimits=true,
        grid style={dashed, gray!60},
        axis x line = bottom,
        axis y line = left,
        axis line style={thin},
        ymax = 8,
        ymin = -2,
        xmax = 15,
        xmin = -5,
        axis lines = middle,
        small,
        clip=false,
        ]
        \addplot[
        draw=black,
        thick,
        smooth,
        samples=2,
        domain=-4:6,
        ]
        {x+2};
        \node[left] at (0, 2){{\tiny{$b_1$}}};
        \node[below] at (-2, 0){{\tiny{$-\frac{b_1}{a_1}$}}};
        \addplot[
        draw=red,
        thick,
        smooth,
        samples=2,
        domain=-4:16,
        ]
        {-.5*x+7};
        \node[left] at (0, 7){{\color{red}\tiny{$b_2$}}};
        \node[below] at (14, 0){{\color{red}\tiny{$-\frac{b_2}{a_2}$}}};
        %
        % Intersection
        %
        \addplot[mark=*] coordinates {(3.3333,5.3333)};
        \addplot[dashed] coordinates {
          (3.3333,0)
          (3.3333,5.3333)
          (0,5.3333)};
        \node[below] at (3.3333, 0){{\tiny{$\frac{b_2-b_1}{a_1-a_2}$}}};
        \node[left] at (0, 5.3333){{\tiny{$a_1\frac{b_2-b_1}{a_1-a_2}+b_1$}}};
      \end{axis}
    \end{tikzpicture}

  \end{center}

\end{frame}


\section{Fonctions polynomiales}


\begin{frame}
  \frametitle{Polynômes, I}
  \framesubtitle{Polynôme de degré $n$}
  \hypertarget{slide_polynome_definition}{}

  \bigskip

  \begin{definition}
    On appelle polynôme de degré $n\in\mathbb N$ la fonction de $\mathbb R$ dans $\mathbb R$ donnée par~:
    \[
      P_n(x) = \alpha_n x^n + \alpha_{n-1}x^{n-1}+\ldots+\alpha_1x+\alpha_0
    \]
    où les coefficients $\{\alpha_i\}_{i=1}^n$ sont réels et $\alpha_n\neq 0$.
  \end{definition}

  \bigskip

  \begin{itemize}

  \item On peut voir le polynôme comme une généralisation de la droite.\newline

  \item Pour $n=1$, on a~:
    \[
      f(x) = \alpha_1 x + \alpha_0
    \]
    c'est-à-dire l'équation de la droite.\newline

  \item Les puissances de $x$, c'est-à-dire les termes $x^i$ pour $i=1,\ldots,n$, sont des monômes.\newline

  \item Un polynôme est une combinaison linéaire de monômes.

  \end{itemize}

\end{frame}


\begin{frame}
  \frametitle{Polynômes, II}
  \framesubtitle{Exemple, polynôme de degré 2 (la parabole)}
  \hypertarget{slide_polynome_exemple_1}{}

  \bigskip
  \begin{columns}
    \begin{column}{0.5\textwidth}
      \begin{tikzpicture}[scale=.9]
        \begin{axis}[
          xticklabels={,,},
          yticklabels={,,},
          enlargelimits=true,
          grid style={dashed, gray!60},
          axis x line = bottom,
          axis y line = left,
          axis line style={thin},
          xmax = 4,
          xmin = -2,
          axis lines = middle,
          small,
          clip=false,
          ]
          \addplot[
          draw=black,
          thick,
          smooth,
          samples=50,
          domain=-2:2,
          ]
          {x^2} node(p1){} ;
          \node [right] at (p1) {$x^2$};
          \addplot[
          draw=black,
          thick,
          smooth,
          samples=50,
          domain=-1:3,
          ]
          {x^2-2*x-1}  node(p2){} ;
          \node [right] at (p2) {$x^2-2x-1$};
        \end{axis}
      \end{tikzpicture}
    \end{column}
    \begin{column}{0.5\textwidth}
      \begin{tikzpicture}[scale=.9]
        \begin{axis}[
          xticklabels={,,},
          yticklabels={,,},
          enlargelimits=true,
          grid style={dashed, gray!60},
          axis x line = bottom,
          axis y line = left,
          axis line style={thin},
          xmax = 4,
          xmin = -4,
          ymax = 2,
          axis lines = middle,
          small,
          clip=false,
          ]
          \addplot[
          draw=black,
          thick,
          smooth,
          samples=50,
          domain=-2:2,
          ]
          {-x^2-1} node(p1){} ;
          \node [right] at (p1) {$-x^2-1$};
          \addplot[
          draw=black,
          thick,
          smooth,
          samples=50,
          domain=-2:3,
          ]
          {-x^2+2*x+1}  node(p2){} ;
          \node [right] at (p2) {$-x^2+2x+1$};
        \end{axis}
      \end{tikzpicture}
    \end{column}
  \end{columns}

\end{frame}


\begin{frame}
  \frametitle{Polynômes, II}
  \framesubtitle{Exemple, polynôme de degré 3}
  \hypertarget{slide_polynome_exemple_2}{}

  \bigskip
  \begin{center}
    \begin{tikzpicture}[scale=1.3]
      \begin{axis}[
        xticklabels={,,},
        yticklabels={,,},
        enlargelimits=true,
        grid style={dashed, gray!60},
        axis x line = bottom,
        axis y line = left,
        axis line style={thin},
        xmax = 4,
        xmin = -2,
        axis lines = middle,
        small,
        clip=false,
        ]
        \addplot[
        draw=black,
        thick,
        smooth,
        samples=50,
        domain=-2:2,
        ]
        {x^3} node(p1){} ;
        \node [right] at (p1) {$x^3$};
      \end{axis}
    \end{tikzpicture}
  \end{center}

\end{frame}


\begin{frame}
  \frametitle{Polynômes, II}
  \framesubtitle{Exemple, polynôme de degré 5}
  \hypertarget{slide_polynome_exemple_3}{}

  \bigskip
  \begin{center}
    \begin{tikzpicture}[scale=1.3]
      \begin{axis}[
        xticklabels={,,},
        yticklabels={,,},
        enlargelimits=true,
        grid style={dashed, gray!60},
        axis x line = bottom,
        axis y line = left,
        axis line style={thin},
        xmax = 3,
        xmin = -3,
        axis lines = middle,
        small,
        clip=false,
        ]
        \addplot[
        draw=black,
        thick,
        smooth,
        samples=50,
        domain=-2.25:2.25,
        ]
        {x^5-5*x^3+4*x} node(p1){} ;
        \node [right] at (p1) {$x^5-5x^3+4x$};
      \end{axis}
    \end{tikzpicture}
  \end{center}

\end{frame}


\begin{frame}
  \frametitle{Polynômes, III}
  \framesubtitle{Racines d'un polynôme}
  \hypertarget{slide_polynome_racines_1}{}

  \bigskip

  \begin{definition}
    Une racine $x^{\star}$ d'un polynôme de degré $n$, $P_n(x)$, est une
    valeur de $x$ telle que $P_n(x^{\star})=0$.
  \end{definition}

  \bigskip

  \begin{itemize}

  \item Un polynôme peut avoir plusieurs racines, on verra plus loin que le nombre de racines est lié au degré du polynôme.\newline

  \item Graphiquement, les racines réelles d'un polynôme correspondent à l'intersection de la courbe représentative du polynôme et de l'axe des abscisses.

  \end{itemize}

\end{frame}


\begin{frame}
  \frametitle{Polynômes, III}
  \framesubtitle{Division euclidienne}
  \hypertarget{slide_division_1}{}

  \bigskip

  \begin{theorem}
    Soient deux polynômes $S(x)$ et $T(x)\neq 0$, alors il existe deux polynômes $Q(x)$ et $R(x)$ uniques tels que $S(x) = Q(x)T(x)+R(x)$, avec $R(x)=0$ ou un polynôme de degré inférieur au polynôme $T(x)$.
  \end{theorem}

  \bigskip

  \begin{itemize}

  \item $Q$ est le quotient de la division euclidienne de $S$ par $T$.\newline

  \item $R$ est le reste de la division euclidienne.\newline

  \item Si le reste de la division euclidienne est nul, on dit qu'on a factorisé le polynôme $S$, puisque $S(x)=Q(x)T(x)$.

  \end{itemize}

\end{frame}


\begin{frame}
  \frametitle{Polynômes, III}
  \framesubtitle{Division euclidienne, exemple 1}
  \hypertarget{slide_division_exemple_1}{}

  \begin{itemize}

  \item Soient les polynômes $S(x) = 6x^3-2x^2+x+3$ et $T(x) = x^2-x+1$.\newline

  \item La division de $S(x)$ par $T(x)$~:\newline

    \begin{Center}
      \polyset{style=D}
      \polylongdiv{6x^3-2x^2+x+3}{x^2-x+1}
    \end{Center}

    \bigskip

  \item Le quotient est $Q(x) = 6x+4$ et le reste $R(x) = -x-1$.

  \end{itemize}

\end{frame}


\begin{frame}
  \frametitle{Polynômes, III}
  \framesubtitle{Division euclidienne, exemple 2}
  \hypertarget{slide_division_exemple_2}{}

  \bigskip

  \begin{itemize}

  \item Soient les polynômes $S(x) = x^3+6x^2-x-30$ et $T(x) = x+5$.\newline

  \item La division de $S(x)$ par $T(x)$~:\newline

    \begin{Center}
      \polyset{style=D}
      \polylongdiv{x^3+6x^2-x-30}{x+5}
    \end{Center}

    \bigskip

  \item Le quotient est $Q(x) = x^2+x-6$ et le reste est nul.\newline

  \item On peut donc écrire~:
    \[
      x^3+6x^2-x-30 = (x+5)(x^2+x-6)
    \]

  \end{itemize}

\end{frame}


\begin{frame}
  \frametitle{Polynômes, III}
  \framesubtitle{Division euclidienne et racines}
  \hypertarget{slide_division_racine}{}

  \bigskip

  \begin{itemize}

  \item Lorsque le reste de la divison euclidienne est nul, c'est-à-dire lorsqu'il est possible d'écrire $S(x)=Q(x)T(x)$, alors les racines de $Q(x)$ sont des racines de $S(x)$ et les racines de $T(x)$ sont des racines de $S(x)$.\newline

  \item Dans l'autre sens, si $x^{\star}$ est une racine de $S(x)$ alors $x^{\star}$ est une racine de $Q(x)$ ou une racine de $T(x)$.\newline

  \item Dans le dernier exemple on a $T(x)=x+5$, donc $-5$ est une racine de $S(x)=x^3+6x^2-x-30$.\newline

  \item Plus généralement, si $x^{\star}$ est la racine d'un polynôme $S(x)$, alors on peut toujours factoriser celui-ci sous la forme~: $S(x) = (x-x^{\star}) Q(x)$.
  \end{itemize}

\end{frame}


\begin{frame}
  \frametitle{Polynômes, IV}
  \framesubtitle{Racines d'un polynôme de degré 2 (a)}
  \hypertarget{slide_polynome_2_racines_1}{}

  \bigskip

  \begin{itemize}

  \item Un polynôme de degré 2 est une fonction de $\mathbb R$ dans $\mathbb R$ définie par~:
    \[
      f(x) = ax^2 + bx + c
    \]
    avec $a\neq 0$, $b$ et $c$ des paramètres réels.\newline

  \item Une racine du polynôme de degré 2 est une valeur de $x$, notée $x^{\star}$, telle que~:
    \[
      a \left.x^{\star}\right.^2 + b x^{\star} + c = 0
    \]

    \bigskip

  \item Cette équation admet au plus deux racines réelles.\newline

  \item Si $x_1^{\star}$ et $x_2^{\star}$ sont deux racines réelles, alors on a~:
    \[
      a x^2 + b x + c = a(x-x_1^{\star})(x-x_2^{\star})
    \]
    pour tout $x\in\mathbb R$

  \end{itemize}

\end{frame}


\begin{frame}
  \frametitle{Polynômes, IV}
  \framesubtitle{Racines d'un polynôme de degré 2 (b)}
  \hypertarget{slide_polynome_2_racines_2}{}

  \bigskip

  \begin{example}
    Soit le polynôme de degré deux $f(x) = x^2-1$. Les racines de ce polynôme sont les solutions de l'équation~:
    \[
      x^2 - 1 = 0
    \]
    c'est-à-dire les valeurs de $x$ telles que $x^2=1$. On voit immédiatement qu'il existe deux solutions~:
    \[
      x_1^{\star} = -1\quad\text{ et }\quad x_2^{\star} = 1
    \]
    et on retrouve donc une identité remarquable bien connue~:
    \[
      x^2-1 = (x+1)(x-1)
    \]
  \end{example}

\end{frame}


\begin{frame}
  \frametitle{Polynômes, IV}
  \framesubtitle{Racines d'un polynôme de degré 2 (c)}
  \hypertarget{slide_polynome_2_racines_3}{}

  \bigskip

  \begin{example}
    Soit le polynôme de degré deux $f(x) = x^2-2x+1$. Les racines de ce polynôme sont les solutions de l'équation~:
    \[
      x^2 -2x + 1 = 0
    \]
    On reconnaît une identité remarquable~:
    \[
      x^2 -2x + 1 = (x-1)^2 = (x-1)(x-1)
    \]
    Ainsi $x^{\star} = 1$ est une racine. On dit qu'il s'agit d'une
    racine double (ou de multiplicité deux) à cause la puissance deux
    le terme $x-1$. On a donc~:
    \[
      x_1^{\star} = 1\quad\text{ et }\quad x_2^{\star} = 1
    \]
  \end{example}

\end{frame}


\begin{frame}
  \frametitle{Polynômes, IV}
  \framesubtitle{Racines d'un polynôme de degré 2 (d)}
  \hypertarget{slide_polynome_2_racines_4}{}

  \bigskip

  \begin{example}
    Soit le polynôme de degré deux $f(x) = x^2+1$. Les racines de ce
    polynôme sont les solutions de l'équation~:
    \[
      x^2 = -1
    \]
    Il n'existe pas de solution réelle à cette équation, car le carré
    d'un nombre réel est toujours positif.
  \end{example}

  \bigskip

  \begin{itemize}

  \item Pour trouver des solutions à cette équation il sortir de l'ensemble des réels.\newline

  \item Peut-on créer un ensemble où cette équation admette une (des) solution(s)~? $\Rightarrow$ Les nombres complexes\ldots

  \end{itemize}

\end{frame}


\begin{frame}
  \frametitle{Polynômes, IV}
  \framesubtitle{Le nombre imaginaire et l'ensemble des nombres complexes}
  \hypertarget{slide_polynome_2_imaginaire_1}{}

  \bigskip

  \begin{itemize}

  \item On définit l'ensemble des nombres complexes noté $\mathbb C$, dont $\mathbb R$ est un sous ensemble, en « imaginant » que l'équation $x^2=-1$ admet une solution que nous noterons $i$.\newline

  \item $i$ est le nombre imaginaire, il est défini comme $i = \sqrt{-1}$.\newline

  \item Un nombre complexe $x\in\mathbb C$ peut s'écrire sous la forme~:
    \[
      x = a + i\cdot b
    \]
    où $a\in\mathbb R$ est la partie réelle de $x$ et $b\in\mathbb R$ la partie imaginaire.\newline

  \end{itemize}

  \begin{center}
    \begin{tikzpicture}[scale=0.6]
      \begin{scope}[thick,font=\scriptsize]
        \draw [->] (-4,0) -- (4,0) node [right]  {$\Re\{x\}$};
        \draw [->] (0,-4) -- (0,4) node [above left] {$\Im\{x\}$};
        \foreach \n in {-3,...,-1,1,2,...,3}{%
          \draw (\n,-3pt) -- (\n,3pt)   node [above] {$\n$};}
        \draw (-3pt,1) -- (3pt,1)   node [right] {$i$};
        \draw (-3pt,2) -- (3pt,2)   node [right] {$2i$};
        \draw (-3pt,3) -- (3pt,3)   node [right] {$3i$};
        \draw [thin, dashed, color=black!80] (0,0) -- (2,3);
        \draw [color=black, fill=black] (2,3) circle(0.05);
        \node [color=black] at (3,3) {$ 2+3i$};
      \end{scope}
    \end{tikzpicture}

  \end{center}

\end{frame}


\begin{frame}
  \frametitle{Polynômes, IV}
  \framesubtitle{Arithmétique avec les nombres complexes}
  \hypertarget{slide_polynome_2_imaginaire_2}{}

  \bigskip

  Soient $x = a_x + i b_x$ et $y = a_y + i b_y$ deux nombres complexes.

  \bigskip

  \begin{itemize}

  \item \textbf{Somme de nombres complexes.}
    \[
      x+y = a_x+a_y + i(b_x+b_y)
    \]

    \bigskip

  \item \textbf{Produit de nombres complexes.}
    \[
      \begin{split}
        x \cdot y &= (a_x + i b_x)(a_y + i b_y)\\
        &= a_xa_y + i a_xb_y + ib_xa_y + i^2b_xb_y\\
        &= a_xa_y-b_xb_y + i(a_xb_y + b_xa_y)
      \end{split}
    \]

    \bigskip

  \item \textbf{Conjugué d'un nombre complexe.}
    \[
      \bar x = a_x - i b_x
    \]

  \end{itemize}

\end{frame}


\begin{frame}
  \frametitle{Polynômes, IV}
  \framesubtitle{Arithmétique avec les nombres complexes (suite)}
  \hypertarget{slide_polynome_2_imaginaire_3}{}

  \bigskip

  \begin{itemize}

  \item \textbf{Le produit d'un nombre complexe et de son conjugué est réel.}
    \[
      \begin{split}
        x \cdot \bar x &= (a_x + i b_x)(a_x - i b_x)\\
        &= a_x^2 + i a_xb_x - i a_x b_x - i^2 b_x^2\\
        &= a_x^2 + b_x^2
      \end{split}
    \]

    \bigskip

  \item \textbf{La norme d'un nombre complexe.}
    \[
      |x| = \sqrt{x \cdot \bar x} = \sqrt{a_x^2 + b_x^2}
    \]
    généralise la valeur absolue dans $\mathbb R$.

  \end{itemize}

\end{frame}


\begin{frame}
  \frametitle{Polynômes, IV}
  \framesubtitle{Arithmétique avec les nombres complexes (suite et fin)}
  \hypertarget{slide_polynome_2_imaginaire_4}{}

  \bigskip

  \begin{itemize}

  \item \textbf{Quotient de deux nombres complexes.}
    \[
      \begin{split}
        \frac{x}{y} &= \frac{x \cdot \bar y}{y \cdot \bar y}\\
        &= \frac{(a_x+ib_x)(a_y-ib_y)}{a_y^2 + b_y^2}\\
        &= \frac{a_xa_y-ia_xb_y+ib_xa_y-i^2b_xb_y}{a_x^2 + b_x^2}\\
        &= \frac{a_xa_y+b_xb_y+i(b_xa_y-a_xb_y)}{a_x^2 + b_x^2}\\
        &= \underbrace{\frac{a_xa_y+b_xb_y}{a_x^2 + b_x^2}}_{\Re\left\{\frac{x}{y}\right\}} + i \underbrace{\frac{b_xa_y-a_xb_y}{a_x^2 + b_x^2}}_{\Im\left\{\frac{x}{y}\right\}}
      \end{split}
    \]

  \end{itemize}

\end{frame}


\begin{frame}
  \frametitle{Polynômes, IV}
  \framesubtitle{Racines d'un polynôme de degré 2 (e)}
  \hypertarget{slide_polynome_2_racines_5}{}

  \bigskip

  \begin{block}{Le discriminant}
    Le discriminant d'un polynôme de degré 2 est défini comme~:
    \[
      \Delta = b^2-4ac
    \]
  \end{block}

  \bigskip

  \begin{itemize}

  \item La nature des racines dépend du signe du discriminant~:\newline

    \begin{itemize}
    \item[$\Delta>0$] Le polynôme possède deux racines réelles,\newline
    \item[$\Delta=0$] Le polynôme possède une racine réelle de multilplicité deux, et\newline
    \item[$\Delta<0$] Le polynôme possède deux racines complexes conjuguées.\newline
    \end{itemize}

  \item Un polynôme de degré deux possède toujours deux racines (dans $\mathbb C$)

  \end{itemize}

\end{frame}


\begin{frame}
  \frametitle{Polynômes, IV}
  \framesubtitle{Racines d'un polynôme de degré 2 (f)}
  \hypertarget{slide_polynome_2_racines_5}{}

  \bigskip

  \begin{theorem}\label{thm:poly_2_roots}
    Les racines d'un polynôme de degré 2 sont~:
    \[
      x^{\star} = \frac{-b \pm \sqrt{\Delta}}{2a}
    \]
    avec $\Delta = b^2-4ac$ le discriminant.
  \end{theorem}

  \bigskip

  \begin{itemize}

  \item Si $\Delta=0$ (une racine de multiplicité deux) on a~: $x^{\star} = -\nicefrac{b}{2a}$.\newline

  \item Les racines complexes conjuguées apparaissent quand $\Delta<0$ à cause de la racine carrée.\newline

  \end{itemize}

\end{frame}


\begin{notes}
  \textbf{Preuve du théorème \hyperlink{slide_polynome_2_racines_5}{\ref{thm:poly_2_roots}}} Le polynôme dont nous cherchons les racines est~:
  \[
    f(x) = a x^2 + b x + c
  \]
  Les racines de $f(x)$ sont aussi les racines de~:
  \[
    g(x) = x^2 +\frac{b}{a} x + \frac{c}{a}
  \]
  on a simplement normalisé le polynôme en factorisant le paramètre $a$. Pour montrer que les racines sont bien celles données dans le théorème, nous allons factoriser $g(x)$ en utilisant successivement deux identités remarquable~: $(\alpha+\beta)^2=\alpha^2+2\alpha\beta+\beta^2$ et $\gamma^2-\delta^2 = (\gamma-\delta)(\gamma+\delta)$. Réécrivons $g(x)$ de façon à pouvoir utiliser la première identité remarquable~:
  \[
    \begin{split}
      g(x) &= x^2 + 2\frac{b}{2a} x +  \frac{b^2}{4a^2} - \frac{b^2}{4a^2} + \frac{c}{a}\\
      &= \left(x+\frac{b}{2a}\right)^2- \frac{b^2}{4a^2} + \frac{c}{a}\\
      &= \left(x+\frac{b}{2a}\right)^2- \left(\frac{b^2}{4a^2} - \frac{c}{a}\right)\\
    \end{split}
  \]
  On peut alors faire apparaître le discriminant dans le dernier terme~:
  \[
    \begin{split}
      g(x) &= \left(x+\frac{b}{2a}\right)^2- \frac{b^2-4ac}{4a^2}\\
      &= \left(x+\frac{b}{2a}\right)^2- \frac{\Delta}{4a^2}
    \end{split}
  \]
  Il ne reste plus qu'à utiliser la seconde identité remarquable~:
  \[
    g(x) = \left(x+\frac{b}{2a}-\frac{\sqrt{\Delta}}{2a}\right)\left(x+\frac{b}{2a}+\frac{\sqrt{\Delta}}{2a}\right) = \left(x+\frac{b-\sqrt{\Delta}}{2a}\right)\left(x+\frac{b+\sqrt{\Delta}}{2a}\right)\qed
  \]

\end{notes}


\begin{frame}
  \frametitle{Polynômes, V}
  \framesubtitle{Racines d'un polynôme de degré supérieur à 2}
  \hypertarget{slide_polynome_2_racines_5_1}{}

  \bigskip

  \begin{itemize}

  \item De la même façon que pour les polynôme d'ordre 2, il existe des formules (par radicaux, c'est-à-dire qui n'utilisent que les opérations usuelles et des racines) pour calculer les solutions des équations  polynomiales de degré 3 ou 4\ldots Mais ces formules sont assez difficile à lire\ldots\newline

  \item En pratique, dans la vie d'un étudiant, on cherche des « racines évidentes » (petits entiers), et factorise le  polynôme pour réduire le degré du polynôme dont il restera à calculer les racines.\newline

  \item En pratique, dans la vraie vie, on utilise un ordinateur pour calculer numériquement les racines.\newline

  \end{itemize}

\end{frame}


\begin{frame}
  \frametitle{Polynômes, V}
  \framesubtitle{Calculer des racines en utilisant des solutions évidentes (a)}
  \hypertarget{slide_polynome_2_racines_5_2}{}

  \bigskip

  \begin{itemize}

  \item Soit le polynôme de degré 3~:
    \[
      f(x) = x^3-\frac{7}{4}x^2 + \frac{7}{8}x - \frac{1}{8}
    \]

  \item On note que pour $x=1$, on a~:
    \[
      f(1) = 1-\frac{7}{4}+\frac{7}{8} -\frac{1}{8} = \frac{8-14+7-1}{8} = 0
    \]

  \item On sait donc qu'on peut factoriser $(x-1)$, c'est-à-dire écrire $f(x)$ comme le produit d'un polynôme de degré 2 et de $(x-1)$.\newline

  \item Pour trouver le polynôme de degré 2 on peut faire une division euclidienne, ou procéder par la méthode dite des « coefficients indéterminés ».\newline

  \item Postulons $f(x) = (x-1)(ax^2+bx+c)$ et identifions les paramètres $a$, $b$ et $c$. Nous devons donc avoir~:
    \[
      x^3-\frac{7}{4}x^2+\frac{7}{8}x -\frac{1}{8} = (x-1)(ax^2+bx+c)
    \]

  \end{itemize}

\end{frame}


\begin{frame}
  \frametitle{Polynômes, V}
  \framesubtitle{Calculer des racines en utilisant des solutions évidentes (b)}
  \hypertarget{slide_polynome_2_racines_5_3}{}

  \bigskip

  \begin{itemize}

  \item C'est-à-dire~:
    \[
      \begin{split}
        x^3-\frac{7}{4}x^2+\frac{7}{8}x -\frac{1}{8} &= ax^3 + bx^2 + cx - ax^2 - bx - c\\
        &= ax^3 + (b-a)x^2 + (c-b)x -c
      \end{split}
    \]

  \item Par identification, on a donc~:
    \[
      \begin{cases}
        1 &= a\\
        -\frac{7}{4} &= b-a\\
        \frac{7}{8} &= c-b\\
        \frac{1}{8} &= c
      \end{cases}
      \quad
      \Leftrightarrow
      \quad
      \begin{cases}
        a &= 1\\
        b &= -\frac{3}{4}\\
        c &= \frac{1}{8}
      \end{cases}
    \]

  \item Et donc~:
    \[
      x^3-\frac{7}{4}x^2+\frac{7}{8}x -\frac{1}{8} = (x-1)\left(x^2-\frac{3}{4}x+\frac{1}{8}\right)
    \]

  \end{itemize}

\end{frame}


\begin{frame}[fragile]
  \frametitle{Polynômes, V}
  \framesubtitle{Calculer les racines numériquement}
  \hypertarget{slide_polynome_2_racines_5_4}{}

  \bigskip

  \begin{itemize}

  \item Avec Python:\newline

    \begin{minted}{python}
      >>> import numpy as np
      >>> np.roots([1, -7.0/4.0, 7.0/8.0, -1.0/8.0])
      array([1.  , 0.5 , 0.25])
    \end{minted}

    \bigskip

  \item[\dbend] Ordre inverse des paramètres par rapport à nos notations.\newline

    \bigskip

  \item Il existe des algorithmes plus généraux pour trouver les zéros d'une fonction, c'est à dire des valeurs de $x$ telles que $f(x)=0$. On verra une version simple de ces algorithmes (qui reposent souvent sur des calculs de dérivées).

  \end{itemize}

\end{frame}


\begin{frame}
  \frametitle{Fonctions rationnelles, I}
  \hypertarget{slide_fonction_rationnelle_definition}{}

  \bigskip

  \begin{definition}
    Une fonction rationnelle est une fonction de la forme $\frac{p(x)}{q(x)}$, où $p(x)$ et $q(x)$ sont des fonctions polynomiales, c'est-à-dire~:
    \[
      f(x) = \frac{\alpha_n x^n + \alpha_{n-1}x^{n-1} + \ldots + \alpha_1 x + \alpha_0}{\beta_m x^m + \beta_{m-1}x^{m-1} + \ldots + \beta_1 x + \beta_0}
    \]
  \end{definition}

  \bigskip

  \begin{itemize}

  \item Si $q(x)$ est un polynôme de degré 0 alors $f(x)$ est une fonction polynomiale.\newline

  \item La fonction $f$ est à valeur dans $\mathbb R$, mais le domaine de la fonction n'est généralement pas $\mathbb R$. Il faut exclure les points où $q(x)$ est nulle, c'est-à-dire les racines de $q(x)$. Ainsi~:
    \[
      \begin{split}
        f: \quad \{x\in\mathbb R|q(x)\neq 0\} &\longrightarrow \mathbb R\\
        x &\longmapsto y = \frac{p(x)}{q(x)}
      \end{split}
    \]

  \end{itemize}

\end{frame}


\begin{frame}
  \frametitle{Fonctions rationnelles, II}
  \hypertarget{slide_fonction_rationnelle_exemple}{}

  \bigskip

  \begin{center}
    \begin{tikzpicture}[scale=1.7]
      \begin{axis}[
        xticklabels={,,},
        yticklabels={,,},
        enlargelimits=true,
        grid style={dashed, gray!60},
        axis x line = bottom,
        axis y line = left,
        axis line style={thin},
        xmax = 5,
        xmin = -5,
        ymax = 7,
        ymin = -9,
        axis lines = middle,
        small,
        clip=false,
        ]
        \addplot[
        draw=black,
        thick,
        smooth,
        samples=50,
        domain=-4:-1.1,
        ]
        {(x-.5)/(x^3-x)} ;
        \addplot[
        draw=black,
        thick,
        smooth,
        samples=50,
        domain=-.9:-.075,
        ]
        {(x-.5)/(x^3-x)} ;
        \addplot[
        draw=black,
        thick,
        smooth,
        samples=50,
        domain=.075:.97,
        ]
        {(x-.5)/(x^3-x)} ;
        \addplot[
        draw=black,
        thick,
        smooth,
        samples=50,
        domain=1.05:4,
        ]
        {(x-.5)/(x^3-x)} ;
        \addplot[
        draw=red,
        dashed,
        ]
        coordinates {(-1,-9) (-1,7)} ;
        \addplot[
        draw=red,
        dashed,
        ]
        coordinates {(1,-9) (1,7)} ;
        \node [right] at (-6,6) {$f(x) = \frac{x-\frac{1}{2}}{x^3-x}$};
      \end{axis}
    \end{tikzpicture}
  \end{center}

\end{frame}


\section{Fonctions puissances}

\begin{frame}
  \frametitle{Fonctions puissances, I}
  \hypertarget{slide_fonctions_puissance_1}{}

  \bigskip

  \begin{definition}
    On appelle fonction puissance les fonctions de la forme~:
    \[
      f(x) = x^{\alpha}
    \]
    où $\alpha$ est une constante réelle.
  \end{definition}

  \bigskip

  \begin{itemize}

  \item Le domaine de défnition dépend de $\alpha$.\newline

  \item Si $\alpha\in\mathbb Q$, c'est-à-dire s'il existe $p$ et $q$ dans $\mathbb N$ tels que $\alpha=\nicefrac{p}{q}$, alors on écrit~: $x^{\alpha} = x^{\frac{p}{q}} = \sqrt[q]{x^p}$.\newline

  \item Si $\alpha$ est un entier naturel, alors le domaine de définition est $\mathbb R$.

  \item Si $\alpha$ est un entier négatif, alors le domaine de définiton est $\mathbb R\setminus \{0\}$.

  \item Si $\alpha=\nicefrac{1}{q}$, le domaine de définiton est $\mathbb R$ si $q$ est impair et $\mathbb R_+$ sinon.

  \end{itemize}

\end{frame}


\begin{frame}
  \frametitle{Fonctions puissances, II}
  \framesubtitle{Exemple}
  \hypertarget{slide_fonctions_puissance_2}{}

  \bigskip

  \begin{center}
    \begin{tikzpicture}[scale=1.7]
      \begin{axis}[
        xticklabels={,,},
        yticklabels={,,},
        enlargelimits=true,
        grid style={dashed, gray!60},
        axis x line = bottom,
        axis y line = left,
        axis line style={thin},
        xmax = 3,
        xmin = 0,
        ymax = 5,
        ymin = 0,
        axis lines = middle,
        small,
        clip=false,
        ]
        \addplot[
        draw=black,
        thick,
        smooth,
        samples=50,
        domain=0:2,
        ]
        {x^2} node(p1){} ;
        \node [right] at (p1) {\tiny $x^2$};
        \addplot[
        draw=black,
        thick,
        smooth,
        samples=2,
        domain=0:2,
        ]
        {x} node(p2){} ;
        \node [right] at (p2) {\tiny $x$};
        \addplot[
        draw=black,
        thick,
        smooth,
        samples=100,
        domain=0:2,
        ]
        {x^.5} node(p3){} ;
        \node [right] at (p3) {\tiny $\sqrt{x}$};
        \addplot[
        draw=black,
        dashed,
        ]
        coordinates {
          (1,0)
          (1,1)
          (0,1)
        } ;
        \node[left] at (0,1) {\tiny 1};
        \node[below] at (1,0) {\tiny 1};
      \end{axis}
    \end{tikzpicture}
  \end{center}

\end{frame}


\begin{frame}
  \frametitle{Fonctions puissances, III}
  \framesubtitle{Propriété 1}
  \hypertarget{slide_fonctions_puissance_3}{}

  \bigskip

  \begin{block}{Produit de fonctions puissances}
    Soit $c$ une constante réelle, soient $\alpha$ et $\beta$ deux paramètres réels, alors~:
    \[
      c^{\alpha}c^{\beta} = c^{\alpha+\beta}
    \]
  \end{block}

  \bigskip

  Par exemple, dans le cas où les puissances sont entières~:
  \[
    \begin{split}
      c^mc^n &= \underbrace{c\times c \times \ldots \times c}_{m\times}\times\underbrace{c\times c \times \ldots \times c}_{n\times}\\
      &= \underbrace{c\times c \times \ldots\ldots \times c}_{(m+n)\times}\\
      &= c^{m+n}
    \end{split}
  \]

\end{frame}


\begin{frame}
  \frametitle{Fonctions puissances, III}
  \framesubtitle{Propriété 2}
  \hypertarget{slide_fonctions_puissance_4}{}

  \bigskip

  \begin{block}{Puissance de puissance}
    Soit $c$ une constante réelle, soient $\alpha$ et $\beta$ deux paramètres réels, alors~:
    \[
      \left(c^{\alpha}\right)^{\beta} = c^{\alpha\beta}
    \]
  \end{block}

  \bigskip

  Par exemple, dans le cas où les puissances sont entières~:
  \[
    \begin{split}
      \left(c^m\right)^n &= \underbrace{c^m\times c^m \times \ldots \times c^m}_{n\times}\\
      &= \underbrace{\underbrace{c\times c \times \times c}_{m\times}\times \ldots \times \underbrace{c\times c \times \times c}_{m\times}}_{n\times}\\
      &=\underbrace{c\times c \times \ldots\ldots \times c}_{(mn)\times}\\
      &= c^{mn}
    \end{split}
  \]

\end{frame}


\section{Fonctions exponentielles}

\begin{frame}
  \frametitle{Fonctions exponentielles, I}
  \hypertarget{slide_fonctions_exponentielles_1}{}

  \bigskip

  \begin{definition}
    On appelle fonction exponentielle de base $a>0$ les fonctions de la forme~:
    \[
      f(x) = a^{x}
    \]
    Il s'agit d'une fonction de $\mathbb R$ dans $\mathbb R_+$.
  \end{definition}

  \bigskip

  \begin{itemize}

  \item D'après les propriétés précédentes, si $f$ est une fonction exponentielle alors~:
    \[
      f(x)f(y) = f(x+y)
    \]

  \item Si on restreint le domaine à $\mathbb N$, cette fonction sert à calculer des intérêts composés ou des facteurs de croissance.\newline

  \item On considère souvent la base~:
    \[
      e=\sum_{k=0}^{\infty}\frac{1}{k!} \approx 2,71828182845904523536028747135266249775724709\ldots
    \]
    une constante irrationnelle (Euler).
  \end{itemize}

\end{frame}


\begin{frame}
  \frametitle{Fonctions exponentielles, II}
  \framesubtitle{Exemple}
  \hypertarget{slide_fonctions_exponentielles_2}{}

  \bigskip

  \begin{center}
    \begin{tikzpicture}[scale=1]
      \begin{axis}[
        xticklabels={,,},
        yticklabels={,,},
        enlargelimits=true,
        grid style={dashed, gray!60},
        axis x line = bottom,
        axis y line = left,
        axis line style={thin},
        xmax = 2,
        xmin = -2,
        ymax = 10,
        ymin = 0,
        axis lines = middle,
        small,
        clip=false,
        ]
        \addplot[
        draw=black,
        thick,
        smooth,
        samples=50,
        domain=-2:2,
        ]
        {3^x} node(p1){} ;
        \node [right] at (p1) {\tiny $3^x$};
        \addplot[
        draw=red,
        thick,
        smooth,
        samples=50,
        domain=-2:2,
        ]
        {2.71828182845904523536028747135266249775724709369996^x} node(p2){} ;
        \node [right, red] at (p2) {\tiny $e^x$};
        \addplot[
        draw=black,
        thick,
        smooth,
        samples=100,
        domain=-2:2,
        ]
        {2^x} node(p3){} ;
        \node [right] at (p3) {\tiny $2^x$};
      \end{axis}
    \end{tikzpicture}
  \end{center}

\end{frame}


\section{Logarithmes}


\begin{frame}
  \frametitle{Logarithmes, I}

  \begin{definition}
    Le logarithme de base $a$ d'un nombre est la puissance de $a$ qui donne ce nombre. On note $\log_a x$ le logarithme base $a$ de $x$.
  \end{definition}

  \bigskip

  \begin{itemize}

  \item Par définition, on doit donc avoir~: $a^{\log_a x} = x$.\newline

  \item On définit la fonction logarithme (base $a$) $f(x) = \log_a(x)$.\newline

  \item Il s'agit de la réciproque de la fonction exponentielle, c'est une fonction de $\mathbb R_+^{\star}$ dans $\mathbb R$.\newline

  \item Quand la base est omise, la base est la constante d'Euler ($e$), on parle alors de logarithme naurel (on aussi $\ln$ parfois).\newline

  \item Les bases fréquemment utilisées sont 10, 2 et $e$.

  \end{itemize}

\end{frame}


\begin{frame}
  \frametitle{Logarithmes, II}
  \framesubtitle{Propriétés}

  \bigskip

  \begin{itemize}

  \item \textbf{Logarithme de 1.} $\log_a 1 = 0$ (puisque $a^0 = 1$).\newline

  \item \textbf{Changement de base.}
    \[
      \log_a(x) = \frac{\log_b(x)}{\log_b(a)}
    \]

    \bigskip

  \item \textbf{Le logarithme d'un produit est la somme des logarithmes}.
    \[
      \log_a(xy) = \log_a(x)+\log_a(y)
    \]
    $\Rightarrow$
    \[
      \log_a \left(x^c\right) = c \log_a (x)
    \]

    \bigskip

  \item \textbf{Le logarithme d'un ratio est la différence des logarithmes}.
    \[
      \log_a(x/y) = \log_a(x)-\log_a(y)
    \]


  \end{itemize}

\end{frame}


\begin{notes}

  \begin{itemize}

  \item \textbf{Formule du logarithme d'un produit.} Par définition du logarithme~:
    \begin{eqnarray*}
      &\text{Si } y_1 = a^{x_1} \text{ alors } \log_a(y_1) = x_1\\
      &\text{Si } y_2 = a^{x_2} \text{ alors } \log_a(y_2) = x_2\\
    \end{eqnarray*}
    Par ailleurs nous savons que~:
    \[
      y_1y_2 = a^{x_1+x_2}
    \]
    Nous avons donc que~:
    \[
      \log_a(y_1y_2) = x_1 + x_2
    \]
    c'est-à-dire~:
    \[
      \log_a(y_1y_2) = \log_a(y_1) + \log_a(y_2)
    \]

  \item \textbf{Formule du change de base.} Notons que si $A = log_a(x)$, alors par définition du logarithme on a $x = a^A$. En base $b$ nous avons donc~:
    \[
      \begin{split}
        \log_b(x) &= \log_b a^A\\
        &= A\log_b a\\
        &= \log_a(x)\log_b(a)
      \end{split}
    \]
    Et donc finalement de façon équivalente~:
    \[
      \log_a(x) = \frac{\log_b(x)}{\log_b(a)}
    \]
    En particulier nous avons~:

  \item \textbf{Logarithme d'un inverse.} On peut montrer que $\log_a \left(\frac{1}{x}\right) = - \log_a(x)$. En effet, on a~:
    \[
      \log_a1 = \log_a \left(x\cdot\frac{1}{x}\right) = \log_a x + \log_a \left(\frac{1}{x}\right)
    \]
    Comme le $\log_a 1 = 0$ pour toute base $a>0$, il vient~:
    \[
      \log_a x + \log_a \left(\frac{1}{x}\right) = 0
    \]
    et donc~:
    \[
      \log_a \left(\frac{1}{x}\right) = -\log_a x
    \]

  \item \textbf{Formule du logarithme d'un quotient.} On a~:
    \[
      \begin{split}
        \log_a \left(\frac{x}{y}\right) &= \log_a \left(x\cdot\frac{1}{y}\right)\\
        &= \log_a x + \log_a \left(\frac{1}{y}\right)\\
        &= \log_a x - \log_a y\\
      \end{split}
    \]

  \end{itemize}

\end{notes}


\begin{frame}
  \frametitle{Logarithmes, III}

  \begin{center}
    \begin{tikzpicture}[scale=1.7]
      \begin{axis}[
        xticklabels={,,},
        yticklabels={,,},
        enlargelimits=true,
        grid style={dashed, gray!60},
        axis x line = bottom,
        axis y line = left,
        axis line style={thin},
        xmax = 3,
        xmin = -3,
        ymax = 5,
        ymin = -5,
        axis lines = middle,
        small,
        clip=false,
        ]
        \addplot[
        draw=black,
        smooth,
        samples=50,
        domain=-2.8:1.8,
        ]
        {2.718281828459045235360287^x} node(p1){} ;
        \node [right] at (p1) {\tiny $e^x$};
        \addplot[
        draw=black,
        dashed,
        samples=2,
        domain=-3:3,
        ]
        {x} node(p2){} ;
        \node [right] at (p2) {\tiny $x$};
        \addplot[
        draw=black,
        thick,
        smooth,
        samples=200,
        domain=0.015:3,
        ]
        {ln(x)} node(p3){} ;
        \node [right] at (p3) {\tiny $\log{x}$};
        \node[above] at (1,0) {\tiny 1};
        \node[left] at (0,1) {\tiny 1};
      \end{axis}
    \end{tikzpicture}
  \end{center}


\end{frame}


\end{document}

% Local Variables:
% ispell-check-comments: exclusive
% ispell-local-dictionary: "francais"
% TeX-master: t
% End: