\synctex=1

\documentclass[10pt,notheorems]{beamer}

\usepackage{etex}
\usepackage{fourier-orns}
\usepackage{ccicons}
\usepackage{amssymb}
\usepackage{amstext}
\usepackage{amsbsy}
\usepackage{amsopn}
\usepackage{amscd}
\usepackage{amsxtra}
\usepackage{amsthm}
\usepackage{float}
\usepackage{color, colortbl}
\usepackage{mathrsfs}
\usepackage{bm}
\usepackage{lastpage}
\usepackage[nice]{nicefrac}
\usepackage{setspace}
\usepackage{ragged2e}
\usepackage{listings}
\usepackage{algorithms/algorithm}
\usepackage{algorithms/algorithmic}
\usepackage[frenchb]{babel}
\usepackage{tikz,pgfplots}
\pgfplotsset{compat=newest}
\usetikzlibrary{patterns, arrows, decorations.pathreplacing, decorations.markings, calc}
\pgfplotsset{plot coordinates/math parser=false}
\newlength\figureheight
\newlength\figurewidth
\usepackage[utf8x]{inputenc}
\usepackage{cancel}
\usepackage{tikz-qtree}
\usepackage{dcolumn}
\usepackage{adjustbox}
\usepackage{environ}
\usepackage[cal=boondox]{mathalfa}
\usepackage{manfnt}
\usepackage{hyperref}
\hypersetup{
  colorlinks=true,
  linkcolor=blue,
  filecolor=black,
  urlcolor=black,
}
\usepackage{venndiagram}

% Git hash
\usepackage{xstring}
\usepackage{catchfile}
\immediate\write18{git rev-parse HEAD > git.hash}
\CatchFileDef{\HEAD}{git.hash}{\endlinechar=-1}
\newcommand{\gitrevision}{\StrLeft{\HEAD}{7}}

\newcommand{\trace}{\mathrm{tr}}
\newcommand{\vect}{\mathrm{vec}}
\newcommand{\tracarg}[1]{\mathrm{tr}\left\{#1\right\}}
\newcommand{\vectarg}[1]{\mathrm{vec}\left(#1\right)}
\newcommand{\vecth}[1]{\mathrm{vech}\left(#1\right)}
\newcommand{\iid}[2]{\mathrm{iid}\left(#1,#2\right)}
\newcommand{\normal}[2]{\mathcal N\left(#1,#2\right)}
\newcommand{\dynare}{\href{http://www.dynare.org}{\color{blue}Dynare}}
\newcommand{\sample}{\mathcal Y_T}
\newcommand{\samplet}[1]{\mathcal Y_{#1}}
\newcommand{\slidetitle}[1]{\fancyhead[L]{\textsc{#1}}}

\newcommand{\R}{{\mathbb R}}
\newcommand{\C}{{\mathbb C}}
\newcommand{\N}{{\mathbb N}}
\newcommand{\Z}{{\mathbb Z}}
\newcommand{\binomial}[2]{\begin{pmatrix} #1 \\ #2 \end{pmatrix}}
\newcommand{\bigO}[1]{\mathcal O \left(#1\right)}
\newcommand{\red}{\color{red}}
\newcommand{\blue}{\color{blue}}

\renewcommand{\qedsymbol}{C.Q.F.D.}

\newcolumntype{d}{D{.}{.}{-1}}
\definecolor{gray}{gray}{0.9}
\newcolumntype{g}{>{\columncolor{gray}}c}

\setbeamertemplate{theorems}[numbered]

\theoremstyle{plain}
\newtheorem{theorem}{Théorème}

\theoremstyle{definition} % insert bellow all blocks you want in normal text
\newtheorem{definition}{Définition}
\newtheorem{properties}{Propriétés}
\newtheorem{lemma}{Lemme}
\newtheorem{property}[properties]{Propriété}
\newtheorem{example}{Exemple}
\newtheorem*{idea}{Éléments de preuve} % no numbered block



\setbeamertemplate{footline}{
  {\hfill\vspace*{1pt}\href{http://creativecommons.org/licenses/by-sa/3.0/legalcode}{\ccbysa}\hspace{.1cm}
    \raisebox{-.075cm}{\href{https://git.adjemian.eu/University/economic-calculus}{\includegraphics[scale=.1]{../img/gitlab.png}}}\enspace
    \href{https://git.adjemian.eu/University/economic-calculus/-/blob/\HEAD/cours/chapitre-2.tex}{\gitrevision}\enspace\today
  }\hspace{1cm}}

\setbeamertemplate{navigation symbols}{}
\setbeamertemplate{blocks}[rounded][shadow=true]
\setbeamertemplate{caption}[numbered]

\NewEnviron{notes}{\justifying\tiny\begin{spacing}{1.0}\BODY\vfill\pagebreak\end{spacing}}

\newenvironment{exercise}[1]
{\bgroup \small\begin{block}{Ex. #1}}
  {\end{block}\egroup}

\newenvironment{defn}[1]
{\bgroup \small\begin{block}{Définition. #1}}
  {\end{block}\egroup}

\newenvironment{exemple}[1]
{\bgroup \small\begin{block}{Exemple. #1}}
  {\end{block}\egroup}

\begin{document}

\title{Calcul Économique\\\small{II. Fonctions}}
\author[S. Adjemian]{Stéphane Adjemian}
\institute{\texttt{stephane.adjemian@univ-lemans.fr}} \date{Septembre 2020}

\begin{frame}
  \titlepage{}
\end{frame}

\begin{frame}
  \frametitle{Plan}
  \tableofcontents
\end{frame}


\section{représentation graphique des fonctions}

\begin{frame}
  \frametitle{Réprésenter graphiquement une fonction, I}
  \hypertarget{slide_plan_cartesien_1}{}

  \begin{itemize}

  \item Une fonction est un ensemble de paires ordonnées, construites
    à partir du \textbf{produit cartésien} de deux ensembles, tel que chaque
    élément de l'ensemble de départ est associé à un et un seul
    élément de l'ensemble d'arrivée.\newline

  \item Dans le chapitre précédent nous avons donné comme exemple de fonction~:
    \[
      B = \{(x,y) | x\in\mathbb N \land y = 2x-1\}
    \]

  \item Pour réprésenter les fonctions on utilise un \textbf{plan cartésien}.\newline

  \item Les éléments de l'ensemble de départ sont représentés sur une ligne horizontale (l'axe des abscisses).\newline

  \item Les éléments de l'ensemble d'arrivée sont représentés sur une ligne verticale (l'axe des ordonnées).\newline

  \item Chaque paire représente les coordonnées d'un point dans le plan.

  \end{itemize}

\end{frame}


\begin{frame}
  \frametitle{Réprésenter graphiquement une fonction, II}
  \framesubtitle{L'ensemble $B$ dans le plan cartésien}
  \hypertarget{slide_plan_cartesien_2}{}

  \begin{center}
    \begin{tikzpicture}[scale=1.3]
      \begin{axis}[
        title={},
        width=10cm,height=7cm,
        xtick = {-4, -3, -2, -1, 1, 2, 3, 4},
        xticklabels={-4, -3, -2, -1, 1, 2, 3, 4},
        ytick = {7, 5, 3, 1, -1, -3, -5, -7},
        yticklabels={7, 5, 3, 1, -1, -3, -5, -7},
        enlargelimits=false,
        axis lines = middle,
        axis line style={thin,<->},
        xmin=-5,
        xmax=5,
        ymax=10,
        ymin=-10,
        ]
        \addplot[mark=*] coordinates {(-4,-9)};
        \addplot[mark=*] coordinates {(-3,-7)};
        \addplot[mark=*] coordinates {(-2,-5)};
        \addplot[mark=*] coordinates {(-1,-3)};
        \addplot[mark=*] coordinates {(0,-1)};
        \addplot[mark=*] coordinates {(1,1)};
        \addplot[mark=*] coordinates {(2,3)};
        \addplot[mark=*] coordinates {(3,5)};
        \addplot[mark=*] coordinates {(4,7)};
        \node[right] at (-.45, 9.1){{\small{$y$}}};
        \node[right] at (4.6, -.6){{\small{$x$}}};
      \end{axis}
    \end{tikzpicture}
  \end{center}

\end{frame}


\section{Les droites}


\begin{frame}
  \frametitle{Les droites, I}
  \hypertarget{slide_droites_1}{}

  \bigskip

  La fonction la plus simple que nous puissions considérer est la
  droite. Celle-ci est caractérisée par l'équation~:
  \[
    y = a x + b
  \]
  où $a$ et $b$ sont des paramètres réels, il s'agit d'une fonction de $\mathbb R$ dans $\mathbb R$.\newline

  \bigskip

  \begin{columns}
    \begin{column}{0.5\textwidth}
      {\small\begin{itemize}
        \item[--] Le paramètre $a$ est la pente de la droite.
        \item[--] Si $f(x) = ax+b$, alors~:
          \[
            a = \frac{f(x_0+h)-f(x_0)}{h}
          \]
          pour tout $x_0\in\mathbb R$ et $h\in\mathbb R^{\star}$
        \item[--] Le paramètre $b$ est l'ordonnée à l'origine.
        \item[--] Si $f(x) = ax+b$, alors~:
          \[
            f(0) = b
          \]
        \end{itemize}}
    \end{column}
    \begin{column}{0.5\textwidth}
      \begin{tikzpicture}[scale=.7]
        \begin{axis}[
          title={Ceci est une droite},
          xlabel=$x$,
          ylabel=$y$,
          xticklabels={,,},
          yticklabels={,,},
          enlargelimits=true,
          grid style={dashed, gray!60},
          axis x line = bottom,
          axis y line = left,
          axis line style={thin},
          ymax = 12,
          ymin = -1,
          xmax = 10,
          xmin = -1,
          axis lines = middle,
          small,
          clip=false,
          ]
          \addplot[
          draw=black,
          thick,
          smooth,
          samples=2,
          domain=-1:10,
          ]
          {x+2};
          \addplot[mark=*] coordinates {(0,2)};
          \node[left] at (0, 2){{\small{$b$}}};
          \addplot[dashed] coordinates {
            (4,0)
            (4,6)
            (0,6)};
          \node[left] at (0, 6){{\small{$f(x_0)$}}};
          \node[below] at (4, 0){{\small{$x_0$}}};
          \addplot[dashed] coordinates {
            (6,0)
            (6,8)
            (0,8)};
          \node[left] at (0, 8){{\small{$f(x_0+h)$}}};
          \node[below] at (6, 0){{\small{$x_0+h$}}};
        \end{axis}
      \end{tikzpicture}
    \end{column}
  \end{columns}

\end{frame}


\begin{frame}
  \frametitle{Les droites, II}
  \hypertarget{slide_droites_2}{}

  \bigskip

  \begin{itemize}

  \item On peut faire beaucoup de choses avec des droites. Vous verrez plus tard que l'on utilise souvent des droites pour approximer des fonctions plus générales (et complexes).\newline

  \item Ce que vous devez savoir faire~:\newline

    \begin{enumerate}

    \item Tracer une droite dans un plan,\newline

    \item Retrouver l'équation d'une droite à partir d'un tracé,\newline

    \item Trouver l'intersection d'une droite et de l'axe des abscisses,\newline

    \item Calculer le point d'intersection de deux droites.\newline

    \end{enumerate}

  \end{itemize}

\end{frame}


\end{document}

% Local Variables:
% ispell-check-comments: exclusive
% ispell-local-dictionary: "francais"
% TeX-master: t
% End: