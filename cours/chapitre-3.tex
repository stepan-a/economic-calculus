\synctex=1

\documentclass[10pt,notheorems]{beamer}

\usepackage{etex}
\usepackage{fourier-orns}
\usepackage{ccicons}
\usepackage{amssymb}
\usepackage{amstext}
\usepackage{amsbsy}
\usepackage{amsopn}
\usepackage{amscd}
\usepackage{amsxtra}
\usepackage{amsthm}
\usepackage{float}
\usepackage{color, colortbl}
\usepackage{mathrsfs}
\usepackage{bm}
\usepackage{lastpage}
\usepackage[nice]{nicefrac}
\usepackage{setspace}
\usepackage{ragged2e}
\usepackage{listings}
\usepackage{polynom}
\usepackage{algorithms/algorithm}
\usepackage{algorithms/algorithmic}
\usepackage[frenchb]{babel}
\usepackage{tikz,pgfplots}
\pgfplotsset{compat=newest}
\usetikzlibrary{patterns, arrows, decorations.pathreplacing, decorations.markings, calc}
\pgfplotsset{plot coordinates/math parser=false}
\newlength\figureheight
\newlength\figurewidth
\usepackage[utf8x]{inputenc}
\usepackage{cancel}
\usepackage{tikz-qtree}
\usepackage{dcolumn}
\usepackage{adjustbox}
\usepackage{environ}
\usepackage[cal=boondox]{mathalfa}
\usepackage{manfnt}
\usepackage{hyperref}
\hypersetup{
  colorlinks=true,
  linkcolor=blue,
  filecolor=black,
  urlcolor=black,
}
\usepackage{venndiagram}
\usepackage{minted}

% Git hash
\usepackage{xstring}
\usepackage{catchfile}
\immediate\write18{git rev-parse HEAD > git.hash}
\CatchFileDef{\HEAD}{git.hash}{\endlinechar=-1}
\newcommand{\gitrevision}{\StrLeft{\HEAD}{7}}

\newcommand{\trace}{\mathrm{tr}}
\newcommand{\vect}{\mathrm{vec}}
\newcommand{\tracarg}[1]{\mathrm{tr}\left\{#1\right\}}
\newcommand{\vectarg}[1]{\mathrm{vec}\left(#1\right)}
\newcommand{\vecth}[1]{\mathrm{vech}\left(#1\right)}
\newcommand{\iid}[2]{\mathrm{iid}\left(#1,#2\right)}
\newcommand{\normal}[2]{\mathcal N\left(#1,#2\right)}
\newcommand{\dynare}{\href{http://www.dynare.org}{\color{blue}Dynare}}
\newcommand{\sample}{\mathcal Y_T}
\newcommand{\samplet}[1]{\mathcal Y_{#1}}
\newcommand{\slidetitle}[1]{\fancyhead[L]{\textsc{#1}}}

\newcommand{\R}{{\mathbb R}}
\newcommand{\C}{{\mathbb C}}
\newcommand{\N}{{\mathbb N}}
\newcommand{\Z}{{\mathbb Z}}
\newcommand{\binomial}[2]{\begin{pmatrix} #1 \\ #2 \end{pmatrix}}
\newcommand{\bigO}[1]{\mathcal O \left(#1\right)}
\newcommand{\red}{\color{red}}
\newcommand{\blue}{\color{blue}}

\renewcommand{\qedsymbol}{C.Q.F.D.}

\newcolumntype{d}{D{.}{.}{-1}}
\definecolor{gray}{gray}{0.9}
\newcolumntype{g}{>{\columncolor{gray}}c}

\setbeamertemplate{theorems}[numbered]

\theoremstyle{plain}
\newtheorem{theorem}{Théorème}

\theoremstyle{definition} % insert bellow all blocks you want in normal text
\newtheorem{definition}{Définition}
\newtheorem{properties}{Propriétés}
\newtheorem{lemma}{Lemme}
\newtheorem{property}[properties]{Propriété}
\newtheorem{example}{Exemple}
\newtheorem*{idea}{Éléments de preuve} % no numbered block



\setbeamertemplate{footline}{
  {\hfill\vspace*{1pt}\href{http://creativecommons.org/licenses/by-sa/3.0/legalcode}{\ccbysa}\hspace{.1cm}
    \raisebox{-.075cm}{\href{https://git.adjemian.eu/University/economic-calculus}{\includegraphics[scale=.1]{../img/gitlab.png}}}\enspace
    \href{https://git.adjemian.eu/University/economic-calculus/-/blob/\HEAD/cours/chapitre-3.tex}{\gitrevision}\enspace\today
  }\hspace{1cm}}

\setbeamertemplate{navigation symbols}{}
\setbeamertemplate{blocks}[rounded][shadow=true]
\setbeamertemplate{caption}[numbered]

\NewEnviron{notes}{\justifying\tiny\begin{spacing}{1.0}\BODY\vfill\pagebreak\end{spacing}}

\newenvironment{exercise}[1]
{\bgroup \small\begin{block}{Ex. #1}}
  {\end{block}\egroup}

\newenvironment{defn}[1]
{\bgroup \small\begin{block}{Définition. #1}}
  {\end{block}\egroup}

\newenvironment{exemple}[1]
{\bgroup \small\begin{block}{Exemple. #1}}
  {\end{block}\egroup}

\begin{document}

\title{Calcul Économique\\\small{III. Suites, limites et continuité}}
\author[S. Adjemian]{Stéphane Adjemian}
\institute{\texttt{stephane.adjemian@univ-lemans.fr}} \date{Septembre 2020}

\begin{frame}
  \titlepage{}
\end{frame}

\begin{frame}
  \frametitle{Plan}
  \tableofcontents
\end{frame}


\section{Suites}

\begin{frame}
  \frametitle{Suites, I}
  \framesubtitle{Définition}
  \hypertarget{slide_suites_1}{}

  \begin{itemize}

  \item Une suite est une famille d'éléments, les « termes » de la suite, indexée par $\mathbb N$.\newline

  \item Lorsque tous les éléments d'une suite appartiennent à un même ensemble $E$, cette suite peut être assimilée à une fonction de $\mathbb N$ dans $E$.\newline

  \item Une suite est dite entière si $E$ est un sous ensemble de $\mathbb Z$, réelle si $E$ est un sous ensemble de $\mathbb R$, ou complexe si $E$ est un sous-ensemble $\mathbb C$.\newline

  \item On note habituellement $u_0$, $u_1$, $u_2$, \ldots les termes d'une suite.\newline

  \item Une suite peut être définie en donnant une expression générale de ses termes, ou à l'aide d'une expression récursive en définissant chaque terme en fonction de termes précédents.\newline

  \end{itemize}

\end{frame}


\begin{frame}
  \frametitle{Suites, I}
  \framesubtitle{Exemple}
  \hypertarget{slide_suites_2}{}

  \begin{itemize}

  \item On définit la suite $(u_n)_{n\in\mathbb N}$~, en donnant l'expression générale du $n$-ième terme~:
    \[
      u_n = (-1)^n
    \]

  \item Cela donne~:\newline
    \begin{center}
      \begin{table}[H]
        \centering
        \begin{tabular}{c|cccccccc}
          $n$ & 0 & 1 & 2 & 3 & 4 & \ldots & $n$ & \ldots\\\hline
          $u_n$ & 1 & -1 & 1 & -1 & 1 &     & $(-1)^n$ &
        \end{tabular}
      \end{table}
    \end{center}

    \bigskip

  \item Alternativement, nous pouvons définir la suite récursivement~:
    \[
      u_n = -1\times u_{n-1}
    \]
    avec la condition initiale $u_0 = 1$.\newline

  \item Quand les signes des termes successifs sont toujours opposés, on dit que la suite est \textbf{alternée}.
  \end{itemize}

\end{frame}


\begin{frame}
  \frametitle{Suites, I}
  \framesubtitle{Monotonie}
  \hypertarget{slide_suites_3}{}

  \begin{itemize}

  \item Une suite est dite \textbf{monotone croissante} si $u_n>u_{n-1}$ pour tout $n\geq 1$.\newline

  \item Une suite est dite \textbf{monotone décroissante} si $u_n<u_{n-1}$ pour tout $n\geq 1$.\newline

  \item On dira qu'une suite est croissante (décroissante) si $u_n\geq u_{n-1}$ (respectivement $u_n\leq u_{n-1}$.\newline

  \end{itemize}

  \begin{example}
    La suite $(u_n)_{n\in\mathbb N}$ de terme général $u_n = \frac{1}{n}$ (pour $n\in\mathbb N^{\star}$) est monotone décroissante. En effet~:
    \[
      u_n - u_{n-1} = \frac{1}{n}-\frac{1}{n-1} = \frac{n-1-n}{n(n-1)} = -\frac{1}{n(n-1)}<0 \forall n\geq 2
    \]
  \end{example}

\end{frame}


\begin{frame}
  \frametitle{Suites, I}
  \framesubtitle{Majorant et minorant}
  \hypertarget{slide_suites_4}{}

  \begin{itemize}

  \item Une suite est \textbf{bornée} s'il existe $k$ et $K$ tels que $k\leq u_n\leq K$ $\forall n\geq 0$.\newline

  \item $k$ ($K$) est un \textbf{minorant} (\textbf{majorant}) de la suite.\newline

  \item Le plus grand minorant (petit majorant) est la \textbf{borne inférieure} (\textbf{borne supérieure}) de la suite.\newline

  \end{itemize}

  \begin{example}
    La suite $(u_n)_{n\in\mathbb N}$ de terme général $u_n = \frac{n-2}{2n}$ (pour $n\in\mathbb N^{\star}$) est bornée. En effet $-\frac{1}{2}\leq u_n\leq \frac{1}{2}$~:
    \[
      u_n - \frac{1}{2} =  \frac{n-2}{2n}-\frac{1}{2}=\frac{1}{2}-\frac{2}{2n}-\frac{1}{2}=-\frac{1}{n}<0 \quad\forall n\geq 1
    \]
    \[
      u_n+\frac{1}{2} = \frac{n-2}{2n}+\frac{1}{2} = 1 - \frac{1}{n} \geq 0\quad \forall n\geq 1
    \]
  \end{example}

\end{frame}


\end{document}

% Local Variables:
% ispell-check-comments: exclusive
% ispell-local-dictionary: "francais"
% TeX-master: t
% End: