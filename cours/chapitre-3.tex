\synctex=1

\documentclass[10pt,notheorems]{beamer}

\usepackage{etex}
\usepackage{fourier-orns}
\usepackage{ccicons}
\usepackage{amssymb}
\usepackage{amstext}
\usepackage{amsbsy}
\usepackage{amsopn}
\usepackage{amscd}
\usepackage{amsxtra}
\usepackage{amsthm}
\usepackage{float}
\usepackage{color, colortbl}
\usepackage{mathrsfs}
\usepackage{bm}
\usepackage{lastpage}
\usepackage[nice]{nicefrac}
\usepackage{setspace}
\usepackage{ragged2e}
\usepackage{listings}
\usepackage{polynom}
\usepackage{algorithms/algorithm}
\usepackage{algorithms/algorithmic}
\usepackage[frenchb]{babel}
\usepackage{tikz,pgfplots}
\pgfplotsset{compat=newest}
\usetikzlibrary{patterns, arrows, decorations.pathreplacing, decorations.markings, calc}
\pgfplotsset{plot coordinates/math parser=false}
\newlength\figureheight
\newlength\figurewidth
\usepackage[utf8x]{inputenc}
\usepackage{cancel}
\usepackage{tikz-qtree}
\usepackage{dcolumn}
\usepackage{adjustbox}
\usepackage{environ}
\usepackage[cal=boondox]{mathalfa}
\usepackage{manfnt}
\usepackage{hyperref}
\hypersetup{
  colorlinks=true,
  linkcolor=blue,
  filecolor=black,
  urlcolor=black,
}
\usepackage{venndiagram}
\usepackage{minted}

% Git hash
\usepackage{xstring}
\usepackage{catchfile}
\immediate\write18{git rev-parse HEAD > git.hash}
\CatchFileDef{\HEAD}{git.hash}{\endlinechar=-1}
\newcommand{\gitrevision}{\StrLeft{\HEAD}{7}}

\newcommand{\trace}{\mathrm{tr}}
\newcommand{\vect}{\mathrm{vec}}
\newcommand{\tracarg}[1]{\mathrm{tr}\left\{#1\right\}}
\newcommand{\vectarg}[1]{\mathrm{vec}\left(#1\right)}
\newcommand{\vecth}[1]{\mathrm{vech}\left(#1\right)}
\newcommand{\iid}[2]{\mathrm{iid}\left(#1,#2\right)}
\newcommand{\normal}[2]{\mathcal N\left(#1,#2\right)}
\newcommand{\dynare}{\href{http://www.dynare.org}{\color{blue}Dynare}}
\newcommand{\sample}{\mathcal Y_T}
\newcommand{\samplet}[1]{\mathcal Y_{#1}}
\newcommand{\slidetitle}[1]{\fancyhead[L]{\textsc{#1}}}

\newcommand{\R}{{\mathbb R}}
\newcommand{\C}{{\mathbb C}}
\newcommand{\N}{{\mathbb N}}
\newcommand{\Z}{{\mathbb Z}}
\newcommand{\binomial}[2]{\begin{pmatrix} #1 \\ #2 \end{pmatrix}}
\newcommand{\bigO}[1]{\mathcal O \left(#1\right)}
\newcommand{\red}{\color{red}}
\newcommand{\blue}{\color{blue}}

\renewcommand{\qedsymbol}{C.Q.F.D.}

\newcolumntype{d}{D{.}{.}{-1}}
\definecolor{gray}{gray}{0.9}
\newcolumntype{g}{>{\columncolor{gray}}c}

\setbeamertemplate{theorems}[numbered]

\theoremstyle{plain}
\newtheorem{theorem}{Théorème}

\theoremstyle{definition} % insert bellow all blocks you want in normal text
\newtheorem{definition}{Définition}
\newtheorem{properties}{Propriétés}
\newtheorem{lemma}{Lemme}
\newtheorem{property}[properties]{Propriété}
\newtheorem{example}{Exemple}
\newtheorem*{idea}{Éléments de preuve} % no numbered block



\setbeamertemplate{footline}{
  {\hfill\vspace*{1pt}\href{http://creativecommons.org/licenses/by-sa/3.0/legalcode}{\ccbysa}\hspace{.1cm}
    \raisebox{-.075cm}{\href{https://git.adjemian.eu/University/economic-calculus}{\includegraphics[scale=.1]{../img/gitlab.png}}}\enspace
    \href{https://git.adjemian.eu/University/economic-calculus/-/blob/\HEAD/cours/chapitre-3.tex}{\gitrevision}\enspace\today
  }\hspace{1cm}}

\setbeamertemplate{navigation symbols}{}
\setbeamertemplate{blocks}[rounded][shadow=true]
\setbeamertemplate{caption}[numbered]

\NewEnviron{notes}{\justifying\tiny\begin{spacing}{1.0}\BODY\vfill\pagebreak\end{spacing}}

\newenvironment{exercise}[1]
{\bgroup \small\begin{block}{Ex. #1}}
  {\end{block}\egroup}

\newenvironment{defn}[1]
{\bgroup \small\begin{block}{Définition. #1}}
  {\end{block}\egroup}

\newenvironment{exemple}[1]
{\bgroup \small\begin{block}{Exemple. #1}}
  {\end{block}\egroup}

\begin{document}

\title{Calcul Économique\\\small{III. Suites, limites et continuité}}
\author[S. Adjemian]{Stéphane Adjemian}
\institute{\texttt{stephane.adjemian@univ-lemans.fr}} \date{Septembre 2020}

\begin{frame}
  \titlepage{}
\end{frame}

\begin{frame}
  \frametitle{Plan}
  \tableofcontents
\end{frame}


\section{Suites}

\begin{frame}
  \frametitle{Suites, I}
  \framesubtitle{Définition}
  \hypertarget{slide_suites_1}{}

  \begin{itemize}

  \item Une suite est une famille d'éléments, les « termes » de la suite, indexée par $\mathbb N$.\newline

  \item Lorsque tous les éléments d'une suite appartiennent à un même ensemble $E$, cette suite peut être assimilée à une fonction de $\mathbb N$ dans $E$.\newline

  \item Une suite est dite entière si $E$ est un sous ensemble de $\mathbb Z$, réelle si $E$ est un sous ensemble de $\mathbb R$, ou complexe si $E$ est un sous-ensemble $\mathbb C$.\newline

  \item On note habituellement $u_0$, $u_1$, $u_2$, \ldots les termes d'une suite.\newline

  \item Une suite peut être définie en donnant une expression générale de ses termes, ou à l'aide d'une expression récursive en définissant chaque terme en fonction de termes précédents.\newline

  \end{itemize}

\end{frame}


\begin{frame}
  \frametitle{Suites, I}
  \framesubtitle{Exemple}
  \hypertarget{slide_suites_2}{}

  \begin{itemize}

  \item On définit la suite $(u_n)_{n\in\mathbb N}$~, en donnant l'expression générale du $n$-ième terme~:
    \[
      u_n = (-1)^n
    \]

  \item Cela donne~:\newline
    \begin{center}
      \begin{table}[H]
        \centering
        \begin{tabular}{c|cccccccc}
          $n$ & 0 & 1 & 2 & 3 & 4 & \ldots & $n$ & \ldots\\\hline
          $u_n$ & 1 & -1 & 1 & -1 & 1 &     & $(-1)^n$ &
        \end{tabular}
      \end{table}
    \end{center}

    \bigskip

  \item Alternativement, nous pouvons définir la suite récursivement~:
    \[
      u_n = -1\times u_{n-1}
    \]
    avec la condition initiale $u_0 = 1$.\newline

  \item Quand les signes des termes successifs sont toujours opposés, on dit que la suite est \textbf{alternée}.
  \end{itemize}

\end{frame}


\begin{frame}
  \frametitle{Suites, I}
  \framesubtitle{Monotonie}
  \hypertarget{slide_suites_3}{}

  \begin{itemize}

  \item Une suite est dite \textbf{monotone croissante} si $u_n>u_{n-1}$ pour tout $n\geq 1$.\newline

  \item Une suite est dite \textbf{monotone décroissante} si $u_n<u_{n-1}$ pour tout $n\geq 1$.\newline

  \item On dira qu'une suite est croissante (décroissante) si $u_n\geq u_{n-1}$ (respectivement $u_n\leq u_{n-1}$.\newline

  \end{itemize}

  \begin{example}
    La suite $(u_n)_{n\in\mathbb N^{\star}}$ de terme général $u_n = \frac{1}{n}$ est monotone décroissante. En effet~:
    \[
      u_n - u_{n-1} = \frac{1}{n}-\frac{1}{n-1} = \frac{n-1-n}{n(n-1)} = -\frac{1}{n(n-1)}<0 \forall n\geq 2
    \]
  \end{example}

\end{frame}


\begin{frame}
  \frametitle{Suites, I}
  \framesubtitle{Majorant et minorant}
  \hypertarget{slide_suites_4}{}

  \begin{itemize}

  \item Une suite est \textbf{bornée} s'il existe $k$ et $K$ tels que $k\leq u_n\leq K$ $\forall n\geq 0$.\newline

  \item $k$ ($K$) est un \textbf{minorant} (\textbf{majorant}) de la suite.\newline

  \item Le plus grand minorant (petit majorant) est la \textbf{borne inférieure} (\textbf{borne supérieure}) de la suite.\newline

  \end{itemize}

  \begin{example}
    La suite $(u_n)_{n\in\mathbb N^{\star}}$ de terme général $u_n = \frac{n-2}{2n}$ est bornée. En effet $-\frac{1}{2}\leq u_n\leq \frac{1}{2}$~:
    \[
      u_n - \frac{1}{2} =  \frac{n-2}{2n}-\frac{1}{2}=\frac{1}{2}-\frac{2}{2n}-\frac{1}{2}=-\frac{1}{n}<0 \quad\forall n\geq 1
    \]
    \[
      u_n+\frac{1}{2} = \frac{n-2}{2n}+\frac{1}{2} = 1 - \frac{1}{n} \geq 0\quad \forall n\geq 1
    \]
  \end{example}

\end{frame}


\begin{frame}
  \frametitle{Suites, I}
  \framesubtitle{Suites arithmétiques (a)}
  \hypertarget{slide_suites_arithmetiques_1}{}

  \bigskip

  \begin{definition}
    Une suite $(u_n)_{n\in\mathbb N}$ est dite arithmétique si ses variations sont constantes, c'est-à-dire si~:
    \[
      \Delta u_n \equiv u_n-u_{n-1} = r\quad \forall n\in\mathbb N^{\star}
    \]
    On dit que la constante $r$ est la raison de la suite.
  \end{definition}

  \bigskip

  \begin{itemize}

  \item La suite arithmétique est monotone croissante si et seulement si sa raison est positive.\newline

  \item Nous avons ici une définition récursive de la suite arithmétique~:
    \[
      u_n = u_{n-1} + r
    \]
    avec une condition initial $u_0$.\newline

  \item Il est possible de définir celle-ci en donnant son terme général.\newline

  \end{itemize}

\end{frame}


\begin{frame}
  \frametitle{Suites, I}
  \framesubtitle{Suites arithmétiques (b)}
  \hypertarget{slide_suites_arithmetiques_2}{}

  \bigskip

  \begin{itemize}

  \item Pour obtenir le terme général on substitue la définition dans le membre de droite, autant de fois que nécessaire pour remonter jusqu'à la condition initiale~:
    \[
      \begin{split}
        u_n &= u_{n-1} + r\\
        &= u_{n-2} + r + r{\color{gray}\text{, car }u_{n-1}=u_{n-2}+r}\\
        \Leftrightarrow u_n &= u_{n-2} + 2r\\
        &\vdots\\
        u_n &= u_0 + nr{\color{gray}\text{, le terme général de la suite.}}
      \end{split}
    \]

    \bigskip

  \item Pour vérifier que le terme général est correct, on peut le substituer dans l'expression récursive.\newline

  \end{itemize}

  \begin{example}
    La suite $(u_n)_{n\in\mathbb N}$ définie par $u_n = n$ est une suite arithmétique de raison 1 et avec pour condition initiale $u_0=0$.
  \end{example}

\end{frame}


\begin{frame}
  \frametitle{Suites, I}
  \framesubtitle{Suites géométriques (a)}
  \hypertarget{slide_suites_geometrique_1}{}

  \bigskip

  \begin{definition}
    Une suite $(u_n)_{n\in\mathbb N}$ est dite géométrique si le rapport de deux termes consécutifs est constant, c'est-à-dire si~:
    \[
      \frac{u_n}{u_{n-1}} = r\quad \forall n\in\mathbb N^{\star}
    \]
    On dit que la constante $r$ est la raison de la suite.
  \end{definition}

  \bigskip

  \begin{itemize}

  \item La suite arithmétique est monotone croissante si et seulement si sa raison est supérieure à 1.\newline

  \item Nous avons ici une définition récursive de la suite arithmétique~:
    \[
      u_n = r u_{n-1}
    \]
    avec une condition initial $u_0$.\newline

  \item Il est possible de définir celle-ci en donnant son terme général.\newline

  \end{itemize}

\end{frame}


\begin{frame}
  \frametitle{Suites, I}
  \framesubtitle{Suites géométriques (b)}
  \hypertarget{slide_suites_geometriques_2}{}

  \bigskip

  \begin{itemize}

  \item Pour obtenir le terme général on substitue la définition dans le membre de droite, autant de fois que nécessaire pour remonter jusqu'à la condition initiale~:
    \[
      \begin{split}
        u_n &= r u_{n-1}\\
        &= r r u_{n-2} {\color{gray}\text{, car }u_{n-1}=r u_{n-2}}\\
        \Leftrightarrow u_n &= r^2 u_{n-2}\\
        &\vdots\\
        u_n &= r^nu_0 {\color{gray}\text{, le terme général de la suite.}}
      \end{split}
    \]

    \bigskip

  \item Pour vérifier que le terme général est correct, on peut le substituer dans l'expression récursive.\newline

  \end{itemize}

  \begin{example}
    La suite $(u_n)_{n\in\mathbb N}$ définie par $u_n = (1+g)^n$ est une suite géométrique de raison $1+g$ et avec pour condition initiale $u_0=1$.
  \end{example}

\end{frame}


\begin{frame}
  \frametitle{Suites, II}
  \framesubtitle{Limite d'une suite (a)}
  \hypertarget{slide_suite_limite_1}{}

  \bigskip

  \begin{definition}
    On dit qu'une suite $(u_n)_{n\in\mathbb N}$ converge vers une limite $u^{\star}$ quand $n$ tend vers l'infini, et on note $\lim_{n\rightarrow\infty}u_n = u^{\star}$, si et seulement si~:
    \[
      \forall \varepsilon>0, \exists N\text{ tel que } |u_n-u^{\star}|<\varepsilon\,\, \forall \, n>N
    \]
  \end{definition}

  \bigskip

  \begin{itemize}

  \item $|u_n-u^{\star}|$ mesure la distance entre le n-ième terme de la suite et la limite $u^{\star}$.\newline

  \item La suite converge vers $u^{\star}$ s'il est possible de rendre la distance entre $u_n$ et $u^{\star}$ arbitrairement petite au delà d'un certain rang ($N$).\newline

  \item Le rang à partir duquel les écarts sont plus petits que $\varepsilon$ dépend a priori de $\varepsilon$. Plus $\varepsilon$ est proche de 0 plus le rang sera élevé (plus loin il faudra aller pour contenir la distance entre $u_n$ et $u^{\star}$ en deçà de $\varepsilon$).

  \end{itemize}

\end{frame}


\begin{frame}
  \frametitle{Suites, II}
  \framesubtitle{Limite d'une suite (b)}
  \hypertarget{slide_suite_limite_2}{}

  \bigskip

  \begin{example}
    Montrons que la suite $(u_n)_{n\in\mathbb N}$ de terme général $u_n = \frac{n-2}{2n}$ (pour $n\in\mathbb N^{\star}$) converge vers $u^{\star} = \frac{1}{2}$ quand $n\rightarrow\infty$. Nous avons~:
    \[
      \begin{split}
        |u_n-u^{\star}| &= \left|\frac{n-2}{2n}-\frac{1}{2}\right|\\
        &= \frac{1}{n}
      \end{split}
    \]
    Ainsi la distance entre $u_n$ et $u^{\star}$ est plus petite que $\varepsilon>0$ si $n>\frac{1}{\varepsilon}\equiv N(\varepsilon)$, et donc~:
    \[
      \forall \varepsilon>0, \exists N(\varepsilon) \text{ tel que } |u_n-u^{\star}|<\varepsilon\,\, \forall \, n>N(\varepsilon)
    \]
  \end{example}

  \bigskip

  \begin{itemize}

  \item On établit la convergence d'une suite vers une limite en trouvant le rang $N(\varepsilon)$

  \end{itemize}

\end{frame}


\begin{frame}
  \frametitle{Suites, II}
  \framesubtitle{Limite d'une suite (c)}
  \hypertarget{slide_suite_limite_3}{}

  \bigskip

  \begin{theorem}
    \medskip
    \begin{enumerate}

    \item Si une suite $(u_n)_{n\in\mathbb N}$ admet $u^{\star}$ pour limite, alors celle-ci est unique,\newline

    \item Toute suite $(u_n)_{n\in\mathbb N}$ convergente est bornée,\newline

    \item Toute suite croissante (décroissante) et majorée (resp. minorée) est convergente.\newline

    \end{enumerate}
  \end{theorem}

\end{frame}


\begin{frame}
  \frametitle{Suites, II}
  \framesubtitle{Limite d'une suite (d)}
  \hypertarget{slide_suite_limite_4}{}

  \bigskip

  \begin{example}

    On peut montrer que la suite $(u_n)_{n\in\mathbb N}$ de terme général $u_n = \frac{n+2}{n}$ est décroissante et minorée par 1. En effet~:
    \[
      u_n-u_{n-1} = \frac{n+2}{n}-\frac{n+1}{n-1} = -\frac{2}{n(n-1)}<0\, \forall\, n>1 \text{ (décroissance)}
    \]
    \[
      u_n-1 = \frac{2}{n} > 0\, \forall\, n\in\mathbb N^{\star} \text{ (la suite est minorée par 1)}
    \]
    Ainsi la suite $(u_n)_{n\in\mathbb N}$ est convergente. Montrons que la limite de la suite est $u^{\star}=1$. Nous avons~:
    \[
      |u_n-1| = \frac{2}{n}
    \]
    Pour que $|u_n-1|$ soit arbitrairement petit, il faut que $\frac{2}{n}<\varepsilon$ c'est-à-dire $n>\frac{2}{\varepsilon}\equiv N(\epsilon)$, et donc~:
    \[
      \forall \varepsilon>0, \exists N(\varepsilon) \text{ tel que } |u_n-u^{\star}|<\varepsilon\,\, \forall \, n>N(\varepsilon)
    \]
  \end{example}

\end{frame}


\begin{frame}
  \frametitle{Suites, II}
  \framesubtitle{Limite d'une suite (e)}
  \hypertarget{slide_suite_limite_5}{}

  \bigskip

  \begin{theorem}
    Soient $(u_n)_{n\in\mathbb N}$ et $(v_n)_{n\in\mathbb N}$ deux suites qui convergent respectivement vers $u^{\star}$ et $v^{\star}$. Alors~:\newline

    \begin{enumerate}

    \item $(u_n+v_n)_{n\in\mathbb N}$ converge vers $u^{\star}+v^{\star}$,\newline

    \item $(\lambda u_n)_{n\in\mathbb N}$ converge vers $\lambda u^{\star}$,\newline

    \item $(u_n\cdot v_n)_{n\in\mathbb N}$ converge vers $u^{\star}\cdot v^{\star}$,\newline

    \item Si $v^{\star}\neq 0$, $\left(\frac{u_n}{v_n}\right)_{n\in\mathbb N}$ converge vers $\frac{u^{\star}}{v^{\star}}$.\newline
    \end{enumerate}
  \end{theorem}

\end{frame}


\begin{frame}
  \frametitle{Suites, II}
  \framesubtitle{Suites extraites}
  \hypertarget{slide_suites_extraites}{}

  \bigskip

  \begin{definition}
    Toute suite $(u_{\varphi(n)})_{n\in\mathbb N}$ où $\varphi$ est une fonction de $\mathbb N$ dans un sous ensemble de $\mathbb N$ strictement croissante est appelée une suite extraite (ou sous suite) de la suite $(u_n)_{n\in\mathbb N}$.
  \end{definition}

  \bigskip

  \begin{example}\label{ex:suite_extraite_divergente_0}
    Soit la suite $(u_n)_{n\in\mathbb N^{\star}}$ définie par le terme général $u_n = (-1)^n\left(1-\frac{1}{n}\right)$. Posons $\varphi_1(k) = 2k$ et $\varphi_2(k)=2k+1$. Ces deux fonctions prennent des entiers naturels en entrée et retourne des entiers naturels pairs, pour $\varphi_1$, ou impairs, pour $\varphi_2$. Ces deux fonctions croissantes nous permettent de considérer une partition de $\mathbb N$.\newline

    La suite extraite $(u_{\varphi_1(n)})_{n\in\mathbb N}$ est positive, alors que la suite extraite $(u_{\varphi_2(n)})_{n\in\mathbb N}$ est négative.
  \end{example}

\end{frame}


\begin{frame}
  \frametitle{Suites, II}
  \framesubtitle{Limite d'une suite (f)}
  \hypertarget{slide_suite_limite_6}{}

  \bigskip

  \begin{theorem}
    Toute suite extraite d'une suite $(u_n)_{n\in\mathbb N}$ convergente converge vers la même limite.
  \end{theorem}

  \bigskip

  \begin{theorem}
    Soit une suite $(u_n)_{n\in\mathbb N}$. Pour que  cette suite converge vers $u^{\star}$ il faut et il suffit que les suites extraites $(u_{2n})_{n\in\mathbb N}$ et $(u_{2n+1})_{n\in\mathbb N}$ convergent vers $u^{\star}$.
  \end{theorem}

  \bigskip

  \begin{theorem}[Bolzano-Weierstrass]
    On peut extraire une sous suite convergente de toute suite bornée.
  \end{theorem}

\end{frame}


\begin{frame}
  \frametitle{Suites, II}
  \framesubtitle{Limite d'une suite (g)}
  \hypertarget{slide_suite_limite_7}{}

  \bigskip

  \begin{example}[suite de l'exemple \hyperlink{slide_suites_extraites}{\ref{ex:suite_extraite_divergente_0}}]
    La suite $(u_n)_{n\in\mathbb N}$ est bornée. En effet on montre
    facilement que $u_n\leq 1$ pour tout $n$ et que que $u_n \geq
    -1$. La première sous suite, définie par $\varphi_1$, sélectionne les
    termes positifs, montrons qu'elle est majorée par 1~:
    \[
      u_{2n}-1 = 1-\frac{1}{2n}-1 = -\frac{1}{2n} < 0 \,\forall\, n\in\mathbb n
    \]
    La seconde sous suite, définie par $\varphi_2$, sélectionne les
    termes négatifs, montrons qu'elle est minorée par -1~:
    \[
      u_{2n+1}+1 = -\left(1-\frac{1}{2n+1}\right)+1 = \frac{1}{2n+1} > 0 \,\forall\, n\in\mathbb n
    \]
    Donc la suite est bornée.
  \end{example}

\end{frame}


\begin{frame}
  \frametitle{Suites, II}
  \framesubtitle{Limite d'une suite (h)}
  \hypertarget{slide_suite_limite_8}{}

  \bigskip

  \addtocounter{example}{-1}
  \begin{example}[suite de l'exemple \hyperlink{slide_suites_extraites}{\ref{ex:suite_extraite_divergente_0}}]
    On peut montrer que chaque sous suite est convergente. On montre
    que $\lim_{n\rightarrow\infty}u_{\varphi_1(n)} = 1$~:
    \[
      |u_{\varphi_1(n)}-1| = \left|-\frac{1}{2n}\right| = \frac{1}{2n}
    \]
    Pour que $|u_{\varphi_1(n)}-1|$ soit inférieur à $\varepsilon$,
    c'est-à-dire $\frac{1}{2n}<\varepsilon$, il faut que $n>\frac{1}{2\varepsilon}\equiv N(\varepsilon)$. Ainsi~:
    \[
      \forall \varepsilon>0, \exists N(\varepsilon) \text{ tel que } |u_{\varphi_1(n)}-1|<\varepsilon\,\, \forall \, n>N(\varepsilon)
    \]
    De la même façon on montre que $\lim_{n\rightarrow\infty}u_{\varphi_2(n)} = -1$.\newline

    \textdbend Les deux suites extraites convergent vers des limites différentes !
    \begin{center}
      $\Rightarrow$ La suite $(u_n)_{n\in\mathbb N}$ diverge.
    \end{center}

  \end{example}

\end{frame}


\begin{frame}
  \frametitle{Suites, II}
  \framesubtitle{Suites divergentes (a)}
  \hypertarget{slide_suite_divergente_1}{}

  \bigskip

  \begin{definition}
    On dit qu'une suite $(u_n)_{n\in\mathbb N}$ est dite divergente si elle ne converge pas~:
    \[
      \forall \ell\in\mathbb R, \exists\varepsilon>0, \text{ tel que } \forall\, N\in\mathbb N\, \text{ on ait }  |u_n-\ell|>\varepsilon
    \]
  \end{definition}

  \bigskip

  \begin{itemize}

  \item Une suite peut diverger vers $\infty$~:
    \[
      \forall \mathcal A \in\mathbb R, \exists N\in\mathbb N\, \text{ tel que } n\geq N \Rightarrow u_n\geq \mathcal A
    \]

  \item Une suite peut diverger vers $-\infty$~:
    \[
      \forall \mathcal A \in\mathbb R, \exists N\in\mathbb N\, \text{ tel que } n\geq N \Rightarrow u_n\leq \mathcal A
    \]

  \end{itemize}

\end{frame}


\begin{frame}
  \frametitle{Suites, II}
  \framesubtitle{Suites divergentes (b)}
  \hypertarget{slide_suite_divergente_2}{}

  \bigskip

  \begin{theorem}

    \begin{enumerate}

    \item Toute suite non bornée est divergente,\newline

    \item Toute suite admettant une sous suite divergente est divergente.\newline

    \item Toute suite admettant des sous suites avec des limites différentes diverge.\newline

    \end{enumerate}

  \end{theorem}

\end{frame}


\end{document}

% Local Variables:
% ispell-check-comments: exclusive
% ispell-local-dictionary: "francais"
% TeX-master: t
% End: