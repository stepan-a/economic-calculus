\synctex=1

\documentclass[10pt,notheorems]{beamer}

\usepackage{etex}
\usepackage{fourier-orns}
\usepackage{ccicons}
\usepackage{amssymb}
\usepackage{amstext}
\usepackage{amsbsy}
\usepackage{amsopn}
\usepackage{amscd}
\usepackage{amsxtra}
\usepackage{amsthm}
\usepackage{float}
\usepackage{color, colortbl}
\usepackage{mathrsfs}
\usepackage{bm}
\usepackage{lastpage}
\usepackage[nice]{nicefrac}
\usepackage{setspace}
\usepackage{ragged2e}
\usepackage{listings}
\usepackage{polynom}
\usepackage{algorithms/algorithm}
\usepackage{algorithms/algorithmic}
\usepackage[frenchb]{babel}
\usepackage{tikz,pgfplots}
\pgfplotsset{compat=newest}
\usetikzlibrary{patterns, arrows, decorations.pathreplacing, decorations.markings, calc}
\pgfplotsset{plot coordinates/math parser=false}
\newlength\figureheight
\newlength\figurewidth
\usepackage[utf8x]{inputenc}
\usepackage{cancel}
\usepackage{tikz-qtree}
\usepackage{dcolumn}
\usepackage{adjustbox}
\usepackage{environ}
\usepackage[cal=boondox]{mathalfa}
\usepackage{manfnt}
\usepackage{hyperref}
\hypersetup{
  colorlinks=true,
  linkcolor=blue,
  filecolor=black,
  urlcolor=black,
}
\usepackage{venndiagram}
\usepackage{minted}

% Git hash
\usepackage{xstring}
\usepackage{catchfile}
\immediate\write18{git rev-parse HEAD > git.hash}
\CatchFileDef{\HEAD}{git.hash}{\endlinechar=-1}
\newcommand{\gitrevision}{\StrLeft{\HEAD}{7}}

% Include pdf page by page
\usepackage{pdfpages}


\newcommand{\trace}{\mathrm{tr}}
\newcommand{\vect}{\mathrm{vec}}
\newcommand{\tracarg}[1]{\mathrm{tr}\left\{#1\right\}}
\newcommand{\vectarg}[1]{\mathrm{vec}\left(#1\right)}
\newcommand{\vecth}[1]{\mathrm{vech}\left(#1\right)}
\newcommand{\iid}[2]{\mathrm{iid}\left(#1,#2\right)}
\newcommand{\normal}[2]{\mathcal N\left(#1,#2\right)}
\newcommand{\dynare}{\href{http://www.dynare.org}{\color{blue}Dynare}}
\newcommand{\sample}{\mathcal Y_T}
\newcommand{\samplet}[1]{\mathcal Y_{#1}}
\newcommand{\slidetitle}[1]{\fancyhead[L]{\textsc{#1}}}

\newcommand{\R}{{\mathbb R}}
\newcommand{\C}{{\mathbb C}}
\newcommand{\N}{{\mathbb N}}
\newcommand{\Z}{{\mathbb Z}}
\newcommand{\binomial}[2]{\begin{pmatrix} #1 \\ #2 \end{pmatrix}}
\newcommand{\bigO}[1]{\mathcal O \left(#1\right)}
\newcommand{\red}{\color{red}}
\newcommand{\blue}{\color{blue}}

\renewcommand{\qedsymbol}{C.Q.F.D.}

\newcolumntype{d}{D{.}{.}{-1}}
\definecolor{gray}{gray}{0.9}
\newcolumntype{g}{>{\columncolor{gray}}c}

\setbeamertemplate{theorems}[numbered]

\theoremstyle{plain}
\newtheorem{theorem}{Théorème}

\theoremstyle{definition} % insert bellow all blocks you want in normal text
\newtheorem{definition}{Définition}
\newtheorem{properties}{Propriétés}
\newtheorem{lemma}{Lemme}
\newtheorem{property}[properties]{Propriété}
\newtheorem{example}{Exemple}
\newtheorem*{idea}{Éléments de preuve} % no numbered block



\setbeamertemplate{footline}{
  {\hfill\vspace*{1pt}\href{http://creativecommons.org/licenses/by-sa/3.0/legalcode}{\ccbysa}\hspace{.1cm}
    \raisebox{-.075cm}{\href{https://git.adjemian.eu/stepan/economic-calculus}{\includegraphics[scale=.1]{../img/gitlab.png}}}\enspace
    \href{https://git.adjemian.eu/stepan/economic-calculus/-/blob/\HEAD/cours/chapitre-3.tex}{\gitrevision}\enspace\today
  }\hspace{1cm}}

\setbeamertemplate{navigation symbols}{}
\setbeamertemplate{blocks}[rounded][shadow=true]
\setbeamertemplate{caption}[numbered]

\NewEnviron{notes}{\justifying\tiny\begin{spacing}{1.0}\BODY\vfill\pagebreak\end{spacing}}

\newenvironment{exercise}[1]
{\bgroup \small\begin{block}{Ex. #1}}
  {\end{block}\egroup}

\newenvironment{defn}[1]
{\bgroup \small\begin{block}{Définition. #1}}
  {\end{block}\egroup}

\newenvironment{exemple}[1]
{\bgroup \small\begin{block}{Exemple. #1}}
  {\end{block}\egroup}

\begin{document}

\title{Calcul Économique\\\small{III. Suites, limites et continuité}}
\author[S. Adjemian]{Stéphane Adjemian}
\institute{\texttt{stephane.adjemian@univ-lemans.fr}} \date{Septembre 2020}

\begin{frame}
  \titlepage{}
\end{frame}

\begin{frame}
  \frametitle{Plan}
  \tableofcontents
\end{frame}


\section{Suites}

\begin{frame}
  \frametitle{Suites, I}
  \framesubtitle{Définition}
  \hypertarget{slide_suites_1}{}

  \begin{itemize}

  \item Une suite est une famille d'éléments, les « termes » de la suite, indexée par $\mathbb N$.\newline

  \item Lorsque tous les éléments d'une suite appartiennent à un même ensemble $E$, cette suite peut être assimilée à une fonction de $\mathbb N$ dans $E$.\newline

  \item Une suite est dite entière si $E$ est un sous ensemble de $\mathbb Z$, réelle si $E$ est un sous ensemble de $\mathbb R$, ou complexe si $E$ est un sous-ensemble $\mathbb C$.\newline

  \item On note habituellement $u_0$, $u_1$, $u_2$, \ldots les termes d'une suite.\newline

  \item Une suite peut être définie en donnant une expression générale de ses termes, ou à l'aide d'une expression récursive en définissant chaque terme en fonction de termes précédents.\newline

  \end{itemize}

\end{frame}


\begin{frame}
  \frametitle{Suites, I}
  \framesubtitle{Exemple}
  \hypertarget{slide_suites_2}{}

  \begin{itemize}

  \item On définit la suite $(u_n)_{n\in\mathbb N}$~, en donnant l'expression générale du $n$-ième terme~:
    \[
      u_n = (-1)^n
    \]

  \item Cela donne~:\newline
    \begin{center}
      \begin{table}[H]
        \centering
        \begin{tabular}{c|cccccccc}
          $n$ & 0 & 1 & 2 & 3 & 4 & \ldots & $n$ & \ldots\\\hline
          $u_n$ & 1 & -1 & 1 & -1 & 1 &     & $(-1)^n$ &
        \end{tabular}
      \end{table}
    \end{center}

    \bigskip

  \item Alternativement, nous pouvons définir la suite récursivement~:
    \[
      u_n = -1\times u_{n-1}
    \]
    avec la condition initiale $u_0 = 1$.\newline

  \item Quand les signes des termes successifs sont toujours opposés, on dit que la suite est \textbf{alternée}.
  \end{itemize}

\end{frame}


\begin{frame}
  \frametitle{Suites, I}
  \framesubtitle{Monotonie}
  \hypertarget{slide_suites_3}{}

  \begin{itemize}

  \item Une suite est dite \textbf{monotone croissante} si $u_n>u_{n-1}$ pour tout $n\geq 1$.\newline

  \item Une suite est dite \textbf{monotone décroissante} si $u_n<u_{n-1}$ pour tout $n\geq 1$.\newline

  \item On dira qu'une suite est croissante (décroissante) si $u_n\geq u_{n-1}$ (respectivement $u_n\leq u_{n-1}$.\newline

  \end{itemize}

  \begin{example}
    La suite $(u_n)_{n\in\mathbb N^{\star}}$ de terme général $u_n = \frac{1}{n}$ est monotone décroissante. En effet~:
    \[
      u_n - u_{n-1} = \frac{1}{n}-\frac{1}{n-1} = \frac{n-1-n}{n(n-1)} = -\frac{1}{n(n-1)}<0 \forall n\geq 2
    \]
  \end{example}

\end{frame}


\begin{frame}
  \frametitle{Suites, I}
  \framesubtitle{Majorant et minorant}
  \hypertarget{slide_suites_4}{}

  \begin{itemize}

  \item Une suite est \textbf{bornée} s'il existe $k$ et $K$ tels que $k\leq u_n\leq K$ $\forall n\geq 0$.\newline

  \item $k$ ($K$) est un \textbf{minorant} (\textbf{majorant}) de la suite.\newline

  \item Le plus grand minorant (petit majorant) est la \textbf{borne inférieure} (\textbf{borne supérieure}) de la suite.\newline

  \end{itemize}

  \begin{example}
    La suite $(u_n)_{n\in\mathbb N^{\star}}$ de terme général $u_n = \frac{n-2}{2n}$ est bornée. En effet $-\frac{1}{2}\leq u_n\leq \frac{1}{2}$~:
    \[
      u_n - \frac{1}{2} =  \frac{n-2}{2n}-\frac{1}{2}=\frac{1}{2}-\frac{2}{2n}-\frac{1}{2}=-\frac{1}{n}<0 \quad\forall n\geq 1
    \]
    \[
      u_n+\frac{1}{2} = \frac{n-2}{2n}+\frac{1}{2} = 1 - \frac{1}{n} \geq 0\quad \forall n\geq 1
    \]
  \end{example}

\end{frame}


\begin{frame}
  \frametitle{Suites, I}
  \framesubtitle{Suites arithmétiques (a)}
  \hypertarget{slide_suites_arithmetiques_1}{}

  \bigskip

  \begin{definition}
    Une suite $(u_n)_{n\in\mathbb N}$ est dite arithmétique si ses variations sont constantes, c'est-à-dire si~:
    \[
      \Delta u_n \equiv u_n-u_{n-1} = r\quad \forall n\in\mathbb N^{\star}
    \]
    On dit que la constante $r$ est la raison de la suite.
  \end{definition}

  \bigskip

  \begin{itemize}

  \item La suite arithmétique est monotone croissante si et seulement si sa raison est positive.\newline

  \item Nous avons ici une définition récursive de la suite arithmétique~:
    \[
      u_n = u_{n-1} + r
    \]
    avec une condition initial $u_0$.\newline

  \item Il est possible de définir celle-ci en donnant son terme général.\newline

  \end{itemize}

\end{frame}


\begin{frame}
  \frametitle{Suites, I}
  \framesubtitle{Suites arithmétiques (b)}
  \hypertarget{slide_suites_arithmetiques_2}{}

  \bigskip

  \begin{itemize}

  \item Pour obtenir le terme général on substitue la définition dans le membre de droite, autant de fois que nécessaire pour remonter jusqu'à la condition initiale~:
    \[
      \begin{split}
        u_n &= u_{n-1} + r\\
        &= u_{n-2} + r + r{\color{gray}\text{, car }u_{n-1}=u_{n-2}+r}\\
        \Leftrightarrow u_n &= u_{n-2} + 2r\\
        &\vdots\\
        u_n &= u_0 + nr{\color{gray}\text{, le terme général de la suite.}}
      \end{split}
    \]

    \bigskip

  \item Pour vérifier que le terme général est correct, on peut le substituer dans l'expression récursive.\newline

  \end{itemize}

  \begin{example}
    La suite $(u_n)_{n\in\mathbb N}$ définie par $u_n = n$ est une suite arithmétique de raison 1 et avec pour condition initiale $u_0=0$.
  \end{example}

\end{frame}


\begin{frame}
  \frametitle{Suites, I}
  \framesubtitle{Suites géométriques (a)}
  \hypertarget{slide_suites_geometrique_1}{}

  \bigskip

  \begin{definition}
    Une suite $(u_n)_{n\in\mathbb N}$ est dite géométrique si le rapport de deux termes consécutifs est constant, c'est-à-dire si~:
    \[
      \frac{u_n}{u_{n-1}} = r\quad \forall n\in\mathbb N^{\star}
    \]
    On dit que la constante $r$ est la raison de la suite.
  \end{definition}

  \bigskip

  \begin{itemize}

  \item La suite arithmétique est monotone croissante si et seulement si sa raison est supérieure à 1.\newline

  \item Nous avons ici une définition récursive de la suite arithmétique~:
    \[
      u_n = r u_{n-1}
    \]
    avec une condition initial $u_0$.\newline

  \item Il est possible de définir celle-ci en donnant son terme général.\newline

  \end{itemize}

\end{frame}


\begin{frame}
  \frametitle{Suites, I}
  \framesubtitle{Suites géométriques (b)}
  \hypertarget{slide_suites_geometriques_2}{}

  \bigskip

  \begin{itemize}

  \item Pour obtenir le terme général on substitue la définition dans le membre de droite, autant de fois que nécessaire pour remonter jusqu'à la condition initiale~:
    \[
      \begin{split}
        u_n &= r u_{n-1}\\
        &= r r u_{n-2} {\color{gray}\text{, car }u_{n-1}=r u_{n-2}}\\
        \Leftrightarrow u_n &= r^2 u_{n-2}\\
        &\vdots\\
        u_n &= r^nu_0 {\color{gray}\text{, le terme général de la suite.}}
      \end{split}
    \]

    \bigskip

  \item Pour vérifier que le terme général est correct, on peut le substituer dans l'expression récursive.\newline

  \end{itemize}

  \begin{example}
    La suite $(u_n)_{n\in\mathbb N}$ définie par $u_n = (1+g)^n$ est une suite géométrique de raison $1+g$ et avec pour condition initiale $u_0=1$.
  \end{example}

\end{frame}


\begin{frame}
  \frametitle{Suites, II}
  \framesubtitle{Limite d'une suite (a)}
  \hypertarget{slide_suite_limite_1}{}

  \bigskip

  \begin{definition}
    On dit qu'une suite $(u_n)_{n\in\mathbb N}$ converge vers une limite $u^{\star}$ quand $n$ tend vers l'infini, et on note $\lim_{n\rightarrow\infty}u_n = u^{\star}$, si et seulement si~:
    \[
      \forall \varepsilon>0, \exists N\text{ tel que } |u_n-u^{\star}|<\varepsilon\,\, \forall \, n>N
    \]
  \end{definition}

  \bigskip

  \begin{itemize}

  \item $|u_n-u^{\star}|$ mesure la distance entre le n-ième terme de la suite et la limite $u^{\star}$.\newline

  \item La suite converge vers $u^{\star}$ s'il est possible de rendre la distance entre $u_n$ et $u^{\star}$ arbitrairement petite au delà d'un certain rang ($N$).\newline

  \item Le rang à partir duquel les écarts sont plus petits que $\varepsilon$ dépend a priori de $\varepsilon$. Plus $\varepsilon$ est proche de 0 plus le rang sera élevé (plus loin il faudra aller pour contenir la distance entre $u_n$ et $u^{\star}$ en deçà de $\varepsilon$).

  \end{itemize}

\end{frame}


\begin{frame}
  \frametitle{Suites, II}
  \framesubtitle{Limite d'une suite (b)}
  \hypertarget{slide_suite_limite_2}{}

  \bigskip

  \begin{example}
    Montrons que la suite $(u_n)_{n\in\mathbb N}$ de terme général $u_n = \frac{n-2}{2n}$ (pour $n\in\mathbb N^{\star}$) converge vers $u^{\star} = \frac{1}{2}$ quand $n\rightarrow\infty$. Nous avons~:
    \[
      \begin{split}
        |u_n-u^{\star}| &= \left|\frac{n-2}{2n}-\frac{1}{2}\right|\\
        &= \frac{1}{n}
      \end{split}
    \]
    Ainsi la distance entre $u_n$ et $u^{\star}$ est plus petite que $\varepsilon>0$ si $n>\frac{1}{\varepsilon}\equiv N(\varepsilon)$, et donc~:
    \[
      \forall \varepsilon>0, \exists N(\varepsilon) \text{ tel que } |u_n-u^{\star}|<\varepsilon\,\, \forall \, n>N(\varepsilon)
    \]
  \end{example}

  \bigskip

  \begin{itemize}

  \item On établit la convergence d'une suite vers une limite en trouvant le rang $N(\varepsilon)$

  \end{itemize}

\end{frame}


\begin{frame}
  \frametitle{Suites, II}
  \framesubtitle{Limite d'une suite (c)}
  \hypertarget{slide_suite_limite_3}{}

  \bigskip

  \begin{theorem}
    \medskip
    \begin{enumerate}

    \item Si une suite $(u_n)_{n\in\mathbb N}$ admet $u^{\star}$ pour limite, alors celle-ci est unique,\newline

    \item Toute suite $(u_n)_{n\in\mathbb N}$ convergente est bornée,\newline

    \item Toute suite croissante (décroissante) et majorée (resp. minorée) est convergente.\newline

    \end{enumerate}
  \end{theorem}

\end{frame}


\begin{frame}
  \frametitle{Suites, II}
  \framesubtitle{Limite d'une suite (d)}
  \hypertarget{slide_suite_limite_4}{}

  \bigskip

  \begin{example}

    On peut montrer que la suite $(u_n)_{n\in\mathbb N}$ de terme général $u_n = \frac{n+2}{n}$ est décroissante et minorée par 1. En effet~:
    \[
      u_n-u_{n-1} = \frac{n+2}{n}-\frac{n+1}{n-1} = -\frac{2}{n(n-1)}<0\, \forall\, n>1 \text{ (décroissance)}
    \]
    \[
      u_n-1 = \frac{2}{n} > 0\, \forall\, n\in\mathbb N^{\star} \text{ (la suite est minorée par 1)}
    \]
    Ainsi la suite $(u_n)_{n\in\mathbb N}$ est convergente. Montrons que la limite de la suite est $u^{\star}=1$. Nous avons~:
    \[
      |u_n-1| = \frac{2}{n}
    \]
    Pour que $|u_n-1|$ soit arbitrairement petit, il faut que $\frac{2}{n}<\varepsilon$ c'est-à-dire $n>\frac{2}{\varepsilon}\equiv N(\epsilon)$, et donc~:
    \[
      \forall \varepsilon>0, \exists N(\varepsilon) \text{ tel que } |u_n-u^{\star}|<\varepsilon\,\, \forall \, n>N(\varepsilon)
    \]
  \end{example}

\end{frame}


\begin{frame}
  \frametitle{Suites, II}
  \framesubtitle{Limite d'une suite (e)}
  \hypertarget{slide_suite_limite_5}{}

  \bigskip

  \begin{theorem}
    Soient $(u_n)_{n\in\mathbb N}$ et $(v_n)_{n\in\mathbb N}$ deux suites qui convergent respectivement vers $u^{\star}$ et $v^{\star}$. Alors~:\newline

    \begin{enumerate}

    \item $(u_n+v_n)_{n\in\mathbb N}$ converge vers $u^{\star}+v^{\star}$,\newline

    \item $(\lambda u_n)_{n\in\mathbb N}$ converge vers $\lambda u^{\star}$,\newline

    \item $(u_n\cdot v_n)_{n\in\mathbb N}$ converge vers $u^{\star}\cdot v^{\star}$,\newline

    \item Si $v^{\star}\neq 0$, $\left(\frac{u_n}{v_n}\right)_{n\in\mathbb N}$ converge vers $\frac{u^{\star}}{v^{\star}}$.\newline
    \end{enumerate}
  \end{theorem}

\end{frame}


\begin{frame}
  \frametitle{Suites, II}
  \framesubtitle{Suites extraites}
  \hypertarget{slide_suites_extraites}{}

  \bigskip

  \begin{definition}
    Toute suite $(u_{\varphi(n)})_{n\in\mathbb N}$ où $\varphi$ est une fonction de $\mathbb N$ dans un sous ensemble de $\mathbb N$ strictement croissante est appelée une suite extraite (ou sous suite) de la suite $(u_n)_{n\in\mathbb N}$.
  \end{definition}

  \bigskip

  \begin{example}\label{ex:suite_extraite_divergente_0}
    Soit la suite $(u_n)_{n\in\mathbb N^{\star}}$ définie par le terme général $u_n = (-1)^n\left(1-\frac{1}{n}\right)$. Posons $\varphi_1(k) = 2k$ et $\varphi_2(k)=2k+1$. Ces deux fonctions prennent des entiers naturels en entrée et retourne des entiers naturels pairs, pour $\varphi_1$, ou impairs, pour $\varphi_2$. Ces deux fonctions croissantes nous permettent de considérer une partition de $\mathbb N$.\newline

    La suite extraite $(u_{\varphi_1(n)})_{n\in\mathbb N}$ est positive, alors que la suite extraite $(u_{\varphi_2(n)})_{n\in\mathbb N}$ est négative.
  \end{example}

\end{frame}


\begin{frame}
  \frametitle{Suites, II}
  \framesubtitle{Limite d'une suite (f)}
  \hypertarget{slide_suite_limite_6}{}

  \bigskip

  \begin{theorem}
    Toute suite extraite d'une suite $(u_n)_{n\in\mathbb N}$ convergente converge vers la même limite.
  \end{theorem}

  \bigskip

  \begin{theorem}
    Soit une suite $(u_n)_{n\in\mathbb N}$. Pour que  cette suite converge vers $u^{\star}$ il faut et il suffit que les suites extraites $(u_{2n})_{n\in\mathbb N}$ et $(u_{2n+1})_{n\in\mathbb N}$ convergent vers $u^{\star}$.
  \end{theorem}

  \bigskip

  \begin{theorem}[Bolzano-Weierstrass]
    On peut extraire une sous suite convergente de toute suite bornée.
  \end{theorem}

\end{frame}


\begin{frame}
  \frametitle{Suites, II}
  \framesubtitle{Limite d'une suite (g)}
  \hypertarget{slide_suite_limite_7}{}

  \bigskip

  \begin{example}[suite de l'exemple \hyperlink{slide_suites_extraites}{\ref{ex:suite_extraite_divergente_0}}]
    La suite $(u_n)_{n\in\mathbb N}$ est bornée. En effet on montre
    facilement que $u_n\leq 1$ pour tout $n$ et que que $u_n \geq
    -1$. La première sous suite, définie par $\varphi_1$, sélectionne les
    termes positifs, montrons qu'elle est majorée par 1~:
    \[
      u_{2n}-1 = 1-\frac{1}{2n}-1 = -\frac{1}{2n} < 0 \,\forall\, n\in\mathbb n
    \]
    La seconde sous suite, définie par $\varphi_2$, sélectionne les
    termes négatifs, montrons qu'elle est minorée par -1~:
    \[
      u_{2n+1}+1 = -\left(1-\frac{1}{2n+1}\right)+1 = \frac{1}{2n+1} > 0 \,\forall\, n\in\mathbb n
    \]
    Donc la suite est bornée.
  \end{example}

\end{frame}


\begin{frame}
  \frametitle{Suites, II}
  \framesubtitle{Limite d'une suite (h)}
  \hypertarget{slide_suite_limite_8}{}

  \bigskip

  \addtocounter{example}{-1}
  \begin{example}[suite de l'exemple \hyperlink{slide_suites_extraites}{\ref{ex:suite_extraite_divergente_0}}]
    On peut montrer que chaque sous suite est convergente. On montre
    que $\lim_{n\rightarrow\infty}u_{\varphi_1(n)} = 1$~:
    \[
      |u_{\varphi_1(n)}-1| = \left|-\frac{1}{2n}\right| = \frac{1}{2n}
    \]
    Pour que $|u_{\varphi_1(n)}-1|$ soit inférieur à $\varepsilon$,
    c'est-à-dire $\frac{1}{2n}<\varepsilon$, il faut que $n>\frac{1}{2\varepsilon}\equiv N(\varepsilon)$. Ainsi~:
    \[
      \forall \varepsilon>0, \exists N(\varepsilon) \text{ tel que } |u_{\varphi_1(n)}-1|<\varepsilon\,\, \forall \, n>N(\varepsilon)
    \]
    De la même façon on montre que $\lim_{n\rightarrow\infty}u_{\varphi_2(n)} = -1$.\newline

    \textdbend Les deux suites extraites convergent vers des limites différentes !
    \begin{center}
      $\Rightarrow$ La suite $(u_n)_{n\in\mathbb N}$ diverge.
    \end{center}

  \end{example}

\end{frame}


\begin{frame}
  \frametitle{Suites, II}
  \framesubtitle{Suite de Cauchy (a)}
  \hypertarget{slide_suite_cauchy_1}{}

  \bigskip

  \begin{definition}
    Une suite $(u_n)_{n\in\mathbb N}$ est dite de Cauchy si et seulement si~:
    \[
      \forall\,\varepsilon>0,\, \exists N\in\mathbb N\,\text{ tel que } \forall m>N,\, \forall n>N\, \text{ on ait }|u_n-u_m|<\varepsilon
    \]
  \end{definition}

  \bigskip

  \begin{theorem}
    Une suite converge si et seulement si elle est une suite de Cauchy.
  \end{theorem}

\end{frame}


\begin{frame}
  \frametitle{Suites, II}
  \framesubtitle{Suite de Cauchy (b)}
  \hypertarget{slide_suite_cauchy_1}{}

  \bigskip

  \begin{example}
    Soit la suite $(u_n)_{n\in\mathbb N}$ définie par le terme général $u_n = \frac{(-1)^n}{n}$. On peut montrer qu'il s'agit d'une suite de Cauchy. On a~:
    \[
      \begin{split}
        |u_n-u_m| &= \left|\frac{(-1)^n}{n}-\frac{(-1)^m}{m}\right|\\
        &\leq \left|\frac{(-1)^n}{n}\right|+\left|\frac{(-1)^m}{m}\right|\\
        &= \frac{1}{n} + \frac{1}{m} < \varepsilon
      \end{split}
    \]
    où la dernière inégalité est vérifiée dès lors que $n$ et $m$ sont plus grands que $N(\varepsilon)=\frac{2}{\varepsilon}$. Ainsi~:
    \[
      \forall\,\varepsilon>0,\, \exists N(\varepsilon)\in\mathbb N\,\text{ tel que } \forall m>N(\varepsilon),\, \forall n>N(\varepsilon)\, \text{ on ait }|u_n-u_m|<\varepsilon
    \]
  \end{example}

\end{frame}


\begin{frame}
  \frametitle{Suites, II}
  \framesubtitle{Suites divergentes (a)}
  \hypertarget{slide_suite_divergente_1}{}

  \bigskip

  \begin{definition}
    On dit qu'une suite $(u_n)_{n\in\mathbb N}$ est dite divergente si elle ne converge pas~:
    \[
      \forall \ell\in\mathbb R, \exists\varepsilon>0, \text{ tel que } \forall\, N\in\mathbb N\, \text{ on ait }  |u_n-\ell|>\varepsilon
    \]
  \end{definition}

  \bigskip

  \begin{itemize}

  \item Une suite peut diverger vers $\infty$~:
    \[
      \forall \mathcal A \in\mathbb R, \exists N\in\mathbb N\, \text{ tel que } n\geq N \Rightarrow u_n\geq \mathcal A
    \]

  \item Une suite peut diverger vers $-\infty$~:
    \[
      \forall \mathcal A \in\mathbb R, \exists N\in\mathbb N\, \text{ tel que } n\geq N \Rightarrow u_n\leq \mathcal A
    \]

  \end{itemize}

\end{frame}


\begin{frame}
  \frametitle{Suites, II}
  \framesubtitle{Suites divergentes (b)}
  \hypertarget{slide_suite_divergente_2}{}

  \bigskip

  \begin{theorem}

    \begin{enumerate}

    \item Toute suite non bornée est divergente,\newline

    \item Toute suite admettant une sous suite divergente est divergente.\newline

    \item Toute suite admettant des sous suites avec des limites différentes diverge.\newline

    \end{enumerate}

  \end{theorem}

\end{frame}


\begin{frame}
  \frametitle{Suites, II}
  \framesubtitle{Suites divergentes (c)}
  \hypertarget{slide_suite_divergente_3}{}

  \bigskip

  \begin{example}

    Soit une suite arithmétique $(u_n)_{n\in\mathbb N}$ de raison
    $r>0$. Montrons que cette suite diverge. Le terme général de cette
    suite est $u_n = u_0+nr$, pour simplifier (et sans perte de
    généralité) on suppose que $u_0=0$ et donc $u_n=nr$.\newline

    Cette suite est monotone croissante, puisque $r=u_n-u_{n-1}>0$
    pour tout $n>1$. Ainsi pour tout $A>0$ on peut toujours trouver un
    rang $N$ ($A/r$) tel que $u_n>A$ pour tout $n>N$. On peut donc
    toujours trouver les termes arbitrairement grands à partir d'un
    certain rang $\Rightarrow$ la suite diverge vers $+\infty$.

  \end{example}

\end{frame}


\begin{frame}
  \frametitle{Suites, II}
  \framesubtitle{limites usuelles}
  \hypertarget{slide_suite_limites_usuelles}{}

  \begin{table}
    \centering
    \begin{tabular}{c|c||c|c}
      $u_n$ & $\lim_{n\rightarrow\infty}u_n$ & $u_n$ & $\lim_{n\rightarrow\infty}u_n$\\\hline
      $\frac{1}{n}$ & 0 & $\sqrt[n]{n}$ & 1\\
      $\frac{1}{n^{\alpha}}$ & 0 & $\left(1+\frac{1}{n}\right)^n$ & $e$\\
      $\sqrt[n]{q},\,q>0$ & 1 & $\frac{\log_b n}{n}$ & 0\\
      $q^n,\,|q|<1$ & 0 & $\frac{\log_b n}{n^\alpha}$ & 0\\
    \end{tabular}
    \caption{Avec $\alpha>0$, $b>0$, $b\neq 1$ et $e$ la constante d'Euler.}
  \end{table}

\end{frame}


\section{Limites d'une fonction}


\begin{frame}
  \frametitle{Limites d'une fonction}

  \begin{itemize}

  \item Une suite réelle est une fonction de $\mathbb N$ dans $\mathbb R$ (ou un sous ensemble de $\mathbb R$).\newline

  \item Étudier le comportement limite d'une suite est relativement simple car il n'y a qu'une seule direction : on caractérise l'évolution de $u_n$ quand $n$ tend vers $+\infty$.\newline

  \item Avec une fonction, c'est un peu plus compliqué. On s'intéresse au comportement de la fonction $f(x)$ quand $x$ tend vers $-\infty$ ou $+\infty$, mais aussi quand $x$ se rapproche des points de singularité de la fonction (voir l'exemple de la fonction rationnelle dans le \href{https://le-mans.adjemian.eu/calcul-économique/cours/chapitre-2.pdf}{chapitre 2}).\newline

  \item Dans cette section on étend la notion de limite aux fonctions.

  \end{itemize}
\end{frame}


\begin{frame}
  \frametitle{Limites d'une fonction}
  \framesubtitle{Limites en $\pm\infty$ (a)}


  \begin{definition}
    La fonction $f(x)$ admet la limite $\ell<\infty$ lorsque $x$ tend vers l'infini pour tout $\varepsilon>0$, il existe $\nu>0$ tel que $|f(x)-\ell|<\varepsilon$ dès que $x>\nu$. On écrit alors~:
    \[
      \lim_{x\rightarrow\infty} f(x) = \ell
    \]
  \end{definition}

  \medskip

  \begin{itemize}

  \item La fonction $f$ converge vers $\ell$ si on peut rendre la distance entre $f(x)$ et $\ell$ arbitrairement petite quand $x$ est assez grand.\newline

  \end{itemize}

  \begin{definition}
    La fonction $f(x)$ admet la limite $\ell<\infty$ lorsque $x$ tend vers moins l'infini pour tout $\varepsilon>0$, il existe $\nu<0$ tel que $|f(x)-\ell|<\varepsilon$ dès que $x<\nu$. On écrit alors~:
    \[
      \lim_{x\rightarrow -\infty} f(x) = \ell
    \]
  \end{definition}

\end{frame}


\begin{frame}
  \frametitle{Limites d'une fonction}
  \framesubtitle{Limites en $\pm\infty$ (b)}

  Plus formellement~:

  \bigskip

  \addtocounter{definition}{-2}
  \begin{definition}
    \[
      \lim_{x\rightarrow\infty} f(x) = \ell<\infty\,\Leftrightarrow\, \forall\varepsilon>0,\, \exists \nu>0\,\text{ tel que }\, |f(x)-\ell|<\varepsilon\, \forall x>\nu
    \]
  \end{definition}

  \bigskip

  \begin{definition}
    \[
      \lim_{x\rightarrow -\infty} f(x) = \ell<\infty\,\Leftrightarrow\, \forall\varepsilon>0,\, \exists \nu<0\,\text{ tel que }\, |f(x)-\ell|<\varepsilon\, \forall x<\nu
    \]
  \end{definition}

\end{frame}


\begin{frame}
  \frametitle{Limites d'une fonction}
  \framesubtitle{Limites en $\pm\infty$ (c)}

  \begin{example}
    Soit la fontion $f(x) = 2-\frac{1}{x}$ définie de $\mathbb R^{\star}$ dans $\mathbb R$. Montrons que~:
    \[
      \lim_{x\rightarrow\infty}f(x) = 2
    \]
    On a~:
    \[
      \begin{split}
        |f(x)-2| &= \left|2-\frac{1}{x}-2\right|\\
        &= \frac{1}{|x|}
      \end{split}
    \]
    Ainsi la proposition $|f(x)-2|<\varepsilon$ est équivalente à $x>\frac{1}{\varepsilon}\,\lor\, x<-\frac{1}{\varepsilon}$. On pose donc $\nu(\varepsilon) = \frac{1}{\varepsilon}$ et on a bien~:
    \[
      \forall\varepsilon>0,\, \exists \nu(\varepsilon)>0\,\text{ tel que }\, |f(x)-2|<\varepsilon\, \forall x>\nu(\varepsilon)
    \]
  \end{example}

\end{frame}


\begin{frame}
  \frametitle{Limites d'une fonction}
  \framesubtitle{Limites en $\pm\infty$ (d)}

  \bigskip

  \begin{definition}[divergence vers l'infini]
    \[
      \lim_{x\rightarrow\infty} f(x) = \infty\,\Leftrightarrow\, \forall \eta>0,\, \exists \nu>0\,\text{ tel que }\, f(x)>\eta\, \forall x>\nu
    \]
    \[
      \lim_{x\rightarrow\infty} f(x) = -\infty\,\Leftrightarrow\, \forall \eta>0,\, \exists \nu>0\,\text{ tel que }\, f(x)<-\eta\, \forall x>\nu
    \]
    \[
      \lim_{x\rightarrow -\infty} f(x) = \infty\,\Leftrightarrow\, \forall \eta>0,\, \exists \nu>0\,\text{ tel que }\, f(x)>\eta\, \forall x<-\nu
    \]
    \[
      \lim_{x\rightarrow -\infty} f(x) = -\infty\,\Leftrightarrow\, \forall \eta>0,\, \exists \nu>0\,\text{ tel que }\, f(x)<-\eta\, \forall x<-\nu
    \]
  \end{definition}

\end{frame}


\begin{frame}
  \frametitle{Limites d'une fonction}
  \framesubtitle{Limites en $\pm\infty$ (e)}

  \begin{example}
    Soit la fontion $f(x) = x^2$ définie de $\mathbb R$ dans $\mathbb R_+$ diverge vers l'infini. On peut montrer que~:
    \[
      \lim_{x\rightarrow\infty}f(x) = \infty \quad \lim_{x\rightarrow -\infty}f(x) = \infty
    \]
    En effet, pour tout $x>\sqrt{\eta}\equiv \nu(\eta)$ on a $f(x)>\eta$ et cela quel que soit $\eta>0$. Aussi pour tout $x<-\sqrt{\eta}\equiv \nu(\eta)$ on a $f(x)>\eta$ et cela quel que soit $\eta>0$.
  \end{example}

\end{frame}


\begin{frame}
  \frametitle{Limites d'une fonction}
  \framesubtitle{Limites en $\pm\infty$ (f)}

  Une fonction peut ne pas converger vers une limite finie et ne pas
  diverger vers l'infini. C'est le cas des fonctions périodidiques,
  comme la fonction $\sin (x)$ définie sur $\mathbb R$~:

  \foreach \n in {1,...,18}{
    \only<\n>{
      \begin{center}
        \includegraphics[scale=1.1,page=\n]{./cos-sin.pdf}
      \end{center}
    }
  }

  Cette fonction est bornée, mais elle ne converge pas vers une limite
  finie, elle oscille entre -1 et 1.

\end{frame}


\begin{frame}
  \frametitle{Limites d'une fonction}
  \framesubtitle{Limites en un point (a)}

  \begin{itemize}

  \item Une fonction peut ne pas être définie en certains points de $\mathbb R$, on parle alors de points de singularité.\newline

  \item Par exemple, l'hyperbole $f(x)=\frac{1}{x}$ est définie sur $\mathbb R$ sauf en $x=0$. Pour caractériser une fonction, il est utile de savoir comment se comporte $f$ quand $x$ se rapproche de ces points de singularité.\newline

  \end{itemize}

  \begin{definition}[limite finie en un point]
    \[
      \lim_{x\rightarrow a}f(x) = \ell<\infty \,\Leftrightarrow\, \forall\varepsilon>0,\, \exists \eta>0\,\text{ tel que }\, |x-a|<\eta\,\Rightarrow\, |f(x)-\ell|<\varepsilon
    \]
  \end{definition}

\end{frame}


\begin{frame}
  \frametitle{Limites d'une fonction}
  \framesubtitle{Limites en un point (b)}

  \begin{example}
    Montrons que $\lim_{x\rightarrow 3} f(x) = 12$, avec $f(x)=2x+6$. La distance de $f$ à sa limite en $x=3$ est~:
    \[
      \begin{split}
        |f(x)-12| &= |2x+6-12|\\
        &=  |2x-6|\\
        &= 2|x-3|
      \end{split}
    \]
    Ainsi, pour tout $\varepsilon>0$ on a~:
    $|f(x)-12|<\varepsilon \,\Leftrightarrow \, |x-3|<\frac{\varepsilon}{2}$.
    Pour chaque $\varepsilon>0$ (arbitrairement petit) il existe donc bien un nombre $\eta(\varepsilon)=\frac{\varepsilon}{2}$ tel que si $|x-3|<\eta$ alors $|f(x)-12|<\varepsilon$.\newline

    Si $x$ est suffisament proche de 3, alors on peut rendre la distance entre $f(x)$ et 12 arbitrairement petite.\newline

    Notons que dans ce cas la limite $\lim_{x\rightarrow a} f(x)=f(a)$ $\Rightarrow$ Continuité.

  \end{example}

\end{frame}


\begin{frame}
  \frametitle{Limites d'une fonction}
  \framesubtitle{Limites en un point (c)}

  \begin{itemize}

  \item La limite en un point n'est pas nécessairement finie.\newline

  \item C'est le cas, par exemple, sur les points de singularité d'une fonction rationnelle.\newline

  \end{itemize}

  \begin{definition}[limites non finies en un point]
    \[
      \lim_{x\rightarrow a}f(x) = \infty \,\Leftrightarrow\, \forall \eta>0,\, \exists \delta>0\,\text{ tel que }\, |x-a|<\delta\,\Rightarrow\, f(x)>\eta
    \]
    \[
      \lim_{x\rightarrow a}f(x) = -\infty \,\Leftrightarrow\, \forall \eta>0,\, \exists \delta>0\,\text{ tel que }\, |x-a|<\delta\,\Rightarrow\, f(x)<-\eta
    \]
  \end{definition}

  \bigskip

  \begin{itemize}

  \item La limite n'est pas finie si on peut rendre $f(x)$ arbitrairement grand en rapprochant $x$ du point $a$.
  \end{itemize}

\end{frame}


\begin{frame}
  \frametitle{Limites d'une fonction}
  \framesubtitle{Limites en un point (d)}

  \begin{example}

    Soit la fonction rationnelle $f(x) = \frac{1}{(x-1)^2}$, il s'agit d'une fonction de $\mathbb R\setminus \{1\}$ dans $\mathbb R_+$. Montrons que $\lim_{x\rightarrow 1}f(x) = \infty$.

    \bigskip

    \begin{columns}[onlytextwidth]
      \begin{column}{.5\textwidth}
        Soit $\eta\in\mathbb R_+$,
        \[
          \begin{split}
            f(x) &>\eta \Leftrightarrow (x-1)^2<\frac{1}{\eta}\\
            &\Leftrightarrow |x-1|<\frac{1}{\sqrt{\eta}}
          \end{split}
        \]
        Ainsi en posant $\delta (\eta) = \frac{1}{\sqrt{\eta}}$, on a bien
        \[
          \begin{split}
            &\forall \eta>0,\, \exists \delta(\eta)>0,\, \text{ tel que }\\
            &|x-1|<\delta(\eta) \Rightarrow f(x)>\eta
          \end{split}
        \]
      \end{column}
      \begin{column}{.5\textwidth}
        \begin{tikzpicture}[scale=.9]
          \begin{axis}[
            xticklabels={,,},
            yticklabels={,,},
            enlargelimits=true,
            grid style={dashed, gray!60},
            axis x line = bottom,
            axis y line = left,
            axis line style={thin},
            xmax = 4,
            xmin = -2,
            ymax = 120,
            ymin = -.5,
            axis lines = middle,
            small,
            clip=false,
            ]
            \addplot[
            draw=black,
            thick,
            smooth,
            samples=50,
            domain=-1:0.9,
            ]
            {1/(x-1)^2};
            \addplot[
            draw=black,
            thick,
            smooth,
            samples=50,
            domain=1.1:3,
            ]
            {1/(x-1)^2};
            \addplot[red, dashed] coordinates { (1,0) (1,110)};
          \end{axis}
        \end{tikzpicture}
      \end{column}
    \end{columns}
  \end{example}

\end{frame}


\begin{frame}
  \frametitle{Limites d'une fonction}
  \framesubtitle{Limites à gauche et à droite (a)}

  \begin{itemize}

  \item Il peut arriver que la limite en un point $a$ dépende de la façon dont on se rapproche de $a$.\newline
  \end{itemize}

  \begin{definition}[Limites finies à gauche et à droite]

    \begin{enumerate}

    \item On dit que $f(x)$ admet une limite à gauche $\ell$ finie ssi $\forall \varepsilon>0$ $\exists\eta>0$ tel que $|x-a|<\eta\,\land\, x<a$ $\Rightarrow$ $|f(x)-\ell|<\varepsilon$, et on note $\lim_{x\rightarrow a^-}f(x) = \ell$.\newline

    \item On dit que $f(x)$ admet une limite à droite $\ell$ finie ssi $\forall \varepsilon>0$ $\exists\eta>0$ tel que $|x-a|<\eta\,\land\, x>a$ $\Rightarrow$ $|f(x)-\ell|<\varepsilon$, et on note $\lim_{x\rightarrow a^+}f(x) = \ell$.

    \end{enumerate}

  \end{definition}

  \bigskip

  \begin{itemize}

  \item Si les limites à gauche et à droite sont différentes ou non définies, la limite de la fonction n'est pas définie.

  \end{itemize}

\end{frame}


\begin{frame}
  \frametitle{Limites d'une fonction}
  \framesubtitle{Limites à gauche et à droite (b)}

  \begin{example}

    Soit la fonction  $f(x) = \frac{x}{|x|}$. Cette fonction n'est pas définie en 0 (puisque nous avons alors une forme indéterminée). Néanmoins, en notant que nous pouvons réécrire la fonctions sous la forme~:
    \begin{columns}[onlytextwidth]
      \begin{column}{.5\textwidth}
        \[
          f(x) =
          \begin{cases}
            1 &\text{ si } x>0\\
            -1 &\text{ si } x<0
          \end{cases}
        \]
        nous avons~:
        \[
          \lim_{x\rightarrow 0^-} f(x) = -1
        \]
        et
        \[
          \lim_{x\rightarrow 0^+} f(x) = 1
        \]
      \end{column}
      \begin{column}{.5\textwidth}
        \begin{tikzpicture}[scale=.9]
          \begin{axis}[
            xticklabels={,,},
            yticklabels={,,},
            enlargelimits=true,
            grid style={dashed, gray!60},
            axis x line = bottom,
            axis y line = left,
            axis line style={thin},
            xmax = 4,
            xmin = -4,
            ymax = 2,
            ymin = -2,
            axis lines = middle,
            small,
            clip=false,
            ]
            \addplot[
            draw=black,
            thick,
            ]
            coordinates {(-3.5,-1) (-0.05, -1)};
            \addplot[
            draw=black,
            thick,
            ]
            coordinates {(0,1) (3.5, 1)};
          \end{axis}
        \end{tikzpicture}
      \end{column}
    \end{columns}
  \end{example}

\end{frame}


\begin{frame}
  \frametitle{Limites d'une fonction}
  \framesubtitle{Limites à gauche et à droite (c)}

  \bigskip

  Les limites à gauche et à droites ne sont pas nécessairement finies.

  \bigskip

  \begin{definition}[Limites non finies à gauche et à droite]
    {\small
      \[
        \lim_{x\rightarrow a+}f(x) = \infty \,\Leftrightarrow\, \forall \eta>0,\, \exists \delta>0\,\text{ tel que }\, |x-a|<\delta\,\land\,x>a\,\Rightarrow\, f(x)>\eta
      \]
      \[
        \lim_{x\rightarrow a+}f(x) = -\infty \,\Leftrightarrow\, \forall \eta>0,\, \exists \delta>0\,\text{ tel que }\, |x-a|<\delta\,\land\,x>a\,\Rightarrow\, f(x)<-\eta
      \]
      \[
        \lim_{x\rightarrow a-}f(x) = \infty \,\Leftrightarrow\, \forall \eta>0,\, \exists \delta>0\,\text{ tel que }\, |x-a|<\delta\,\land\,x<a\,\Rightarrow\, f(x)>\eta
      \]
      \[
        \lim_{x\rightarrow a-}f(x) = -\infty \,\Leftrightarrow\, \forall \eta>0,\, \exists \delta>0\,\text{ tel que }\, |x-a|<\delta\,\land\,x<a\,\Rightarrow\, f(x)<-\eta
      \]
    }
  \end{definition}

\end{frame}


\begin{frame}
  \frametitle{Limites d'une fonction}
  \framesubtitle{Limites à gauche et à droite (d)}

  \bigskip

  Les limites à gauche et à droite en zéro de l'hyperbole sont différentes~:

  \bigskip

  \begin{center}
    \begin{tikzpicture}[scale=1.4]
      \begin{axis}[
        xticklabels={,,},
        yticklabels={,,},
        enlargelimits=true,
        grid style={dashed, gray!60},
        axis x line = bottom,
        axis y line = left,
        axis line style={thin},
        xmax = 5,
        xmin = -5,
        ymax = 22,
        ymin = -22,
        axis lines = middle,
        small,
        clip=false,
        ]
        \addplot[
        draw=black,
        thick,
        smooth,
        samples=500,
        domain=-4.5:-0.05,
        ]
        {1/x};
        \addplot[
        draw=black,
        thick,
        smooth,
        samples=500,
        domain=.05:4.5,
        ]
        {1/x};
        \node[right] (-3,20) {$f(x) = \frac{1}{x}$};
      \end{axis}
    \end{tikzpicture}
  \end{center}

\end{frame}


\begin{frame}
  \frametitle{Limites d'une fonction}
  \framesubtitle{Propriétés}

  \begin{theorem}
    Soient deux fonctions $f(x)$ et $g(x)$ deux fonctions telles que $\lim_{x\rightarrow a}f(x) = \ell_1$ et $\lim_{x\rightarrow a}g(x) = \ell_2$, où les constantes $\ell_1$ et $\ell_2$ sont des constantes finies, alors~:\newline
    \begin{enumerate}

    \item $\lim_{x\rightarrow a} c = c$  pour toute constante $c$\newline

    \item $\lim_{x\rightarrow a} c f(x) = c\ell_1$\newline

    \item $\lim_{x\rightarrow a} f(x)\pm g(x) = \ell_1\pm\ell_2$\newline

    \item $\lim_{x\rightarrow a} f(x)\cdot g(x) = \ell_1\ell_2$\newline

    \item $\lim_{x\rightarrow a} \frac{f(x)}{g(x)} = \frac{\ell_1}{\ell_2}$, si $\ell_2\neq 0$.

    \end{enumerate}
  \end{theorem}

\end{frame}


\begin{frame}
  \frametitle{Limites d'une fonction}
  \framesubtitle{Limites usuelles}

  \bigskip

  \newcounter{enumcounter}

  \begin{columns}
    \begin{column}{.5\textwidth}
      \begin{enumerate}
      \item $\lim_{x\rightarrow 0}\log x = -\infty$
      \item $\lim_{x\rightarrow 0}\frac{e^x-1}{x} = 1$
      \item $\lim_{x\rightarrow 0}\frac{\log(1+x)}{x} = 1$
      \item $\lim_{x\rightarrow 0} \left(1+x\right)^{\frac{1}{x}} = e$
      \item $\lim_{x\rightarrow\infty}\left(1+\frac{1}{x}\right)^x = e$
      \item $\lim_{x\rightarrow\infty}\left(1+\frac{a}{x}\right)^x = e^a$
      \end{enumerate}
      \setcounter{enumcounter}{\value{enumi}}
    \end{column}
    \begin{column}{.5\textwidth}
      \begin{enumerate}
        \setcounter{enumi}{\value{enumcounter}}
      \item $\lim_{x\rightarrow \infty}\log x = \infty$
      \item $\lim_{x\rightarrow \infty}e^x = \infty$
      \item $\lim_{x\rightarrow -\infty}e^x = 0$
      \item $\lim_{x\rightarrow\infty}\frac{x^k}{a^x} = 0$, $\forall\, a>0$, $k\in \mathbb R$
      \item $\lim_{x\rightarrow\infty}\frac{a^x}{x^x} = 0$, $\forall\, a>0$
      \item $\lim_{x\rightarrow\infty} = \frac{\log x^k}{x^m} = 0$, $\forall\, m>0$
      \end{enumerate}
    \end{column}
  \end{columns}

\end{frame}


\begin{frame}
  \frametitle{Limites d'une fonction}

  \begin{example}

    \[
      \begin{split}
        \lim_{x\rightarrow 2}\frac{x^2+3}{x} =& \frac{\lim_{x\rightarrow 2} x^2 + 3}{\lim_{x\rightarrow 2} x}\\
        &= \frac{\lim_{x\rightarrow 2} x^2 + \lim_{x\rightarrow 2} 3}{\lim_{x\rightarrow 2} x}\\
        &= \frac{\lim_{x\rightarrow 2} x \cdot \lim_{x\rightarrow 2} x + \lim_{x\rightarrow 2} 3}{\lim_{x\rightarrow 2} x}\\
        &= \frac{2\cdot 2 + 3}{2} = \frac{7}{2}
      \end{split}
    \]

  \end{example}

\end{frame}


\begin{frame}
  \frametitle{Limites d'une fonction}

  \begin{example}
    Considérons la limite quand $x$ tend vers 4 de la fonction $f(x) = \frac{x^2-16}{x-4}$. Cette fonction n'est pas définie en $x=4$ car on a alors une forme indéterminée de la forme $\nicefrac{0}{0}$. Néanmoins, tant que $x\neq 4$, et c'est le cas quand on considère une limite ($x$ se rapproche de 4 mais n'est pas égal à 4), on peut diviser le numérateur et le dénominateur par $x-4$:
    \[
      \begin{split}
        \lim_{x\rightarrow 4}\frac{x^2-16}{x-4} &= \lim_{x\rightarrow 4} \frac{(x+4)(x-4)}{x-4}\\
        &= \lim_{x\rightarrow 4} x+4\\
        &= \left(\lim_{x\rightarrow 4} x\right)+4\\
        &= 8
      \end{split}
    \]

  \end{example}

\end{frame}


\begin{frame}
  \frametitle{Limites d'une fonction}

  \begin{example}
    On s'intéresse à la limite de la fraction rationnelle  $f(x) = \frac{2x^3+5x^2+6}{x^3-3x+9}$ lorsque $x$ tend vers l'infini. À nouveau nous sommmes en présence d'une forme indéterminée (de la forme $\nicefrac{\infty}{\infty}$). En supposant, sans perte de généralité puisque nous nous intéressons au comportement de $f(x)$ quand $x$ est très grand, que $x\neq$ on peut diviser le numérateur et le dénominateur par la plus grande puissance de $x$:
    \[
      \begin{split}
        \lim_{x\rightarrow \infty}\frac{2x^3+5x^2+6}{x^3-3x+9} &= \lim_{x\rightarrow \infty}\frac{2+\frac{5}{x}+\frac{6}{x^3}}{1-\frac{3}{x^2}+\frac{1}{x^3}}\\
        &= \frac{ \lim_{x\rightarrow \infty}2+ 5\lim_{x\rightarrow \infty}\frac{1}{x}+6\lim_{x\rightarrow \infty}\frac{1}{x^3}}{\lim_{x\rightarrow \infty} 1-3\lim_{x\rightarrow \infty}\frac{1}{x^2}+\lim_{x\rightarrow \infty}\frac{1}{x^3}}\\
        &= \frac{2+5\times 0 + 6\times 0}{1-3\times 0 + 9\times 0}\\
        &= 2
      \end{split}
    \]

  \end{example}

\end{frame}

\section{Continuité des fonctions}

\begin{frame}
  \frametitle{Continuité en un point, I}

  \begin{itemize}

  \item Une fonction continue sur un intervalle est une fonction que l'on peut tracer sur une page sans lever le crayon, elle ne s'interrompt nulle part.\newline

  \item Afin d'obtenir une définition plus opérationnelle, on commence par définir la continuité en un point.\newline

  \item Pour qu'une fonction soit continue en un point il faut qu'elle soit définie en ce point, et que l'on puisse s'en approcher.\newline

  \end{itemize}

  \begin{definition}
    Une fonction $f(x)$ est dite continue en $x=a$ si les trois conditions suivantes sont satisfaites~:
    \begin{enumerate}
    \item $f(a)$ est défini,
    \item $\lim_{x\rightarrow a}f(x)$ existe, et
    \item $\lim_{x\rightarrow a}f(x) = f(a)$.
    \end{enumerate}
  \end{definition}

\end{frame}


\begin{frame}
  \frametitle{Continuité en un point, II}

  En utilisant la définition de la limite, on a alternativement~:\newline

  \begin{definition}
    Une fonction $f(x)$ est continue en $x=a$ ssi:
    \[
      \forall \varepsilon>0,\, \exists \delta>0\quad |\quad |x-a|<\delta\, \Rightarrow\, |f(x)-f(a)|<\varepsilon
    \]
  \end{definition}

  \bigskip

  \begin{itemize}

  \item Une fonction $f$ est continue en $x=a$ si on peut rendre arbirairement petite la distance entre $f(x)$ et $f(a)$ dès lors que $x$ est assez proche de $a$.\newline

  \item Si une fonction n'est pas continue en un point on dit qu'elle  est discontinue.

  \end{itemize}

\end{frame}


\begin{frame}
  \frametitle{Continuité en un point, III}

  \begin{itemize}

  \item Puisque la continuité est définie à partir du concept de limite\ldots\newline

  \item On peut définir la continuité à droite et la continuité à gauche.\newline

  \item Ce rafinement est nécessaire si la fonction est définie sur un intervalle fermé.\newline

  \end{itemize}


  \begin{definition}
    \begin{enumerate}
    \item Une fonction $f(x)$ est continue à droite en $x=a$ ssi:
      \[
        \forall \varepsilon>0,\, \exists \delta>0\quad |\quad |x-a|<\delta\,\land\,x>a\, \Rightarrow\, |f(x)-f(a)|<\varepsilon
      \]
    \item Une fonction $f(x)$ est continue à gauche en $x=a$ ssi:
      \[
        \forall \varepsilon>0,\, \exists \delta>0\quad |\quad |x-a|<\delta\,\land\,x<a\, \Rightarrow\, |f(x)-f(a)|<\varepsilon
      \]
    \end{enumerate}
  \end{definition}

\end{frame}


\begin{frame}
  \frametitle{Continuité en un point, IV}

  \begin{itemize}

  \item Si une fonction n'est pas continue en un point à droite \textbf{et} à gauche, alors elle est discontinue en ce point.\newline

  \item Si une fonction est dite continue sur $E$ si et seulement si elle est continue en tout point $x\in E$.\newline

  \item Au sens strict, une fonction ne peut donc être continue sur un interval fermé $[a, b]$ puisqu'elle n'est pas continue à gauche de $a$ et à droite de $b$.\newline

  \item On dira qu'une fonction est continue sur un intervalle $[a,b]$ si elle est continue sur l'intervalle ouvert $]a,b[$ et si elle continue à droite en $a$ et continue à gauche en $b$.

  \end{itemize}

\end{frame}


\begin{frame}
  \frametitle{Continuité, I}

  \begin{example}

    Soit $f(x) = |x|$ une fonction défine sur $\mathbb R$ à valeurs dans $\mathbb R^+$. Montrons que cette fonction est continue en $x=0$.\newline

    \begin{columns}[onlytextwidth]
      \begin{column}{.5\textwidth}
        {\small
          \begin{itemize}
          \item $\lim_{x\rightarrow 0^-} |x| = 0$, en effet on a~:
            \[
              \begin{split}
                \lim_{x\rightarrow 0^-} |x| &= \lim_{x\rightarrow 0^-} -x\\
                &= -\lim_{x\rightarrow 0^-} x = 0
              \end{split}
            \]
          \item $\lim_{x\rightarrow 0^+} |x| = 0$, en effet on a~:
            \[
              \begin{split}
                \lim_{x\rightarrow 0^+} |x| &= \lim_{x\rightarrow 0^+} x\\
                &= 0
              \end{split}
            \]
          \item La limite existe (les limites à droite et à gauche existent et sont identiques) et est égale à $f(0)$ $\Rightarrow$ Continuité en 0
          \end{itemize}}
      \end{column}
      \begin{column}{.5\textwidth}
        \begin{tikzpicture}[scale=1]
          \begin{axis}[
            xticklabels={,,},
            yticklabels={,,},
            enlargelimits=true,
            grid style={dashed, gray!60},
            axis x line = bottom,
            axis y line = left,
            axis line style={thin},
            xmax = 5,
            xmin = -5,
            ymax = 5.5,
            ymin = -0.5,
            axis lines = middle,
            small,
            clip=false,
            ]
            \addplot[
            draw=black,
            thick,
            domain=-5:0,
            ]
            {-x};
            \addplot[
            draw=black,
            thick,
            domain=0:5,
            ]
            {x};
          \end{axis}
        \end{tikzpicture}
      \end{column}
    \end{columns}
  \end{example}

\end{frame}


\begin{frame}
  \frametitle{Continuité, II}

  \begin{example}
    Montrons que la droite $f(x) = ax+b$ est continue (pour des paramètres $a$ et $b$ réels) sur $\mathbb R$. Nous savons déjà que pour tout $\alpha$ réel fini $f(\alpha) = a\alpha+b$ est défini, il nous reste à montrer que pour tout $\alpha\in\mathbb R$ $\lim_{x\rightarrow\alpha}f(x)=f(\alpha)$. Nous avons~:
    \[
      \begin{split}
        \lim_{x\rightarrow\alpha}f(x) &= \lim_{x\rightarrow\alpha} ax+b\\
        &= a\lim_{x\rightarrow\alpha} x + \lim_{x\rightarrow\alpha} b\\
        &= a\alpha + b = f(\alpha)
      \end{split}
    \]
    La fonction est donc continue en tout point $\alpha\in\mathbb R$, elle est donc continue sur $\mathbb R$.
  \end{example}

\end{frame}


\begin{frame}
  \frametitle{Continuité, III}

  \begin{example}

    Soit $f(x) = \sqrt{x}$ une fonction de  $\mathbb R^+$ à valeurs dans $\mathbb R^+$. Cette fonction n'est pas continue en 0. En effet en 0 on a $f(0) = 0$ ($f$ est donc finie en ce point) et $\lim_{x\rightarrow 0^+}f(x) = 0$, mais la limite à gauche n'est pas définie puisque la fonction racine carrée n'est pas définie pour les valeurs négatives de $x$.\newline

    Soit $\alpha>0$, montrons que $\lim_{x\rightarrow a}f(x) = \sqrt{a}$ (comme la racine carrée de $\alpha$ est finie pour tout $\alpha<\infty$ cela suffit à montrer la continuité en $\alpha$). Nous avons~:
    \[
      |f(x)-f(\alpha)| = |\sqrt{x}-\sqrt{\alpha}| \leq \sqrt{|x-\alpha|}
    \]
    Ainsi, si $|x-\alpha|<\varepsilon^2$ on a nécessairement $|f(x)-f(\alpha)|<\varepsilon$. En posant $\delta(\varepsilon) = \varepsilon^2$ nous avons donc pour tout $\alpha\in\mathbb R_+^{\star}$~:
    \[
      \forall \varepsilon>0, \exists \delta(\varepsilon)>0\,\text{tel que}\, |x-a|<\delta(\varepsilon)\Rightarrow |f(x)-f(\alpha)|<\varepsilon
    \]
  \end{example}

\end{frame}


\begin{notes}
  Revenons sur l'inégalité utilisée pour montrer la continuité de la fonction racine carrée, et montrons que l'on a~:
  \[
    |\sqrt{a}-\sqrt{b}| \leq \sqrt{|a-b|}
  \]
  De façon équivalente~:
  \[
    \begin{split}
      (\sqrt{a}-\sqrt{b})^2&\leq |a-b|\\
      \Leftrightarrow a+b-2\sqrt{a}\sqrt{b} &\leq |a-b|
    \end{split}
  \]
  Distinguons de cas~:
  \medskip
  \begin{itemize}
  \item [$a \geq b$] On a donc~:
    \[
      \begin{split}
        a+b-2\sqrt{a}\sqrt{b} &\leq a-b\\
        \Leftrightarrow 2b &\leq 2\sqrt{a}\sqrt{b}\\
        \Leftrightarrow b &\leq \sqrt{a}\sqrt{b}\\
      \end{split}
    \]
    ce qui est vrai dès lors que $a\geq b$. En effet $a\geq b \Leftrightarrow \sqrt{a}\geq\sqrt{b} \Leftrightarrow \sqrt{a}\sqrt{b} \geq b$.

  \item [$a \leq b$] On a donc~:
    \[
      \begin{split}
        a+b-2\sqrt{a}\sqrt{b} &\leq b-a\\
        \Leftrightarrow 2a &\leq 2\sqrt{a}\sqrt{b}\\
        \Leftrightarrow a &\leq \sqrt{a}\sqrt{b}\\
      \end{split}
    \]
    ce qui est vrai dès lors que $a\leq b$.
  \end{itemize}

  \medskip

  Ainsi quel que soit $a$ et $b$ positifs on a~:
  \[
    a+b-2\sqrt{a}\sqrt{b} \leq |a-b|
  \]
  ou de façon équivalente~:
  \[
    |\sqrt{a}-\sqrt{b}| \leq \sqrt{|a-b|}
  \]

\end{notes}


\begin{frame}
  \frametitle{Fonctions lipschitzienne sur un intervalle, I}

  \hypertarget{slide_fonction_lipschitzienne}{}

  \begin{definition}
    La fonction $f$ de $E$ dans $\mathbb R$ est dite k-lipschitzienne sur $E$ si $\forall (x,y)\in\mathbb E^2$ on a $|f(x)-f(y)|\leq k|x-y|$ (avec $k\in\mathbb R_+$).
  \end{definition}

  \bigskip

  \begin{itemize}
  \item La fonction constante $f(x)=c\,\forall c\in\mathbb R_+$ est 0-lipschitzienne.\newline
  \item Une fonction $f$ est lipschitzienne si $\exists k>0$ tel que $f$ est k-lipschitzienne.\newline
  \end{itemize}

  \begin{theorem}\label{thm:lipschitz}
    Si $f$ est une fonction lipschitzienne sur $E$ alors elle est continue sur le même intervalle.
  \end{theorem}

\end{frame}


\begin{notes}

  \textbf{Preuve du théorème \hyperlink{slide_fonction_lipschitzienne}{\ref{thm:lipschitz}}.} Si $f$ est une fonction lipschitzienne sur $E$, alors il existe $k\in\mathbb R_+$ tel que~:
  \[
    \forall (x,y)\in\mathbb E^2 on ait |f(x)-f(y)|\leq k|x-y|
  \]
  En particulier, on doit avoir pour $a\in E$~:
  \[
    \forall x\in E,\, |f(x)-f(a)|\leq k|x-a|
  \]
  Or $\forall k\in\mathbb R_+$ on a~:
  \[
    \lim_{x\rightarrow a} |x-a| = 0
  \]
  Autreent dit on peut rendre arbitrairement petite la distance entre $f(x)$ et $f(a)$ pourvu que $x$ soit assez proche de $a$. La fonction $f$ est donc continue en $a$. Comme ceci est valable pour tout $a$ dans $E$, la fonction est continue sur $E$.

\end{notes}


\begin{frame}
  \frametitle{Fonctions lipschitzienne sur un intervalle, II}

  \begin{example}
    La droite $f(x) = ax+b$ est une fonction lipschitzienne sur
    $\mathbb R$ (et donc continue, mais nous le savons déjà). Pour
    tout $(x,y)\in\mathbb R$ nous avons~:
    \[
      \begin{split}
        |f(x)-f(y)| &= |ax+b-ay-b|\\
        &= |a|\cdot|x-y|
      \end{split}
    \]
    La doite de pente $a$ est donc $|a|$-lipschitzienne
  \end{example}

\end{frame}


\begin{frame}
  \frametitle{Fonctions discontinues}
  \framesubtitle{Typologie}

  \begin{itemize}

  \item Une fonction n'est pas continue sur $E$, si il existe un point dans $E$ où la fonction est discontinue.\newline

  \item Une fonction peut être discontinue en un point $a$ pour trois raisons~:\newline

    \begin{enumerate}
    \item La fonction $f$ n'est pas définie en $a$,
      \medskip
    \item la limite $\lim_{x \rightarrow a}f(x)$ n'existe pas, ou
      \medskip
    \item la limite $\lim_{x \rightarrow a}f(x)$ est différente de $f(a)$.\newline
    \end{enumerate}

  \item On distingue donc plusieurs types de discontinuités.
  \end{itemize}

\end{frame}


\begin{frame}
  \frametitle{Fonctions discontinues}
  \framesubtitle{Discontinuité infinie (a)}

  \begin{itemize}

  \item Si la fonction admet un point de singularité $a$ sur son domaine où~:
    \[
      \lim_{x\rightarrow a}f(x) = \pm \infty
    \]
    \bigskip

  \item \ldots Ou si la fonction en un point $a$ les limites à gauche et/ou à droite ne sont pas finies~:
    \[
      \lim_{x\rightarrow a^-}f(x) = \pm \infty \quad\quad \lim_{x\rightarrow a^+}f(x) = \pm \infty
    \]
    \bigskip

  \item Possible avec des fonctions rationnelles lorsque le polynôme au dénominateur admet des racines réelles.\newline

  \item \ldots Ou autres.

  \end{itemize}

\end{frame}


\begin{frame}
  \frametitle{Fonctions discontinues}
  \framesubtitle{Discontinuité infinie (b)}

  \begin{example}

    La fonction $f(x) = \frac{1}{x^2-2x+1}$ admet un point de singularité, elle n'est définie que sur $\mathbb R\setminus \{1\}$

    \bigskip

    \begin{columns}[onlytextwidth]
      \begin{column}{.5\textwidth}
        De façon équivalente~:
        \[
          f(x) = \frac{1}{(x-1)^2}
        \]
        Nous avons déjà montré que~:
        \[
          \lim_{x\rightarrow 1}f(x) = \infty
        \]
        La fonction $f$ est donc discontinue en $x=1$. Elle n'est pas
        continue sur $\mathbb R$, mais elle est continue sur
        $]-\infty, 1[$ et $]1,\infty[$.
      \end{column}
      \begin{column}{.5\textwidth}
        \begin{tikzpicture}[scale=.9]
          \begin{axis}[
            xticklabels={,,},
            yticklabels={,,},
            enlargelimits=true,
            grid style={dashed, gray!60},
            axis x line = bottom,
            axis y line = left,
            axis line style={thin},
            xmax = 4,
            xmin = -2,
            ymax = 120,
            ymin = -.5,
            axis lines = middle,
            small,
            clip=false,
            ]
            \addplot[
            draw=black,
            thick,
            smooth,
            samples=50,
            domain=-1:0.9,
            ]
            {1/(x-1)^2};
            \addplot[
            draw=black,
            thick,
            smooth,
            samples=50,
            domain=1.1:3,
            ]
            {1/(x-1)^2};
            \addplot[red, dashed] coordinates { (1,0) (1,110)};
          \end{axis}
        \end{tikzpicture}
      \end{column}
    \end{columns}

  \end{example}

\end{frame}


\begin{frame}
  \frametitle{Fonctions discontinues}
  \framesubtitle{Discontinuité finie (a)}

  \begin{itemize}

  \item Si la fonction « saute » en un point.\newline

  \item Une fonction admet un saut en un point $x = a$, si la limite en ce point n'existe pas car les limites à gauche et à droite sont différentes.\newline

  \item Si $\lim_{x\rightarrow a^-}f(x) = \underline{\ell}$ et $\lim_{x\rightarrow a^+}f(x) = \overline{\ell}$, avec $\underline{\ell}\neq\overline{\ell}$, la fonction « saute » de $\underline{\ell}$ à $\overline{\ell}$ en $x=a$.\newline

  \end{itemize}

\end{frame}


\begin{frame}
  \frametitle{Fonctions discontinues}
  \framesubtitle{Discontinuité finie (b)}

  \begin{example}

    La fonction $f(x) = \frac{2}{1+3^{\frac{1}{x}}}$ est définie sur $\mathbb R\setminus {0}$ et saute en 0.

    \bigskip

    \begin{columns}[onlytextwidth]
      \begin{column}{.5\textwidth}
        \[
          \lim_{x\rightarrow 0^-}f(x) = 2
        \]
        et
        \[
          \lim_{x\rightarrow 0^+}f(x) = 0
        \]
      \end{column}
      \begin{column}{.5\textwidth}
        \begin{tikzpicture}[scale=.9]
          \begin{axis}[
            xticklabels={,,},
            yticklabels={,,},
            enlargelimits=true,
            grid style={dashed, gray!60},
            axis x line = bottom,
            axis y line = left,
            axis line style={thin},
            xmin = -4,
            xmax = 4,
            ymax = 3,
            ymin = -1,
            axis lines = middle,
            small,
            clip=false,
            ]
            \addplot[
            draw=black,
            thick,
            smooth,
            samples=500,
            domain=-3.5:-0.01,
            ]
            {2/(1+3^(1/x))};
            \addplot[
            draw=black,
            thick,
            smooth,
            samples=500,
            domain=0.01:3.5,
            ]
            {2/(1+3^(1/x))};
            \addplot[red, dashed] coordinates { (-3.5,1) (3.5,1)};
          \end{axis}
        \end{tikzpicture}
      \end{column}
    \end{columns}

  \end{example}

\end{frame}


\begin{frame}
  \frametitle{Fonctions discontinues}
  \framesubtitle{Discontinuité réparable (a)}

  \begin{itemize}

  \item Soit $f$ une fonction définie sur $\mathbb R\setminus \{a\}$ et à valeurs dans $\mathbb R$.\newline

  \item Supposons que $f$ ait une limite définie en $a$, c'est-à-dire que $\lim_{x\rightarrow a}f(x) = \ell$ avec $|\ell|<\infty$.\newline

  \item On dit alors que $f$ admet une discontinuité réparable en $a$, ou que $f$ est prolongeable par continuité en $a$.\newline

  \item On pose~:
    \[
      \overline{f}(x) =
      \begin{cases}
        f(x) \quad&\forall x\in\mathbb R\setminus \{a\},\\
        \ell  &\text{si $x=a$.}
      \end{cases}
    \]
    la fonction $\overline{f}$ est continue sur $\mathbb R$.\newline

  \item Une discontinuité peut être réparable à gauche (droite) seulement si la limite à gauche (droite) est définie mais pas la limite à droite (gauche).

  \end{itemize}

\end{frame}


\begin{frame}
  \frametitle{Fonctions discontinues}
  \framesubtitle{Discontinuité réparable (b)}

  \begin{example}

    Soit la fonction $f(x) = e^{-\frac{1}{x^2}}$ de $x\in \mathbb R\setminus \{0\}$ dans $\mathbb R_+$.

    \begin{columns}[onlytextwidth]
      \begin{column}{.5\textwidth}
        \begin{itemize}

        \item Cette fonction n'est pas définie en $x=0$ à cause de la fraction sous l'exponentielle.\newline

        \item Pourtant $\lim_{x\rightarrow 0}f(x) = 0$.\newline

        \item On prolonge la fonction $f(x)$~:
          \[
            \overline{f}(x) =
            \begin{cases}
              e^{-\frac{1}{x^2}} \quad &\forall x\in \mathbb R\setminus \{0\}\\
              0                     &\text{sinon.}
            \end{cases}
          \]
        \end{itemize}

      \end{column}
      \begin{column}{.5\textwidth}
        \begin{tikzpicture}[scale=.9]
          \begin{axis}[
            xticklabels={,,},
            yticklabels={,,},
            enlargelimits=true,
            grid style={dashed, gray!60},
            axis x line = bottom,
            axis y line = left,
            axis line style={thin},
            xmin = -2,
            xmax = 2,
            ymax = 1.5,
            ymin = -.5,
            axis lines = middle,
            small,
            clip=false,
            ]
            \addplot[
            draw=black,
            thick,
            smooth,
            samples=500,
            domain=-2:-0.01,
            ]
            {exp(-1/x^2)};
            \addplot[
            draw=black,
            thick,
            smooth,
            samples=500,
            domain=0.01:2,
            ]
            {exp(-1/x^2)};
          \end{axis}
        \end{tikzpicture}
      \end{column}
    \end{columns}

  \end{example}

\end{frame}


\begin{frame}
  \frametitle{Fonctions discontinues}
  \framesubtitle{Discontinuité apparente}

  \begin{itemize}

  \item Soit $f$ une fonction définie sur $E$ à valeurs dans $\mathbb R$.\newline

  \item Si il existe $a\in E$ tel que la limite en $a$ soit définie mais différente de $f(a)$, alors on parle de discontinuuité apparente.\newline

  \item Si il existe $a\in E$ tel que $\lim_{x\rightarrow a}f(x)=\ell\neq f(a)$, tout se passe comme si il y avait une erreur dans la définition de la fonction.\newline

  \item Ce type de discontinuité est réparable par prolongement.

  \end{itemize}

\end{frame}


\begin{frame}
  \frametitle{Fonctions continues}
  \framesubtitle{Propriétés (a)}

  \begin{theorem}
    Soient $f$ et $g$ deux fonctions continues sur $E$ à valeurs dans $\mathbb R$, alors~:\newline
    \begin{enumerate}
    \item $f+g$ et $f-g$ sont des fonctions continues,\newline
    \item $f\cdot g$ est une fonction continue,\newline
    \item $f/g$ est une fonction continue en tout point de $E$ où $g$ est non nulle.\newline
    \end{enumerate}
  \end{theorem}

  \bigskip

  \begin{theorem}
    Soient $f$ une fonction continue de $E$ dans $F$ et $g$ une fonction continue de $F$ dans $G$, alors la composition $g\circ f$ est continue tant qu'elle est définie.
  \end{theorem}

\end{frame}


\begin{frame}
  \frametitle{Fonctions continues}
  \framesubtitle{Propriétés (b)}

  \begin{theorem}
    Soit $f$ une fonction continue et strictement monotone sur un intervalle $E$, alors $f$ admet une réciproque $f^{-1}$, celle-ci est continue sur l'image de $E$ par $f$, strictement monotone et de même sens de monotonie que $f$.
  \end{theorem}

\end{frame}


\begin{frame}
  \frametitle{Fonctions continues}
  \framesubtitle{Propriétés (c)}

  \begin{theorem}
    Soit $f$ une fonction continue sur un intervalle fermé $[a,b]$, alors $f$ a une valeur maximale $M$ et une valeur minimale $m$ sur l'intervalle $[a,b]$, et $f$ prend sur $[a,b]$ toute valeur comprise entre $m$ et $M$
  \end{theorem}

  \bigskip

  \begin{itemize}

  \item Ce résultat assure l'existence des extrema d'une fonction.\newline

  \item Il faut que la fonction soit continue sur un intervalle \textbf{fermé}.\newline

  \item Pour tout $k\in[m,M]$, il existe $c\in[a,b]$ tel que $f(c)=k$.

  \end{itemize}

\end{frame}


\begin{frame}
  \frametitle{Fonctions continues}
  \framesubtitle{Propriétés (d)}

  \begin{theorem}
    Soit $f$ une fonction continue sur un intervalle fermé $[a,b]$, alors si $f(a)$ et $f(b)$ sont de signes opposés il existe au moins un $x_0\in[a,b]$ tel que $f(x_0)=0$.
  \end{theorem}

  \bigskip

  \begin{itemize}

  \item Ce résultat nous donne une condition suffisante d'existence de solutions pour les équations de la forme $f(x)=0$.\newline

  \item Si $P(x)$ est une fonction polynomiale d'ordre impair, avec un coefficient positif sur le monôme de degré le plus élevé, on sait que $\lim_{x\rightarrow -\infty}P(x) = -\infty $ et $\lim_{x\rightarrow \infty}P(x) = \infty $. On peut donc toujours trouver un intervalle $[a,b]$ tel que $P(a)<0$ et $P(b)>0$. Et on sait que dans cet intervalle on trouvera au moins une racine réelle du polynôme.\newline

  \item Méthode de bissection pour résoudre l'équation $f(x)=0$.

  \end{itemize}

\end{frame}


\begin{frame}
  \frametitle{Résoudre f(x)=0 par bissection}

  \begin{columns}[onlytextwidth]
    \begin{column}{.5\textwidth}
      \begin{itemize}

      \item On cherche $a$ et $b$ tels que $f(a)f(b)<0$ (signes différents).\newline

      \item Tant que $|f(a)-f(b)|>\varepsilon$~:\newline

        \begin{enumerate}
        \item $m = \frac{a+b}{2}$ (moyenne)
          \medskip
        \item Si $f(a)f(m)<0$ alors on pose $b = m$ et on retourne en 1.
          \medskip
        \item Si $f(a)f(m)<0$ alors on pose $a = m$ et on retourne en 1.
          \medskip
        \end{enumerate}

      \end{itemize}

    \end{column}
    \begin{column}{.5\textwidth}
      \begin{center}
        \includegraphics[scale = .2]{../img/bisection-wikipedia.pdf}
      \end{center}
    \end{column}
  \end{columns}

\end{frame}





\begin{frame}
  \frametitle{Fonctions continues}
  \framesubtitle{Retour sur les limites}

  \begin{theorem}
    Soit $f$ une fonction continue de $\mathbb R$ dans $\mathbb R$. Alors si la suite $(u_n)_{n\in\mathbb N}$ converge vers $u^{\star}$ alors la suite $(v_n)_{n\in\mathbb N}$ de terme général $v_n = f(u_n)$ converge vers $f(u^{\star})$ quand $n$ tend vers l'infini.
  \end{theorem}

  \bigskip

  \begin{itemize}

  \item Idem pour les limites des fonctions.

  \end{itemize}

\end{frame}


\begin{frame}
  \frametitle{Fonctions continues}
  \framesubtitle{Retour sur les équations récurrentes}

  \begin{theorem}[Point fixe]
    Soit une suite $(u_n)_{n\in\mathbb N}$ définie de façon récursive par $u_{n+1} = f(u_n)$, avec la condition initale $u_0$ donnée. Supposons que la suite converge vers $u^{\star}$. Si la fonction $f$ est continue en $u^{\star}$ alors la limite de la suite est aussi la solution de l'équation $f(x) = x$.
  \end{theorem}

  \bigskip

  \begin{itemize}

  \item Donne une méthode générale pour calculer de façon itérative la solution de l'équation $f(x)=x$.\newline

  \item Ou de façon plus générale une équation de la forme $g(x) = 0$.\newline

  \item Plus efficace que la méthode de bissection (car on n'exploite pas seulement le signe de la fonction mais aussi sa valeur)... Mais on peut faire mieux en exploitant les dérivées de la fonction.

  \end{itemize}

\end{frame}



\end{document}

% Local Variables:
% ispell-check-comments: exclusive
% ispell-local-dictionary: "francais"
% TeX-master: t
% End: