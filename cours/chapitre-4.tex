\synctex=1

\documentclass[10pt,notheorems]{beamer}

\usepackage{etex}
\usepackage{fourier-orns}
\usepackage{ccicons}
\usepackage{amssymb}
\usepackage{amstext}
\usepackage{amsbsy}
\usepackage{amsopn}
\usepackage{amscd}
\usepackage{amsxtra}
\usepackage{amsthm}
\usepackage{multirow}
\usepackage{float}
\usepackage{color, colortbl}
\usepackage{mathrsfs}
\usepackage{bm}
\usepackage{lastpage}
\usepackage[nice]{nicefrac}
\usepackage{setspace}
\usepackage{ragged2e}
\usepackage{listings}
\usepackage{polynom}
\usepackage{algorithms/algorithm}
\usepackage{algorithms/algorithmic}
\usepackage[frenchb]{babel}
\usepackage{tikz,pgfplots}
\pgfplotsset{compat=newest}
\usetikzlibrary{patterns, arrows, decorations.pathreplacing, decorations.markings, calc}
\pgfplotsset{plot coordinates/math parser=false}
\newlength\figureheight
\newlength\figurewidth
\usepackage{cancel}
\usepackage{tikz-qtree}
\usepackage{dcolumn}
\usepackage{adjustbox}
\usepackage{environ}
\usepackage[cal=boondox]{mathalfa}
\usepackage{manfnt}
\usepackage{hyperref}
\hypersetup{
  colorlinks=true,
  linkcolor=blue,
  filecolor=black,
  urlcolor=black,
}
\usepackage{venndiagram}
\usepackage{minted}

% Git hash
\usepackage{xstring}
\usepackage{catchfile}
\immediate\write18{git rev-parse HEAD > git.hash}
\CatchFileDef{\HEAD}{git.hash}{\endlinechar=-1}
\newcommand{\gitrevision}{\StrLeft{\HEAD}{7}}

% Include pdf page by page
\usepackage{pdfpages}


\newcommand{\trace}{\mathrm{tr}}
\newcommand{\vect}{\mathrm{vec}}
\newcommand{\tracarg}[1]{\mathrm{tr}\left\{#1\right\}}
\newcommand{\vectarg}[1]{\mathrm{vec}\left(#1\right)}
\newcommand{\vecth}[1]{\mathrm{vech}\left(#1\right)}
\newcommand{\iid}[2]{\mathrm{iid}\left(#1,#2\right)}
\newcommand{\normal}[2]{\mathcal N\left(#1,#2\right)}
\newcommand{\dynare}{\href{http://www.dynare.org}{\color{blue}Dynare}}
\newcommand{\sample}{\mathcal Y_T}
\newcommand{\samplet}[1]{\mathcal Y_{#1}}
\newcommand{\slidetitle}[1]{\fancyhead[L]{\textsc{#1}}}

\newcommand{\R}{{\mathbb R}}
\newcommand{\C}{{\mathbb C}}
\newcommand{\N}{{\mathbb N}}
\newcommand{\Z}{{\mathbb Z}}
\newcommand{\binomial}[2]{\begin{pmatrix} #1 \\ #2 \end{pmatrix}}
\newcommand{\bigO}[1]{\mathcal O \left(#1\right)}
\newcommand{\red}{\color{red}}
\newcommand{\blue}{\color{blue}}

\renewcommand{\qedsymbol}{C.Q.F.D.}

\newcolumntype{d}{D{.}{.}{-1}}
\definecolor{gray}{gray}{0.9}
\newcolumntype{g}{>{\columncolor{gray}}c}

\setbeamertemplate{theorems}[numbered]

\theoremstyle{plain}
\newtheorem{theorem}{Théorème}

\theoremstyle{definition} % insert bellow all blocks you want in normal text
\newtheorem{definition}{Définition}
\newtheorem{properties}{Propriétés}
\newtheorem{lemma}{Lemme}
\newtheorem{property}[properties]{Propriété}
\newtheorem{example}{Exemple}
\newtheorem*{idea}{Éléments de preuve} % no numbered block
\newtheorem{corollary}{Corollaire}%[theorem]


\setbeamertemplate{footline}{
  {\hfill\vspace*{1pt}\href{http://creativecommons.org/licenses/by-sa/3.0/legalcode}{\ccbysa}\hspace{.1cm}
    \raisebox{-.075cm}{\href{https://git.adjemian.eu/stepan/economic-calculus}{\includegraphics[scale=.1]{../img/gitlab.png}}}\enspace
    \href{https://git.adjemian.eu/stepan/economic-calculus/-/blob/\HEAD/cours/chapitre-4.tex}{\gitrevision}\enspace\today
  }\hspace{1cm}}

\setbeamertemplate{navigation symbols}{}
\setbeamertemplate{blocks}[rounded][shadow=true]
\setbeamertemplate{caption}[numbered]

\NewEnviron{notes}{\justifying\tiny\begin{spacing}{1.0}\BODY\vfill\pagebreak\end{spacing}}

\newenvironment{exercise}[1]
{\bgroup \small\begin{block}{Ex. #1}}
  {\end{block}\egroup}

\newenvironment{defn}[1]
{\bgroup \small\begin{block}{Définition. #1}}
  {\end{block}\egroup}

\newenvironment{exemple}[1]
{\bgroup \small\begin{block}{Exemple. #1}}
  {\end{block}\egroup}

\begin{document}

\title{Calcul Économique\\\small{IV. Dérivées}}
\author[S. Adjemian]{Stéphane Adjemian}
\institute{\texttt{stephane.adjemian@univ-lemans.fr}} \date{Octobre 2020}

\begin{frame}
  \titlepage{}
\end{frame}

\begin{frame}
  \frametitle{Plan}
  \tableofcontents
\end{frame}


\section{Dérivée d'une fonction en un point}

\begin{frame}
  \frametitle{Variation d'une fonction, I}
  \hypertarget{slide_variation_1}{}

  \begin{itemize}

  \item Soit $f$ une fonction de $E$ dans $F$.\newline

  \item On veut prédire l'effet d'une variation de $x\in E$ sur son image par $f$.\newline

  \item Soit $x_0\in E$, son image par $f$ est $y_0 = f(x_0)\in F$.\newline

  \item Pertubons $x_0$ en lui rajoutant $\Delta x$. On suppose que la perturbation est « assez petite » pour que~: $x_1=x_0+\Delta x \in E$.\newline

  \item On peut donc réécrire $\Delta x = x_1-x_0$, et définir $y_1 = f(x_0+\Delta x)\in F$ la conséquence sur l'image par $f$.\newline

  \item On pose $\Delta f(x) = f(x_1)-f(x_0)$ la variation de l'image par $f$ induite par la variation $\Delta x$ de $x$.

  \end{itemize}

\end{frame}


\begin{frame}
  \frametitle{Variation d'une fonction, II}
  \hypertarget{slide_variation_2}{}


  \begin{example}
    Soit $f(x) = x^3$ une fonction de $\mathbb R$ dans $\mathbb R$. Posons comme valeur initiale de x $x_0 = 2$, son image par $f$ est $y_0 = f(x_0) = 8$. Donnons-nous une perturbation $\Delta x = 1$. On a alors $x_1=x_0+\Delta x = 3$ et son image $y_1 = 3^3 = 27$. La variation induite de l'image est donc $\Delta f(x) = y_1-y_0$, soit $\Delta f(x) = 19$. Si nous changeons la perturbation sur $x$ en posant $\Delta x = -1$, nous obtenons $x_1 = 1$, $y_1 = 1$ et donc $\Delta f(x) = -7$.\newline

    Pour ces deux exemples $\Delta f(x)$ et $\Delta x$ ont le même signe~: quand on augmente (diminue) $x$ cela induit une augmentation (baisse) de $f(x)$. Cela suggère que $f$ est croissante autour de $x=1$. Ce résultat n'est évidemment pas général.\newline

    Si maintenant $x_0 = 1$ et $\Delta x = 1$, on a $x_1 = 2$, $y_0 = 1$, $y_1 = 8$ et donc $\Delta f(x) = 7$.    On note que l'amplitude de l'effet sur l'image n'est pas constant (dépend du niveau initial $x_0$)\newline

  \end{example}

\end{frame}


\begin{frame}
  \frametitle{Variation d'une fonction, III}
  \hypertarget{slide_variation_3}{}

  \begin{example}

    Soit la fonction de $\mathbb R$ dans $\mathbb R$~: $f(x) = ax+b$, une droite, où $a$ et $b$ sont des paramètres réels. Si on perturbe $x_0$ avec $\Delta x$, on a~:
    \[
      y_1 = a (\underbrace{x_0+\Delta x}_{x_1}) + b
    \]
    sachant que $y_0 = a x_0 + b$, on a donc~:
    \[
      \begin{split}
        \Delta f(x) &= y_1-y_0\\
        &= a(x_1-x_0)\\
        &= a \Delta x
      \end{split}
    \]
    Si on normalise la variation induite $\Delta f(x)$ par la variation $\Delta$, on obtient~:
    \[
      \frac{\Delta f(x)}{\Delta x} = a \quad \forall x_0\in\mathbb R
    \]
    Pour une droite, le \textbf{taux de variation} ne dépend pas de $x_0$.

  \end{example}

\end{frame}


\begin{frame}
  \frametitle{Variation d'une fonction, IV}
  \hypertarget{slide_variation_4}{}

  \begin{columns}[onlytextwidth]
    \begin{column}{.5\textwidth}
      \begin{itemize}
      \item Plus généralement~:
        \[
          f(x) + \Delta f(x) = f(x+\Delta x)
        \]
      \item Soit de façon équivalente~:
        \[
          \Delta f(x) = f(x+\Delta x)-f(x)
        \]
      \item En divisant par $\Delta x\neq 0$, on  obtient le taux de variation~:
        \[
          \frac{\Delta f(x)}{\Delta x} = \frac{f(x+\Delta x)-f(x)}{\Delta x}
        \]
        qui correspond à la pente de l'arc passant par les points $A_0$ et $A_1$ sur le graphique.
      \end{itemize}
    \end{column}
    \begin{column}{.5\textwidth}
      \begin{tikzpicture}[scale=1.05]
        \begin{axis}[
          title={},
          xlabel= {},
          ylabel= {},
          xticklabels={,,},
          yticklabels={,,},
          enlargelimits=true,
          grid style={dashed, gray!60},
          axis x line = bottom,
          axis y line = left,
          axis lines = middle,
          axis line style={thin},
          xmin = -.1,
          xmax = 3,
          ymin = -.1,
          ymax = 5,
          small,
          clip=false,
          ]
          \addplot[
          draw=black,
          thick,
          smooth,
          samples=500,
          domain=0:2.1,
          ]
          {x^2} node[right] {\tiny $f(x)$} ;
          \addplot[
          draw=red,
          domain=.2:2.5
          ]
          {3*x-2};
          \node[draw=red, circle, fill=red, scale=.2] at (1,1) {};
          \node[draw=red, circle, fill=red, scale=.2] at (2,4) {};
          \node[below] at (1,0) {\tiny{\color{red}$x_0$}};
          \node[below] at (2,0) {\tiny{\color{red}$x_1 = x_0+\Delta x$}};
          \node[left] at (0,1) {\tiny{\color{red}$f(x_0)$}};
          \node[left] at (0,4) {\tiny{\color{red}$f(x_0+\Delta x)$}};
          \node[above] at (1,1.05) {\tiny\color{red} $A_0$};
          \node[right] at (2.05,4) {\tiny\color{red} $A_1$};
          \addplot[draw=red, dotted] coordinates {(2,4) (0,4)};
          \addplot[draw=red, dotted] coordinates {(2,4) (2,0)};
          \addplot[draw=red, dotted] coordinates {(1,1) (0,1)};
          \addplot[draw=red, dotted] coordinates {(1,1) (1,0)};
        \end{axis}
      \end{tikzpicture}
    \end{column}
  \end{columns}

\end{frame}


\begin{frame}
  \frametitle{Variation d'une fonction, V}
  \hypertarget{slide_variation_5}{}

  \begin{columns}[onlytextwidth]
    \begin{column}{.5\textwidth}
      \begin{itemize}

      \item Quel est le taux de variation pour $\Delta x = 0$~? On a une forme indéterminée de type 0/0.\newline

      \item Si la limite existe, la dérivée de $f$ en $x_0$ est la limite du taux de variation quand $\Delta x$ tend vers 0~:
        \[
          \frac{\mathrm d}{\mathrm dx}f(x_0) = \lim_{\Delta x\rightarrow 0}\frac{f(x+\Delta x)-f(x)}{\Delta x}
        \]
        \medskip
      \item On a remplacé $\Delta x$ par $\mathrm dx$ $\rightarrow$ variations infinitésimales.\newline

      \item Si la dérivée en $x_0$ existe on la notera aussi $f'(x_0)$.\newline
      \end{itemize}
    \end{column}
    \begin{column}{.5\textwidth}
      \begin{center}
        \begin{tikzpicture}[scale=1]
          \begin{axis}[
            title={},
            xlabel= {},
            ylabel= {},
            xticklabels={,,},
            yticklabels={,,},
            enlargelimits=true,
            grid style={dashed, gray!60},
            axis x line = bottom,
            axis y line = left,
            axis lines = middle,
            axis line style={thin},
            xmin = -.1,
            xmax = 3,
            ymin = -.1,
            ymax = 5,
            small,
            clip=false,
            ]
            \addplot[
            draw=black,
            thick,
            smooth,
            samples=500,
            domain=0:2.1,
            ]
            {x^2} node[right] {\tiny $f(x)$} ;
            \addplot[
            draw=red,
            domain=.2:2.5
            ]
            {2*x-1};
            \node[draw=red, circle, fill=red, scale=.2] at (1,1) {};
            \node[below] at (1,0) {\tiny{\color{red}$x_0$}};
            \node[left] at (0,1) {\tiny{\color{red}$f(x_0)$}};
            \node[above] at (1,1.05) {\tiny\color{red} $A_0$};
            \addplot[draw=red, dotted] coordinates {(1,1) (0,1)};
            \addplot[draw=red, dotted] coordinates {(1,1) (1,0)};
          \end{axis}
        \end{tikzpicture}
      \end{center}
    \end{column}
  \end{columns}

\end{frame}


\begin{frame}
  \frametitle{Dérivée en un point, I}
  \hypertarget{slide_derivee_1}{}

  \begin{definition}
    Soit $f$ une fonction de $E$ dans $\mathbb R$. On note $f'(x_0)$ la dérivée de $f$ au point $x_0\in E$, celle-ci est définie par~:
    \[
      f'(x_0) = \lim_{h\rightarrow 0} \frac{f(x_0+h)-f(x_0)}{h}
    \]
  \end{definition}

  \bigskip

  \begin{itemize}

  \item La dérivée est définie par une limite.\newline

  \item Pour que la dérivée existe, il faut que la limite soit définie.\newline

  \item La dérivée de la fonction $f$ en un point $x_0$ est la pente de la tangente à la courbe représentative de $f$ en $x_0$.
  \end{itemize}

\end{frame}


\begin{frame}
  \frametitle{Dérivée en un point, II}
  \hypertarget{slide_derivee_2}{}

  \begin{center}
    \begin{tikzpicture}[scale=1.6]
      \begin{axis}[
        title={},
        xlabel= {},
        ylabel= {},
        xticklabels={,,},
        yticklabels={,,},
        enlargelimits=true,
        grid style={dashed, gray!60},
        axis x line = bottom,
        axis y line = left,
        axis lines = middle,
        axis line style={thin},
        xmin = -.1,
        xmax = 1.7,
        ymin = -1.5,
        ymax = 3,
        small,
        clip=false,
        ]
        \addplot[
        draw=black,
        thick,
        smooth,
        samples=500,
        domain=0:1.5,
        ]
        {x^2} node[right] {\tiny $f(x)$} ;
        \addplot[
        draw=red,
        domain=-.3:1.6
        ]
        {2*x-1};
        \node[draw=red, circle, fill=red, scale=.2] at (1,1) {};
        \node[below] at (1,0) {\tiny{\color{red}$x_0$}};
        \node[left] at (0,1) {\tiny{\color{red}$f(x_0)$}};
        \addplot[draw=red, dotted] coordinates {(1,1) (0,1)};
        \addplot[draw=red, dotted] coordinates {(1,1) (1,0)};
        \node[draw=red, circle, fill=red, scale=.2] at (0,-1) {};
        \node[left] at (-0.05,-1) {\tiny{\color{red}$f(x_0)-f'(x_0)x_0$}};
        \node[draw=red, circle, fill=red, scale=.2] at (0.5,0) {};
        \node[right] at (.4,-.275) {\tiny{\color{red}$x_0-\frac{f(x_0)}{f'(x_0)}$}};
        \node[right] at (.1,-1.3) {\tiny{\color{red}Équation de la tangente: $y = f(x_0)+f'(x_0)(x-x_0)$}};
      \end{axis}
    \end{tikzpicture}
  \end{center}

\end{frame}


\begin{notes}
  Pour déterminer l'équation de la tangente on exploite deux informations~:\newline
  \begin{itemize}

  \item La pente de la tangente est $f'(x_0)$,\newline

  \item La tangente passe par le point $(x_0, f(x_0))$.\newline

  \end{itemize}

  L'équation de la tangente est donc de la forme~:
  \[
    y = f'(x_0)x + b
  \]
  Il ne reste plus qu'à choisir $b$ pour s'assurer que la tangente passe bien par le point $(x_0, f(x_0))$. On a~:
  \[
    f(x_0) = f'(x_0)x_0 + b
  \]
  soit de façon équivalente~:
  \[
    b = f(x_0) - f'(x_0)x_0
  \]
  et donc~:
  \[
    y = f'(x_0) x + f(x_0) - f'(x_0)x_0
  \]
  ou encore en factorisant~:
  \[
    y = f(x_0) + f'(x_0)(x-x_0)\hfil\qed
  \]
\end{notes}


\begin{frame}
  \frametitle{Dérivée en un point, III}
  \hypertarget{slide_derivee_3}{}

  \begin{itemize}

  \item La dérivée est définie par une limite.\newline

  \item On peut donc définir une dérivée à droite et une dérivée à gauche\newline

  \item Si les dérivées à droite et à gauche en $x_0$ sont différentes, on dit que la fonction n'est pas dérivable en $x_0$ (comme pour l'existence de la limite).\newline

  \end{itemize}

  \begin{definition}
    Soit $f$ une fonction de $E$ dans $\mathbb R$. On note $f'_-(x_0)$ et $f'_+(x_0)$  les dérivées à gauche et à droite de $f$ au point $x_0\in E$, celle-ci sont définies par~:
    \[
      f'_-(x_0) = \lim_{h\rightarrow 0^-} \frac{f(x_0+h)-f(x_0)}{h}
    \]
    et
    \[
      f'_+(x_0) = \lim_{h\rightarrow 0^+} \frac{f(x_0+h)-f(x_0)}{h}
    \]
  \end{definition}

\end{frame}


\begin{frame}
  \frametitle{Dérivée en un point, IV}
  \hypertarget{slide_derivee_4}{}

  \begin{example}

    Soit $f(x) = x^2$ une fonction de $\mathbb R$ dans $\mathbb R_+$. Calculons la dérivée en $x=1$. Nous avons~:
    \[
      \begin{split}
        f'(1) &= \lim_{h\rightarrow 0}\frac{f(1+h)-f(1)}{h}\\
        &= \lim_{h\rightarrow 0}\frac{(1+h)^2-1}{h}\\
        &= \lim_{h\rightarrow 0}\frac{h^2+2h+1-1}{h}\\
        &= \lim_{h\rightarrow 0}\frac{h^2+2h}{h}\\
        &= \lim_{h\rightarrow 0}h+2\\
        &= 2
      \end{split}
    \]
  \end{example}

\end{frame}


\begin{frame}
  \frametitle{Dérivée en un point, V}
  \hypertarget{slide_derivee_5}{}

  \begin{itemize}

  \item Pour qu'une fonction soit dérivable en un point, il faut que la fonction soit définie en ce point.\newline

  \item Si une fonction n'est pas définie en un point $x_0$, alors la fonction n'admet pas de dérivée en ce point.\newline

  \item[\dbend] Ce n'est pas parce qu'une fonction est définie en un point que la dérivée en ce point existe.\newline

  \end{itemize}

  \begin{theorem}\label{thm:derivable-continue}
    Soit $f$ une fonction de $E$ dans $\mathbb R$. Si la fonction $f$ est dérivable en $x_0\in E$, alors la fonction $f$ est continue en $x_0$.
  \end{theorem}

  \bigskip

  \begin{itemize}

  \item[\dbend] La réciproque n'est pas vraie (voir l'exemple suivant)~: une fonction continue n'est pas nécessairement dérivable.

  \end{itemize}

\end{frame}


\begin{notes}
  \textbf{Preuve du théorème
    \hyperlink{slide_derivee_5}{\ref{thm:derivable-continue}}.}
  Puisque la fonction est supposée dérivable en $x_0\in E$, nous avons
  par définition de la dérivée~:
  \[
    f'(x_0) = \lim_{h\rightarrow 0} \frac{f(x_0+h)-f(x_0)}{h}
  \]
  Posons~:
  \[
    g(h) =  \frac{f(x_0+h)-f(x_0)}{h}
  \]
  Nous avons donc~:
  \[
    h g(h) = f(x_0+h)-f(x_0)
  \]
  avec $\lim_{h\rightarrow 0}g(h) = f'(x_0)$. Ainsi~:
  \[
    \lim_{h\rightarrow 0} hg(h) = \lim_{h\rightarrow 0}f(x_0+h) - f(x_0)
  \]
  \[
    \Leftrightarrow f'(x_0)\lim_{h\rightarrow 0} h = \lim_{h\rightarrow 0}f(x_0+h) - f(x_0)
  \]
  \[
    \Leftrightarrow 0 = \lim_{h\rightarrow 0}f(x_0+h) - f(x_0)
  \]
  \[
    \Leftrightarrow 0 = \lim_{x\rightarrow x_0}f(x) - f(x_0)
  \]
  d'où finalement~:
  \[
    \lim_{x\rightarrow x_0}f(x) = f(x_0)
  \]
  La fonction est donc bien continue en $x_0$.

\end{notes}


\begin{frame}
  \frametitle{Dérivée en un point, VI}
  \hypertarget{slide_derivee_6}{}

  \begin{example}
    Soit $f(x) = |x|$ une fonction défine sur $\mathbb R$ à valeurs dans $\mathbb R^+$. Cette fonction est continue en $x=0$ mais n'est pas dérivable.\newline

    \begin{columns}[onlytextwidth]
      \begin{column}{.5\textwidth}
        {\small
          \begin{itemize}
          \item $f'_-(0)=\lim_{h\rightarrow 0^-} \frac{|0+h|-0}{h} = -1$, en effet on a~:
            \[
              \begin{split}
                f'_-(0) &= \lim_{h\rightarrow 0^-} \frac{|h|}{h}\\
                &= \lim_{h\rightarrow 0^-} \frac{-h}{h}\\
                &= -\lim_{h\rightarrow 0^-} 1\\
                &= -1\\
              \end{split}
            \]
          \item Mais $f'_+(0) = 1$

          \item Les dérivées à droite et à gauche sont différentes, la fonction n'est donc pas dérivable en $x=0$.
          \end{itemize}}
      \end{column}
      \begin{column}{.5\textwidth}
        \begin{tikzpicture}[scale=1]
          \begin{axis}[
            xticklabels={,,},
            yticklabels={,,},
            enlargelimits=true,
            grid style={dashed, gray!60},
            axis x line = bottom,
            axis y line = left,
            axis line style={thin},
            xmax = 5,
            xmin = -5,
            ymax = 5.5,
            ymin = -0.5,
            axis lines = middle,
            small,
            clip=false,
            ]
            \addplot[
            draw=black,
            thick,
            domain=-5:0,
            ]
            {-x};
            \addplot[
            draw=black,
            thick,
            domain=0:5,
            ]
            {x};
          \end{axis}
        \end{tikzpicture}
      \end{column}
    \end{columns}
  \end{example}

\end{frame}


\begin{frame}
  \frametitle{Dérivée sur un intervalle}
  \hypertarget{slide_derivee_7}{}


  \begin{definition}\label{dfn:derivable-intervalle}
    Soit $f$ une fonction de $E$ dans $\mathbb R$. Si la fonction $f$ est dérivable sur $E$ si ell est dérivable en tout point  $x\in E$.
  \end{definition}

  \bigskip

  \begin{example}
    La fonction $f(x) = x^2$ de $\mathbb R$ dans $\mathbb R_+$ est dérivable sur $\mathbb R$. En effet, pour tout $x\in\mathbb R$ nous avons~:
    \[
      \begin{split}
        f'(x) &= \lim_{h\rightarrow 0}\frac{(x+h)^2-x^2}{h}\\
        &= \lim_{h\rightarrow 0}\frac{x^2+2xh+h^2-x^2}{h} = \lim_{h\rightarrow 0}\frac{2xh+h^2}{h}\\
        &= \lim_{h\rightarrow 0} 2x+h = 2x + \lim_{h\rightarrow 0} h\\
        &= 2x
      \end{split}
    \]

  \end{example}

\end{frame}


\section{Règles de dérivation}


\begin{frame}
  \frametitle{Règles de dérivation}
  \framesubtitle{Dérivée d'une somme}
  \hypertarget{slide_derivee_somme_1}{}

  \begin{theorem}
    Soient $f(x)$ et $g(x)$ deux fonctions de $E$ dans $F$ dérivables sur $E$, alors~:
    \[
      (f(x)+g(x))' = f'(x) + g'(x)
    \]
  \end{theorem}

  \bigskip

  {\small \textbf{Preuve.} Notons $\ell(x) = f(x)+g(x)$, nous avons par définition~:
    \[
      \begin{split}
        \ell'(x) &= \lim_{h\rightarrow 0} \frac{\ell(x+h)-\ell(x)}{h}\\
        &= \lim_{h\rightarrow 0} \frac{f(x+h)+g(x+h)-f(x)-g(x)}{h}\\
        &= \lim_{h\rightarrow 0} \frac{f(x+h)-f(x)}{h} + \lim_{h\rightarrow 0} \frac{g(x+h)-g(x)}{h}\\
        &= f'(x)+g'(x) \qed
      \end{split}
    \]
  }

\end{frame}


\begin{frame}
  \frametitle{Règles de dérivation}
  \framesubtitle{Dérivée d'un produit (a)}
  \hypertarget{slide_derivee_produit_1}{}

  \begin{theorem}
    Soient $f(x)$ et $g(x)$ deux fonctions de $E$ dans $F$ dérivables sur $E$, alors~:
    \[
      (f(x)\cdot g(x))' = f'(x)g(x) + f(x)g'(x)
    \]
  \end{theorem}

  \bigskip

  {\small \textbf{Preuve.} Notons $\ell(x) = f(x) \cdot g(x)$, nous avons par définition~:
    \[
      \begin{split}
        \ell'(x) &= \lim_{h\rightarrow 0} \frac{\ell(x+h)-\ell(x)}{h}\\
        &= \lim_{h\rightarrow 0} \frac{f(x+h)g(x+h)-f(x)g(x)}{h} \\
        &= \lim_{h\rightarrow 0} \frac{(f(x+h)-f(x))g(x+h)+f(x)g(x+h)-f(x)g(x)}{h}\\
        &= \lim_{h\rightarrow 0} \frac{(f(x+h)-f(x))g(x+h)+f(x)(g(x+h)-g(x))}{h}\\
      \end{split}
    \]
  }

\end{frame}


\begin{frame}
  \frametitle{Règles de dérivation}
  \framesubtitle{Dérivée d'un produit (b)}
  \hypertarget{slide_derivee_produit_2}{}

  {\small On a donc~:
    \[
      \begin{split}
        \ell'(x) &= \lim_{h\rightarrow 0} \frac{(f(x+h)-f(x))g(x+h)}{h} + \lim_{h\rightarrow 0} \frac{f(x)(g(x+h)-g(x))}{h}\\
        &= \lim_{h\rightarrow 0} \frac{f(x+h)-f(x)}{h}\lim_{h\rightarrow 0} g(x+h) + f(x)\lim_{h\rightarrow 0} \frac{g(x+h)-g(x)}{h}\\
        &= \lim_{h\rightarrow 0} \frac{f(x+h)-f(x)}{h}g(x) + f(x)\lim_{h\rightarrow 0} \frac{g(x+h)-g(x)}{h}\\
        &= f'(x)g(x) + f(x)g'(x) \qed
      \end{split}
    \]
  }

\end{frame}


\begin{frame}
  \frametitle{Règles de dérivation}
  \framesubtitle{Dérivée d'un quotient}
  \hypertarget{slide_derivee_quotient_1}{}

  \begin{theorem}
    Soient $f(x)$ et $g(x)$ deux fonctions de $E$ dans $F$ dérivables sur $E$, avec $g(x)\neq 0\,\forall x\in E$, alors~:
    \[
      \left(\frac{f(x)}{g(x)}\right)' = \frac{f'(x)g(x) - f(x)g'(x)}{g(x)^2}
    \]
  \end{theorem}

  \bigskip

  {\small \textbf{Preuve.} Notons $\ell(x) = \frac{f(x)}{g(x)}$, nous avons de façon équivalente~:
    \[
      \ell(x)g(x) = f(x)
    \]
    en dérivant, et en exploitant la règle de dérivation d'un produit, il vient~:
    \[
      \ell'(x)g(x)+\ell(x)g'(x) = f'(x)
    \]
    soit~:
    \[
      \ell'(x) = \frac{f'(x)-\ell(x)g'(x)}{g(x)} = \frac{f'(x)g(x)-f(x)g'(x)}{g(x)^2} \qed
    \]
  }

\end{frame}


\begin{frame}
  \frametitle{Règles de dérivation}
  \framesubtitle{Dérivée d'une composition}
  \hypertarget{slide_derivee_composition_1}{}

  \begin{theorem}\label{thm:composition}
    Soient $f(x)$ une fonction dérivable de $E$ dans $F$ et $g(x)$ une fonction dérivable de $F$ dans $G$, alors~:
    \[
      g(f(x))' = g'(f(x))f'(x)
    \]
  \end{theorem}

  \bigskip

  \begin{example}
    Soit $f(x)$ une fonction de $\mathbb R$ dans $\mathbb R$ dérivable. Alors $g(x) = f(x)^2$ est une fonction dérivable sur $\mathbb R$ et~:
    \[
      g'(x) = 2f(x)f'(x)
    \]
  \end{example}

\end{frame}


\begin{notes}
  \textbf{Preuve du théorème \hyperlink{slide_derivee_composition_1}{\ref{thm:composition}}.} Commençons par noter que, par définition de la dérivée, si les fonctions $f$ et $g$ sont dérivables, alors on peut écrire~:
  \[
    f(x+h) =_0 f(x) + hf'(x) + \varepsilon(h)
  \]
  et
  \[
    g(y+k) =_0 g(y) + kg'(y) + \nu(k)
  \]
  avec $\lim_{h\rightarrow 0}\varepsilon(h)=0$ et $\lim_{k\rightarrow 0}\nu(k)=0$. Il s'agit d'un développement limité à l'ordre 1 des fonctions $f$ et $g$. La première équation nous dit que pour de petites valeurs de $h$ on peut approximer $f(x+h)$ par $f(x) + hf'(x)$, l'équation de la tangente à $f$ au point $x$, l'erreur d'approximation $\varepsilon(h)$ tend vers 0 quand $h$ tend vers 0. On a donc~:
  \[
    g\circ f(x+h) = g\Bigl(\underbrace{f(x)}_{y}+\underbrace{hf'(x) + \varepsilon(h)}_{k(h)}\Bigr)
  \]
  avec $\lim_{h\rightarrow 0}k(h) = 0$. En exploitant le dévelopement limité de $g$, on obtient~:
  \[
    g\circ f(x+h) = g(f(x)) + \Bigl( hf'(x) + \varepsilon(h) \Bigr) g'(f(x)) + \nu\Bigl( hf'(x) + \varepsilon(h) \Bigr)
  \]
  \[
    \Leftrightarrow g\circ f(x+h) = g(f(x)) + h g'(f(x))f'(x) + \underbrace{\nu\Bigl( hf'(x) + \varepsilon(h) \Bigr)+\varepsilon(h)g'(f(x))}_{\eta(h)}
  \]
  avec $\lim_{h\rightarrow 0}\eta(h) = 0$. On a donc~:
  \[
    g\circ f(x+h) = g(f(x)) + h g'(f(x))f'(x) + \eta(h)
  \]
  et par identification~:
  \[
    g(f(x))' = g'(f(x))f'(x)
  \]

\end{notes}


\begin{frame}
  \frametitle{Règles de dérivation}
  \framesubtitle{Dérivée d'une fonction réciproque}
  \hypertarget{slide_derivee_reciproque_1}{}

  \begin{theorem}\label{thm:composition}
    Soit $f$ une fonction dérivable et bijective de $E$ dans $F$, alors $f^{-1}$ est dérivable en tout point $x\in F$ tel que $f'(f^{-1}(x))\neq 0$ et on a~:
    \[
      (f^{-1})'(x) = \frac{1}{f'(f^{-1}(x))}
    \]
  \end{theorem}

  \bigskip

  {\small \textbf{Preuve.} Nous savons déjà que la composition d'une fonction et de sa réciproque (celle-ci existe car la fonction $f$ est supposée bijective) est égale à la fonction identité~:
    \[
      f\Bigl(f^{-1}(x)\Bigr) = x
    \]
    en dérivant les deux membres et utilisant le théorème \hyperlink{slide_derivee_composition_1}{\ref{thm:composition}} sur la dérivation des fonctions composées, nous avons~:
    \[
      f'\Bigl(f^{-1}(x)\Bigr)(f^{-1})'(x)  = 1
    \]
    \[
      (f^{-1})'(x)  = \frac{1}{f'\Bigl(f^{-1}(x)\Bigr)}\qed
    \]
  }

\end{frame}


\section{Dérivées de fonctions usuelles}


\begin{frame}
  \frametitle{Fonctions exponentielle et logarithme, I}
  \hypertarget{slide_derivee_exp_log_1}{}

  \begin{definition}
    L'équation fonctionnelle $f'(x) = f(x)$ avec $f(0)=1$ admet une unique solution~:
    \[
      f(x) = e^x
    \]
    la fonction exponentielle.
  \end{definition}

  \bigskip

  \begin{theorem}\label{thm:log_derivee_1}
    La dérivée de $f(x) = log(x)$, pour $x\in\mathbb R_+^*$ est $f'(x) = \frac{1}{x}$.
  \end{theorem}

  \bigskip

  \begin{theorem}\label{thm:log_derivee_2}
    Soit $f$ une fonction à valeurs dans $\mathbb R_+^{\star}$, alors~:
    \[
      \Bigl(\log f(x)\Bigr)' = \frac{f'(x)}{f(x)}
    \]
  \end{theorem}

\end{frame}


\begin{notes}

  \textbf{Preuve du théorème \hyperlink{slide_derivee_exp_log_1}{\ref{thm:log_derivee_1}}.} On sait que la fonction logarithme népérien est la fonction réciproque de la fonction expolnentielle, on a donc~:
  \[
    e^{\log x} = x
  \]
  En dérivant~:
  \[
    \left(e^{\log x}\right)' = 1
  \]
  Pour dériver le l'exponentielle d'une fonction, on utilise le théorème \hyperlink{slide_derivee_composition_1}{\ref{thm:composition}} de dérivation des fonctions composées. De façon générale, on a donc si $f(x)$ est une fonction dérivable~:
  \[
    \left(e^{f(x)}\right)' = f'(x)e^{f(x)}
  \]
  puisque par définition la dérivée de la fonction exponentielle est la fonction exponentielle. Dans le cas qui nous intéresse, il vient~:
  \[
    \Bigl( \log x \Bigr)' e^{\log x} = 1
  \]
  et donc~:
  \[
    x \Bigl( \log x \Bigr)' = 1
  \]
  d'où~:
  \[
    \Bigl( \log x \Bigr)' = \frac{1}{x}
  \]

  \bigskip

  \textbf{Preuve du théorème \hyperlink{slide_derivee_exp_log_1}{\ref{thm:log_derivee_2}}.} direct avec le théorème \hyperlink{slide_derivee_composition_1}{\ref{thm:composition}}.

\end{notes}


\begin{frame}
  \frametitle{Fonctions exponentielle et logarithme, II}
  \hypertarget{slide_derivee_exp_log_1}{}

  \bigskip

  \begin{corollary}
    Si $f(x)$ est une fonction positive, alors~:
    \[
      f'(x) = f(x) \Bigl(\log f(x)\Bigr)'
    \]
  \end{corollary}

  \bigskip

  \begin{example}
    Soit la fonction $f(x) = x^x$ définie sur $\mathbb R_+^{\star}$. On a~:
    \[
      \begin{split}
        f'(x) &= x^x \Bigl(\log x^x\Bigr)'\\
        &= x^x \Bigl(x\log x\Bigr)'\\
        &= x^x \Bigl(\frac{x}{x}+\log x\Bigr)\\
        &= x^x \Bigl(1+\log x\Bigr)
      \end{split}
    \]
  \end{example}

\end{frame}


\begin{frame}
  \frametitle{Dérivées des fonctions usuelles}
  \hypertarget{slide_derivee_usuelles}{}

  \bigskip
  \renewcommand{\arraystretch}{1.8}
  \begin{table}[H]
    \centering
    {\small
      \begin{tabular}{l|c|l|c|l}
        \hline
        $D_f$ & $f(x)$ & $D_{f'}$ & $f'(x)$ & Remarques\\ \hline
        $\R$ & $k$ & $\R$ & 0 & $k\in\R$\\
        $\R$ & $kx$ & $\R$ & $k$ & $k\in\R$\\
        $\R$ & $x^n$ & $\R$ & $nx^{n-1}$ & $n\in\N$\\
        $\R^{\star}$ & $\frac{1}{x^n}$ & $\R^{\star}$ & $-\frac{n}{x^{n+1}}$ & $n\in\N$\\
        $\R_+^\star$ & $x^{\alpha}$  & $\R_+^{\star}$ &  $\alpha x^{\alpha - 1}$ & $\alpha\in\R$\\
        % </math> constante réelle. Fonction prolongeable par continuité en 0 si {{math|''α'' ≥ 0}}, et de prolongée dérivable en 0 si {{math|''α'' ≥ 1}}.
        $\R^\star$ & $\log |x|$ & $\R^\star$ & $\frac{1}{x}$ & \\
        $\R^\star$ & $\log_a |x|$ & $\R^\star$ & $\frac{1}{x \log a}$ & $a>0$ et $a\neq 1$\\
        $\R$ & $e^x$  & $\R$  $e^x$ & \\
        $\R$ & $a^x$ & $\R$ & $a^x \log a$ & $a > 0$ \\ \hline\hline
      \end{tabular}}
  \end{table}

\end{frame}


\begin{frame}
  \frametitle{Fonction puissance: $x^n$, avec $n\in\mathbb N$}

  \begin{itemize}

  \item Montrons que $\left(x^n\right)' = nx^{n-1}$ par récurrence.\newline

  \item Au rang 1, nous avons $x' = 1\times x^0 = 1$.\newline

  \item Supposons que $\left(x^n\right)' = nx^{n-1}$ et montrons que l'on doit alors avoir $\left(x^{n+1}\right)' = (n+1)x^{n}$~:
    \[
      \begin{split}
        \left(x^{n+1}\right)' &= \left(x^{n+1}\right)'\\
        &=\left(xx^{n}\right)'\\
        &=x^n + x n x^{n-1}\\
        &=x^n + n x^{n} = (n+1)x^{n}\qed
      \end{split}
    \]

  \end{itemize}
\end{frame}


\section{Dérivées d'ordre supérieur}

\begin{frame}
  \frametitle{Dérivées d'ordre supérieur}
  \hypertarget{slide_derivees_ordre_n_1}{}

  \begin{definition}
    Soit $f$ une fonction dérivable sur $E$, et telle que $f'$ est elle-même dérivable sur $E$, alors la dérivée de $f'$ est appelée \textbf{dérivée seconde} de la fonction $f$, et notée $f''$. On note de même $f'''$ la dérivée tierce de $f$ si elle existe, plus généralement on note $f^{(n)}$ la dérivée n-ième de la fonction $f$.
  \end{definition}

  \bigskip

  \begin{itemize}
  \item Par convention $f^{(0)} = f$.\newline
  \item Par construction $\left(f^{(k)}\right)' = f^{(k+1)}$.\newline
  \end{itemize}


  \begin{definition}
    Soit $f$ une fonction de $E\rightarrow F\subseteq \mathbb R$. Si $f$ est $k$ fois dérivable sur $E$, et sa dérivée $k$-ième est continue sur $E$, on dit que $f$ est de classe $\mathcal C^k$ sur $E$. Par convention, si $f$ est continue sur $E$ on dit que $f$ est de classe $\mathcal C^0$ sur $E$; si jamais $f$ est de classe $\mathcal C^k$ sur $E$ pour tout entier $k$ alors on dit que $f$ est de classe $\mathcal C^{\infty}$.
  \end{definition}

\end{frame}


\begin{frame}
  \frametitle{Dérivées d'ordre supérieur}
  \framesubtitle{Règles de dérivation}
  \hypertarget{slide_derivees_ordre_n_2}{}

  \begin{theorem}
    Soient $n\in\mathbb N$, $\alpha\in \mathbb R$ et $f$ et $g$ deux fonctions de classe $\mathcal C^n$ sur un intervalle $E$, alors~:\newline
    \begin{enumerate}
    \item $\alpha f$ est de classe $\mathcal C^n$ sur $E$ et $\left(\alpha f\right)^{(n)} = \alpha f^{(n)}$\newline
    \item $f+g$  est de classe $\mathcal C^n$ sur $E$ et $\left(f+g\right)^{(n)} = f^{(n)} + g^{(n)}$
    \end{enumerate}
  \end{theorem}

  \bigskip

  \begin{corollary}
    La dérivée $n$-ième d'une fonction polynomiale de degré inférieur à $n$ est nulle.
  \end{corollary}

\end{frame}


\begin{frame}
  \frametitle{Dérivées d'ordre supérieur}
  \framesubtitle{Fonctions usuelles}
  \hypertarget{slide_derivees_ordre_n_3}{}

  \bigskip

  On peut établir les formules suivantes par récurrence~:

  \bigskip

  \renewcommand{\arraystretch}{2}
  \begin{table}[H]
    \centering
    {\small
      \begin{tabular}{c|l|c}
        \hline
        $f(x)$ & $D_{f^{(n)}}$ & $f^{(n)}(x)$\\ \hline
        $e^x$  & $\mathbb R$ & $e^x$\\
        \multirow{2}{*}{$x^p$, $p\in\mathbb N$}  & \multirow{2}{*}{$\mathbb R$} & \multirow{2}{*}{$\begin{cases}\frac{p!}{(n-p)!}x^{b-n} & \text{ si }n\leq p\\ 0 & \text{ sinon.}\end{cases}$}\\
               & & \\
        $x^{\alpha}$, $\alpha\in\mathbb R\setminus\mathbb N$ & $\mathbb R_+^{\star}$ & $x^{\alpha-n}\prod_{i=0}^{n-1}(\alpha-i)$ \\
        $\frac{1}{a+x}$ & $\mathbb R \setminus \{-a\}$ & $\frac{(-1)^nn!}{(a+x)^{n+1}}$ \\
        $\frac{1}{a-x}$ & $\mathbb R \setminus \{a\}$ & $\frac{(-1)^nn!}{(a-x)^{n+1}}$ \\ \hline\hline
      \end{tabular}}
  \end{table}

\end{frame}


\begin{frame}
  \frametitle{Dérivées d'ordre supérieur}
  \framesubtitle{Règles de dérivation: formule de Leibniz}
  \hypertarget{slide_derivees_ordre_n_4}{}

  \begin{theorem}\label{thm:leibniz}
    Soit $n\in\mathbb N$. Soient $f$ et $g$ deux fonctions de classe
    $\mathcal C^n$ sur un intervalle $E$, alors la fonction $f\cdot g$
    est de classe $\mathcal C^n$ sur $E$ et~:
    \[
      (f\cdot g)^{(n)} = \sum_{k=0}^n \binom{n}{k}f^{(k)}\cdot g^{(n-k)}
    \]
  \end{theorem}

  \bigskip

  {\textbf{Rappels.}} Le coefficient binomial $\binom{n}{k}$ est défini par~:
  \[
    \binom{n}{k} = \frac{n!}{k!(n-k)!}
  \]
  Ce coefficient est aussi utilisé en combinatoire pour dénombrer le
  nombre de façons de choisir $k$ objets parmi $n$ objets distincts et
  que l'ordre dans lesquel les objets sont énumérés n'a pas
  d'importance. On peut montrer que~: $\binom{n}{1} = n$, $\binom{n}{0} = 1$, $\binom{0}{0} = 1$, $\binom{n}{n} = 1$, $\binom{n}{k} = \binom{n}{n-k}$, et
  \[
    \binom{n}{k} = \binom{n-1}{k-1} + \binom{n-1}{k}
  \]

\end{frame}


\begin{notes}

  \textbf{Preuve du théorème \hyperlink{slide_derivees_ordre_n_4}{\ref{thm:leibniz}}.} On utilise une preuve par récurrence. Pour $n=0$, on a~:
  \[
    \begin{split}
      (f\cdot g)^{(0)} &= \sum_{k=0}^0 \binom{n}{k}f^{(k)}\cdot g^{(n-k)}\\
      &= \binom{0}{0}f^{(0)}\cdot g^{(0)} = f \cdot g
    \end{split}
  \]
  et on sait que le produit de deux fonctions continues est une fonction continue. Nous pourrions nous arrêter là est vérifier que si le théorème est vrai au rang $n$ alors il est vrai au rang $n+1$, mais vérifions tout de même que le théorème \hyperlink{slide_derivees_ordre_n_4}{\ref{thm:leibniz}} nous permet bien de retrouver le résultat connu pour la dérivée d'ordre 1. Nous avons~:
  \[
    \begin{split}
      (f\cdot g)^{(1)} &= \sum_{k=0}^1 \binom{1}{k}f^{(k)}\cdot g^{(1-k)}\\
      &= \binom{1}{0}f^{(0)}g^{(1)}+\binom{1}{1}f^{(1)}g^{(0)}\\
      &= f'g+fg'
    \end{split}
  \]
  Si $f$ et $g$ sont de classe $\mathcal C^1$ alors $f'g+fg'$ est une fonction continue (somme et produits de fonctions continues). Montrons que le théorème est vrai au rang $n+1$, c'est-à-dire que si $f$ et $g$ sont de classe $\mathcal C^{n+1}$ alors $f\cdot g$ est de classe $\mathcal C^{n+1}$ et~:
  \[
    (f\cdot g)^{(n+1)} = \sum_{k=0}^{n+1} \binom{n+1}{k}f^{(k)}\cdot g^{(n+1-k)}
  \]
  Nous avons~:
  \[
    \begin{split}
      (f\cdot g)^{(n+1)} &= \left((f\cdot g)^{(n)}\right)'\\
      &= \left(\sum_{k=0}^{n} \binom{n}{k}f^{(k)}\cdot g^{(n-k)}\right)'\\
      &= \sum_{k=0}^{n} \binom{n}{k}f^{(k+1)}\cdot g^{(n-k)}+ \sum_{k=0}^{n}\binom{n}{k}f^{(k)}\cdot g^{(n+1-k)}\\
      &= \sum_{k=1}^{n+1} \binom{n}{k-1}f^{(k)}\cdot g^{(n+1-k)}+ \sum_{k=0}^{n}\binom{n}{k}f^{(k)}\cdot g^{(n+1-k)}\\
      &= \binom{n}{0}f\cdot g^{(n+1)} + \sum_{k=1}^n\left(\binom{n}{k-1}+\binom{n}{k}\right)f^{(k)}\cdot g^{(n+1-k)} + \binom{n}{n}f^{(n+1)}\cdot g\\
      &= f\cdot g^{(n+1)} + \sum_{k=1}^n\binom{n+1}{k}f^{(k)}\cdot g^{(n+1-k)} + f^{(n+1)}\cdot g\\
      &= \binom{n+1}{0}f\cdot g^{(n+1)} + \sum_{k=1}^n\binom{n+1}{k}f^{(k)}\cdot g^{(n+1-k)} + \binom{n+1}{n+1}f^{(n+1)}\cdot g\\
      &= \sum_{k=0}^{n+1}\binom{n+1}{k}f^{(k)}\cdot g^{(n+1-k)}\quad\quad \qed
    \end{split}
  \]

\end{notes}


\begin{frame}
  \frametitle{Dérivées d'ordre supérieur}
  \framesubtitle{Règles de dérivation: formule de Faa di Bruno}
  \hypertarget{slide_derivees_ordre_n_5}{}

  \begin{theorem}\label{thm:faa-di-bruno}
    Soit $n\in\mathbb N$. Soient $f$ et $g$ deux fonctions de classe
    $\mathcal C^n$ sur un intervalle $E$, alors la fonction $f\circ g$
    est de classe $\mathcal C^n$ sur $E$ et~:
    {\small\[
        (f\circ g)^{(n)}(x) = \sum_{\sum_{i=1}^n im_i=n} \frac{n!}{m_1!m_2!\cdots m_n!} f^{(m_1+m_2+\dots+m_n)}(g(x)) \prod_{j=1}^n \left(\frac{g^{(j)}(x)}{j!}\right)^{m_j}
      \]}
  \end{theorem}

  \begin{example}
    Soient $f$ et $g$ deux fonctions de classe $\mathcal C^2$ sur un $E$, alors~:
    \[
      \begin{split}
        (f\circ g)''(x) &= \Bigl(f'(g(x))g'(x)\Bigr)'\\
        &= f''(g(x))g'(x)g'(x)+f'(g(x))g''(x)\\
        &= f''(g(x))g'(x)^2+f'(g(x))g''(x)\\
      \end{split}
    \]
  \end{example}
\end{frame}


\section{Extrema et sens de variation d'une fonction}

\begin{frame}
  \frametitle{Extrema, I}
  \hypertarget{slide_extrema_1}{}
  \begin{definition}
    Soit $f: E\rightarrow \mathbb R$ une fonction. On dit que $a\in E$ est un~:
    \begin{description}
    \item[\textbf{maximum}] de $f$ sur $E$ si pour tout $x\in E$ on a $f(x)\leq f(a)$,
    \item[\textbf{minimum}] de $f$ sur $E$ si pour tout $x\in E$ on a $f(x)\geq f(a)$.
    \end{description}
    On dira que $a$ est un \textbf{extremum} de $f$ sur $E$ si $a$ est un maximum ou minimum de $f$ sur $E$.
  \end{definition}

  \bigskip

  \begin{definition}
    Soient $E$ un intervalle ouvert de $\mathbb R$, $f$ une fonction de $E$ dans $\mathbb R$ et $a\in E$. On dit que $a$ ets un~:
    \begin{description}
    \item[\textbf{maximum local}] de $f$ sur $E$ s'il existe $\delta>0$ pour tout $x\in E$ on ait
      $|x-a|\leq\delta \Rightarrow f(x)\leq f(a)$,
    \item[\textbf{minimum local}] de $f$ sur $E$ s'il existe $\delta>0$ pour tout $x\in E$ on ait
      $|x-a|\leq\delta \Rightarrow f(x)\geq f(a)$.
    \end{description}
  \end{definition}

\end{frame}


\begin{frame}
  \frametitle{Extrema, II}
  \hypertarget{slide_extrema_2}{}

  \begin{theorem}\label{thm:local_extrema}
    Soit $E$  un intervalle ouvert de $\mathbb R$, $f: E\rightarrow \mathbb R$ une fonction et $a$ un extremum local de $f$. \underline{Si $f$ est dérivable en $a$} alors on doit avoir $f'(a)=0$.
  \end{theorem}

  \bigskip

  \begin{itemize}
  \item[\dbend] Le théorème \hyperlink{slide_extrema_2}{\ref{thm:local_extrema}} ne donne qu'une condition nécessaire. Ce n'est pas parce que la dérivée d'une fonction est nulle en un point qu'elle admet un extremum en ce point. Par exemple, la fonction $f(x)=x^3$ de $\mathbb R$ dans $\mathbb R$ admet une dérivée nulle en 0 mais pas d'extremum local (en 0 ou ailleurs).\newline
  \end{itemize}

  \begin{theorem}[Théorème de Rolle]\label{thm:rolle}
    Soit $a < b$ deux réels, et $f$  une fonction continue sur $[a, b]$ et dérivablesur $]a, b[$ telle que $f(a) = f(b)$. Alors il existe $c\in]a, b[$ tel que $f'(c) = 0$.
  \end{theorem}

\end{frame}


\begin{frame}
  \frametitle{Extrema, II}
  \framesubtitle{Illustration du théorème de Rolle}
  \hypertarget{slide_extrema_3}{}

  \begin{center}
    \begin{tikzpicture}[scale=1.5]
      \begin{axis}[
        xticklabels={,,},
        yticklabels={,,},
        enlargelimits=true,
        grid style={dashed, gray!60},
        axis x line = bottom,
        axis y line = left,
        axis line style={thin},
        xmax = 9,
        xmin = -3,
        ymax = 8,
        ymin = -0.5,
        axis lines = middle,
        small,
        clip=false,
        ]
        \addplot[smooth, thick,samples=100,domain=1.4:8.5] plot(\x,{0.05*((\x)-3)^2*((\x)-9)^2+1});
        \addplot[draw=red, dotted] coordinates {(2,0) (2,3.45) (0,3.45)};
        \addplot[draw=red, dotted] coordinates {(7.414213562373062,0) (7.414213562373062,3.45) (2,3.45)};
        \node[color=red, left] at (0,3.45) {\tiny{$f(a)=f(b)$}};
        \fill [color=red] (2,3.45) circle (1.5pt);
        \fill [color=red] (7.414213562373062,3.45) circle (1.5pt);
        \node[color=red, below] at (2,0) {\tiny{$a$}};
        \node[color=red, below] at (7.414213562373062,0) {\tiny{$b$}};
        \addplot[draw=blue, dotted] coordinates {(3,0) (3,1) (0,1)};
        \node[color=blue, left] at (0,1) {\tiny{$m$}};
        \node[color=blue, below] at (3,0) {\tiny{$c_1$}};
        \addplot[draw=blue, dotted] coordinates {(6,0) (6,5.05) (0,5.05)};
        \node[color=blue, left] at (0,5.05) {\tiny{$M$}};
        \node[color=blue, below] at (6,0) {\tiny{$c_2$}};
        \addplot[draw=blue, <->, thick] coordinates {(2.3,1) (3.7,1)};
        \addplot[draw=blue, <->, thick] coordinates {(5.3,5.05) (6.7,5.05)};
      \end{axis}
    \end{tikzpicture}
  \end{center}

\end{frame}


\begin{notes}

  \textbf{Preuve du théorème \hyperlink{slide_extrema_2}{\ref{thm:local_extrema}}.} Supposons que $a$ soit tel que $f'(a)>0$. Comme $f'(a)$ est la limite du taux d'accroissement $\frac{f(x)-f(a)}{x-a}$ quand $x$ tend vers $a$, celui-ci doit être positif pour $x$ assez proche de $a$, c'est-à-dire~:
  \[
    \exists \delta>0\,|\,\forall x\in E, |x-a|\leq\delta \Rightarrow \frac{f(x)-f(a)}{x-a}>0
  \]
  Ainsi pour tout $x$ tel que $0<x-a\leq\delta$ on a $f(x)>f(a)$ et pour tout $x$ tel que $-\delta\leq x-a <0$ on a $f(x)<f(a)$, donc $a$ n'est pas un extremum local (car ni maximum ni minimum). On peut suivre le même argument si $f'(a)<0$.\newline

  \textbf{Preuve du théorème \hyperlink{slide_extrema_2}{\ref{thm:rolle}}.} Si $f$ est une fonction constante sur $[a,b]$, alors la dérivée de la fonction est nulle sur $]a,b[$ et le théorème est évidemment vérifié. Intéressons nous au cas où la fonction n'est pas constante. On sait alors, puisque $f$ est continue et par le théorème 13 du chapitre III, que la fonction admet un maximum $M$ et un minimum $m$ sur $[a,b]$ et que l'un des deux doit être différent de $f(a)$\footnote{Notons que si $M$ ou $m$ sont différents de $f(a)$ alors ils sont aussi différents de $f(b)$, puisque par hypothèse $f(a)=f(b)$.} (puisque la fonction n'est pas constante). Supposons que $M>f(a)$ et posons $c\in ]a,b[$ tel que $f(c)=M$. Alors $f(c)$ est le maximum de $f$ sur $]a, b[$~: c est un maximum local de $f$, et donc $f'(c) = 0$ par le théorème \hyperlink{slide_extrema_2}{\ref{thm:local_extrema}}.
\end{notes}


\begin{frame}
  \frametitle{Accroissements finis}
  \hypertarget{slide_accroissements_finis_1}{}

  \begin{theorem}\label{thm:accroissements_finis}
    Soient $a<b$ deux réels, et $f: [a,b]\rightarrow \mathbb R$ une fonction continue sur $[a, b]$ et dérivable sur $]a, b[$. Alors $\exists\, c\in]a, b[$ tel que $f(b)-f(a) = f'(c)(b-a)$.
  \end{theorem}

  \bigskip

  \begin{columns}[onlytextwidth]
    \begin{column}{0.5\textwidth}
      {\small
        \begin{itemize}
        \item $\frac{f(b)-f(a)}{b-a}$ est la pente de la corde entre les points $A$ et $B$.\newline
        \item $f'(c)$ est la pente de la tangente à la courbe représentative de $f$ en un point $c$.\newline
        \item Le théorème nous dit qu'il existe au moins un point $c$ où la tangente à la courbe est parallèle à la corde.
        \end{itemize}}
    \end{column}
    \begin{column}{0.5\textwidth}
      \begin{center}
        {\small
          \begin{tikzpicture}[scale=.7]
            \begin{scope}
              \clip (-3,-2) rectangle (3,2);
              \draw[thick,smooth,domain=-3:3] plot (\x,{\x^3/3 - \x});
              \node[black,right] at (-3,-1.8) {{\small$\mathcal  C_f$}};
            \end{scope}
            \node[circle,fill=red,scale=.3] (a) at (-2,-2/3) {};
            \node[circle,fill=red,scale=.3] (b) at (2,2/3) {};
            \node[red] at (-2.2,-1.2/3) {$A$};
            \node[red] at (2.2,1.2/3) {$B$};
            \node[circle,fill=red,scale=.3] (b) at (2,2/3) {};
            \draw[red] (a) -- (b);
            \coordinate (origin) at (-4,-3);
            \coordinate (topright) at (3.4,2);
            \draw[<->] (topright -| origin) -- (origin) -- (origin -| topright);
            \draw[dashed, red] (a) -- (a|-origin) node[below] {$a$};
            \draw[dashed, red] (b) -- (b|-origin) node[below] {$b$};

            \node[circle,fill=blue,scale=.3] (x0) at ({-2/sqrt(3)},{(1/3)*(-2/sqrt(3))^3+2/sqrt(3)}) {};
            \draw[blue] (x0) +(-1,-1/3) -- +(1,1/3);
            \node[circle,fill=blue,scale=.3] (x1) at ({2/sqrt(3)},{(1/3)*(2/sqrt(3))^3-2/sqrt(3)}) {};
            \draw[blue] (x1) +(-1,-1/3) -- +(1,1/3);
            \draw[dashed, blue] (x0) -- (x0 |- origin) node[below]{$c_1$};
            \draw[dashed, blue] (x1) -- (x1 |- origin) node[below]{$c_2$};
          \end{tikzpicture}}
      \end{center}
    \end{column}
  \end{columns}
\end{frame}

\begin{notes}

  \textbf{Preuve du théorème \hyperlink{slide_accroissements_finis_1}{\ref{thm:accroissements_finis}}.} Posons la fonction $g$ définie sur le même intervalle que $f$ à valeurs dans $\mathbb R$~:
  \[
    g(x) = f(x)-\frac{f(b)-f(a)}{b-a}(x-a)
  \]
  Par construction, $g$ est une fonction continue sur $[a,b]$ et dérivable sur $]a,b[$. La fonction $g$ est la différence de la fonction $f$ et d'une droite parallèle à la corde entre $(a,f(a))$ et $(b,f(b))$ passant par le point $(a,0)$ sur l'axe des abscisses. On note que la fonction $g$ vérifie $g(a) = g(b) = f(a)$, il s'agit en fait d'une rotation de la fonction d'origine $f$ qui nous permet d'utiliser le théorème de Rolle (dans l'illustration graphique ci-dessous, on passe de la courbe en trait plein à la courbe en tirets verts). En effet par le théorème \hyperlink{slide_extrema_2}{\ref{thm:rolle}} on sait qu'il existe $c\in]a,b[$ tel que~:
  \[
    g'(c) = 0
  \]
  \[
    \Leftrightarrow f'(c) = \frac{f(b)-f(a)}{b-a}\quad\quad\qed
  \]

  \begin{center}
    \begin{tikzpicture}[scale=1]
      \begin{scope}
        \clip (-3,-2) rectangle (3,2);
        \draw[thick,smooth,domain=-3:3] plot (\x,{\x^3/3 - \x});
        \node[black,right] at (-3,-1.8) {{\small$\mathcal  C_f$}};
        \draw[thick,green,dashed,domain=-3:3] plot (\x,{\x^3/3 - \x - (1/3)*(\x+2)});
        \node[green,left] at (2.9,0.7) {{\small$\mathcal  C_g$}};
      \end{scope}
      \node[circle,fill=red,scale=.5] (a) at (-2,-2/3) {};
      \node[circle,fill=red,scale=.5] (b) at (2,2/3) {};
      \node[circle,fill=green,scale=.5] (bb) at (2,-2/3) {};
      \node[red] at (-2.2,-1.2/3) {$A$};
      \node[red] at (2.2,1.2/3) {$B$};
      \node[green] at (2.2,-2/3) {$B'$};
      \node[circle,fill=red,scale=.5] (b) at (2,2/3) {};
      \draw[red] (a) -- (b);
      \draw[green] (a) -- (bb);
      \coordinate (origin) at (-4,-3);
      \coordinate (topright) at (3.4,2);
      \draw[<->] (topright -| origin) -- (origin) -- (origin -| topright);
      \draw[dashed, red] (a) -- (a|-origin) node[below] {$a$};
      \draw[dashed, red] (b) -- (b|-origin) node[below] {$b$};

      \node[circle,fill=blue,scale=.5] (x0) at ({-2/sqrt(3)},{(1/3)*(-2/sqrt(3))^3+2/sqrt(3)}) {};
      \node[circle,fill=green,scale=.5] (xx0) at ({-2/sqrt(3)},{(1/3)*(-2/sqrt(3))^3+2/sqrt(3)-(1/3)*(2-2/sqrt(3))}) {};
      \draw[blue] (x0) +(-1,-1/3) -- +(1,1/3);
      \node[circle,fill=blue,scale=.5] (x1) at ({2/sqrt(3)},{(1/3)*(2/sqrt(3))^3-2/sqrt(3)}) {};
      \node[circle,fill=green,scale=.5] (xx1) at ({2/sqrt(3)},{(1/3)*(2/sqrt(3))^3-2/sqrt(3)-(1/3)*(2+2/sqrt(3))}) {};
      \draw[blue] (x1) +(-1,-1/3) -- +(1,1/3);
      \draw[dashed, blue] (x0) -- (x0 |- origin) node[below]{$c_1$};
      \draw[dashed, blue] (x1) -- (x1 |- origin) node[below]{$c_2$};
      \draw[green] (xx0) +(-1,0) -- +(1,0);
      \draw[green] (xx1) +(-1,0) -- +(1,0);
    \end{tikzpicture}
  \end{center}

\end{notes}


\begin{frame}
  \frametitle{Sens de variation}
  \hypertarget{slide_sens_de_variation_1}{}

  \begin{theorem}\label{thm:sens_de_variation}
    Soit $E$ un intervalle de $\mathbb R$ et $f: E\rightarrow \mathbb R$ une fonction dérivable. Alors $f$ est croissante sur $E$ si et seulement si, on a $f'(x)\geq0$ pour tout $x\in E$, et $f$ est décroissante sur $E$ si et seulement si on a $f'(x)\leq 0$ pour tout $x\in E$. De plus, si $f'(x)>0$ pour tout $x\in E$  alors $f$  est strictement croissante, et si $f'(x)<0$ pour tout $x\in E$ alors $f$ est strictement décroissante.
  \end{theorem}

  \bigskip

  \begin{itemize}

  \item Ce théorème (voir la preuve) est une conséquence direct du théorème \hyperlink{slide_accroissements_finis_1}{\ref{thm:accroissements_finis}}, dit des accroissements finis.\newline

  \item Si la dérivée est nulle pour tout $x\in E$ alors la fonction est constante (simultanément croissante et décroissante).\newline

  \item C'est ce théorème qui nous permet de construire des tableaux de variation pour décrire les fonctions.

  \end{itemize}

\end{frame}


\begin{notes}

  \textbf{Preuve du théorème \hyperlink{slide_sens_de_variation_1}{\ref{thm:sens_de_variation}}.} Si $f$ est croissante sur $E$ alors pour tout $(x,y)\in E^2$ avec $y\neq x$ le signe de $f(y)-f(x)$ doit être identique au signe de $y-x$ (de sorte que $f(y)>f(x)$ si et seulement si $y>x$). Ainsi la dérivée~:
  \[
    f'(x) = \lim_{y\rightarrow x}\frac{f(y)-f(x)}{y-x}
  \]
  doit être positive ou nulle puisque le taux de variation sous la limite est positif. Pour la réciproque, si $f'(x)\geq 0$ pour tout $x\in E$, alors considérons $(x,y)\in E^2$ avec $y\neq x$. Par le théorème \hyperlink{slide_accroissements_finis_1}{\ref{thm:accroissements_finis}} on sait qu'il existe $c\in]x,y[$ tel que~:
  \[
    f(y)-f(x) = f'(c)(y-x)
  \]
  ainsi le signe de $f(y)\geq f(x)$ si $y>x$ et $f(y)\leq f(x)$ si $y<x$, puisque $f'(c)\geq 0$, $f$ est donc une fonction croissante. Si la fonction est strictement positive alors la fonction est strictement croissante, c'est-à-dire~: $f(y)>f(x)$ si $y>x$ et $f(y)<f(x)$ si $y<x$.

\end{notes}


\begin{frame}
  \frametitle{Inégalité des accroissements finis}
  \hypertarget{slide_inegalite_accroissements_finis_1}{}

  \begin{theorem}\label{thm:inegalite_accroissements_finis}
    Soient $a<b$ deux réels, et $f: [a,b]\rightarrow \mathbb R$ une fonction continue sur $[a, b]$ et dérivable sur $]a, b[$. Si $\sup_{c\in]a,b[}|f'(c)|<\infty$, alors pour tout $(x,y)\in[a,b]^2$ on a~:
    \[
      |f(x)-f(y)| \leq k |x-y|
    \]
  \end{theorem}

  \bigskip

  \begin{itemize}

  \item Si la fonction $f$ est de classe $\mathcal C^1$, alors $f'$ est une fonction continue et donc bornée. Ainsi le théorème \hyperlink{slide_inegalite_accroissements_finis_1}{\ref{thm:inegalite_accroissements_finis}} s'applique à toute fonction de classe $\mathcal C^1$.\newline

  \item Sous les conditions du théorème \hyperlink{slide_inegalite_accroissements_finis_1}{\ref{thm:inegalite_accroissements_finis}}, la fonction $f$ est $k$-lipschitzienne.\newline

  \item Une fonction de classe $\mathcal C^1$ est $k$-lipschitzienne.

  \end{itemize}

\end{frame}


\begin{notes}
  \textbf{preuve du théorème \hyperlink{slide_inegalite_accroissements_finis_1}{\ref{thm:inegalite_accroissements_finis}}.} Direct par le théorème \hyperlink{slide_accroissements_finis_1}{\ref{thm:accroissements_finis}}. Soit $(x,y)\in[a,b]^2$, sans perte de généralité on suppose que $y>x$. On peut appliquer l’égalité des accroissements finis sur l'intervalle $[x, y]$, on sait qu'il existe $c\in]x, y[$ tel que $f(y)-f(x) = f'(c)(y-x)$. Soit, en prenant la valeur absolue, $|f(y)-f(x)| = |f'(c)||y-x|$. Puisque la dérivée est finie on sait qu'il existe $k\in\mathbb R_+$ fini tel que $|f(x)-f(y)|<=k|x-y|$.\qed

\end{notes}


\begin{frame}
  \frametitle{Extrema, III}

  \begin{itemize}

  \item Le théorème \hyperlink{slide_extrema_2}{\ref{thm:local_extrema}} donne une condition pour un extremum local, mais (\textit{i}) cette condition est seulement nécessaire, et (\textit{ii}) celle-ci suppose que la fonction est dérivable en l'extremum\ldots\newline

  \item En admettant que la fonction soit dérivable en l'extremum, pour que cette condition permette efectivement d'identifier un extremum local il faut qu'on ait un changement de signe de la dérivée autour de l'extremum~:\newline

    \begin{itemize}

    \item Pour que $a$ soit un maximum local de $f$, si $f$ est dérivable en $a$, il faut et il suffit que $f'(a)=0$, $\exists \delta_->0$ tel que $f'(x)>0$ pour tout $x\in[a-\delta_-,a[$ et $\exists \delta_+>0$ tel que $f'(x)<0$ pour tout $x\in]a, a-\delta_+]$.\newline

    \item Pour que $a$ soit un minimum local de $f$, si $f$ est dérivable en $a$, il faut et il suffit que $f'(a)=0$, $\exists \delta_->0$ tel que $f'(x)<0$ pour tout $x\in[a-\delta_-,a[$ et $\exists \delta_+>0$ tel que $f'(x)>0$ pour tout $x\in]a, a-\delta_+]$.\newline

    \end{itemize}

  \end{itemize}

\end{frame}


\begin{frame}
  \frametitle{Extrema, IV}
  \framesubtitle{Exemple avec une fonction non dérivable en l'extremum}

  \begin{columns}[onlytextwidth]
    \begin{column}{.5\textwidth}
      \begin{itemize}

      \item $f(x) = x^{\frac{2}{3}}$\newline

      \item $f'(x) = \frac{2}{3}x^{-\frac{1}{3}}$\newline

      \item $f'$ a une discontinuité (infinie) en $x=0$\ldots\newline

      \item La fonction admet un minimum global en 0, pourtant la dérivée
        en 0 n'est pas définie.\newline

      \item Même si $f'(0) \neq 0$, on a bien le changement de signe
        de la dérivée autour de l'extremum.
      \end{itemize}
    \end{column}
    \begin{column}{.5\textwidth}
      \begin{center}
        \begin{tikzpicture}[scale=1]
          \begin{axis}[
            xticklabels={,,},
            yticklabels={,,},
            enlargelimits=true,
            grid style={dashed, gray!60},
            axis x line = bottom,
            axis y line = left,
            axis line style={thin},
            xmax = 4,
            xmin = -4,
            ymax = 3,
            ymin = -0.1,
            axis lines = middle,
            small,
            clip=false,
            ]
            \addplot[
            draw=black,
            samples=5000,
            thick,
            domain=-3:3,
            ]
            {(x^2)^(1/3)};
          \end{axis}
        \end{tikzpicture}
      \end{center}
    \end{column}
  \end{columns}

\end{frame}


\begin{frame}
  \frametitle{Extrema, V}
  \framesubtitle{Comment identifier les extrema d'une fonction~?}

  Il suffit de suivre les étapes suivantes~:

  \bigskip

  \begin{itemize}

  \item[1.] Calculer la dérivée première $f'(x)$.\newline

  \item[2.] Identifier les valeurs de $x$ pour lesquelles $f'$ est nulle.\newline

  \item[3.] Identifier les valeurs de $x$ pour lesquelles $f'$ est discontinue.\newline

  \item[4.] Pour chaque valeur $a$ identifiée en 2.a et 2.b, déterminer si $f'$ change de signe autour de $a$~:\newline
    \begin{itemize}
    \item $f'$ passe du signe $-$ à $+$: minimum local,
    \item $f'$ passe du signe $+$ à $-$: maximum local,

    \end{itemize}

  \end{itemize}

  \bigskip

  Ensuite il ne reste qu'à évaluer la fonction $f$ sur les extrema identifiés.

\end{frame}


\begin{frame}
  \frametitle{Extrema, VI}
  \framesubtitle{Exemple (a)}

  \medskip

  \begin{itemize}

  \item Soit la fonction $f(x) = (1+x)^{\frac{2}{3}}(2-x)^{\frac{1}{3}}$. On cherche les extrema de cette fonction.\newline

  \item Calculons la dérivée de la fonction~:
    \[
      \begin{split}
        f'(x) &= -\frac{1}{3}(1+x)^{\frac{2}{3}}(2-x)^{-\frac{2}{3}}+\frac{2}{3}(1+x)^{-\frac{1}{3}}(2-x)^{\frac{1}{3}}\\
        &= \frac{1}{3}(1+x)^{-\frac{1}{3}}(2-x)^{-\frac{2}{3}}\Bigl(-(1+x)+2(2-x)\Bigr)\\
        &=\frac{1-x}{(1+x)^{\frac{1}{3}}(2-x)^{\frac{2}{3}}}
      \end{split}
    \]

    \bigskip

  \item La dérivée $f'$ est nulle lorsque $x=1$.\newline

  \item La dérivée $f'$ est discontinue en $x=-1$ et $x=2$ car le dénominateur est alors nul. Il s'agit de discontinuités infinies, nous avons~:
    \[
      \lim_{x\rightarrow 1^-}f'(x) = -\infty \quad \text{ et } \quad \lim_{x\rightarrow 1^+}f'(x) = \infty
    \]
    \[
      \lim_{x\rightarrow 2^-}f'(x) = -\infty \quad \text{ et } \quad \lim_{x\rightarrow 2^+}f'(x) = -\infty
    \]

  \end{itemize}

\end{frame}


\begin{frame}
  \frametitle{Extrema, VI}
  \framesubtitle{Exemple (b)}

  \medskip

  \begin{itemize}

  \item Déterminons les changements de signe de la dérivée $f'$ en $x=-1$, $x=1$ et $x=2$~:

    \[
      \begin{cases}
        f'(x)<0 &\text{ si }x<-1\\
        f'(x)>0 &\text{ si }-1<x<1
      \end{cases}
      \Rightarrow
      \text{ minimum local en }x=-1
    \]

    \[
      \begin{cases}
        f'(x)>0 &\text{ si }-1<x<1\\
        f'(x)<0 &\text{ si }1<x<2\\
      \end{cases}
      \Rightarrow
      \text{ maximum local en }x=1
    \]

    \bigskip

    \[
      \begin{cases}
        f'(x)<0 &\text{ si }1<x<2\\
        f'(x)<0 &\text{ si }x>2
      \end{cases}
      \Rightarrow
      \text{ pas d'extremum en }x=2
    \]

    \bigskip

  \item $f(-1) = f(2) = 0$ et $f(1) = \sqrt[3]{4}$.

  \end{itemize}

\end{frame}


\begin{frame}
  \frametitle{Extrema, VI}
  \framesubtitle{Exemple (c)}

  \begin{center}
    \begin{tikzpicture}[scale=1.8]
      \begin{axis}[
        xticklabels={,,},
        yticklabels={,,},
        enlargelimits=true,
        grid style={dashed, gray!60},
        axis x line = bottom,
        axis y line = left,
        axis line style={thin},
        xmax = 3,
        xmin = -3,
        ymax = 4,
        ymin = -3,
        axis lines = middle,
        small,
        clip=false,
        ]
        \addplot[
        draw=black,
        samples=50,
        thick,
        domain=-3.5:-1.2,
        ]
        {(((1+x)^2)^(1/3))*(((2-x)^2)^(1/6))};
        \addplot[
        draw=black,
        samples=1000,
        thick,
        domain=-1.2:-0.8,
        ]
        {(((1+x)^2)^(1/3))*(((2-x)^2)^(1/6))};
        \addplot[
        draw=black,
        samples=200,
        thick,
        domain=-0.8:1.8,
        ]
        {(((1+x)^2)^(1/3))*(((2-x)^2)^(1/6))};
        \addplot[
        draw=black,
        samples=1000,
        thick,
        domain=1.8:2,
        ]
        {(((1+x)^2)^(1/3))*(((2-x)^2)^(1/6))};
        \addplot[
        draw=black,
        samples=1000,
        thick,
        domain=2:2.2,
        ]
        {-(((1+x)^2)^(1/3))*(((2-x)^2)^(1/6))};
        \addplot[
        draw=black,
        samples=200,
        thick,
        domain=2:3.5,
        ]
        {-(((1+x)^2)^(1/3))*(((2-x)^2)^(1/6))};
      \end{axis}
    \end{tikzpicture}
  \end{center}

\end{frame}


\section{Convexité et concavité}


\begin{frame}
  \frametitle{Fonctions convexes et concaves}
  \framesubtitle{Définitions}
  \hypertarget{slide_fonctions_convexes_1}{}
  \bigskip Une fonction $f$ est convexe (concave) si et seulement si
  les cordes sont au dessus (dessous) de la courbe représentative de
  $f$.

  {\small
  \begin{definition}\label{def:convexe}
    Une fonction $f:E\rightarrow \mathbb R$ est convexe sur $E$ si et seulement si $\forall (x,y)\in E^2$ et $\forall \lambda\in[0,1]$~:
    \[
      f(\lambda x + (1-\lambda)y) \leq \underbrace{\lambda f(x) + (1-\lambda)f(y)}_{\text{un point sur la corde}}
    \]
    On parle de convexité stricte si l'inégalité est stricte.
  \end{definition}

  \begin{definition}\label{def:concave}
    Une fonction $f:E\rightarrow \mathbb R$ est concave sur $E$ si et seulement si $\forall (x,y)\in E^2$ et $\forall \lambda\in[0,1]$~:
    \[
      f(\lambda x + (1-\lambda)y) \geq \lambda f(x) + (1-\lambda)f(y)
    \]
    On parle de concavité stricte si l'inégalité est stricte.
  \end{definition}
}
\end{frame}


\begin{frame}
  \frametitle{Fonctions convexes et concaves}
  \framesubtitle{Illustration graphique}

  \begin{center}
    \begin{tikzpicture}[scale=1.6]
      \begin{axis}[
        title={},
        xlabel= {},
        ylabel= {},
        xticklabels={,,},
        yticklabels={,,},
        enlargelimits=true,
        grid style={dashed, gray!60},
        axis x line = bottom,
        axis y line = left,
        axis lines = middle,
        axis line style={thin},
        xmin = -.1,
        xmax = 3,
        ymin = -.1,
        ymax = 3,
        small,
        clip=false,
        extra x ticks={1.25},
        extra tick style={blue},
        extra x tick labels={\color{red}\tiny$\lambda x + (1-\lambda)y$},
        ]
        \addplot[
        draw=black,
        thick,
        smooth,
        samples=200,
        domain=-.1:2.2,
        ]
        {exp((x-1))} node[right] {\tiny $\mathcal C_f$} ;
        \coordinate (x) at (.5,0.6065306597126334);
        \coordinate (y) at (2,2.718281828459045);
        \node[draw=red, circle, fill=red, scale=.2] at (x) {};
        \node[draw=red, circle, fill=red, scale=.2] at (y) {};
        \draw[red] (x) -- (y);
        \draw[red, dotted] (.5,0) -- (x);
        \draw[red, dotted] (0,0.6065306597126334) -- (x);
        \node[red, left] at (0,0.6065306597126334) {\tiny $f(x)$};
        \node[below, red] at (0.5,0) {\tiny $x$};
        \draw[red, dotted] (2,0) -- (y);
        \draw[red, dotted] (0,2.718281828459045) -- (y);
        \node[red, left] at (0,2.718281828459045) {\tiny $f(y)$};
        \node[below, red] at (2,0) {\tiny $y$};
        \draw[dashed, red] (1.25,0) -- (1.25,1.6624062440858394);
        \draw[dashed, red] (0,1.6624062440858394) -- (1.25,1.6624062440858394);
        \node[red, left] at (0,1.6624062440858394) {\tiny $\lambda f(x)+(1-\lambda)f(y)$};
        \draw[dashed, black] (0,1.2840254166877414) -- (1.25,1.2840254166877414);
        \node[black, left] at (0,1.2840254166877414) {\tiny $f(\lambda x+(1-\lambda)y)$};
      \end{axis}
    \end{tikzpicture}
  \end{center}

\end{frame}


\begin{frame}
  \frametitle{Fonctions convexes et concaves}
  \framesubtitle{Inégalité des pentes (a)}
  \hypertarget{slide_fonctions_convexes_3}{}

  \begin{theorem}\label{thm:trois_pentes}
    Soit une fonction $f:E\rightarrow \mathbb R$ convexe sur $E$, alors $\forall (a,c,b)\in E^3$ tels que $a<c<b$ on a~:
    \[
      \frac{f(c)-f(a)}{c-a} \leq \frac{f(b)-f(a)}{b-a} \leq \frac{f(b)-f(c)}{b-c}
    \]
    et donc~:
    \[
      \frac{f(c)-f(a)}{c-a} \leq \frac{f(b)-f(c)}{b-c}
    \]
    Réciproquement si l'une des trois inégalités est vérifiée $\forall (a,b,c)\in E^3$ tels que $a<c<b$ alors $f$ est convexe sur $E$.
  \end{theorem}

\end{frame}


\begin{frame}
  \frametitle{Fonctions convexes et concaves}
  \framesubtitle{Inégalité des pentes (b)}

  \begin{center}
    \begin{tikzpicture}[scale=1.4]
      \begin{axis}[
        title={},
        xlabel= {},
        ylabel= {},
        xticklabels={,,},
        yticklabels={,,},
        enlargelimits=true,
        grid style={dashed, gray!60},
        axis x line = bottom,
        axis y line = left,
        axis lines = middle,
        axis line style={thin},
        xmin = -.1,
        xmax = 3.5,
        ymin = -.1,
        ymax = 5.5,
        small,
        clip=false,
        ]
        \addplot[
        draw=black,
        thick,
        smooth,
        samples=200,
        domain=-.1:3.1,
        ]
        {(x-1)^2+1} node[right] {\tiny $\mathcal C_f$} ;
        \coordinate (A) at (.5,1.25);
        \coordinate (C) at (1.25,1.0625);
        \coordinate (B) at (3,5);
        \draw[red] (A) -- (C);
        \draw[red] (C) -- (B);
        \draw[red] (A) -- (B);
        \node[draw=red, circle, fill=red, scale=.2] at (A) {};
        \node[draw=red, circle, fill=red, scale=.2] at (C) {};
        \node[draw=red, circle, fill=red, scale=.2] at (B) {};
        \node[red, below] at (.5, 1.2) {\tiny $\mathbf{A}$};
        \node[red, below] at (1.25, 1) {\tiny $\mathbf{C}$};
        \node[red, below] at (3, 4.8) {\tiny $\mathbf{B}$};
      \end{axis}
    \end{tikzpicture}
  \end{center}
  $\mathcal C_f$ est la courbe représentative d'une fonction convexe. La pente entre $A$ et $C$ est inférieure à la pente entre $A$ et $B$ qui est inférieure à la pente entre $C$ et $B$.

\end{frame}


\begin{notes}
  \textbf{preuve du théorème \hyperlink{slide_fonctions_convexes_3}{\ref{thm:trois_pentes}}.}
  $\forall (a,b,c)\in E^3$ tels que $a<c<b$ chacune de ces trois inégalités peut se réécrire de façon équivalente~:
  \[
    f(c) \leq \frac{b-c}{b-a}f(a) + \frac{c-a}{b-a}f(b)
  \]
  Elles sont donc équivalentes. Posons $\lambda = \frac{b-c}{b-a}$, on a~:
  \[
    1-\lambda = \frac{b-a-b+c}{b-a} = \frac{c-a}{b-a}
  \]
  On note que $\lambda\in]0,1[$ puisque $c$ est strictement compris entre $a$ et $b$ et que~:
  \[
    \lambda a + (1-\lambda) b = \frac{(b-c)a+(c-a)b}{b-a} = \frac{c(b-a)}{b-a} = c
  \]
  Ainsi, nous avons~:
  \[
    f(\lambda a + (1-\lambda) b) \leq \lambda f(a) + (1-\lambda) f(b)
  \]
  et donc la fonction $f$ est convexe.
\end{notes}


\begin{frame}
  \frametitle{Fonctions convexes et concaves}
  \framesubtitle{Inégalité des pentes (c)}
  \hypertarget{slide_fonctions_convexes_5}{}

  \begin{theorem}\label{thm:pentes}
    Soit une fonction $f:E\rightarrow \mathbb R$ une fonction. Pour tout $a\in E$ on peut définir la fonction « pente »~:
    \[
      \begin{split}
        \Delta_a: E\setminus\{a\}&\rightarrow \mathbb R\\
        x &\mapsto \frac{f(x)-f(a)}{x-a}
      \end{split}
    \]
    Alors les assertions suivantes son équivalentes~:
    \begin{enumerate}[(i)]
    \item $f$ est convexe sur $E$,
    \item $\forall a\in E$, $\Delta_a$ est une fonction croissante sur $\{x\in E| x>a\}$,
    \item $\forall a\in E$, $\Delta_a$ est une fonction croissante sur $\{x\in E| x<a\}$,
    \item $\forall a\in E$, $\Delta_a$ est une fonction croissante sur $E\setminus\{a\}$.
    \end{enumerate}
  \end{theorem}

\end{frame}

\begin{frame}
  \frametitle{Fonctions convexes et concaves}
  \framesubtitle{Dérivées (a)}
  \hypertarget{slide_fonctions_convexes_6}{}

  \begin{itemize}

  \item Les théorèmes \hyperlink{slide_fonctions_convexes_3}{\ref{thm:trois_pentes}} et \hyperlink{slide_fonctions_convexes_5}{\ref{thm:pentes}} permettent de montrer qu'une fonction $f$ convexe possède des dérivées à droite et à gauche en tout point à l'intérieur de $E$ et donc assure la continuité de la fonction à l'intérieur de $E$.\newline

  \item En calculant les pentes sur des intervalles de plus en plus petit\ldots On en déduit à la limite une condition sur le sens de variation de la dérivée.\newline

  \end{itemize}

  \begin{theorem}
    Une fonction $f:\, E\rightarrow \mathbb R$ est convexe si et seulement si la dérivée $f'$ est une fonction croissante (en supposant que la dérivée première existe), c'est-à-dire si et seulement si la dérivée seconde est $f''(x)>0\,\forall x\in E$ (en supposant que la dérivée seconde existe).
  \end{theorem}

\end{frame}


\begin{frame}
  \frametitle{Fonctions convexes et concaves}
  \framesubtitle{Dérivées (b)}
  \hypertarget{slide_fonctions_convexes_7}{}

  \begin{itemize}

  \item $f$ est concave si et seulement si $-f$ est convexe (voir les définitions \hyperlink{slide_fonctions_convexes_1}{\ref{def:convexe}} et \hyperlink{slide_fonctions_convexes_1}{\ref{def:concave}})\newline

  \item On peut reprendre le théorèmes \hyperlink{slide_fonctions_convexes_3}{\ref{thm:trois_pentes}} et \hyperlink{slide_fonctions_convexes_5}{\ref{thm:pentes}} en inversant les inégalités pour les fonctions concaves.\newline

  \item Si $f$ est deux fois dérivable sur $E$ alors elle est concave sur $E$ si et seulement si sa dérivée seconde est négative sur $E$.\newline

  \end{itemize}

\end{frame}


\begin{frame}
  \frametitle{Fonctions convexes et concaves}
  \framesubtitle{Extrema (a)}
  \hypertarget{slide_fonctions_convexes_8}{}

  \begin{itemize}

  \item Pour qu'un extremum local $x_0$ de $f$ soit un maximum local,
    il faut que, dans un voisinage de $x_0$ la fonction $f$ soit
    croissante puis décroissante à partir de $x_0$.\newline

  \item Si la dérivée $f'$ existe dans un voisinage de $x_0$, pour
    qu'un extremum local $x_0$ de $f$ soit un maximum local, il faut
    que, dans un voisinage de $x_0$ la dérivée $f'$ soit positive
    puis négative à partir de $x_0$ (la dérivée décroît).\newline

  \item Si la dérivée seconde $f''$ existe dans un voisinage de $x_0$, pour
    qu'un extremum local $x_0$ de $f$ soit un maximum local, il faut
    que, dans un voisinage de $x_0$ la dérivée $f''$ soit négative\newline

  \end{itemize}

\end{frame}


\begin{frame}
  \frametitle{Fonctions convexes et concaves}
  \framesubtitle{Extrema (b)}
  \hypertarget{slide_fonctions_convexes_9}{}

  \begin{theorem}
    Soit $f$ une fonction de $E$ dans $\mathbb R$ telle que les dérivées $f'$ et $f''$ soient continues en $x_0\in E$. Alors~:
    \begin{enumerate}[(i)]
    \item $f'(x_0) = 0$ et $f''(x_0)<0$ $\Rightarrow$ maximum local en $x_0$,
    \item $f'(x_0) = 0$ et $f''(x_0)>0$ $\Rightarrow$ minimum local en $x_0$.
    \end{enumerate}
  \end{theorem}

  \bigskip

  \begin{itemize}

  \item Si la fonction est concave (convexe) sur $E$, alors
    $f'(x_0)=0$ $\Rightarrow$ $x_0$ est un maximum (minimum) global
    sur $E$.\newline

  \item Si la dérivée seconde est nulle, il faut aller chercher des
    dérivées d'ordre supérieur pour conclure. Si les dérivées
    $f'(x_0)$, $f''(x_0)$, \ldots, $f^{(n-1)}(x_0)$ sont nulles et
    $f^{(n)}(x_0)\neq 0$, alors quand $n$ est impair il n'y a pas
    d'extremum en $x_0$ et quand $n$ est pair~:

    \begin{itemize}
    \item $f^{(n)}(x_0)>0$ $\Rightarrow$ minimum local en $x_0$,
    \item $f^{(n)}(x_0)<0$ $\Rightarrow$ maximum local en $x_0$.
    \end{itemize}

  \end{itemize}
\end{frame}


\begin{frame}
  \frametitle{Fonctions convexes et concaves}
  \framesubtitle{Exemple}
  \hypertarget{slide_fonctions_convexes_10}{}

  \begin{itemize}

  \item Soit la fonction polynomiale $f(x) = x^3-2x^2+x+1$.\newline

  \item On a~: $f'(x) = 3x^2-4x+1$ et $f''(x) = 6x-4$.\newline

  \end{itemize}

  \begin{columns}[onlytextwidth]
    \begin{column}{.5\textwidth}
      \begin{itemize}
      \item La dérivée première est nulle en $x=1$ et $x=\frac{1}{3}$\newline
      \item La dérivée seconde est positive ssi $x>\frac{2}{3}$\newline
      \item La fonction $f$ est concave sur $]-\infty,\frac{2}{3}]$ et convexe sur $[\frac{2}{3},\infty[$\newline
      \item On a un maximum local en $x=\frac{1}{3}$ et minimum local en $x=1$.
      \end{itemize}
    \end{column}
    \begin{column}{.5\textwidth}
      \begin{center}
        \begin{tikzpicture}[scale=1]
          \begin{axis}[
            xticklabels={,,},
            yticklabels={,,},
            enlargelimits=true,
            grid style={dashed, gray!60},
            axis x line = bottom,
            axis y line = left,
            axis line style={thin},
            xmax = 2,
            xmin = -2,
            ymax = 2.5,
            ymin = -2,
            axis lines = middle,
            small,
            clip=false,
            ]
            \addplot[
            draw=black,
            samples=5000,
            thick,
            domain=-1:1.9,
            ]
            {x^3-2*x^2+x+1};
            \coordinate (A) at (.333333,1.1481481481481481);
            \coordinate (B) at (1,1);
            \node[draw=red, circle, fill=red, scale=.2] at (A) {};
            \node[draw=red, circle, fill=red, scale=.2] at (B) {};
            \draw[red,dotted] (A) -- (.333333,0);
            \draw[red,dotted] (B) -- (1,0);
            \node[below] at (.333333,0) {\tiny $\frac{1}{3}$};
            \node[below] at (1,0) {\tiny $1$};
          \end{axis}
        \end{tikzpicture}
      \end{center}
    \end{column}
  \end{columns}
\end{frame}



\begin{frame}
  \frametitle{Fonctions convexes et concaves}
  \framesubtitle{Point d'inflexion (a)}
  \hypertarget{slide_fonctions_convexes_11}{}

  \begin{itemize}

  \item Dans l'exemple, la fonction est concave puis convexe au delà de $x=\frac{2}{3}$.\newline

  \item Un point où la courbure d'une fonction change (concavité/convexité) est appelé un point d'infléxion.\newline

  \item Si la fonction $f\in\mathcal C^2$, alors on trouve les points d'inflexions en (i) identifiant les points $\bar x$ tels que $f''(\bar x)=0$, (ii) vérifiant si $f''$ change de signe en $\bar x$.\newline

  \item[\dbend] Si la fonction $f$ n'est pas de classe $\mathcal C^2$, il faut aussi s'intéresser aux points $\tilde x$ où $f''$ est discontinue et vérifier s'il y a un changement de signe de $f''$ autour de ces points.

  \end{itemize}

\end{frame}


\begin{frame}
  \frametitle{Fonctions convexes et concaves}
  \framesubtitle{Point d'inflexion (b)}
  \hypertarget{slide_fonctions_convexes_12}{}

  \begin{example}
    Soit la fonction de $\mathbb R$ dans $\mathbb R$ $f(x) = x^5$

    \begin{columns}[onlytextwidth]
      \begin{column}{.5\textwidth}
        \begin{itemize}
        \item On a $f'(x) = 5x^4>0$.\newline
        \item La fonction $f$ est donc croissante.\newline
        \item On a aussi $f''(x) = 20x^3$.\newline
        \item $f''(x) = 0$ ssi $x=0$, et $f''(x)<0$ ssi $x<0$.\newline
        \item On a donc un point d'inflexion en 0.
        \end{itemize}
      \end{column}
      \begin{column}{.5\textwidth}
        \begin{center}
          \begin{tikzpicture}[scale=1]
            \begin{axis}[
              xticklabels={,,},
              yticklabels={,,},
              enlargelimits=true,
              grid style={dashed, gray!60},
              axis x line = bottom,
              axis y line = left,
              axis line style={thin},
              xmax = 1.1,
              xmin = -1.1,
              ymax = 1.1,
              ymin = -1.1,
              axis lines = middle,
              small,
              clip=false,
              ]
              \addplot[
              draw=black,
              samples=500,
              thick,
              domain=-1:1,
              ]
              {x^5};
              \node[below] at (-.1,0) {\tiny $0$};
            \end{axis}
          \end{tikzpicture}
        \end{center}
      \end{column}
    \end{columns}
  \end{example}

\end{frame}


\begin{frame}
  \frametitle{Fonctions convexes et concaves}
  \framesubtitle{Point d'inflexion (c)}
  \hypertarget{slide_fonctions_convexes_13}{}

  \begin{example}
    Soit la fonction de $\mathbb R$ dans $\mathbb R$ $f(x) = \sqrt[3]{x}$

    \begin{columns}[onlytextwidth]
      \begin{column}{.5\textwidth}
        \begin{itemize}
        \item On a $f'(x) = \frac{1}{3}x^{-\frac{2}{3}}>0$.\newline
        \item La fonction $f$ est donc croissante.\newline
        \item On a aussi $f''(x) = -\frac{2}{9}x^{-\frac{5}{3}}\neq 0$ $\forall x\in\mathbb R$.\newline
        \item La dérivée seconde (comme la dérivée première) n'est pas définie en zéro.\newline
        \item Comme $f''(x)>0$ ssi $x<0$, on a un point d'inflexion en 0.
        \end{itemize}
      \end{column}
      \begin{column}{.5\textwidth}
        \begin{center}
          \begin{tikzpicture}[scale=1]
            \begin{axis}[
              xticklabels={,,},
              yticklabels={,,},
              enlargelimits=true,
              grid style={dashed, gray!60},
              axis x line = bottom,
              axis y line = left,
              axis line style={thin},
              xmax = 1.1,
              xmin = -1.1,
              ymax = 1.1,
              ymin = -1.1,
              axis lines = middle,
              small,
              clip=false,
              ]
              \addplot[
              draw=black,
              smooth,
              samples=500,
              thick,
              domain=0:1,
              ]
              {x^.333333};
              \addplot[
              draw=black,
              smooth,
              samples=5000,
              thick,
              domain=-1:0,
              ]
              {-1*(-x)^.333333};
              \node[below] at (-.1,0) {\tiny $0$};
            \end{axis}
          \end{tikzpicture}
        \end{center}
      \end{column}
    \end{columns}
  \end{example}

\end{frame}

\section{Approximation polynomiale}

\begin{frame}
  \frametitle{Approximation polynomiale en un point}
  \framesubtitle{Le problème}
  \hypertarget{slide_taylor_1}{}

  \begin{itemize}

  \item On peut appliquer le calcul des dérivées pour approximer une fonction en un point.\newline

  \item Supposons que nous souhaitions approximer une fonction $f: E\rightarrow \mathbb R$ de classe $\mathcal C^n$ dans un voisinage d'un point $x_0\in E$ à l'aide d'une fonction polynomiale $p_n(x) = \alpha_0+\alpha_1x+\alpha_2x^2+\ldots+\alpha_nx^n$\newline

  \item En exploitant seulement des informations sur la fonction $f$ au point $x_0$~: $f(x_0)$, $f'(x_0)$, $f''(x_0)$, \ldots, $f^{(n)}(x_0)$ (par exemple parce qu'il est trop coûteux d'évaluer la fonction $f$ en un autre point).\newline

  \item On cherche la fonction polynomiale, c'est-à-dire les paramètres $(\alpha_i)_{i=1}^n$, la plus proche possible de la fonction $f$ (dans un voisinage de $x_0$).\newline

  \item Le degré du polynôme $n$ est l'ordre d'appoximation.

  \end{itemize}

\end{frame}

\begin{frame}
  \frametitle{Approximation polynomiale en un point}
  \framesubtitle{Approximation de $f(x) = \frac{1}{1-x}$ autour de 0}
  \hypertarget{slide_taylor_2}{}

  \begin{center}
    \begin{tikzpicture}[scale=1.5]
      \begin{axis}[
        xticklabels={,,},
        yticklabels={,,},
        enlargelimits=true,
        grid style={dashed, gray!60},
        axis x line = bottom,
        axis y line = left,
        axis line style={thin},
        xmax = 1,
        xmin = -1,
        ymax = 5,
        ymin = -.5,
        axis lines = middle,
        small,
        clip=false,
        ]
        \addplot[
        draw=red,
        smooth,
        samples=100,
        ultra thick,
        domain=-.9:.8,
        ]
        {1/(1-x)};
        \addplot[
        draw=red!10!blue,
        samples=2,
        domain=-.9:.9,
        ]
        {1};
        \node[red!10!blue, right] at (.975, 1) {\tiny $n=0$};
        \addplot[
        draw=red!20!blue,
        samples=2,
        domain=-.9:.9,
        ]
        {1+x};
        \node[red!20!blue, right] at (.975, 1.9) {\tiny $n=1$};
        \addplot[
        draw=red!30!blue,
        samples=200,
        smooth,
        domain=-.9:.9,
        ]
        {1+x+x^2};
        \node[red!30!blue, right] at (.975, 2.71) {\tiny $n=2$};
        \addplot[
        draw=red!40!blue,
        samples=200,
        smooth,
        domain=-.9:.9,
        ]
        {1+x+x^2+x^3};
        \node[red!40!blue, right] at (.975, 3.439) {\tiny $n=3$};
        \addplot[
        draw=red!50!blue,
        samples=200,
        smooth,
        domain=-.9:.9,
        ]
        {1+x+x^2+x^3+x^4};
        \node[red!50!blue, right] at (.975, 4.0951) {\tiny $n=4$};
        \addplot[
        draw=red!60!blue,
        samples=200,
        smooth,
        domain=-.9:.9,
        ]
        {1+x+x^2+x^3+x^4+x^5};
        \node[red!60!blue, right] at (.975, 4.68559) {\tiny $n=5$};
        \addplot[
        draw=red!80!blue,
        samples=200,
        smooth,
        domain=-.9:.9,
        ]
        {1+x+x^2+x^3+x^4+x^5+x^6+x^7+x^8+x^9+x^10};
        \node[red!80!blue, right] at (.975, 6.861894039100001) {\tiny $n=10$};
      \end{axis}
    \end{tikzpicture}
  \end{center}
\end{frame}


\begin{frame}
  \frametitle{Approximation polynomiale en un point}
  \framesubtitle{Approximation d'ordre 0}
  \hypertarget{slide_taylor_3}{}

  \begin{itemize}

  \item On veut approximer $f(x)$ dans un voisinag!e de $x_0$ par la fonction polynomiale $p_0(x) = \alpha_0$, une constante.\newline

  \item Nous n'avons qu'un paramètre à déterminer ($\alpha_0$).\newline

  \item Étant donnée l'information disponible (les propriétés de $f$ en $x_0$) on obtient la meilleure approximation possible en s'assurant qu'en $x_0$ les niveaux de $p_0(x)$ et $f(x)$ sont identiques.\newline

  \item On pose $\alpha_0 = f(x_0)$, de sorte que notre approximation de $f(x)$ est~:
    \[
      p_0(x) = f(x_0)
    \]

  \item Il y a de fortes chances pour que cette approximation ne soit pas très bonne, sauf si le niveau de $f$ est peu variable (ou mieux si $f$ est constante auquel cas il n'y a pas d'erreur d'appoximation).

  \end{itemize}

\end{frame}


\begin{frame}
  \frametitle{Approximation polynomiale en un point}
  \framesubtitle{Approximation d'ordre 1}
  \hypertarget{slide_taylor_4}{}

  \bigskip

  \begin{itemize}

  \item On veut approximer $f(x)$ dans un voisinage de $x_0$ par la fonction polynomiale $p_1(x) = \alpha_0+\alpha_1 x$, une droite.\newline

  \item Nous avons deux paramètres à déterminer ($\alpha_0$ et $\alpha_1$).\newline

  \item Étant donnée l'information disponible (les propriétés de $f$ en $x_0$) on obtient la meilleure approximation possible en s'assurant qu'en $x_0$ les pentes de $p_0(x)$ et $f(x)$ sont identiques.\newline

  \item On veut que $p_1(x)$ soit tel que $p_1(x_0) = f(x_0)$ et $p_1'(x_0) = f'(x_0)$, c'est-à-dire~:
    \[
      \begin{cases}
        \alpha_0+\alpha_1x_0 &= f(x_0)\\
        \alpha_1 &= f'(x_0)
      \end{cases}
      \quad\Leftrightarrow\quad
      \begin{cases}
        \alpha_0 &= f(x_0)-f'(x_0)x_0\\
        \alpha_1 &= f'(x_0)
      \end{cases}
    \]

    \bigskip

  \item La fonction $f$ est donc approximée par~:
    \[
      p_1(x) = f(x_0) + f'(x_0)(x-x_0)
    \]

  \end{itemize}

\end{frame}


\begin{frame}
  \frametitle{Approximation polynomiale en un point}
  \framesubtitle{Approximation d'ordre 2}
  \hypertarget{slide_taylor_5}{}

  \bigskip

  \begin{itemize}

  \item On veut approximer $f(x)$ dans un voisinage de $x_0$ par la fonction polynomiale $p_2(x) = \alpha_0+\alpha_1 x + \alpha_2 x^2$, une parabole.\newline

  \item Les trois paramètres $\alpha_0$, $\alpha_1$ et $\alpha_2$ sont identifiés en égalisant les dérivées d'ordre 0, 1 et 2 de $f$ et $p_2$ en $x_0$.\newline
    {\small
      \[
        \begin{cases}
          \alpha_0+\alpha_1x_0+\alpha_2x_0^2 &= f(x_0)\\
          \alpha_1+2\alpha_2x_0 &= f'(x_0)\\
          2\alpha_2 &= f''(x_0)
        \end{cases}
        \quad\Leftrightarrow\quad
        \begin{cases}
          \alpha_0 &= f(x_0)-f'(x_0)x_0-\frac{1}{2}f''(x_0)x_0^2\\
          \alpha_1 &= f'(x_0)-f''(x_0)x_0\\
          \alpha_2 &= \frac{1}{2}f''(x_0)
        \end{cases}
      \]}

    \medskip

  \item La fonction $f$ est donc approximée par~:
    \[
      p_2(x) = f(x_0) + f'(x_0)(x-x_0) + \frac{1}{2}f''(x_0)(x-x_0)^2
    \]

    \medskip

  \item On note que le coefficient associé à $(x-x_0)$ n'a pas changé
    entre les approximations à l'ordre 1 et 2 (de même pour la
    constante).

  \end{itemize}

\end{frame}


\begin{frame}
  \frametitle{Approximation polynomiale en un point}
  \framesubtitle{Approximation d'ordre 3}
  \hypertarget{slide_taylor_6}{}

  \bigskip

  \begin{itemize}

  \item On veut approximer $f(x)$ dans un voisinage de $x_0$ par la fonction polynomiale $p_3(x) = \alpha_0+\alpha_1 x + \alpha_2 x^2 + \alpha_3 x^3$.\newline

  \item Les quatre paramètres $\alpha_0$, $\alpha_1$, $\alpha_2$ et $\alpha_3$ sont identifiés en égalisant les dérivées d'ordre 0, 1, 2 et 3 de $f$ et $p_3$ en $x_0$.\newline

  \item On peut montrer que $f$ doit être approximée par~:
    \[
      p_3(x) = f(x_0) + f'(x_0)(x-x_0) + \frac{1}{2}f''(x_0)(x-x_0)^2 + \frac{1}{3\cdot 2}f'''(x_0)(x-x_0)^3
    \]

    \medskip

  \item À nouveau, on note que le coefficient associé à $(x-x_0)$ est identique avec les approximations à l'ordre 1, 2 et 3, ou que le coefficient associé à $(x-x_0)^2$ est identique pour les approximations aux ordres 2 et 3.

  \end{itemize}

\end{frame}


\begin{notes}
  En identifiant les dérivées de $f$ et de $p_3$ on obtient le système suivant~:
  \[
    \begin{cases}
      \alpha_0+\alpha_1x_0+\alpha_2x_0^2 +\alpha_3x_0^3&= f(x_0)\\
      \alpha_1+2\alpha_2x_0+3\alpha_3 x_0^2 &= f'(x_0)\\
      2\alpha_2+6\alpha_3 x_0 &= f''(x_0)\\
      6\alpha_3 &= f'''(x_0)
    \end{cases}
  \]
  On peut résoudre ce système en remontant par le bas. Avec la dernière équation on obtient directement $\alpha_3 = \frac{1}{3\cdot 2}f'''(x_0)$. En substituant dans la troisième équation il vient $\alpha_2 = \frac{1}{2}f''(x_0)-\frac{1}{2}f'''(x_0)x_0$. Par substitution dans la deuxième équation on obtient~:
  \[
    \alpha_1 = f'(x_0)-f''(x_0)x_0+\frac{1}{2}f'''(x_0)x_0^2
  \]
  Et finalement, en substituant le tout dans la première équation~:
  \[
    \alpha_0 = f(x_0)-f'(x_0)x_0+\frac{1}{2}f''(x_0)x_0^2-\frac{1}{6}f'''(x_0)x_0^3
  \]
  Le polynôme de degré trois est donc~:
  \[
    p_3(x) = f(x_0)+f'(x_0)(x-x_0)+\frac{1}{2}f''(x_0)(x^2-2x_0x+x_0^2)+\frac{1}{6}f'''(x_0)(x^3-3x_0x^2+3x_0^2x-x_0^3)
  \]
  d'où le résultât annoncé en reconnaissant les identités remarquables.
\end{notes}


\begin{frame}
  \frametitle{Approximation polynomiale en un point}
  \framesubtitle{Approximation d'ordre $n$}
  \hypertarget{slide_taylor_7}{}

  \bigskip

  \begin{itemize}

  \item On veut approximer $f(x)$ dans un voisinage de $x_0$ par la fonction polynomiale $p_n(x) = \alpha_0+\alpha_1 x + \alpha_2 x^2 +\ldots+ \alpha_n x^n$.\newline

  \item Les $n+1$ paramètres $\alpha_0$, $\alpha_1$, $\alpha_2$, \ldots $\alpha_n$ sont identifiés en égalisant les dérivées d'ordre 0 à $n$ de $f$ et $p_n$ en $x_0$.\newline

  \item On peut montrer que $f$ peut alors être approximée par~:
    \[
      p_n(x) = \sum_{i=0}^n\frac{f^{(i)}(x_0)}{i!}(x-x_0)^i
    \]

  \item[\dbend] Il ne faut pas oublier que si $x\neq x_0$ alors $f(x)\neq p_n(x)$... Il ne s'agit que d'une approximation.
  \end{itemize}

\end{frame}


\begin{notes}
  Postulons que la fonction polynomiale utilisée pour approximer la fonction $f$ a la forme suivante~:
  \[
    p_n(x) = \sum_{i=0}^n c_i (x-x_0)^i
  \]
  Les dérivées de cette fonction sont~:
  \[
    p_n'(x) = \sum_{i=1}^nc_i i (x-x_0)^{i-1}
  \]
  \[
    p_n''(x) = \sum_{i=2}^nc_i i (i-1)(x-x_0)^{i-2}
  \]
  \[
    \vdots
  \]
  \[
    p_n^{(j)}(x) = \sum_{i=j}^nc_i i (i-1)\ldots(i-j+1)(x-x_0)^{i-j}
  \]
  \[
    \vdots
  \]
  \[
    p_n^{(n)}(x) = c_n n (n-1) \ldots 2
  \]
  En évaluant ces dérivées en $x_0$, on trouve $p'(x_0)=c_1$, $p''(x_0)=2c_2$, $p'''(x_0)=6c_3$, ..., $p^{(j)}(x_0) = c_j j!$, ..., $p^{(n)}(x_0) = c_n n!$. En identifiant avec les dérivées de $f$ évaluées en $x_0$, on trouve directement la formule données sur la page précédente.
\end{notes}


\begin{frame}
  \frametitle{Approximation polynomiale en un point}
  \framesubtitle{Formule de Taylor Young}
  \hypertarget{slide_taylor_8}{}

  \bigskip

  \begin{theorem}
    Soit $f$ de $E$ dans $\mathbb R$ une fonction de classe $\mathcal C^n$. Soit $a\in E$, alors il existe une fonction $\epsilon$ de $E$ dans $\mathbb R$ vérifiant $\lim_{x\rightarrow a}\epsilon(x) = 0$ telle que pour tout $x\in E$~:
    \[
      f(x) = f(a) + \sum_{k=1}^n\frac{f^{(k)}(a)}{k!}(x-a)^k + (x-a)^n\epsilon(x)
    \]
  \end{theorem}

  \begin{example}
    On sait que si $|x|<1$ alors $\sum_{i=0}^\infty x^i = \frac{1}{1-x}$ (il s'agit d'une série gémétrique). On a donc~:
    {\small
      \[
        \frac{1}{1-x} = \underbrace{\sum_{i=0}^nx^i}_{p_n(x)} + \sum_{i=n+1}^\infty x^i
        = \sum_{i=0}^nx^i + x^{n+1}\sum_{i=0}^\infty x^i
        = \sum_{i=0}^nx^i + x^{n}\underbrace{\frac{x}{1-x}}_{\epsilon(x)}
      \]}
  \end{example}

\end{frame}


\begin{frame}
  \frametitle{Approximation polynomiale en un point}
  \framesubtitle{Une expression pour le reste}
  \hypertarget{slide_taylor_9}{}
  \bigskip

  \begin{theorem}\label{thm:taylor}
    Soit $f$ de $E$ dans $\mathbb R$ une fonction de classe $\mathcal C^{n+1}$. Soit $a\in E$, alors pour tout $x\in E$~:
    \[
      f(x) = f(a) + \sum_{k=1}^n\frac{f^{(k)}(a)}{k!}(x-a)^k + R_n(x)
    \]
    avec le reste~:
    \[
      R_n(x) = \frac{f^{(n+1)}(c)}{(n+1)!}(x-a)^{n+1}
    \]
    où $c$ est compris entre $a$ et $x$.
  \end{theorem}

\end{frame}


\begin{notes}
  \textbf{Preuve du théorème \hyperlink{slide_taylor_9}{\ref{thm:taylor}}.} On utilise le théorème de Rolle \hyperlink{slide_extrema_2}{\ref{thm:rolle}}. Posons~:
  \[
    g(\lambda) = f(\lambda) - p_n(\lambda) - \frac{f(x)-p_n(x)}{(x-a)^{n+1}}(\lambda-a)^{n+1}
  \]
  avec
  \[
    p_n(x) = \sum_{k=1}^n\frac{f^{(k)}(a)}{k!}(x-a)^k
  \]
  La fonction $g$, continue sur l'intervalle compris entre $x$ et $a$, vérifie~:
  \[
    g(x) = f(x) - p_n(x) - \frac{f(x)-p_n(x)}{(x-a)^{n+1}}(x-a)^{n+1} = f(x) - p_n(x) - f(x) + p_n(x) = 0
  \]
  et
  \[
    g(a) = f(a) - p_n(a) = 0
  \]
  par construction du polynôme $p_n(x)$. On peut donc appliquer le théorème de Rolle~: on sait qu'il existe $c_1$ compris entre $x$ et $a$ tel que $g'(c_1)=0$. La dérivée de $g$ est~:
  \[
    g'(\lambda) = f'(\lambda) - p_n'(\lambda) - (n+1)\frac{f(x)-p_n(x)}{(x-a)^{n+1}}(\lambda-a)^{n}
  \]
  et on observe que l'on a encore $g'(a)=0$ (par construction de $p_n(x)$ en identifiant ses dérivées en $a$ avec celles de $f$ en $a$) et bien sûr $g'(c_1)=0$, puisque la fonction est continue, on peut à nouveau appliquer le théorème de Rolle. On sait donc qu'il existe $c_2$ entre $c_1$ et $a$ tel que $g''(c_2) = 0$, avec~:
  \[
    g''(\lambda) = f''(\lambda) - p_n''(\lambda) - (n+1)n\frac{f(x)-p_n(x)}{(x-a)^{n+1}}(\lambda-a)^{n-1}
  \]
  En notant que $g''(a)=0$, on sait qu'il existe $c_3$ compris entre $a$ et $c_2$ tel que $g'''(c_3) = 0$ par le théorème de Rolle, avec~:
  \[
    g'''(\lambda) = f'''(\lambda) - p_n'''(\lambda) - (n+1)n(n-1)\frac{f(x)-p_n(x)}{(x-a)^{n+1}}(\lambda-a)^{n-2}
  \]
  On peut continuer ainsi jusqu'à la dérivée $n$-ième de $g$, et conclure qu'il existe $c$ tel que~:
  \[
    g^{(n+1)}(c) = 0 = f^{(n+1)}(c) - p_n^{(n+1)}(c) - (n+1)!\frac{f(x)-p_n(x)}{(x-a)^{n+1}}
  \]
  Comme $p_n(x)$ est une fonction polynomiale de degré $n$, on sait que $p_n^{(n+1)}(x) = 0$ pour tout $x$, on a donc~:
  \[
    f^{(n+1)}(c) - (n+1)!\frac{f(x)-p_n(x)}{(x-a)^{n+1}} = 0
  \]
  soit encore~:
  \[
    \underbrace{f(x)-p_n(x)}_{R_n(x)} = \frac{f^{(n+1)}(c)}{(n+1)!}\qed
  \]
\end{notes}


\begin{frame}
  \frametitle{Approximation polynomiale en un point}
  \framesubtitle{Série de Taylor}

  \bigskip

  \begin{itemize}

  \item Si la fonction $f$ est de classe $\mathcal C^{\infty}$, on peut aller plus loin\ldots\newline

  \item Une \textbf{série entière} en $x$ est une série de terme général $c_nx^n$, on note~: $\sum_n c_nx^n$.\newline

  \item La série converge si la somme $\sum_{n=0}^{\infty} c_nx^n$ est définie. Le \textbf{radius de convergence} $r>0$ est un réel tel que la série converge pour tout $x$ tel que $|x|<r$.\newline

  \item Soit $f:\, E\rightarrow \mathbb R$ une fonction indéfiniment dérivable en $x_0\in E$, la série de Taylor associée à $f$ est~:
    \[
      \sum_{n=0}^{\infty} \frac{f^{(n)}(a)}{n!}(x-a)^n
    \]
    On peut montrer, sous certaines conditions, que si la série converge alors elle converge vers $f(x)$.

  \end{itemize}

\end{frame}


\begin{frame}
  \frametitle{Approximation polynomiale en un point}
  \framesubtitle{Application~: convexité et tangentes}

  \bigskip

  \begin{itemize}

  \item La courbe représentative d'une fonction convexe (concave) est au dessus (dessous) de toutes ses tangentes.\newline

  \item Pour le comprendre, considérons un développement de Taylor à l'ordre 2 d'une fonction $f$ dans un voisinage de $a$:
    \[
      f(x) = f(a) +  f'(a)(x-a) + \frac{1}{2}f''(a)(x-a)^2 + (x-a)^2\epsilon(x)
    \]

  \item En considérant des valeurs de $x$ suffisament proches de $a$, on peut ommettre le reste~:
    \[
      f(x) \approx_a f(a) +  f'(a)(x-a) + \frac{1}{2}f''(a)(x-a)^2
    \]

  \item Si la fonction est convexe le dernier terme est positif, et donc~:
    \[
      f(x) \gtrapprox_a f(a) +  f'(a)(x-a)
    \]
    où sur la droite nous reconnaissons l'équation de la tangente à $f$ au point $a$.

  \end{itemize}

\end{frame}


\section{Règle de l'Hôpital}


\begin{frame}
  \frametitle{Règle de l'Hôpital, I}

  \begin{itemize}

  \item Quand on s'intéresse à la limite d'une fonction de la forme
    $\frac{f(x)}{g(x)}$ lorsque $x\rightarrow a$, il arrive que l'on
    soit confronté à une forme indéterminée de type $0/0$ ou
    $\infty/\infty$.\newline

  \item La règle de l'Hôpital peut nous tirer de l'embarras.\newline

  \end{itemize}

  \begin{block}{Règle de l'Hôpital}
    Soient $f$ et $g$ deux fonctions dérivables (on suppose que $g'$ ne s'annule pas). Alors si $\lim_{x\rightarrow a}f(x) = 0$ et $\lim_{x\rightarrow a}g(x) = 0$ (ou si les limites de $f$ et $g$ ne sont pas finies) on a~:
    \[
      \lim_{x\rightarrow a} \frac{f(x)}{g(x)} = \lim_{x\rightarrow a} \frac{f'(x)}{g'(x)}
    \]
  \end{block}

  \medskip

  \begin{itemize}

  \item On éventuellement appliquer cette règle plusieurs fois (si les dérivées d'ordre supérieurs existent).

  \end{itemize}

\end{frame}


\begin{notes}
  Nous ne montrerons pas la validité de la règle de l'Hôpital ici (il faut utiliser une généralisation du théorème de Rolle que nous n'avons pas présentée). On peut néanmoins, au moins dans le cas des formes indéterminées de type $0/0$, donner l'intuition et voir que cette règle est directement liée à la définiton même de la dérivée.\newline

  Soient $f$ et $g$ deux fonctions dérivables telles que $\lim_{x\rightarrow a}f(x) = 0$ et $\lim_{x\rightarrow a}g(x) = 0$. Nous avons donc~:
  \[
    \frac{f(x)}{g(x)} = \frac{f(x)-f(a)}{g(x)-g(a)}
  \]
  Puisque les deux fonctions sont dérivables, elles sont aussi continues et donc $f(a)=g(a)=0$. En divisant le numérateur et dénominateur par $x-a$ il vient~:
  \[
    \frac{f(x)}{g(x)} = \frac{\frac{f(x)-f(a)}{x-a}}{\frac{g(x)-g(a)}{x-a}}
  \]
  On reconnaît au numérateur et dénominateur les taux de variation de $f$ et $g$, ce qui explique pourquoi on peut éventuellement lever l'indétermination en considérant la limite du ratio des dérivées.
  \[
    \begin{split}
      \lim_{x\rightarrow a}\frac{f(x)}{g(x)} &= \lim_{x\rightarrow a}\frac{\frac{f(x)-f(a)}{x-a}}{\frac{g(x)-g(a)}{x-a}}\\
      &= \frac{\lim_{x\rightarrow a}\frac{f(x)-f(a)}{x-a}}{\lim_{x\rightarrow a}\frac{g(x)-g(a)}{x-a}}\\
      &= \frac{f'(a)}{g'(a)}\\
      &= \lim_{x\rightarrow a}\frac{f'(x)}{g'(x)}
    \end{split}
  \]

\end{notes}


\begin{frame}
  \frametitle{Règle de l'Hôpital, II}

  \begin{example}
    Soit la fonction~:
    \[
      f(x) = \frac{6x^2}{3e^x-x^3-3x-3}
    \]
    On cherche la limite quand $x$ tend vers 0, et nous sommes
    confrontés à une forme indéterminée de type $0/0$. Par la règle de l'Hôpital, nous avons~:
    \[
      \begin{split}
        \lim_{x\rightarrow 0} f(x) &= \lim_{x\rightarrow 0} \frac{12x}{3e^x-3x^2-3} \quad \text{Il s'agit encore d'une forme intéterminée.}\\
        &= \lim_{x\rightarrow 0} \frac{12}{3e^x-6x} = \frac{12}{\lim_{x\rightarrow 0} 3e^x-6x} = 4
      \end{split}
    \]
  \end{example}

\end{frame}


\begin{frame}
  \frametitle{Règle de l'Hôpital, III}

  \begin{example}
    Soit la fonction~:
    \[
      f(x) = \frac{\log x}{5x}
    \]
    On cherche la limite quand $x$ tend vers l'infini, et nous sommes
    confrontés à une forme indéterminée de type $\infty/\infty$. Par la règle de l'Hôpital, nous avons~:
    \[
      \begin{split}
        \lim_{x\rightarrow \infty} f(x) &= \lim_{x\rightarrow \infty} \frac{\frac{1}{x}}{5}\\
        &= \frac{1}{5}\lim_{x\rightarrow \infty} \frac{1}{x} = 0
      \end{split}
    \]
  \end{example}

\end{frame}


\begin{frame}
  \frametitle{Règle de l'Hôpital, IV}

  \begin{example}
    Soit la fonction~:
    \[
      f(x) = \frac{2}{x}-\frac{2}{e^x-1}
    \]
    On cherche la limite quand $x$ tend vers 0, et nous sommes
    confrontés à une forme indéterminée de type $\infty-\infty$. Posons $u(x) = \frac{2}{x}$ et $v(x)=\frac{2}{e^x-1}$, on a~:
    \[
      \begin{split}
        \lim_{x\rightarrow 0} f(x) &= \lim_{x\rightarrow 0} \frac{\frac{1}{g(x)}-\frac{1}{f(x)}}{\frac{1}{g(x)f(x)}}\\
        &= \lim_{x\rightarrow 0} \frac{e^x-x-1}{\frac{x}{2}\left(e^x-1\right)}\quad \text{une forme indéterminée de type $0/0$.}\\
        &= \lim_{x\rightarrow 0} \frac{e^x-1}{\frac{1}{2}\left(xe^x+e^x-1\right)}\\
        &= \lim_{x\rightarrow 0} \frac{e^x}{\frac{1}{2}\left(xe^x+e^x+e^x\right)} = 1\\
      \end{split}
    \]
  \end{example}

\end{frame}


\end{document}

% Local Variables:
% ispell-check-comments: exclusive
% ispell-local-dictionary: "francais"
% TeX-master: t
% End: