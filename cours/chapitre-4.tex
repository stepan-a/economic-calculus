\synctex=1

\documentclass[10pt,notheorems]{beamer}

\usepackage{etex}
\usepackage{fourier-orns}
\usepackage{ccicons}
\usepackage{amssymb}
\usepackage{amstext}
\usepackage{amsbsy}
\usepackage{amsopn}
\usepackage{amscd}
\usepackage{amsxtra}
\usepackage{amsthm}
\usepackage{multirow}
\usepackage{float}
\usepackage{color, colortbl}
\usepackage{mathrsfs}
\usepackage{bm}
\usepackage{lastpage}
\usepackage[nice]{nicefrac}
\usepackage{setspace}
\usepackage{ragged2e}
\usepackage{listings}
\usepackage{polynom}
\usepackage{algorithms/algorithm}
\usepackage{algorithms/algorithmic}
\usepackage[frenchb]{babel}
\usepackage{tikz,pgfplots}
\pgfplotsset{compat=newest}
\usetikzlibrary{patterns, arrows, decorations.pathreplacing, decorations.markings, calc}
\pgfplotsset{plot coordinates/math parser=false}
\newlength\figureheight
\newlength\figurewidth
\usepackage[utf8x]{inputenc}
\usepackage{cancel}
\usepackage{tikz-qtree}
\usepackage{dcolumn}
\usepackage{adjustbox}
\usepackage{environ}
\usepackage[cal=boondox]{mathalfa}
\usepackage{manfnt}
\usepackage{hyperref}
\hypersetup{
  colorlinks=true,
  linkcolor=blue,
  filecolor=black,
  urlcolor=black,
}
\usepackage{venndiagram}
\usepackage{minted}

% Git hash
\usepackage{xstring}
\usepackage{catchfile}
\immediate\write18{git rev-parse HEAD > git.hash}
\CatchFileDef{\HEAD}{git.hash}{\endlinechar=-1}
\newcommand{\gitrevision}{\StrLeft{\HEAD}{7}}

% Include pdf page by page
\usepackage{pdfpages}


\newcommand{\trace}{\mathrm{tr}}
\newcommand{\vect}{\mathrm{vec}}
\newcommand{\tracarg}[1]{\mathrm{tr}\left\{#1\right\}}
\newcommand{\vectarg}[1]{\mathrm{vec}\left(#1\right)}
\newcommand{\vecth}[1]{\mathrm{vech}\left(#1\right)}
\newcommand{\iid}[2]{\mathrm{iid}\left(#1,#2\right)}
\newcommand{\normal}[2]{\mathcal N\left(#1,#2\right)}
\newcommand{\dynare}{\href{http://www.dynare.org}{\color{blue}Dynare}}
\newcommand{\sample}{\mathcal Y_T}
\newcommand{\samplet}[1]{\mathcal Y_{#1}}
\newcommand{\slidetitle}[1]{\fancyhead[L]{\textsc{#1}}}

\newcommand{\R}{{\mathbb R}}
\newcommand{\C}{{\mathbb C}}
\newcommand{\N}{{\mathbb N}}
\newcommand{\Z}{{\mathbb Z}}
\newcommand{\binomial}[2]{\begin{pmatrix} #1 \\ #2 \end{pmatrix}}
\newcommand{\bigO}[1]{\mathcal O \left(#1\right)}
\newcommand{\red}{\color{red}}
\newcommand{\blue}{\color{blue}}

\renewcommand{\qedsymbol}{C.Q.F.D.}

\newcolumntype{d}{D{.}{.}{-1}}
\definecolor{gray}{gray}{0.9}
\newcolumntype{g}{>{\columncolor{gray}}c}

\setbeamertemplate{theorems}[numbered]

\theoremstyle{plain}
\newtheorem{theorem}{Théorème}

\theoremstyle{definition} % insert bellow all blocks you want in normal text
\newtheorem{definition}{Définition}
\newtheorem{properties}{Propriétés}
\newtheorem{lemma}{Lemme}
\newtheorem{property}[properties]{Propriété}
\newtheorem{example}{Exemple}
\newtheorem*{idea}{Éléments de preuve} % no numbered block
\newtheorem{corollary}{Corollaire}%[theorem]


\setbeamertemplate{footline}{
  {\hfill\vspace*{1pt}\href{http://creativecommons.org/licenses/by-sa/3.0/legalcode}{\ccbysa}\hspace{.1cm}
    \raisebox{-.075cm}{\href{https://git.adjemian.eu/stepan/economic-calculus}{\includegraphics[scale=.1]{../img/gitlab.png}}}\enspace
    \href{https://git.adjemian.eu/stepan/economic-calculus/-/blob/\HEAD/cours/chapitre-4.tex}{\gitrevision}\enspace\today
  }\hspace{1cm}}

\setbeamertemplate{navigation symbols}{}
\setbeamertemplate{blocks}[rounded][shadow=true]
\setbeamertemplate{caption}[numbered]

\NewEnviron{notes}{\justifying\tiny\begin{spacing}{1.0}\BODY\vfill\pagebreak\end{spacing}}

\newenvironment{exercise}[1]
{\bgroup \small\begin{block}{Ex. #1}}
  {\end{block}\egroup}

\newenvironment{defn}[1]
{\bgroup \small\begin{block}{Définition. #1}}
  {\end{block}\egroup}

\newenvironment{exemple}[1]
{\bgroup \small\begin{block}{Exemple. #1}}
  {\end{block}\egroup}

\begin{document}

\title{Calcul Économique\\\small{IV. Dérivées}}
\author[S. Adjemian]{Stéphane Adjemian}
\institute{\texttt{stephane.adjemian@univ-lemans.fr}} \date{Octobre 2020}

\begin{frame}
  \titlepage{}
\end{frame}

\begin{frame}
  \frametitle{Plan}
  \tableofcontents
\end{frame}


\section{Dérivée d'une fonction en un point}

\begin{frame}
  \frametitle{Variation d'une fonction, I}
  \hypertarget{slide_variation_1}{}

  \begin{itemize}

  \item Soit $f$ une fonction de $E$ dans $F$.\newline

  \item On veut prédire l'effet d'une variation de $x\in E$ sur son image par $f$.\newline

  \item Soit $x_0\in E$, son image par $f$ est $y_0 = f(x_0)\in F$.\newline

  \item Pertubons $x_0$ en lui rajoutant $\Delta x$. On suppose que la perturbation est « assez petite » pour que~: $x_1=x_0+\Delta x \in E$.\newline

  \item On peut donc réécrire $\Delta x = x_1-x_0$, et définir $y_1 = f(x_0+\Delta x)\in F$ la conséquence sur l'image par $f$.\newline

  \item On pose $\Delta f(x) = f(x_1)-f(x_0)$ la variation de l'image par $f$ induite par la variation $\Delta x$ de $x$.

  \end{itemize}

\end{frame}


\begin{frame}
  \frametitle{Variation d'une fonction, II}
  \hypertarget{slide_variation_2}{}


  \begin{example}
    Soit $f(x) = x^3$ une fonction de $\mathbb R$ dans $\mathbb R$. Posons comme valeur initiale de x $x_0 = 2$, son image par $f$ est $y_0 = f(x_0) = 8$. Donnons-nous une perturbation $\Delta x = 1$. On a alors $x_1=x_0+\Delta x = 3$ et son image $y_1 = 3^3 = 27$. La variation induite de l'image est donc $\Delta f(x) = y_1-y_0$, soit $\Delta f(x) = 19$. Si nous changeons la perturbation sur $x$ en posant $\Delta x = -1$, nous obtenons $x_1 = 1$, $y_1 = 1$ et donc $\Delta f(x) = -7$.\newline

    Pour ces deux exemples $\Delta f(x)$ et $\Delta x$ ont le même signe~: quand on augmente (diminue) $x$ cela induit une augmentation (baisse) de $f(x)$. Cela suggère que $f$ est croissante autour de $x=1$. Ce résultat n'est évidemment pas général.\newline

    Si maintenant $x_0 = 1$ et $\Delta x = 1$, on a $x_1 = 2$, $y_0 = 1$, $y_1 = 8$ et donc $\Delta f(x) = 7$.    On note que l'amplitude de l'effet sur l'image n'est pas constant (dépend du niveau initial $x_0$)\newline

  \end{example}

\end{frame}


\begin{frame}
  \frametitle{Variation d'une fonction, III}
  \hypertarget{slide_variation_3}{}

  \begin{example}

    Soit la fonction de $\mathbb R$ dans $\mathbb R$~: $f(x) = ax+b$, une droite, où $a$ et $b$ sont des paramètres réels. Si on perturbe $x_0$ avec $\Delta x$, on a~:
    \[
      y_1 = a (\underbrace{x_0+\Delta x}_{x_1}) + b
    \]
    sachant que $y_0 = a x_0 + b$, on a donc~:
    \[
      \begin{split}
        \Delta f(x) &= y_1-y_0\\
        &= a(x_1-x_0)\\
        &= a \Delta x
      \end{split}
    \]
    Si on normalise la variation induite $\Delta f(x)$ par la variation $\Delta$, on obtient~:
    \[
      \frac{\Delta f(x)}{\Delta x} = a \quad \forall x_0\in\mathbb R
    \]
    Pour une droite, le \textbf{taux de variation} ne dépend pas de $x_0$.

  \end{example}

\end{frame}


\begin{frame}
  \frametitle{Variation d'une fonction, IV}
  \hypertarget{slide_variation_4}{}

  \begin{columns}[onlytextwidth]
    \begin{column}{.5\textwidth}
      \begin{itemize}
      \item Plus généralement~:
        \[
          f(x) + \Delta f(x) = f(x+\Delta x)
        \]
      \item Soit de façon équivalente~:
        \[
          \Delta f(x) = f(x+\Delta x)-f(x)
        \]
      \item En divisant par $\Delta x\neq 0$, on  obtient le taux de variation~:
        \[
          \frac{\Delta f(x)}{\Delta x} = \frac{f(x+\Delta x)-f(x)}{\Delta x}
        \]
        qui correspond à la pente de l'arc passant par les points $A_0$ et $A_1$ sur le graphique.
      \end{itemize}
    \end{column}
    \begin{column}{.5\textwidth}
      \begin{tikzpicture}[scale=1.05]
        \begin{axis}[
          title={},
          xlabel= {},
          ylabel= {},
          xticklabels={,,},
          yticklabels={,,},
          enlargelimits=true,
          grid style={dashed, gray!60},
          axis x line = bottom,
          axis y line = left,
          axis lines = middle,
          axis line style={thin},
          xmin = -.1,
          xmax = 3,
          ymin = -.1,
          ymax = 5,
          small,
          clip=false,
          ]
          \addplot[
          draw=black,
          thick,
          smooth,
          samples=500,
          domain=0:2.1,
          ]
          {x^2} node[right] {\tiny $f(x)$} ;
          \addplot[
          draw=red,
          domain=.2:2.5
          ]
          {3*x-2};
          \node[draw=red, circle, fill=red, scale=.2] at (1,1) {};
          \node[draw=red, circle, fill=red, scale=.2] at (2,4) {};
          \node[below] at (1,0) {\tiny{\color{red}$x_0$}};
          \node[below] at (2,0) {\tiny{\color{red}$x_1 = x_0+\Delta x$}};
          \node[left] at (0,1) {\tiny{\color{red}$f(x_0)$}};
          \node[left] at (0,4) {\tiny{\color{red}$f(x_0+\Delta x)$}};
          \node[above] at (1,1.05) {\tiny\color{red} $A_0$};
          \node[right] at (2.05,4) {\tiny\color{red} $A_1$};
          \addplot[draw=red, dotted] coordinates {(2,4) (0,4)};
          \addplot[draw=red, dotted] coordinates {(2,4) (2,0)};
          \addplot[draw=red, dotted] coordinates {(1,1) (0,1)};
          \addplot[draw=red, dotted] coordinates {(1,1) (1,0)};
        \end{axis}
      \end{tikzpicture}
    \end{column}
  \end{columns}

\end{frame}


\begin{frame}
  \frametitle{Variation d'une fonction, V}
  \hypertarget{slide_variation_5}{}

  \begin{columns}[onlytextwidth]
    \begin{column}{.5\textwidth}
      \begin{itemize}

      \item Quel est le taux de variation pour $\Delta x = 0$~? On a une forme indéterminée de type 0/0.\newline

      \item Si la limite existe, la dérivée de $f$ en $x_0$ est la limite du taux de variation quand $\Delta x$ tend vers 0~:
        \[
          \frac{\mathrm d}{\mathrm dx}f(x_0) = \lim_{\Delta x\rightarrow 0}\frac{f(x+\Delta x)-f(x)}{\Delta x}
        \]
        \medskip
      \item On a remplacé $\Delta x$ par $\mathrm dx$ $\rightarrow$ variations infinitésimales.\newline

      \item Si la dérivée en $x_0$ existe on la notera aussi $f'(x_0)$.\newline
      \end{itemize}
    \end{column}
    \begin{column}{.5\textwidth}
      \begin{center}
        \begin{tikzpicture}[scale=1]
          \begin{axis}[
            title={},
            xlabel= {},
            ylabel= {},
            xticklabels={,,},
            yticklabels={,,},
            enlargelimits=true,
            grid style={dashed, gray!60},
            axis x line = bottom,
            axis y line = left,
            axis lines = middle,
            axis line style={thin},
            xmin = -.1,
            xmax = 3,
            ymin = -.1,
            ymax = 5,
            small,
            clip=false,
            ]
            \addplot[
            draw=black,
            thick,
            smooth,
            samples=500,
            domain=0:2.1,
            ]
            {x^2} node[right] {\tiny $f(x)$} ;
            \addplot[
            draw=red,
            domain=.2:2.5
            ]
            {2*x-1};
            \node[draw=red, circle, fill=red, scale=.2] at (1,1) {};
            \node[below] at (1,0) {\tiny{\color{red}$x_0$}};
            \node[left] at (0,1) {\tiny{\color{red}$f(x_0)$}};
            \node[above] at (1,1.05) {\tiny\color{red} $A_0$};
            \addplot[draw=red, dotted] coordinates {(1,1) (0,1)};
            \addplot[draw=red, dotted] coordinates {(1,1) (1,0)};
          \end{axis}
        \end{tikzpicture}
      \end{center}
    \end{column}
  \end{columns}

\end{frame}


\begin{frame}
  \frametitle{Dérivée en un point, I}
  \hypertarget{slide_derivee_1}{}

  \begin{definition}
    Soit $f$ une fonction de $E$ dans $\mathbb R$. On note $f'(x_0)$ la dérivée de $f$ au point $x_0\in E$, celle-ci est définie par~:
    \[
      f'(x_0) = \lim_{h\rightarrow 0} \frac{f(x_0+h)-f(x_0)}{h}
    \]
  \end{definition}

  \bigskip

  \begin{itemize}

  \item La dérivée est définie par une limite.\newline

  \item Pour que la dérivée existe, il faut que la limite soit définie.\newline

  \item La dérivée de la fonction $f$ en un point $x_0$ est la pente de la tangente à la courbe représentative de $f$ en $x_0$.
  \end{itemize}

\end{frame}


\begin{frame}
  \frametitle{Dérivée en un point, II}
  \hypertarget{slide_derivee_2}{}

  \begin{center}
    \begin{tikzpicture}[scale=1.6]
      \begin{axis}[
        title={},
        xlabel= {},
        ylabel= {},
        xticklabels={,,},
        yticklabels={,,},
        enlargelimits=true,
        grid style={dashed, gray!60},
        axis x line = bottom,
        axis y line = left,
        axis lines = middle,
        axis line style={thin},
        xmin = -.1,
        xmax = 1.7,
        ymin = -1.5,
        ymax = 3,
        small,
        clip=false,
        ]
        \addplot[
        draw=black,
        thick,
        smooth,
        samples=500,
        domain=0:1.5,
        ]
        {x^2} node[right] {\tiny $f(x)$} ;
        \addplot[
        draw=red,
        domain=-.3:1.6
        ]
        {2*x-1};
        \node[draw=red, circle, fill=red, scale=.2] at (1,1) {};
        \node[below] at (1,0) {\tiny{\color{red}$x_0$}};
        \node[left] at (0,1) {\tiny{\color{red}$f(x_0)$}};
        \addplot[draw=red, dotted] coordinates {(1,1) (0,1)};
        \addplot[draw=red, dotted] coordinates {(1,1) (1,0)};
        \node[draw=red, circle, fill=red, scale=.2] at (0,-1) {};
        \node[left] at (-0.05,-1) {\tiny{\color{red}$f(x_0)-f'(x_0)x_0$}};
        \node[draw=red, circle, fill=red, scale=.2] at (0.5,0) {};
        \node[right] at (.4,-.275) {\tiny{\color{red}$x_0-\frac{f(x_0)}{f'(x_0)}$}};
        \node[right] at (.1,-1.3) {\tiny{\color{red}Équation de la tangente: $y = f(x_0)+f'(x_0)(x-x_0)$}};
      \end{axis}
    \end{tikzpicture}
  \end{center}

\end{frame}


\begin{notes}
  Pour déterminer l'équation de la tangente on exploite deux informations~:\newline
  \begin{itemize}

  \item La pente de la tangente est $f'(x_0)$,\newline

  \item La tangente passe par le point $(x_0, f(x_0))$.\newline

  \end{itemize}

  L'équation de la tangente est donc de la forme~:
  \[
    y = f'(x_0)x + b
  \]
  Il ne reste plus qu'à choisir $b$ pour s'assurer que la tangente passe bien par le point $(x_0, f(x_0))$. On a~:
  \[
    f(x_0) = f'(x_0)x_0 + b
  \]
  soit de façon équivalente~:
  \[
    b = f(x_0) - f'(x_0)x_0
  \]
  et donc~:
  \[
    y = f'(x_0) x + f(x_0) - f'(x_0)x_0
  \]
  ou encore en factorisant~:
  \[
    y = f(x_0) + f'(x_0)(x-x_0)\hfil\qed
  \]
\end{notes}


\begin{frame}
  \frametitle{Dérivée en un point, III}
  \hypertarget{slide_derivee_3}{}

  \begin{itemize}

  \item La dérivée est définie par une limite.\newline

  \item On peut donc définir une dérivée à droite et une dérivée à gauche\newline

  \item Si les dérivées à droite et à gauche en $x_0$ sont différentes, on dit que la fonction n'est pas dérivable en $x_0$ (comme pour l'existence de la limite).\newline

  \end{itemize}

  \begin{definition}
    Soit $f$ une fonction de $E$ dans $\mathbb R$. On note $f'_-(x_0)$ et $f'_+(x_0)$  les dérivées à gauche et à droite de $f$ au point $x_0\in E$, celle-ci sont définies par~:
    \[
      f'_-(x_0) = \lim_{h\rightarrow 0^-} \frac{f(x_0+h)-f(x_0)}{h}
    \]
    et
    \[
      f'_+(x_0) = \lim_{h\rightarrow 0^+} \frac{f(x_0+h)-f(x_0)}{h}
    \]
  \end{definition}

\end{frame}


\begin{frame}
  \frametitle{Dérivée en un point, IV}
  \hypertarget{slide_derivee_4}{}

  \begin{example}

    Soit $f(x) = x^2$ une fonction de $\mathbb R$ dans $\mathbb R_+$. Calculons la dérivée en $x=1$. Nous avons~:
    \[
      \begin{split}
        f'(1) &= \lim_{h\rightarrow 0}\frac{f(1+h)-f(1)}{h}\\
        &= \lim_{h\rightarrow 0}\frac{(1+h)^2-1}{h}\\
        &= \lim_{h\rightarrow 0}\frac{h^2+2h+1-1}{h}\\
        &= \lim_{h\rightarrow 0}\frac{h^2+2h}{h}\\
        &= \lim_{h\rightarrow 0}h+2\\
        &= 2
      \end{split}
    \]
  \end{example}

\end{frame}


\begin{frame}
  \frametitle{Dérivée en un point, V}
  \hypertarget{slide_derivee_5}{}

  \begin{itemize}

  \item Pour qu'une fonction soit dérivable en un point, il faut que la fonction soit définie en ce point.\newline

  \item Si une fonction n'est pas définie en un point $x_0$, alors la fonction n'admet pas de dérivée en ce point.\newline

  \item[\dbend] Ce n'est pas parce qu'une fonction est définie en un point que la dérivée en ce point existe.\newline

  \end{itemize}

  \begin{theorem}\label{thm:derivable-continue}
    Soit $f$ une fonction de $E$ dans $\mathbb R$. Si la fonction $f$ est dérivable en $x_0\in E$, alors la fonction $f$ est continue en $x_0$.
  \end{theorem}

  \bigskip

  \begin{itemize}

  \item[\dbend] La réciproque n'est pas vraie (voir l'exemple suivant)~: une fonction continue n'est pas nécessairement dérivable.

  \end{itemize}

\end{frame}


\begin{notes}
  \textbf{Preuve du théorème
    \hyperlink{slide_derivee_5}{\ref{thm:derivable-continue}}.}
  Puisque la fonction est supposée dérivable en $x_0\in E$, nous avons
  par définition de la dérivée~:
  \[
    f'(x_0) = \lim_{h\rightarrow 0} \frac{f(x_0+h)-f(x_0)}{h}
  \]
  Posons~:
  \[
    g(h) =  \frac{f(x_0+h)-f(x_0)}{h}
  \]
  Nous avons donc~:
  \[
    h g(h) = f(x_0+h)-f(x_0)
  \]
  avec $\lim_{h\rightarrow 0}g(h) = f'(x_0)$. Ainsi~:
  \[
    \lim_{h\rightarrow 0} hg(h) = \lim_{h\rightarrow 0}f(x_0+h) - f(x_0)
  \]
  \[
    \Leftrightarrow f'(x_0)\lim_{h\rightarrow 0} h = \lim_{h\rightarrow 0}f(x_0+h) - f(x_0)
  \]
  \[
    \Leftrightarrow 0 = \lim_{h\rightarrow 0}f(x_0+h) - f(x_0)
  \]
  \[
    \Leftrightarrow 0 = \lim_{x\rightarrow x_0}f(x) - f(x_0)
  \]
  d'où finalement~:
  \[
    \lim_{x\rightarrow x_0}f(x) = f(x_0)
  \]
  La fonction est donc bien continue en $x_0$.

\end{notes}


\begin{frame}
  \frametitle{Dérivée en un point, VI}
  \hypertarget{slide_derivee_6}{}

  \begin{example}
    Soit $f(x) = |x|$ une fonction défine sur $\mathbb R$ à valeurs dans $\mathbb R^+$. Cette fonction est continue en $x=0$ mais n'est pas dérivable.\newline

    \begin{columns}[onlytextwidth]
      \begin{column}{.5\textwidth}
        {\small
          \begin{itemize}
          \item $f'_-(0)=\lim_{h\rightarrow 0^-} \frac{|0+h|-0}{h} = -1$, en effet on a~:
            \[
              \begin{split}
                f'_-(0) &= \lim_{h\rightarrow 0^-} \frac{|h|}{h}\\
                &= \lim_{h\rightarrow 0^-} \frac{-h}{h}\\
                &= -\lim_{h\rightarrow 0^-} 1\\
                &= -1\\
              \end{split}
            \]
          \item Mais $f'_+(0) = 1$

          \item Les dérivées à droite et à gauche sont différentes, la fonction n'est donc pas dérivable en $x=0$.
          \end{itemize}}
      \end{column}
      \begin{column}{.5\textwidth}
        \begin{tikzpicture}[scale=1]
          \begin{axis}[
            xticklabels={,,},
            yticklabels={,,},
            enlargelimits=true,
            grid style={dashed, gray!60},
            axis x line = bottom,
            axis y line = left,
            axis line style={thin},
            xmax = 5,
            xmin = -5,
            ymax = 5.5,
            ymin = -0.5,
            axis lines = middle,
            small,
            clip=false,
            ]
            \addplot[
            draw=black,
            thick,
            domain=-5:0,
            ]
            {-x};
            \addplot[
            draw=black,
            thick,
            domain=0:5,
            ]
            {x};
          \end{axis}
        \end{tikzpicture}
      \end{column}
    \end{columns}
  \end{example}

\end{frame}


\begin{frame}
  \frametitle{Dérivée sur un intervalle}
  \hypertarget{slide_derivee_7}{}


  \begin{definition}\label{dfn:derivable-intervalle}
    Soit $f$ une fonction de $E$ dans $\mathbb R$. Si la fonction $f$ est dérivable sur $E$ si ell est dérivable en tout point  $x\in E$.
  \end{definition}

  \bigskip

  \begin{example}
    La fonction $f(x) = x^2$ de $\mathbb R$ dans $\mathbb R_+$ est dérivable sur $\mathbb R$. En effet, pour tout $x\in\mathbb R$ nous avons~:
    \[
      \begin{split}
        f'(x) &= \lim_{h\rightarrow 0}\frac{(x+h)^2-x^2}{h}\\
        &= \lim_{h\rightarrow 0}\frac{x^2+2xh+h^2-x^2}{h} = \lim_{h\rightarrow 0}\frac{2xh+h^2}{h}\\
        &= \lim_{h\rightarrow 0} 2x+h = 2x + \lim_{h\rightarrow 0} h\\
        &= 2x
      \end{split}
    \]

  \end{example}

\end{frame}


\section{Règles de dérivation}


\begin{frame}
  \frametitle{Règles de dérivation}
  \framesubtitle{Dérivée d'une somme}
  \hypertarget{slide_derivee_somme_1}{}

  \begin{theorem}
    Soient $f(x)$ et $g(x)$ deux fonctions de $E$ dans $F$ dérivables sur $E$, alors~:
    \[
      (f(x)+g(x))' = f'(x) + g'(x)
    \]
  \end{theorem}

  \bigskip

  {\small \textbf{Preuve.} Notons $\ell(x) = f(x)+g(x)$, nous avons par définition~:
    \[
      \begin{split}
        \ell'(x) &= \lim_{h\rightarrow 0} \frac{\ell(x+h)-\ell(x)}{h}\\
        &= \lim_{h\rightarrow 0} \frac{f(x+h)+g(x+h)-f(x)-g(x)}{h}\\
        &= \lim_{h\rightarrow 0} \frac{f(x+h)-f(x)}{h} + \lim_{h\rightarrow 0} \frac{g(x+h)-g(x)}{h}\\
        &= f'(x)+g'(x) \qed
      \end{split}
    \]
  }

\end{frame}


\begin{frame}
  \frametitle{Règles de dérivation}
  \framesubtitle{Dérivée d'un produit (a)}
  \hypertarget{slide_derivee_produit_1}{}

  \begin{theorem}
    Soient $f(x)$ et $g(x)$ deux fonctions de $E$ dans $F$ dérivables sur $E$, alors~:
    \[
      (f(x)\cdot g(x))' = f'(x)g(x) + f(x)g'(x)
    \]
  \end{theorem}

  \bigskip

  {\small \textbf{Preuve.} Notons $\ell(x) = f(x) \cdot g(x)$, nous avons par définition~:
    \[
      \begin{split}
        \ell'(x) &= \lim_{h\rightarrow 0} \frac{\ell(x+h)-\ell(x)}{h}\\
        &= \lim_{h\rightarrow 0} \frac{f(x+h)g(x+h)-f(x)g(x)}{h} \\
        &= \lim_{h\rightarrow 0} \frac{(f(x+h)-f(x))g(x+h)+f(x)g(x+h)-f(x)g(x)}{h}\\
        &= \lim_{h\rightarrow 0} \frac{(f(x+h)-f(x))g(x+h)+f(x)(g(x+h)-g(x))}{h}\\
      \end{split}
    \]
  }

\end{frame}


\begin{frame}
  \frametitle{Règles de dérivation}
  \framesubtitle{Dérivée d'un produit (b)}
  \hypertarget{slide_derivee_produit_2}{}

  {\small On a donc~:
    \[
      \begin{split}
        \ell'(x) &= \lim_{h\rightarrow 0} \frac{(f(x+h)-f(x))g(x+h)}{h} + \lim_{h\rightarrow 0} \frac{f(x)(g(x+h)-g(x))}{h}\\
        &= \lim_{h\rightarrow 0} \frac{f(x+h)-f(x)}{h}\lim_{h\rightarrow 0} g(x+h) + f(x)\lim_{h\rightarrow 0} \frac{g(x+h)-g(x)}{h}\\
        &= \lim_{h\rightarrow 0} \frac{f(x+h)-f(x)}{h}g(x) + f(x)\lim_{h\rightarrow 0} \frac{g(x+h)-g(x)}{h}\\
        &= f'(x)g(x) + f(x)g'(x) \qed
      \end{split}
    \]
  }

\end{frame}


\begin{frame}
  \frametitle{Règles de dérivation}
  \framesubtitle{Dérivée d'un quotient}
  \hypertarget{slide_derivee_quotient_1}{}

  \begin{theorem}
    Soient $f(x)$ et $g(x)$ deux fonctions de $E$ dans $F$ dérivables sur $E$, avec $g(x)\neq 0\,\forall x\in E$, alors~:
    \[
      \left(\frac{f(x)}{g(x)}\right)' = \frac{f'(x)g(x) - f(x)g'(x)}{g(x)^2}
    \]
  \end{theorem}

  \bigskip

  {\small \textbf{Preuve.} Notons $\ell(x) = \frac{f(x)}{g(x)}$, nous avons de façon équivalente~:
    \[
      \ell(x)g(x) = f(x)
    \]
    en dérivant, et en exploitant la règle de dérivation d'un produit, il vient~:
    \[
      \ell'(x)g(x)+\ell(x)g'(x) = f'(x)
    \]
    soit~:
    \[
      \ell'(x) = \frac{f'(x)-\ell(x)g'(x)}{g(x)} = \frac{f'(x)g(x)-f(x)g'(x)}{g(x)^2} \qed
    \]
  }

\end{frame}


\begin{frame}
  \frametitle{Règles de dérivation}
  \framesubtitle{Dérivée d'une composition}
  \hypertarget{slide_derivee_composition_1}{}

  \begin{theorem}\label{thm:composition}
    Soient $f(x)$ une fonction dérivable de $E$ dans $F$ et $g(x)$ une fonction dérivable de $F$ dans $G$, alors~:
    \[
      g(f(x))' = g'(f(x))f'(x)
    \]
  \end{theorem}

  \bigskip

  \begin{example}
    Soit $f(x)$ une fonction de $\mathbb R$ dans $\mathbb R$ dérivable. Alors $g(x) = f(x)^2$ est une fonction dérivable sur $\mathbb R$ et~:
    \[
      g'(x) = 2f(x)f'(x)
    \]
  \end{example}

\end{frame}


\begin{notes}
  \textbf{Preuve du théorème \hyperlink{slide_derivee_composition_1}{\ref{thm:composition}}.} Commençons par noter que, par définition de la dérivée, si les fonctions $f$ et $g$ sont dérivables, alors on peut écrire~:
  \[
    f(x+h) =_0 f(x) + hf'(x) + \varepsilon(h)
  \]
  et
  \[
    g(y+k) =_0 g(y) + kg'(y) + \nu(k)
  \]
  avec $\lim_{h\rightarrow 0}\varepsilon(h)=0$ et $\lim_{k\rightarrow 0}\nu(k)=0$. Il s'agit d'un développement limité à l'ordre 1 des fonctions $f$ et $g$. La première équation nous dit que pour de petites valeurs de $h$ on peut approximer $f(x+h)$ par $f(x) + hf'(x)$, l'équation de la tangente à $f$ au point $x$, l'erreur d'approximation $\varepsilon(h)$ tend vers 0 quand $h$ tend vers 0. On a donc~:
  \[
    g\circ f(x+h) = g\Bigl(\underbrace{f(x)}_{y}+\underbrace{hf'(x) + \varepsilon(h)}_{k(h)}\Bigr)
  \]
  avec $\lim_{h\rightarrow 0}k(h) = 0$. En exploitant le dévelopement limité de $g$, on obtient~:
  \[
    g\circ f(x+h) = g(f(x)) + \Bigl( hf'(x) + \varepsilon(h) \Bigr) g'(f(x)) + \nu\Bigl( hf'(x) + \varepsilon(h) \Bigr)
  \]
  \[
    \Leftrightarrow g\circ f(x+h) = g(f(x)) + h g'(f(x))f'(x) + \underbrace{\nu\Bigl( hf'(x) + \varepsilon(h) \Bigr)+\varepsilon(h)g'(f(x))}_{\eta(h)}
  \]
  avec $\lim_{h\rightarrow 0}\eta(h) = 0$. On a donc~:
  \[
    g\circ f(x+h) = g(f(x)) + h g'(f(x))f'(x) + \eta(h)
  \]
  et par identification~:
  \[
    g(f(x))' = g'(f(x))f'(x)
  \]

\end{notes}


\begin{frame}
  \frametitle{Règles de dérivation}
  \framesubtitle{Dérivée d'une fonction réciproque}
  \hypertarget{slide_derivee_reciproque_1}{}

  \begin{theorem}\label{thm:composition}
    Soit $f$ une fonction dérivable et bijective de $E$ dans $F$, alors $f^{-1}$ est dérivable en tout point $x\in J$ tel que $f'(f^{-1}(x))\neq 0$ et on a~:
    \[
      (f^{-1})'(x) = \frac{1}{f'(f^{-1}(x))}
    \]
  \end{theorem}

  \bigskip

  {\small \textbf{Preuve.} Nous savons déjà que la composition d'une fonction et de sa réciproque (celle-ci existe car la fonction $f$ est supposée bijective) est égale à la fonction identité~:
    \[
      f\Bigl(f^{-1}(x)\Bigr) = x
    \]
    en dérivant les deux membres et utilisant le théorème \hyperlink{slide_derivee_composition_1}{\ref{thm:composition}} sur la dérivation des fonctions composées, nous avons~:
    \[
      f'\Bigl(f^{-1}(x)\Bigr)(f^{-1})'(x)  = 1
    \]
    \[
      (f^{-1})'(x)  = \frac{1}{f'\Bigl(f^{-1}(x)\Bigr)}\qed
    \]
  }

\end{frame}


\section{Dérivées de fonctions usuelles}


\begin{frame}
  \frametitle{Fonctions exponentielle et logarithme, I}
  \hypertarget{slide_derivee_exp_log_1}{}

  \begin{definition}
    L'équation fonctionnelle $f'(x) = f(x)$ avec $f(0)=1$ admet une unique solution~:
    \[
      f(x) = e^x
    \]
    la fonction exponentielle.
  \end{definition}

  \bigskip

  \begin{theorem}\label{thm:log_derivee_1}
    La dérivée de $f(x) = log(x)$, pour $x\in\mathbb R_+^*$ est $f'(x) = \frac{1}{x}$.
  \end{theorem}

  \bigskip

  \begin{theorem}\label{thm:log_derivee_2}
    Soit $f$ une fonction à valeurs dans $\mathbb R_+^{\star}$, alors~:
    \[
      \Bigl(\log f(x)\Bigr)' = \frac{f'(x)}{f(x)}
    \]
  \end{theorem}

\end{frame}


\begin{notes}

  \textbf{Preuve du théorème \hyperlink{slide_derivee_exp_log_1}{\ref{thm:log_derivee_1}}.} On sait que la fonction logarithme népérien est la fonction réciproque de la fonction expolnentielle, on a donc~:
  \[
    e^{\log x} = x
  \]
  En dérivant~:
  \[
    \left(e^{\log x}\right)' = 1
  \]
  Pour dériver le l'exponentielle d'une fonction, on utilise le théorème \hyperlink{slide_derivee_composition_1}{\ref{thm:composition}} de dérivation des fonctions composées. De façon générale, on a donc si $f(x)$ est une fonction dérivable~:
  \[
    \left(e^{f(x)}\right)' = f'(x)e^{f(x)}
  \]
  puisque par définition la dérivée de la fonction exponentielle est la fonction exponentielle. Dans le cas qui nous intéresse, il vient~:
  \[
    \Bigl( \log x \Bigr)' e^{\log x} = 1
  \]
  et donc~:
  \[
    x \Bigl( \log x \Bigr)' = 1
  \]
  d'où~:
  \[
    \Bigl( \log x \Bigr)' = \frac{1}{x}
  \]

  \bigskip

  \textbf{Preuve du théorème \hyperlink{slide_derivee_exp_log_1}{\ref{thm:log_derivee_2}}.} direct avec le théorème \hyperlink{slide_derivee_composition_1}{\ref{thm:composition}}.

\end{notes}


\begin{frame}
  \frametitle{Fonctions exponentielle et logarithme, II}
  \hypertarget{slide_derivee_exp_log_1}{}

  \bigskip

  \begin{corollary}
    Si $f(x)$ est une fonction positive, alors~:
    \[
      f'(x) = f(x) \Bigl(\log f(x)\Bigr)'
    \]
  \end{corollary}

  \bigskip

  \begin{example}
    Soit la fonction $f(x) = x^x$ définie sur $\mathbb R_+^{\star}$. On a~:
    \[
      \begin{split}
        f'(x) &= x^x \Bigl(\log x^x\Bigr)'\\
        &= x^x \Bigl(x\log x\Bigr)'\\
        &= x^x \Bigl(\frac{x}{x}+\log x\Bigr)\\
        &= x^x \Bigl(1+\log x\Bigr)
      \end{split}
    \]
  \end{example}

\end{frame}


\begin{frame}
  \frametitle{Dérivées des fonctions usuelles}
  \hypertarget{slide_derivee_usuelles}{}

  \bigskip
  \renewcommand{\arraystretch}{1.8}
  \begin{table}[H]
    \centering
    {\small
      \begin{tabular}{l|c|l|c|l}
        \hline
        $D_f$ & $f(x)$ & $D_{f'}$ & $f'(x)$ & Remarques\\ \hline
        $\R$ & $k$ & $\R$ & 0 & $k\in\R$\\
        $\R$ & $kx$ & $\R$ & $k$ & $k\in\R$\\
        $\R$ & $x^n$ & $\R$ & $nx^{n-1}$ & $n\in\N$\\
        $\R^{\star}$ & $\frac{1}{x^n}$ & $\R^{\star}$ & $-\frac{n}{x^{n+1}}$ & $n\in\N$\\
        $\R_+^\star$ & $x^{\alpha}$  & $\R_+^{\star}$ &  $\alpha x^{\alpha - 1}$ & $\alpha\in\R$\\
        % </math> constante réelle. Fonction prolongeable par continuité en 0 si {{math|''α'' ≥ 0}}, et de prolongée dérivable en 0 si {{math|''α'' ≥ 1}}.
        $\R^\star$ & $\log |x|$ & $\R^\star$ & $\frac{1}{x}$ & \\
        $\R^\star$ & $\log_a |x|$ & $\R^\star$ & $\frac{1}{x \log a}$ & $a>0$ et $a\neq 1$\\
        $\R$ & $e^x$  & $\R$  $e^x$ & \\
        $\R$ & $a^x$ & $\R$ & $a^x \log a$ & $a > 0$ \\ \hline\hline
      \end{tabular}}
  \end{table}

\end{frame}


\begin{frame}
  \frametitle{Fonction puissance: $x^n$, avec $n\in\mathbb N$}

  \begin{itemize}

  \item Montrons que $\left(x^n\right)' = nx^{n-1}$ par récurrence.\newline

  \item Au rang 1, nous avons $x' = 1\times x^0 = 1$.\newline

  \item Supposons que $\left(x^n\right)' = nx^{n-1}$ et montrons que l'on doit alors avoir $\left(x^{n+1}\right)' = (n+1)x^{n}$~:
    \[
      \begin{split}
        \left(x^{n+1}\right)' &= \left(x^{n+1}\right)'\\
        &=\left(xx^{n}\right)'\\
        &=x^n + x n x^{n-1}\\
        &=x^n + n x^{n} = (n+1)x^{n}\qed
      \end{split}
    \]

  \end{itemize}
\end{frame}


\section{Dérivées d'ordre supérieur}

\begin{frame}
  \frametitle{Dérivées d'ordre supérieur}
  \hypertarget{slide_derivees_ordre_n_1}{}

  \begin{definition}
    Soit $f$ une fonction dérivable sur $E$, et telle que $f'$ est elle-même dérivable sur $E$, alors la dérivée de $f'$ est appelée \textbf{dérivée seconde} de la fonction $f$, et notée $f''$. On note de même $f'''$ la dérivée tierce de $f$ si elle existe, plus généralement on note $f^{(n)}$ la dérivée n-ième de la fonction $f$.
  \end{definition}

  \bigskip

  \begin{itemize}
  \item Par convention $f^{(0)} = f$.\newline
  \item Par construction $\left(f^{(k)}\right)' = f^{(k+1)}$.\newline
  \end{itemize}


  \begin{definition}
    Soit $f$ une fonction de $E\rightarrow F\subseteq \mathbb R$. Si $f$ est $k$ fois dérivable sur $E$, et sa dérivée $k$-ième est continue sur $E$, on dit que $f$ est de classe $\mathcal C^k$ sur $E$. Par convention, si $f$ est continue sur $E$ on dit que $f$ est de classe $\mathcal C^0$ sur $E$; si jamais $f$ est de classe $\mathcal C^k$ sur $E$ pour tout entier $k$ alors on dit que $f$ est de classe $\mathcal C^{\infty}$.
  \end{definition}

\end{frame}


\begin{frame}
  \frametitle{Dérivées d'ordre supérieur}
  \framesubtitle{Règles de dérivation}
  \hypertarget{slide_derivees_ordre_n_2}{}

  \begin{theorem}
    Soient $n\in\mathbb N$, $\alpha\in \mathbb R$ et $f$ et $g$ deux fonctions de classe $\mathcal C^n$ sur un intervalle $E$, alors~:\newline
    \begin{enumerate}
    \item $\alpha f$ est de classe $\mathcal C^n$ sur $E$ et $\left(\alpha f\right)^{(n)} = \alpha f^{(n)}$\newline
    \item $f+g$  est de classe $\mathcal C^n$ sur $E$ et $\left(f+g\right)^{(n)} = f^{(n)} + g^{(n)}$
    \end{enumerate}
  \end{theorem}

  \bigskip

  \begin{corollary}
    La dérivée $n$-ième d'une fonction polynomiale de degré inférieur à $n$ est nulle.
  \end{corollary}

\end{frame}


\begin{frame}
  \frametitle{Dérivées d'ordre supérieur}
  \framesubtitle{Fonctions usuelles}
  \hypertarget{slide_derivees_ordre_n_3}{}

  \bigskip

  On peut établir les formules suivantes par récurrence~:

  \bigskip

  \renewcommand{\arraystretch}{2}
  \begin{table}[H]
    \centering
    {\small
      \begin{tabular}{c|l|c}
        \hline
        $f(x)$ & $D_{f^{(n)}}$ & $f^{(n)}(x)$\\ \hline
        $e^x$  & $\mathbb R$ & $e^x$\\
        \multirow{2}{*}{$x^p$, $p\in\mathbb N$}  & \multirow{2}{*}{$\mathbb R$} & \multirow{2}{*}{$\begin{cases}\frac{p!}{(n-p)!}x^{b-n} & \text{ si }n\leq p\\ 0 & \text{ sinon.}\end{cases}$}\\
               & & \\
        $x^{\alpha}$, $\alpha\in\mathbb R\setminus\mathbb N$ & $\mathbb R_+^{\star}$ & $x^{\alpha-n}\prod_{i=0}^{n-1}(\alpha-i)$ \\
        $\frac{1}{a+x}$ & $\mathbb R \setminus \{-a\}$ & $\frac{(-1)^nn!}{(a+x)^{n+1}}$ \\
        $\frac{1}{a-x}$ & $\mathbb R \setminus \{a\}$ & $\frac{(-1)^nn!}{(a-x)^{n+1}}$ \\ \hline\hline
      \end{tabular}}
  \end{table}

\end{frame}


\begin{frame}
  \frametitle{Dérivées d'ordre supérieur}
  \framesubtitle{Règles de dérivation: formule de Leibniz}
  \hypertarget{slide_derivees_ordre_n_4}{}

  \begin{theorem}\label{thm:leibniz}
    Soit $n\in\mathbb N$. Soient $f$ et $g$ deux fonctions de classe
    $\mathcal C^n$ sur un intervalle $E$, alors la fonction $f\cdot g$
    est de classe $\mathcal C^n$ sur $E$ et~:
    \[
      (f\cdot g)^{(n)} = \sum_{k=0}^n \binom{n}{k}f^{(k)}\cdot g^{(n-k)}
    \]
  \end{theorem}

  \bigskip

  {\textbf{Rappels.}} Le coefficient binomial $\binom{n}{k}$ est défini par~:
  \[
    \binom{n}{k} = \frac{n!}{k!(n-k)!}
  \]
  Ce coefficient est aussi utilisé en combinatoire pour dénombrer le
  nombre de façons de choisir $k$ objets parmi $n$ objets distincts et
  que l'ordre dans lesquel les objets sont énumérés n'a pas
  d'importance. On peut montrer que~: $\binom{n}{1} = n$, $\binom{n}{0} = 1$, $\binom{0}{0} = 1$, $\binom{n}{n} = 1$, $\binom{n}{k} = \binom{n}{n-k}$, et
  \[
    \binom{n}{k} = \binom{n-1}{k-1} + \binom{n-1}{k}
  \]

\end{frame}


\begin{notes}

  \textbf{Preuve du théorème \hyperlink{slide_derivees_ordre_n_4}{\ref{thm:leibniz}}.} On utilise une preuve par récurrence. Pour $n=0$, on a~:
  \[
    \begin{split}
      (f\cdot g)^{(0)} &= \sum_{k=0}^0 \binom{n}{k}f^{(k)}\cdot g^{(n-k)}\\
      &= \binom{0}{0}f^{(0)}\cdot g^{(0)} = f \cdot g
    \end{split}
  \]
  et on sait que le produit de deux fonctions continues est une fonction continue. Nous pourrions nous arrêter là est vérifier que si le théorème est vrai au rang $n$ alors il est vrai au rang $n+1$, mais vérifions tout de même que le théorème \hyperlink{slide_derivees_ordre_n_4}{\ref{thm:leibniz}} nous permet bien de retrouver le résultat connu pour la dérivée d'ordre 1. Nous avons~:
  \[
    \begin{split}
      (f\cdot g)^{(1)} &= \sum_{k=0}^1 \binom{1}{k}f^{(k)}\cdot g^{(1-k)}\\
      &= \binom{1}{0}f^{(0)}g^{(1)}+\binom{1}{1}f^{(1)}g^{(0)}\\
      &= f'g+fg'
    \end{split}
  \]
  Si $f$ et $g$ sont de classe $\mathcal C^1$ alors $f'g+fg'$ est une fonction continue (somme et produits de fonctions continues). Montrons que le théorème est vrai au rang $n+1$, c'est-à-dire que si $f$ et $g$ sont de classe $\mathcal C^{n+1}$ alors $f\cdot g$ est de classe $\mathcal C^{n+1}$ et~:
  \[
    (f\cdot g)^{(n+1)} = \sum_{k=0}^{n+1} \binom{n+1}{k}f^{(k)}\cdot g^{(n+1-k)}
  \]
  Nous avons~:
  \[
    \begin{split}
      (f\cdot g)^{(n+1)} &= \left((f\cdot g)^{(n)}\right)'\\
      &= \left(\sum_{k=0}^{n} \binom{n}{k}f^{(k)}\cdot g^{(n-k)}\right)'\\
      &= \sum_{k=0}^{n} \binom{n}{k}f^{(k+1)}\cdot g^{(n-k)}+ \sum_{k=0}^{n}\binom{n}{k}f^{(k)}\cdot g^{(n+1-k)}\\
      &= \sum_{k=1}^{n+1} \binom{n}{k-1}f^{(k)}\cdot g^{(n+1-k)}+ \sum_{k=0}^{n}\binom{n}{k}f^{(k)}\cdot g^{(n+1-k)}\\
      &= \binom{n}{0}f\cdot g^{(n+1)} + \sum_{k=1}^n\left(\binom{n}{k-1}+\binom{n}{k}\right)f^{(k)}\cdot g^{(n+1-k)} + \binom{n+1}{n+1}f^{(n+1)}\cdot g\\
      &= f\cdot g^{(n+1)} + \sum_{k=1}^n\binom{n+1}{k}f^{(k)}\cdot g^{(n+1-k)} + f^{(n+1)}\cdot g\\
      &= \sum_{k=0}^{n+1}\binom{n+1}{k}f^{(k)}\cdot g^{(n+1-k)}\quad\quad \qed
    \end{split}
  \]

\end{notes}


\begin{frame}
  \frametitle{Dérivées d'ordre supérieur}
  \framesubtitle{Règles de dérivation: formule de Faa di Bruno}
  \hypertarget{slide_derivees_ordre_n_5}{}

  \begin{theorem}\label{thm:faa-di-bruno}
    Soit $n\in\mathbb N$. Soient $f$ et $g$ deux fonctions de classe
    $\mathcal C^n$ sur un intervalle $E$, alors la fonction $f\circ g$
    est de classe $\mathcal C^n$ sur $E$ et~:
    {\small\[
        (f\circ g)^{(n)}(x) = \sum_{\sum_{i=1}^n im_i=n} \frac{n!}{m_1!m_2!\cdots m_n!} f^{(m_1+m_2+\dots+m_n)}(g(x)) \prod_{j=1}^n \left(\frac{g^{(j)}(x)}{j!}\right)^{m_j}
      \]}
  \end{theorem}

  \begin{example}
    Soient $f$ et $g$ deux fonctions de classe $\mathcal C^2$ sur un $E$, alors~:
    \[
      \begin{split}
        (f\circ g)''(x) &= \Bigl(f'(g(x))g'(x)\Bigr)'\\
        &= f''(g(x))g'(x)g'(x)+f'(g(x))g''(x)\\
        &= f''(g(x))g'(x)^2+f'(g(x))g''(x)\\
      \end{split}
    \]
  \end{example}
\end{frame}


\section{Extrema et sens de variation d'une fonction}

\begin{frame}
  \frametitle{Extrema, I}
  \hypertarget{slide_extrema_1}{}
  \begin{definition}
    Soit $f: E\rightarrow \mathbb R$ une fonction. On dit que $a\in E$ est un~:
    \begin{description}
    \item[\textbf{maximum}] de $f$ sur $E$ si pour tout $x\in E$ on a $f(x)\leq f(a)$,
    \item[\textbf{minimum}] de $f$ sur $E$ si pour tout $x\in E$ on a $f(x)\geq f(a)$.
    \end{description}
    On dira que $a$ est un \textbf{extremum} de $f$ sur $E$ si $a$ est un maximum ou minimum de $f$ sur $E$.
  \end{definition}

  \bigskip

  \begin{definition}
    Soient $E$ un intervalle ouvert de $\mathbb R$, $f$ une fonction de $E$ dans $\mathbb R$ et $a\in E$. On dit que $a$ ets un~:
    \begin{description}
    \item[\textbf{maximum local}] de $f$ sur $E$ s'il existe $\delta>0$ pour tout $x\in E$ on ait
      $|x-a|\leq\delta \Rightarrow f(x)\leq f(a)$,
    \item[\textbf{minimum local}] de $f$ sur $E$ s'il existe $\delta>0$ pour tout $x\in E$ on ait
      $|x-a|\leq\delta \Rightarrow f(x)\geq f(a)$.
    \end{description}
  \end{definition}

\end{frame}


\begin{frame}
  \frametitle{Extrema, II}
  \hypertarget{slide_extrema_2}{}

  \begin{theorem}\label{thm:local_extrema}
    Soit $E$  un intervalle ouvert de $\mathbb R$, $f: E\rightarrow \mathbb R$ une fonction et $a$ un extremum local de $f$. \underline{Si $f$ est dérivable en $a$} alors on doit avoir $f'(a)=0$.
  \end{theorem}

  \bigskip

  \begin{itemize}
  \item[\dbend] Le théorème \hyperlink{slide_extrema_2}{\ref{thm:local_extrema}} ne donne qu'une condition nécessaire. Ce n'est pas parce que la dérivée d'une fonction est nulle en un point qu'elle admet un extremum en ce point. Par exemple, la fonction $f(x)=x^3$ de $\mathbb R$ dans $\mathbb R$ admet une dérivée nulle en 0 mais pas d'extremum local (en 0 ou ailleurs).\newline
  \end{itemize}

  \begin{theorem}[Théorème de Rolle]\label{thm:rolle}
    Soit $a < b$ deux réels, et $f$  une fonction continue sur $[a, b]$ et dérivablesur $]a, b[$ telle que $f(a) = f(b)$. Alors il existe $c\in]a, b[$ tel que $f'(c) = 0$.
  \end{theorem}

\end{frame}


\begin{frame}
  \frametitle{Extrema, II}
  \framesubtitle{Illustration du théorème de Rolle}
  \hypertarget{slide_extrema_3}{}

  \begin{center}
    \begin{tikzpicture}[scale=1.5]
      \begin{axis}[
        xticklabels={,,},
        yticklabels={,,},
        enlargelimits=true,
        grid style={dashed, gray!60},
        axis x line = bottom,
        axis y line = left,
        axis line style={thin},
        xmax = 9,
        xmin = -3,
        ymax = 8,
        ymin = -0.5,
        axis lines = middle,
        small,
        clip=false,
        ]
        \addplot[smooth, thick,samples=100,domain=1.4:8.5] plot(\x,{0.05*((\x)-3)^2*((\x)-9)^2+1});
        \addplot[draw=red, dotted] coordinates {(2,0) (2,3.45) (0,3.45)};
        \addplot[draw=red, dotted] coordinates {(7.414213562373062,0) (7.414213562373062,3.45) (2,3.45)};
        \node[color=red, left] at (0,3.45) {\tiny{$f(a)=f(b)$}};
        \fill [color=red] (2,3.45) circle (1.5pt);
        \fill [color=red] (7.414213562373062,3.45) circle (1.5pt);
        \node[color=red, below] at (2,0) {\tiny{$a$}};
        \node[color=red, below] at (7.414213562373062,0) {\tiny{$b$}};
        \addplot[draw=blue, dotted] coordinates {(3,0) (3,1) (0,1)};
        \node[color=blue, left] at (0,1) {\tiny{$m$}};
        \node[color=blue, below] at (3,0) {\tiny{$c_1$}};
        \addplot[draw=blue, dotted] coordinates {(6,0) (6,5.05) (0,5.05)};
        \node[color=blue, left] at (0,5.05) {\tiny{$M$}};
        \node[color=blue, below] at (6,0) {\tiny{$c_2$}};
        \addplot[draw=blue, <->, thick] coordinates {(2.3,1) (3.7,1)};
        \addplot[draw=blue, <->, thick] coordinates {(5.3,5.05) (6.7,5.05)};
      \end{axis}
    \end{tikzpicture}
  \end{center}

\end{frame}


\begin{notes}

  \textbf{Preuve du théorème \hyperlink{slide_extrema_2}{\ref{thm:local_extrema}}.} Supposons que $a$ soit tel que $f'(a)>0$. Comme $f'(a)$ est la limite du taux d'accroissement $\frac{f(x)-f(a)}{x-a}$ quand $x$ tend vers $a$, celui-ci doit être positif pour $x$ assez proche de $a$, c'est-à-dire~:
  \[
    \exists \delta>0\,|\,\forall x\in E, |x-a|\leq\delta \Rightarrow \frac{f(x)-f(a)}{x-a}>0
  \]
  Ainsi pour tout $x$ tel que $0<x-a\leq\delta$ on a $f(x)>f(a)$ et pour tout $x$ tel que $-\delta\leq x-a <0$ on a $f(x)<f(a)$, donc $a$ n'est pas un extremum local (car ni maximum ni minimum). On peut suivre le même argument si $f'(a)<0$.\newline

  \textbf{Preuve du théorème \hyperlink{slide_extrema_2}{\ref{thm:rolle}}.} Si $f$ est une fonction constante sur $[a,b]$, alors la dérivée de la fonction est nulle sur $]a,b[$ et le théorème est évidemment vérifié. Intéressons nous au cas où la fonction n'est pas constante. On sait alors, puisque $f$ est continue et par le théorème 13 du chapitre III, que la fonction admet un maximum $M$ et un minimum $m$ sur $[a,b]$ et que l'un des deux doit être différent de $f(a)$\footnote{Notons que si $M$ ou $m$ sont différents de $f(a)$ alors ils sont aussi différents de $f(b)$, puisque par hypothèse $f(a)=f(b)$.} (puisque la fonction n'est pas constante). Supposons que $M>f(a)$ et posons $c\in ]a,b[$ tel que $f(c)=M$. Alors $f(c)$ est le maximum de $f$ sur $]a, b[$~: c est un maximum local de $f$, et donc $f'(c) = 0$ par le théorème \hyperlink{slide_extrema_2}{\ref{thm:local_extrema}}.
\end{notes}


\begin{frame}
  \frametitle{Accroissements finis}
  \hypertarget{slide_accroissements_finis_1}{}

  \begin{theorem}\label{thm:accroissements_finis}
    Soient $a<b$ deux réels, et $f: [a,b]\rightarrow \mathbb R$ une fonction continue sur $[a, b]$ et dérivable sur $]a, b[$. Alors $\exists\, c\in]a, b[$ tel que $f(b)−f(a) = f'(c)(b−a)$.
  \end{theorem}

  \bigskip

  \begin{columns}[onlytextwidth]
    \begin{column}{0.5\textwidth}
      {\small
        \begin{itemize}
        \item $\frac{f(b)-f(a)}{b-a}$ est la pente de la corde entre les points $A$ et $B$.\newline
        \item $f'(c)$ est la pente de la tangente à la courbe représentative de $f$ en un point $c$.\newline
        \item Le théorème nous dit qu'il existe au moins un point $c$ où la tangente à la courbe est parallèle à la corde.
        \end{itemize}}
    \end{column}
    \begin{column}{0.5\textwidth}
      \begin{center}
        {\small
          \begin{tikzpicture}[scale=.7]
            \begin{scope}
              \clip (-3,-2) rectangle (3,2);
              \draw[thick,smooth,domain=-3:3] plot (\x,{\x^3/3 - \x});
              \node[black,right] at (-3,-1.8) {{\small$\mathcal  C_f$}};
            \end{scope}
            \node[circle,fill=red,scale=.3] (a) at (-2,-2/3) {};
            \node[circle,fill=red,scale=.3] (b) at (2,2/3) {};
            \node[red] at (-2.2,-1.2/3) {$A$};
            \node[red] at (2.2,1.2/3) {$B$};
            \node[circle,fill=red,scale=.3] (b) at (2,2/3) {};
            \draw[red] (a) -- (b);
            \coordinate (origin) at (-4,-3);
            \coordinate (topright) at (3.4,2);
            \draw[<->] (topright -| origin) -- (origin) -- (origin -| topright);
            \draw[dashed, red] (a) -- (a|-origin) node[below] {$a$};
            \draw[dashed, red] (b) -- (b|-origin) node[below] {$b$};

            \node[circle,fill=blue,scale=.3] (x0) at ({-2/sqrt(3)},{(1/3)*(-2/sqrt(3))^3+2/sqrt(3)}) {};
            \draw[blue] (x0) +(-1,-1/3) -- +(1,1/3);
            \node[circle,fill=blue,scale=.3] (x1) at ({2/sqrt(3)},{(1/3)*(2/sqrt(3))^3-2/sqrt(3)}) {};
            \draw[blue] (x1) +(-1,-1/3) -- +(1,1/3);
            \draw[dashed, blue] (x0) -- (x0 |- origin) node[below]{$c_1$};
            \draw[dashed, blue] (x1) -- (x1 |- origin) node[below]{$c_2$};
          \end{tikzpicture}}
      \end{center}
    \end{column}
  \end{columns}
\end{frame}


\begin{notes}

  \textbf{Preuve du théorème \hyperlink{slide_accroissements_finis_1}{\ref{thm:accroissements_finis}}.} Posons la fonction $g$ définie sur le même intervalle que $f$ à valeurs dans $\mathbb R$~:
  \[
    g(x) = f(x)-\frac{f(b)-f(a)}{b-a}(x-a)
  \]
  Par construction, $g$ est une fonction continue sur $[a,b]$ et dérivable sur $]a,b[$. La fonction $g$ est la différence de la fonction $f$ et d'une droite parallèle à la corde entre $(a,f(a))$ et $(b,f(b))$ passant par le point $(a,0)$ sur l'axe des abscisses. On note que la fonction $g$ vérifie $g(a) = g(b) = f(a)$, il s'agit en fait d'une rotation de la fonction d'origine $f$ qui nous permet d'utiliser le théorème de Rolle (dans l'illustration graphique ci-dessous, on passe de la courbe en trait plein à la courbe en tirets verts). En effet par le théorème \hyperlink{slide_extrema_2}{\ref{thm:rolle}} on sait qu'il existe $c\in]a,b[$ tel que~:
  \[
    g'(c) = 0
  \]
  \[
    \Leftrightarrow f'(c) = \frac{f(b)-f(a)}{b-a}\quad\quad\qed
  \]

  \begin{center}
    \begin{tikzpicture}[scale=1]
      \begin{scope}
        \clip (-3,-2) rectangle (3,2);
        \draw[thick,smooth,domain=-3:3] plot (\x,{\x^3/3 - \x});
        \node[black,right] at (-3,-1.8) {{\small$\mathcal  C_f$}};
        \draw[thick,green,dashed,domain=-3:3] plot (\x,{\x^3/3 - \x - (1/3)*(\x+2)});
        \node[green,left] at (2.9,0.7) {{\small$\mathcal  C_g$}};
      \end{scope}
      \node[circle,fill=red,scale=.5] (a) at (-2,-2/3) {};
      \node[circle,fill=red,scale=.5] (b) at (2,2/3) {};
      \node[circle,fill=green,scale=.5] (bb) at (2,-2/3) {};
      \node[red] at (-2.2,-1.2/3) {$A$};
      \node[red] at (2.2,1.2/3) {$B$};
      \node[green] at (2.2,-2/3) {$B'$};
      \node[circle,fill=red,scale=.5] (b) at (2,2/3) {};
      \draw[red] (a) -- (b);
      \draw[green] (a) -- (bb);
      \coordinate (origin) at (-4,-3);
      \coordinate (topright) at (3.4,2);
      \draw[<->] (topright -| origin) -- (origin) -- (origin -| topright);
      \draw[dashed, red] (a) -- (a|-origin) node[below] {$a$};
      \draw[dashed, red] (b) -- (b|-origin) node[below] {$b$};

      \node[circle,fill=blue,scale=.5] (x0) at ({-2/sqrt(3)},{(1/3)*(-2/sqrt(3))^3+2/sqrt(3)}) {};
      \node[circle,fill=green,scale=.5] (xx0) at ({-2/sqrt(3)},{(1/3)*(-2/sqrt(3))^3+2/sqrt(3)-(1/3)*(2-2/sqrt(3))}) {};
      \draw[blue] (x0) +(-1,-1/3) -- +(1,1/3);
      \node[circle,fill=blue,scale=.5] (x1) at ({2/sqrt(3)},{(1/3)*(2/sqrt(3))^3-2/sqrt(3)}) {};
      \node[circle,fill=green,scale=.5] (xx1) at ({2/sqrt(3)},{(1/3)*(2/sqrt(3))^3-2/sqrt(3)-(1/3)*(2+2/sqrt(3))}) {};
      \draw[blue] (x1) +(-1,-1/3) -- +(1,1/3);
      \draw[dashed, blue] (x0) -- (x0 |- origin) node[below]{$c_1$};
      \draw[dashed, blue] (x1) -- (x1 |- origin) node[below]{$c_2$};
      \draw[green] (xx0) +(-1,0) -- +(1,0);
      \draw[green] (xx1) +(-1,0) -- +(1,0);
    \end{tikzpicture}
  \end{center}

\end{notes}


\begin{frame}
  \frametitle{Sens de variation}
  \hypertarget{slide_sens_de_variation_1}{}

  \begin{theorem}\label{thm:sens_de_variation}
    Soit $E$ un intervalle de $\mathbb R$ et $f: E\rightarrow \mathbb R$ une fonction dérivable. Alors $f$ est croissante sur $E$ si et seulement si, on a $f'(x)\geq0$ pour tout $x\in E$, et $f$ est décroissante sur $E$ si et seulement si on a $f'(x)\leq 0$ pour tout $x\in E$. De plus, si $f'(x)>0$ pour tout $x\in E$  alors $f$  est strictement croissante, et si $f'(x)<0$ pour tout $x\in E$ alors $f$ est strictement décroissante.
  \end{theorem}

  \bigskip

  \begin{itemize}

  \item Ce théorème (voir la preuve) est une conséquence direct du théorème \hyperlink{slide_accroissements_finis_1}{\ref{thm:accroissements_finis}}, dit des accroissements finis.\newline

  \item Si la dérivée est nulle pour tout $x\in E$ alors la fonction est constante (simultanément croissante et décroissante).\newline

  \item C'est ce théorème qui nous permet de construire des tableaux de variation pour décrire les fonctions.

  \end{itemize}

\end{frame}


\begin{notes}

  \textbf{Preuve du théorème \hyperlink{slide_sens_de_variation_1}{\ref{thm:sens_de_variation}}.} Si $f$ est croissante sur $E$ alors pour tout $(x,y)\in E^2$ avec $y\neq x$ le signe de $f(y)-f(x)$ doit être identique au signe de $y-x$ (de sorte que $f(y)>f(x)$ si et seulement si $y>x$). Ainsi la dérivée~:
  \[
    f'(x) = \lim_{y\rightarrow x}\frac{f(y)-f(x)}{y-x}
  \]
  doit être positive ou nulle puisque le taux de variation sous la limite est positif. Pour la réciproque, si $f'(x)\geq 0$ pour tout $x\in E$, alors considérons $(x,y)\in E^2$ avec $y\neq x$. Par le théorème \hyperlink{slide_accroissements_finis_1}{\ref{thm:accroissements_finis}} on sait qu'il existe $c\in]x,y[$ tel que~:
  \[
    f(y)-f(x) = f'(c)(y-x)
  \]
  ainsi le signe de $f(y)\geq f(x)$ si $y>x$ et $f(y)\leq f(x)$ si $y<x$, puisque $f'(c)\geq 0$, $f$ est donc une fonction croissante. Si la fonction est strictement positive alors la fonction est strictement croissante, c'est-à-dire~: $f(y)>f(x)$ si $y>x$ et $f(y)<f(x)$ si $y<x$.

\end{notes}


\begin{frame}
  \frametitle{Inégalité des accroissements finis}
  \hypertarget{slide_inegalite_accroissements_finis_1}{}

  \begin{theorem}\label{thm:inegalite_accroissements_finis}
    Soient $a<b$ deux réels, et $f: [a,b]\rightarrow \mathbb R$ une fonction continue sur $[a, b]$ et dérivable sur $]a, b[$. Si $\sup_{c\in]a,b[}|f'(c)|<\infty$, alors pour tout $(x,y)\in[a,b]^2$ on a~:
    \[
      |f(x)-f(y)| \leq k |x-y|
    \]
  \end{theorem}

  \bigskip

  \begin{itemize}

  \item Si la fonction $f$ est de classe $\mathcal C^1$, alors $f'$ est une fonction continue et donc bornée. Ainsi le théorème \hyperlink{slide_inegalite_accroissements_finis_1}{\ref{thm:inegalite_accroissements_finis}} s'applique à toute fonction de classe $\mathcal C^1$.\newline

  \item Sous les conditions du théorème \hyperlink{slide_inegalite_accroissements_finis_1}{\ref{thm:inegalite_accroissements_finis}}, la fonction $f$ est $k$-lipschitzienne.\newline

  \item Une fonction de classe $\mathcal C^1$ est $k$-lipschitzienne.

  \end{itemize}

\end{frame}


\begin{notes}
  \textbf{preuve du théorème \hyperlink{slide_inegalite_accroissements_finis_1}{\ref{thm:inegalite_accroissements_finis}}.} Direct par le théorème \hyperlink{slide_accroissements_finis_1}{\ref{thm:accroissements_finis}}. Soit $(x,y)\in[a,b]^2$, sans perte de généralité on suppose que $y>x$. On peut appliquer l’égalité des accroissements finis sur l'intervalle $[x, y]$, on sait qu'il existe $c\in]x, y[$ tel que $f(y)− f(x) = f'(c)(y−x)$. Soit, en prenant la valeur absolue, $|f(y)− f(x)| = |f'(c)||y−x|$. Puisque la dérivée est finie on sait qu'il existe $k\in\mathbb R_+$ fini tel que $|f(x)− f(y)|≤ k|x− y|$.\qed

\end{notes}


\begin{frame}
  \frametitle{Extrema, III}

  \begin{itemize}

  \item Le théorème \hyperlink{slide_extrema_2}{\ref{thm:local_extrema}} donne une condition pour un extremum local, mais (\textit{i}) cette condition est seulement nécessaire, et (\textit{ii}) celle-ci suppose que la fonction est dérivable en l'extremum\ldots\newline

  \item En admettant que la fonction soit dérivable en l'extremum, pour que cette condition permette efectivement d'identifier un extremum local il faut qu'on ait un changement de signe de la dérivée autour de l'extremum~:\newline

    \begin{itemize}

    \item Pour que $a$ soit un maximum local de $f$, si $f$ est dérivable en $a$, il faut et il suffit que $f'(a)=0$, $\exists \delta_->0$ tel que $f'(x)>0$ pour tout $x\in[a-\delta_-,a[$ et $\exists \delta_+>0$ tel que $f'(x)<0$ pour tout $x\in]a, a-\delta_+]$.\newline

    \item Pour que $a$ soit un minimum local de $f$, si $f$ est dérivable en $a$, il faut et il suffit que $f'(a)=0$, $\exists \delta_->0$ tel que $f'(x)<0$ pour tout $x\in[a-\delta_-,a[$ et $\exists \delta_+>0$ tel que $f'(x)>0$ pour tout $x\in]a, a-\delta_+]$.\newline

    \end{itemize}

  \end{itemize}

\end{frame}


\begin{frame}
  \frametitle{Extrema, IV}
  \framesubtitle{Exemple avec une fonction non dérivable en l'extremum}

  \begin{columns}[onlytextwidth]
    \begin{column}{.5\textwidth}
      \begin{itemize}

      \item $f(x) = x^{\frac{2}{3}}$\newline

      \item $f'(x) = \frac{2}{3}x^{-\frac{1}{3}}$\newline

      \item $f'$ a une discontinuité (infinie) en $x=0$\ldots\newline

      \item La fonction admet un minimum global en 0, pourtant la dérivée
        en 0 n'est pas définie.\newline

      \item Même si $f'(0) \neq 0$, on a bien le changement de signe
        de la dérivée autour de l'extremum.
      \end{itemize}
    \end{column}
    \begin{column}{.5\textwidth}
      \begin{center}
        \begin{tikzpicture}[scale=1]
          \begin{axis}[
            xticklabels={,,},
            yticklabels={,,},
            enlargelimits=true,
            grid style={dashed, gray!60},
            axis x line = bottom,
            axis y line = left,
            axis line style={thin},
            xmax = 4,
            xmin = -4,
            ymax = 3,
            ymin = -0.1,
            axis lines = middle,
            small,
            clip=false,
            ]
            \addplot[
            draw=black,
            samples=5000,
            thick,
            domain=-3:3,
            ]
            {(x^2)^(1/3)};
          \end{axis}
        \end{tikzpicture}
      \end{center}
    \end{column}
  \end{columns}

\end{frame}


\begin{frame}
  \frametitle{Extrema, V}
  \framesubtitle{Comment identifier les extrema d'une fonction~?}

  Il suffit de suivre les étapes suivantes~:

  \bigskip

  \begin{itemize}

  \item[1.] Calculer la dérivée première $f'(x)$.\newline

  \item[2.] Identifier les valeurs de $x$ pour lesquelles $f'$ est nulle.\newline

  \item[3.] Identifier les valeurs de $x$ pour lesquelles $f'$ est discontinue.\newline

  \item[4.] Pour chaque valeur $a$ identifiée en 2.a et 2.b, déterminer si $f'$ change de signe autour de $a$~:\newline
    \begin{itemize}
    \item $f'$ passe du signe $-$ à $+$: minimum local,
    \item $f'$ passe du signe $+$ à $-$: maximum local,

    \end{itemize}

  \end{itemize}

  \bigskip

  Ensuite il ne reste qu'à évaluer la fonction $f$ sur les extrema identifiés.

\end{frame}


\begin{frame}
  \frametitle{Extrema, VI}
  \framesubtitle{Exemple (a)}

  \medskip

  \begin{itemize}

  \item Soit la fonction $f(x) = (1+x)^{\frac{2}{3}}(2-x)^{\frac{1}{3}}$. On cherche les extrema de cette fonction.\newline

  \item Calculons la dérivée de la fonction~:
    \[
      \begin{split}
        f'(x) &= -\frac{1}{3}(1+x)^{\frac{2}{3}}(2-x)^{-\frac{2}{3}}+\frac{2}{3}(1+x)^{-\frac{1}{3}}(2-x)^{\frac{1}{3}}\\
        &= \frac{1}{3}(1+x)^{-\frac{1}{3}}(2-x)^{-\frac{2}{3}}\Bigl(-(1+x)+2(2-x)\Bigr)\\
        &=\frac{1-x}{(1+x)^{\frac{1}{3}}(2-x)^{\frac{2}{3}}}
      \end{split}
    \]

    \bigskip

  \item La dérivée $f'$ est nulle lorsque $x=1$.\newline

  \item La dérivée $f'$ est discontinue en $x=-1$ et $x=2$ car le dénominateur est alors nul. Il s'agit de discontinuités infinies, nous avons~:
    \[
      \lim_{x\rightarrow 1^-}f'(x) = -\infty \quad \text{ et } \quad \lim_{x\rightarrow 1^+}f'(x) = \infty
    \]
    \[
      \lim_{x\rightarrow 2^-}f'(x) = -\infty \quad \text{ et } \quad \lim_{x\rightarrow 2^+}f'(x) = -\infty
    \]

  \end{itemize}

\end{frame}


\begin{frame}
  \frametitle{Extrema, VI}
  \framesubtitle{Exemple (b)}

  \medskip

  \begin{itemize}

  \item Déterminons les changements de signe de la dérivée $f'$ en $x=-1$, $x=1$ et $x=2$~:

    \[
      \begin{cases}
        f'(x)<0 &\text{ si }x<-1\\
        f'(x)>0 &\text{ si }-1<x<1
      \end{cases}
      \Rightarrow
      \text{ minimum local en }x=-1
    \]

    \[
      \begin{cases}
        f'(x)>0 &\text{ si }-1<x<1\\
        f'(x)<0 &\text{ si }1<x<2\\
      \end{cases}
      \Rightarrow
      \text{ maximum local en }x=1
    \]

    \bigskip

    \[
      \begin{cases}
        f'(x)<0 &\text{ si }1<x<2\\
        f'(x)<0 &\text{ si }x>2
      \end{cases}
      \Rightarrow
      \text{ pas d'extremum en }x=2
    \]

    \bigskip

  \item $f(-1) = f(2) = 0$ et $f(1) = \sqrt[3]{4}$.

  \end{itemize}

\end{frame}


\begin{frame}
  \frametitle{Extrema, VI}
  \framesubtitle{Exemple (c)}

  \begin{center}
    \begin{tikzpicture}[scale=1.8]
      \begin{axis}[
        xticklabels={,,},
        yticklabels={,,},
        enlargelimits=true,
        grid style={dashed, gray!60},
        axis x line = bottom,
        axis y line = left,
        axis line style={thin},
        xmax = 3,
        xmin = -3,
        ymax = 4,
        ymin = -3,
        axis lines = middle,
        small,
        clip=false,
        ]
        \addplot[
        draw=black,
        samples=50,
        thick,
        domain=-3.5:-1.2,
        ]
        {(((1+x)^2)^(1/3))*(((2-x)^2)^(1/6))};
        \addplot[
        draw=black,
        samples=1000,
        thick,
        domain=-1.2:-0.8,
        ]
        {(((1+x)^2)^(1/3))*(((2-x)^2)^(1/6))};
        \addplot[
        draw=black,
        samples=200,
        thick,
        domain=-0.8:1.8,
        ]
        {(((1+x)^2)^(1/3))*(((2-x)^2)^(1/6))};
        \addplot[
        draw=black,
        samples=1000,
        thick,
        domain=1.8:2,
        ]
        {(((1+x)^2)^(1/3))*(((2-x)^2)^(1/6))};
        \addplot[
        draw=black,
        samples=1000,
        thick,
        domain=2:2.2,
        ]
        {-(((1+x)^2)^(1/3))*(((2-x)^2)^(1/6))};
        \addplot[
        draw=black,
        samples=200,
        thick,
        domain=2:3.5,
        ]
        {-(((1+x)^2)^(1/3))*(((2-x)^2)^(1/6))};
      \end{axis}
    \end{tikzpicture}
  \end{center}

\end{frame}


\end{document}

% Local Variables:
% ispell-check-comments: exclusive
% ispell-local-dictionary: "francais"
% TeX-master: t
% End: