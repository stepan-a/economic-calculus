\synctex=1

\documentclass[10pt,notheorems]{beamer}

\usepackage{etex}
\usepackage{fourier-orns}
\usepackage{ccicons}
\usepackage{amssymb}
\usepackage{amstext}
\usepackage{amsbsy}
\usepackage{amsopn}
\usepackage{amscd}
\usepackage{amsxtra}
\usepackage{amsthm}
\usepackage{float}
\usepackage{color, colortbl}
\usepackage{mathrsfs}
\usepackage{bm}
\usepackage{lastpage}
\usepackage[nice]{nicefrac}
\usepackage{setspace}
\usepackage{ragged2e}
\usepackage{listings}
\usepackage{polynom}
\usepackage{algorithms/algorithm}
\usepackage{algorithms/algorithmic}
\usepackage[frenchb]{babel}
\usepackage{tikz,pgfplots}
\pgfplotsset{compat=newest}
\usetikzlibrary{patterns, arrows, decorations.pathreplacing, decorations.markings, calc}
\pgfplotsset{plot coordinates/math parser=false}
\newlength\figureheight
\newlength\figurewidth
\usepackage[utf8x]{inputenc}
\usepackage{cancel}
\usepackage{tikz-qtree}
\usepackage{dcolumn}
\usepackage{adjustbox}
\usepackage{environ}
\usepackage[cal=boondox]{mathalfa}
\usepackage{manfnt}
\usepackage{hyperref}
\hypersetup{
  colorlinks=true,
  linkcolor=blue,
  filecolor=black,
  urlcolor=black,
}
\usepackage{venndiagram}
\usepackage{minted}

% Git hash
\usepackage{xstring}
\usepackage{catchfile}
\immediate\write18{git rev-parse HEAD > git.hash}
\CatchFileDef{\HEAD}{git.hash}{\endlinechar=-1}
\newcommand{\gitrevision}{\StrLeft{\HEAD}{7}}

% Include pdf page by page
\usepackage{pdfpages}


\newcommand{\trace}{\mathrm{tr}}
\newcommand{\vect}{\mathrm{vec}}
\newcommand{\tracarg}[1]{\mathrm{tr}\left\{#1\right\}}
\newcommand{\vectarg}[1]{\mathrm{vec}\left(#1\right)}
\newcommand{\vecth}[1]{\mathrm{vech}\left(#1\right)}
\newcommand{\iid}[2]{\mathrm{iid}\left(#1,#2\right)}
\newcommand{\normal}[2]{\mathcal N\left(#1,#2\right)}
\newcommand{\dynare}{\href{http://www.dynare.org}{\color{blue}Dynare}}
\newcommand{\sample}{\mathcal Y_T}
\newcommand{\samplet}[1]{\mathcal Y_{#1}}
\newcommand{\slidetitle}[1]{\fancyhead[L]{\textsc{#1}}}

\newcommand{\R}{{\mathbb R}}
\newcommand{\C}{{\mathbb C}}
\newcommand{\N}{{\mathbb N}}
\newcommand{\Z}{{\mathbb Z}}
\newcommand{\binomial}[2]{\begin{pmatrix} #1 \\ #2 \end{pmatrix}}
\newcommand{\bigO}[1]{\mathcal O \left(#1\right)}
\newcommand{\red}{\color{red}}
\newcommand{\blue}{\color{blue}}

\renewcommand{\qedsymbol}{C.Q.F.D.}

\newcolumntype{d}{D{.}{.}{-1}}
\definecolor{gray}{gray}{0.9}
\newcolumntype{g}{>{\columncolor{gray}}c}

\setbeamertemplate{theorems}[numbered]

\theoremstyle{plain}
\newtheorem{theorem}{Théorème}

\theoremstyle{definition} % insert bellow all blocks you want in normal text
\newtheorem{definition}{Définition}
\newtheorem{properties}{Propriétés}
\newtheorem{lemma}{Lemme}
\newtheorem{property}[properties]{Propriété}
\newtheorem{example}{Exemple}
\newtheorem*{idea}{Éléments de preuve} % no numbered block
\newtheorem{corollary}{Corollaire}%[theorem]


\setbeamertemplate{footline}{
  {\hfill\vspace*{1pt}\href{http://creativecommons.org/licenses/by-sa/3.0/legalcode}{\ccbysa}\hspace{.1cm}
    \raisebox{-.075cm}{\href{https://git.adjemian.eu/stepan/economic-calculus}{\includegraphics[scale=.1]{../img/gitlab.png}}}\enspace
    \href{https://git.adjemian.eu/stepan/economic-calculus/-/blob/\HEAD/cours/chapitre-4.tex}{\gitrevision}\enspace\today
  }\hspace{1cm}}

\setbeamertemplate{navigation symbols}{}
\setbeamertemplate{blocks}[rounded][shadow=true]
\setbeamertemplate{caption}[numbered]

\NewEnviron{notes}{\justifying\tiny\begin{spacing}{1.0}\BODY\vfill\pagebreak\end{spacing}}

\newenvironment{exercise}[1]
{\bgroup \small\begin{block}{Ex. #1}}
  {\end{block}\egroup}

\newenvironment{defn}[1]
{\bgroup \small\begin{block}{Définition. #1}}
  {\end{block}\egroup}

\newenvironment{exemple}[1]
{\bgroup \small\begin{block}{Exemple. #1}}
  {\end{block}\egroup}

\begin{document}

\title{Calcul Économique\\\small{IV. Dérivées}}
\author[S. Adjemian]{Stéphane Adjemian}
\institute{\texttt{stephane.adjemian@univ-lemans.fr}} \date{Octobre 2020}

\begin{frame}
  \titlepage{}
\end{frame}

\begin{frame}
  \frametitle{Plan}
  \tableofcontents
\end{frame}


\section{Dérivée d'une fonction en un point}

\begin{frame}
  \frametitle{Variation d'une fonction, I}
  \hypertarget{slide_variation_1}{}

  \begin{itemize}

  \item Soit $f$ une fonction de $E$ dans $F$.\newline

  \item On veut prédire l'effet d'une variation de $x\in E$ sur son image par $f$.\newline

  \item Soit $x_0\in E$, son image par $f$ est $y_0 = f(x_0)\in F$.\newline

  \item Pertubons $x_0$ en lui rajoutant $\Delta x$. On suppose que la perturbation est « assez petite » pour que~: $x_1=x_0+\Delta x \in E$.\newline

  \item On peut donc réécrire $\Delta x = x_1-x_0$, et définir $y_1 = f(x_0+\Delta x)\in F$ la conséquence sur l'image par $f$.\newline

  \item On pose $\Delta f(x) = f(x_1)-f(x_0)$ la variation de l'image par $f$ induite par la variation $\Delta x$ de $x$.

  \end{itemize}

\end{frame}


\begin{frame}
  \frametitle{Variation d'une fonction, II}
  \hypertarget{slide_variation_2}{}


  \begin{example}
    Soit $f(x) = x^3$ une fonction de $\mathbb R$ dans $\mathbb R$. Posons comme valeur initiale de x $x_0 = 2$, son image par $f$ est $y_0 = f(x_0) = 8$. Donnons-nous une perturbation $\Delta x = 1$. On a alors $x_1=x_0+\Delta x = 3$ et son image $y_1 = 3^3 = 27$. La variation induite de l'image est donc $\Delta f(x) = y_1-y_0$, soit $\Delta f(x) = 19$. Si nous changeons la perturbation sur $x$ en posant $\Delta x = -1$, nous obtenons $x_1 = 1$, $y_1 = 1$ et donc $\Delta f(x) = -7$.\newline

    Pour ces deux exemples $\Delta f(x)$ et $\Delta x$ ont le même signe~: quand on augmente (diminue) $x$ cela induit une augmentation (baisse) de $f(x)$. Cela suggère que $f$ est croissante autour de $x=1$. Ce résultat n'est évidemment pas général.\newline

    Si maintenant $x_0 = 1$ et $\Delta x = 1$, on a $x_1 = 2$, $y_0 = 1$, $y_1 = 8$ et donc $\Delta f(x) = 7$.    On note que l'amplitude de l'effet sur l'image n'est pas constant (dépend du niveau initial $x_0$)\newline

  \end{example}

\end{frame}


\begin{frame}
  \frametitle{Variation d'une fonction, III}
  \hypertarget{slide_variation_3}{}

  \begin{example}

    Soit la fonction de $\mathbb R$ dans $\mathbb R$~: $f(x) = ax+b$, une droite, où $a$ et $b$ sont des paramètres réels. Si on perturbe $x_0$ avec $\Delta x$, on a~:
    \[
      y_1 = a (\underbrace{x_0+\Delta x}_{x_1}) + b
    \]
    sachant que $y_0 = a x_0 + b$, on a donc~:
    \[
      \begin{split}
        \Delta f(x) &= y_1-y_0\\
        &= a(x_1-x_0)\\
        &= a \Delta x
      \end{split}
    \]
    Si on normalise la variation induite $\Delta f(x)$ par la variation $\Delta$, on obtient~:
    \[
      \frac{\Delta f(x)}{\Delta x} = a \quad \forall x_0\in\mathbb R
    \]
    Pour une droite, le \textbf{taux de variation} ne dépend pas de $x_0$.

  \end{example}

\end{frame}


\begin{frame}
  \frametitle{Variation d'une fonction, IV}
  \hypertarget{slide_variation_4}{}

  \begin{columns}[onlytextwidth]
    \begin{column}{.5\textwidth}
      \begin{itemize}
      \item Plus généralement~:
        \[
          f(x) + \Delta f(x) = f(x+\Delta x)
        \]
      \item Soit de façon équivalente~:
        \[
          \Delta f(x) = f(x+\Delta x)-f(x)
        \]
      \item En divisant par $\Delta x\neq 0$, on  obtient le taux de variation~:
        \[
          \frac{\Delta f(x)}{\Delta x} = \frac{f(x+\Delta x)-f(x)}{\Delta x}
        \]
        qui correspond à la pente de l'arc passant par les points $A_0$ et $A_1$ sur le graphique.
      \end{itemize}
    \end{column}
    \begin{column}{.5\textwidth}
      \begin{tikzpicture}[scale=1.05]
        \begin{axis}[
          title={},
          xlabel= {},
          ylabel= {},
          xticklabels={,,},
          yticklabels={,,},
          enlargelimits=true,
          grid style={dashed, gray!60},
          axis x line = bottom,
          axis y line = left,
          axis lines = middle,
          axis line style={thin},
          xmin = -.1,
          xmax = 3,
          ymin = -.1,
          ymax = 5,
          small,
          clip=false,
          ]
          \addplot[
          draw=black,
          thick,
          smooth,
          samples=500,
          domain=0:2.1,
          ]
          {x^2} node[right] {\tiny $f(x)$} ;
          \addplot[
          draw=red,
          domain=.2:2.5
          ]
          {3*x-2};
          \node[draw=red, circle, fill=red, scale=.2] at (1,1) {};
          \node[draw=red, circle, fill=red, scale=.2] at (2,4) {};
          \node[below] at (1,0) {\tiny{\color{red}$x_0$}};
          \node[below] at (2,0) {\tiny{\color{red}$x_1 = x_0+\Delta x$}};
          \node[left] at (0,1) {\tiny{\color{red}$f(x_0)$}};
          \node[left] at (0,4) {\tiny{\color{red}$f(x_0+\Delta x)$}};
          \node[above] at (1,1.05) {\tiny\color{red} $A_0$};
          \node[right] at (2.05,4) {\tiny\color{red} $A_1$};
          \addplot[draw=red, dotted] coordinates {(2,4) (0,4)};
          \addplot[draw=red, dotted] coordinates {(2,4) (2,0)};
          \addplot[draw=red, dotted] coordinates {(1,1) (0,1)};
          \addplot[draw=red, dotted] coordinates {(1,1) (1,0)};
        \end{axis}
      \end{tikzpicture}
    \end{column}
  \end{columns}

\end{frame}


\begin{frame}
  \frametitle{Variation d'une fonction, V}
  \hypertarget{slide_variation_5}{}

  \begin{columns}[onlytextwidth]
    \begin{column}{.5\textwidth}
      \begin{itemize}

      \item Quel est le taux de variation pour $\Delta x = 0$~? On a une forme indéterminée de type 0/0.\newline

      \item Si la limite existe, la dérivée de $f$ en $x_0$ est la limite du taux de variation quand $\Delta x$ tend vers 0~:
        \[
          \frac{\mathrm d}{\mathrm dx}f(x_0) = \lim_{\Delta x\rightarrow 0}\frac{f(x+\Delta x)-f(x)}{\Delta x}
        \]
        \medskip
      \item On a remplacé $\Delta x$ par $\mathrm dx$ $\rightarrow$ variations infinitésimales.\newline

      \item Si la dérivée en $x_0$ existe on la notera aussi $f'(x_0)$.\newline
      \end{itemize}
    \end{column}
    \begin{column}{.5\textwidth}
      \begin{center}
        \begin{tikzpicture}[scale=1]
          \begin{axis}[
            title={},
            xlabel= {},
            ylabel= {},
            xticklabels={,,},
            yticklabels={,,},
            enlargelimits=true,
            grid style={dashed, gray!60},
            axis x line = bottom,
            axis y line = left,
            axis lines = middle,
            axis line style={thin},
            xmin = -.1,
            xmax = 3,
            ymin = -.1,
            ymax = 5,
            small,
            clip=false,
            ]
            \addplot[
            draw=black,
            thick,
            smooth,
            samples=500,
            domain=0:2.1,
            ]
            {x^2} node[right] {\tiny $f(x)$} ;
            \addplot[
            draw=red,
            domain=.2:2.5
            ]
            {2*x-1};
            \node[draw=red, circle, fill=red, scale=.2] at (1,1) {};
            \node[below] at (1,0) {\tiny{\color{red}$x_0$}};
            \node[left] at (0,1) {\tiny{\color{red}$f(x_0)$}};
            \node[above] at (1,1.05) {\tiny\color{red} $A_0$};
            \addplot[draw=red, dotted] coordinates {(1,1) (0,1)};
            \addplot[draw=red, dotted] coordinates {(1,1) (1,0)};
          \end{axis}
        \end{tikzpicture}
      \end{center}
    \end{column}
  \end{columns}

\end{frame}


\begin{frame}
  \frametitle{Dérivée en un point, I}
  \hypertarget{slide_derivee_1}{}

  \begin{definition}
    Soit $f$ une fonction de $E$ dans $\mathbb R$. On note $f'(x_0)$ la dérivée de $f$ au point $x_0\in E$, celle-ci est définie par~:
    \[
      f'(x_0) = \lim_{h\rightarrow 0} \frac{f(x_0+h)-f(x_0)}{h}
    \]
  \end{definition}

  \bigskip

  \begin{itemize}

  \item La dérivée est définie par une limite.\newline

  \item Pour que la dérivée existe, il faut que la limite soit définie.\newline

  \item La dérivée de la fonction $f$ en un point $x_0$ est la pente de la tangente à la courbe représentative de $f$ en $x_0$.
  \end{itemize}

\end{frame}


\begin{frame}
  \frametitle{Dérivée en un point, II}
  \hypertarget{slide_derivee_2}{}

  \begin{center}
    \begin{tikzpicture}[scale=1.6]
      \begin{axis}[
        title={},
        xlabel= {},
        ylabel= {},
        xticklabels={,,},
        yticklabels={,,},
        enlargelimits=true,
        grid style={dashed, gray!60},
        axis x line = bottom,
        axis y line = left,
        axis lines = middle,
        axis line style={thin},
        xmin = -.1,
        xmax = 1.7,
        ymin = -1.5,
        ymax = 3,
        small,
        clip=false,
        ]
        \addplot[
        draw=black,
        thick,
        smooth,
        samples=500,
        domain=0:1.5,
        ]
        {x^2} node[right] {\tiny $f(x)$} ;
        \addplot[
        draw=red,
        domain=-.3:1.6
        ]
        {2*x-1};
        \node[draw=red, circle, fill=red, scale=.2] at (1,1) {};
        \node[below] at (1,0) {\tiny{\color{red}$x_0$}};
        \node[left] at (0,1) {\tiny{\color{red}$f(x_0)$}};
        \addplot[draw=red, dotted] coordinates {(1,1) (0,1)};
        \addplot[draw=red, dotted] coordinates {(1,1) (1,0)};
        \node[draw=red, circle, fill=red, scale=.2] at (0,-1) {};
        \node[left] at (-0.05,-1) {\tiny{\color{red}$f(x_0)-f'(x_0)x_0$}};
        \node[draw=red, circle, fill=red, scale=.2] at (0.5,0) {};
        \node[right] at (.4,-.275) {\tiny{\color{red}$x_0-\frac{f(x_0)}{f'(x_0)}$}};
        \node[right] at (.1,-1.3) {\tiny{\color{red}Équation de la tangente: $y = f(x_0)+f'(x_0)(x-x_0)$}};
      \end{axis}
    \end{tikzpicture}
  \end{center}

\end{frame}


\begin{notes}
  Pour déterminer l'équation de la tangente on exploite deux informations~:\newline
  \begin{itemize}

  \item La pente de la tangente est $f'(x_0)$,\newline

  \item La tangente passe par le point $(x_0, f(x_0))$.\newline

  \end{itemize}

  L'équation de la tangente est donc de la forme~:
  \[
    y = f'(x_0)x + b
  \]
  Il ne reste plus qu'à choisir $b$ pour s'assurer que la tangente passe bien par le point $(x_0, f(x_0))$. On a~:
  \[
    f(x_0) = f'(x_0)x_0 + b
  \]
  soit de façon équivalente~:
  \[
    b = f(x_0) - f'(x_0)x_0
  \]
  et donc~:
  \[
    y = f'(x_0) x + f(x_0) - f'(x_0)x_0
  \]
  ou encore en factorisant~:
  \[
    y = f(x_0) + f'(x_0)(x-x_0)\hfil\qed
  \]
\end{notes}


\begin{frame}
  \frametitle{Dérivée en un point, III}
  \hypertarget{slide_derivee_3}{}

  \begin{itemize}

  \item La dérivée est définie par une limite.\newline

  \item On peut donc définir une dérivée à droite et une dérivée à gauche\newline

  \item Si les dérivées à droite et à gauche en $x_0$ sont différentes, on dit que la fonction n'est pas dérivable en $x_0$ (comme pour l'existence de la limite).\newline

  \end{itemize}

  \begin{definition}
    Soit $f$ une fonction de $E$ dans $\mathbb R$. On note $f'_-(x_0)$ et $f'_+(x_0)$  les dérivées à gauche et à droite de $f$ au point $x_0\in E$, celle-ci sont définies par~:
    \[
      f'_-(x_0) = \lim_{h\rightarrow 0^-} \frac{f(x_0+h)-f(x_0)}{h}
    \]
    et
    \[
      f'_+(x_0) = \lim_{h\rightarrow 0^+} \frac{f(x_0+h)-f(x_0)}{h}
    \]
  \end{definition}

\end{frame}


\begin{frame}
  \frametitle{Dérivée en un point, IV}
  \hypertarget{slide_derivee_4}{}

  \begin{example}

    Soit $f(x) = x^2$ une fonction de $\mathbb R$ dans $\mathbb R_+$. Calculons la dérivée en $x=1$. Nous avons~:
    \[
      \begin{split}
        f'(1) &= \lim_{h\rightarrow 0}\frac{f(1+h)-f(1)}{h}\\
        &= \lim_{h\rightarrow 0}\frac{(1+h)^2-1}{h}\\
        &= \lim_{h\rightarrow 0}\frac{h^2+2h+1-1}{h}\\
        &= \lim_{h\rightarrow 0}\frac{h^2+2h}{h}\\
        &= \lim_{h\rightarrow 0}h+2\\
        &= 2
      \end{split}
    \]
  \end{example}

\end{frame}


\begin{frame}
  \frametitle{Dérivée en un point, V}
  \hypertarget{slide_derivee_5}{}

  \begin{itemize}

  \item Pour qu'une fonction soit dérivable en un point, il faut que la fonction soit définie en ce point.\newline

  \item Si une fonction n'est pas définie en un point $x_0$, alors la fonction n'admet pas de dérivée en ce point.\newline

  \item[\dbend] Ce n'est pas parce qu'une fonction est définie en un point que la dérivée en ce point existe.\newline

  \end{itemize}

  \begin{theorem}\label{thm:derivable-continue}
    Soit $f$ une fonction de $E$ dans $\mathbb R$. Si la fonction $f$ est dérivable en $x_0\in E$, alors la fonction $f$ est continue en $x_0$.
  \end{theorem}

  \bigskip

  \begin{itemize}

  \item[\dbend] La réciproque n'est pas vraie (voir l'exemple suivant)~: une fonction continue n'est pas nécessairement dérivable.

  \end{itemize}

\end{frame}


\begin{notes}
  \textbf{Preuve du théorème
    \hyperlink{slide_derivee_5}{\ref{thm:derivable-continue}}.}
  Puisque la fonction est supposée dérivable en $x_0\in E$, nous avons
  par définition de la dérivée~:
  \[
    f'(x_0) = \lim_{h\rightarrow 0} \frac{f(x_0+h)-f(x_0)}{h}
  \]
  Posons~:
  \[
    g(h) =  \frac{f(x_0+h)-f(x_0)}{h}
  \]
  Nous avons donc~:
  \[
    h g(h) = f(x_0+h)-f(x_0)
  \]
  avec $\lim_{h\rightarrow 0}g(h) = f'(x_0)$. Ainsi~:
  \[
    \lim_{h\rightarrow 0} hg(h) = \lim_{h\rightarrow 0}f(x_0+h) - f(x_0)
  \]
  \[
    \Leftrightarrow f'(x_0)\lim_{h\rightarrow 0} h = \lim_{h\rightarrow 0}f(x_0+h) - f(x_0)
  \]
  \[
    \Leftrightarrow 0 = \lim_{h\rightarrow 0}f(x_0+h) - f(x_0)
  \]
  \[
    \Leftrightarrow 0 = \lim_{x\rightarrow x_0}f(x) - f(x_0)
  \]
  d'où finalement~:
  \[
    \lim_{x\rightarrow x_0}f(x) = f(x_0)
  \]
  La fonction est donc bien continue en $x_0$.

\end{notes}


\begin{frame}
  \frametitle{Dérivée en un point, VI}
  \hypertarget{slide_derivee_6}{}

  \begin{example}
    Soit $f(x) = |x|$ une fonction défine sur $\mathbb R$ à valeurs dans $\mathbb R^+$. Cette fonction est continue en $x=0$ mais n'est pas dérivable.\newline

    \begin{columns}[onlytextwidth]
      \begin{column}{.5\textwidth}
        {\small
          \begin{itemize}
          \item $f'_-(0)=\lim_{h\rightarrow 0^-} \frac{|0+h|-0}{h} = -1$, en effet on a~:
            \[
              \begin{split}
                f'_-(0) &= \lim_{h\rightarrow 0^-} \frac{|h|}{h}\\
                &= \lim_{h\rightarrow 0^-} \frac{-h}{h}\\
                &= -\lim_{h\rightarrow 0^-} 1\\
                &= -1\\
              \end{split}
            \]
          \item Mais $f'_+(0) = 1$

          \item Les dérivées à droite et à gauche sont différentes, la fonction n'est donc pas dérivable en $x=0$.
          \end{itemize}}
      \end{column}
      \begin{column}{.5\textwidth}
        \begin{tikzpicture}[scale=1]
          \begin{axis}[
            xticklabels={,,},
            yticklabels={,,},
            enlargelimits=true,
            grid style={dashed, gray!60},
            axis x line = bottom,
            axis y line = left,
            axis line style={thin},
            xmax = 5,
            xmin = -5,
            ymax = 5.5,
            ymin = -0.5,
            axis lines = middle,
            small,
            clip=false,
            ]
            \addplot[
            draw=black,
            thick,
            domain=-5:0,
            ]
            {-x};
            \addplot[
            draw=black,
            thick,
            domain=0:5,
            ]
            {x};
          \end{axis}
        \end{tikzpicture}
      \end{column}
    \end{columns}
  \end{example}

\end{frame}


\begin{frame}
  \frametitle{Dérivée sur un intervalle}
  \hypertarget{slide_derivee_7}{}


  \begin{definition}\label{dfn:derivable-intervalle}
    Soit $f$ une fonction de $E$ dans $\mathbb R$. Si la fonction $f$ est dérivable sur $E$ si ell est dérivable en tout point  $x\in E$.
  \end{definition}

  \bigskip

  \begin{example}
    La fonction $f(x) = x^2$ de $\mathbb R$ dans $\mathbb R_+$ est dérivable sur $\mathbb R$. En effet, pour tout $x\in\mathbb R$ nous avons~:
    \[
      \begin{split}
        f'(x) &= \lim_{h\rightarrow 0}\frac{(x+h)^2-x^2}{h}\\
        &= \lim_{h\rightarrow 0}\frac{x^2+2xh+h^2-x^2}{h} = \lim_{h\rightarrow 0}\frac{2xh+h^2}{h}\\
        &= \lim_{h\rightarrow 0} 2x+h = 2x + \lim_{h\rightarrow 0} h\\
        &= 2x
      \end{split}
    \]

  \end{example}

\end{frame}


\section{Règles de dérivation}


\begin{frame}
  \frametitle{Règles de dérivation}
  \framesubtitle{Dérivée d'une somme}
  \hypertarget{slide_derivee_somme_1}{}

  \begin{theorem}
    Soient $f(x)$ et $g(x)$ deux fonctions de $E$ dans $F$ dérivables sur $E$, alors~:
    \[
      (f(x)+g(x))' = f'(x) + g'(x)
    \]
  \end{theorem}

  \bigskip

  {\small \textbf{Preuve.} Notons $\ell(x) = f(x)+g(x)$, nous avons par définition~:
    \[
      \begin{split}
        \ell'(x) &= \lim_{h\rightarrow 0} \frac{\ell(x+h)-\ell(x)}{h}\\
        &= \lim_{h\rightarrow 0} \frac{f(x+h)+g(x+h)-f(x)-g(x)}{h}\\
        &= \lim_{h\rightarrow 0} \frac{f(x+h)-f(x)}{h} + \lim_{h\rightarrow 0} \frac{g(x+h)-g(x)}{h}\\
        &= f'(x)+g'(x) \qed
      \end{split}
    \]
  }

\end{frame}


\begin{frame}
  \frametitle{Règles de dérivation}
  \framesubtitle{Dérivée d'un produit (a)}
  \hypertarget{slide_derivee_produit_1}{}

  \begin{theorem}
    Soient $f(x)$ et $g(x)$ deux fonctions de $E$ dans $F$ dérivables sur $E$, alors~:
    \[
      (f(x)\cdot g(x))' = f'(x)g(x) + f(x)g'(x)
    \]
  \end{theorem}

  \bigskip

  {\small \textbf{Preuve.} Notons $\ell(x) = f(x) \cdot g(x)$, nous avons par définition~:
    \[
      \begin{split}
        \ell'(x) &= \lim_{h\rightarrow 0} \frac{\ell(x+h)-\ell(x)}{h}\\
        &= \lim_{h\rightarrow 0} \frac{f(x+h)g(x+h)-f(x)g(x)}{h} \\
        &= \lim_{h\rightarrow 0} \frac{(f(x+h)-f(x))g(x+h)+f(x)g(x+h)-f(x)g(x)}{h}\\
        &= \lim_{h\rightarrow 0} \frac{(f(x+h)-f(x))g(x+h)+f(x)(g(x+h)-g(x))}{h}\\
      \end{split}
    \]
  }

\end{frame}


\begin{frame}
  \frametitle{Règles de dérivation}
  \framesubtitle{Dérivée d'un produit (b)}
  \hypertarget{slide_derivee_produit_2}{}

  {\small On a donc~:
    \[
      \begin{split}
        \ell'(x) &= \lim_{h\rightarrow 0} \frac{(f(x+h)-f(x))g(x+h)}{h} + \lim_{h\rightarrow 0} \frac{f(x)(g(x+h)-g(x))}{h}\\
        &= \lim_{h\rightarrow 0} \frac{f(x+h)-f(x)}{h}\lim_{h\rightarrow 0} g(x+h) + f(x)\lim_{h\rightarrow 0} \frac{g(x+h)-g(x)}{h}\\
        &= \lim_{h\rightarrow 0} \frac{f(x+h)-f(x)}{h}g(x) + f(x)\lim_{h\rightarrow 0} \frac{g(x+h)-g(x)}{h}\\
        &= f'(x)g(x) + f(x)g'(x) \qed
      \end{split}
    \]
  }

\end{frame}


\begin{frame}
  \frametitle{Règles de dérivation}
  \framesubtitle{Dérivée d'un quotient}
  \hypertarget{slide_derivee_quotient_1}{}

  \begin{theorem}
    Soient $f(x)$ et $g(x)$ deux fonctions de $E$ dans $F$ dérivables sur $E$, avec $g(x)\neq 0\,\forall x\in E$, alors~:
    \[
      \left(\frac{f(x)}{g(x)}\right)' = \frac{f'(x)g(x) - f(x)g'(x)}{g(x)^2}
    \]
  \end{theorem}

  \bigskip

  {\small \textbf{Preuve.} Notons $\ell(x) = \frac{f(x)}{g(x)}$, nous avons de façon équivalente~:
    \[
      \ell(x)g(x) = f(x)
    \]
    en dérivant, et en exploitant la règle de dérivation d'un produit, il vient~:
    \[
      \ell'(x)g(x)+\ell(x)g'(x) = f'(x)
    \]
    soit~:
    \[
      \ell'(x) = \frac{f'(x)-\ell(x)g'(x)}{g(x)} = \frac{f'(x)g(x)-f(x)g'(x)}{g(x)^2} \qed
    \]
  }

\end{frame}


\begin{frame}
  \frametitle{Règles de dérivation}
  \framesubtitle{Dérivée d'une composition}
  \hypertarget{slide_derivee_composition_1}{}

  \begin{theorem}\label{thm:composition}
    Soient $f(x)$ une fonction dérivable de $E$ dans $F$ et $g(x)$ une fonction dérivable de $F$ dans $G$, alors~:
    \[
      g(f(x))' = g'(f(x))f'(x)
    \]
  \end{theorem}

  \bigskip

  \begin{example}
    Soit $f(x)$ une fonction de $\mathbb R$ dans $\mathbb R$ dérivable. Alors $g(x) = f(x)^2$ est une fonction dérivable sur $\mathbb R$ et~:
    \[
      g'(x) = 2f(x)f'(x)
    \]
  \end{example}

\end{frame}


\begin{notes}
  \textbf{Preuve du théorème \hyperlink{slide_derivee_composition_1}{\ref{thm:composition}}.} Commençons par noter que, par définition de la dérivée, si les fonctions $f$ et $g$ sont dérivables, alors on peut écrire~:
  \[
    f(x+h) =_0 f(x) + hf'(x) + \varepsilon(h)
  \]
  et
  \[
    g(y+k) =_0 g(y) + kg'(y) + \nu(k)
  \]
  avec $\lim_{h\rightarrow 0}\varepsilon(h)=0$ et $\lim_{k\rightarrow 0}\nu(k)=0$. Il s'agit d'un développement limité à l'ordre 1 des fonctions $f$ et $g$. La première équation nous dit que pour de petites valeurs de $h$ on peut approximer $f(x+h)$ par $f(x) + hf'(x)$, l'équation de la tangente à $f$ au point $x$, l'erreur d'approximation $\varepsilon(h)$ tend vers 0 quand $h$ tend vers 0. On a donc~:
  \[
    g\circ f(x+h) = g\Bigl(\underbrace{f(x)}_{y}+\underbrace{hf'(x) + \varepsilon(h)}_{k(h)}\Bigr)
  \]
  avec $\lim_{h\rightarrow 0}k(h) = 0$. En exploitant le dévelopement limité de $g$, on obtient~:
  \[
    g\circ f(x+h) = g(f(x)) + \Bigl( hf'(x) + \varepsilon(h) \Bigr) g'(f(x)) + \nu\Bigl( hf'(x) + \varepsilon(h) \Bigr)
  \]
  \[
    \Leftrightarrow g\circ f(x+h) = g(f(x)) + h g'(f(x))f'(x) + \underbrace{\nu\Bigl( hf'(x) + \varepsilon(h) \Bigr)+\varepsilon(h)g'(f(x))}_{\eta(h)}
  \]
  avec $\lim_{h\rightarrow 0}\eta(h) = 0$. On a donc~:
  \[
    g\circ f(x+h) = g(f(x)) + h g'(f(x))f'(x) + \eta(h)
  \]
  et par identification~:
  \[
    g(f(x))' = g'(f(x))f'(x)
  \]

\end{notes}


\begin{frame}
  \frametitle{Règles de dérivation}
  \framesubtitle{Dérivée d'une fonction réciproque}
  \hypertarget{slide_derivee_reciproque_1}{}

  \begin{theorem}\label{thm:composition}
    Soit $f$ une fonction dérivable et bijective de $E$ dans $F$, alors $f^{-1}$ est dérivable en tout point $x\in J$ tel que $f'(f^{-1}(x))\neq 0$ et on a~:
    \[
      (f^{-1})'(x) = \frac{1}{f'(f^{-1}(x))}
    \]
  \end{theorem}

  \bigskip

  {\small \textbf{Preuve.} Nous savons déjà que la composition d'une fonction et de sa réciproque (celle-ci existe car la fonction $f$ est supposée bijective) est égale à la fonction identité~:
    \[
      f\Bigl(f^{-1}(x)\Bigr) = x
    \]
    en dérivant les deux membres et utilisant le théorème \hyperlink{slide_derivee_composition_1}{\ref{thm:composition}} sur la dérivation des fonctions composées, nous avons~:
    \[
      f'\Bigl(f^{-1}(x)\Bigr)(f^{-1})'(x)  = 1
    \]
    \[
      (f^{-1})'(x)  = \frac{1}{f'\Bigl(f^{-1}(x)\Bigr)}\qed
    \]
  }

\end{frame}


\section{Dérivées de fonctions usuelles}


\begin{frame}
  \frametitle{Fonctions exponentielle et logarithme, I}
  \hypertarget{slide_derivee_exp_log_1}{}

  \begin{definition}
    L'équation fonctionnelle $f'(x) = f(x)$ avec $f(0)=1$ admet une unique solution~:
    \[
      f(x) = e^x
    \]
    la fonction exponentielle.
  \end{definition}

  \bigskip

  \begin{theorem}\label{thm:log_derivee_1}
    La dérivée de $f(x) = log(x)$, pour $x\in\mathbb R_+^*$ est $f'(x) = \frac{1}{x}$.
  \end{theorem}

  \bigskip

  \begin{theorem}\label{thm:log_derivee_2}
    Soit $f$ une fonction à valeurs dans $\mathbb R_+^{\star}$, alors~:
    \[
      \Bigl(\log f(x)\Bigr)' = \frac{f'(x)}{f(x)}
    \]
  \end{theorem}

\end{frame}


\begin{notes}

  \textbf{Preuve du théorème \hyperlink{slide_derivee_exp_log_1}{\ref{thm:log_derivee_1}}.} On sait que la fonction logarithme népérien est la fonction réciproque de la fonction expolnentielle, on a donc~:
  \[
    e^{\log x} = x
  \]
  En dérivant~:
  \[
    \left(e^{\log x}\right)' = 1
  \]
  Pour dériver le l'exponentielle d'une fonction, on utilise le théorème \hyperlink{slide_derivee_composition_1}{\ref{thm:composition}} de dérivation des fonctions composées. De façon générale, on a donc si $f(x)$ est une fonction dérivable~:
  \[
    \left(e^{f(x)}\right)' = f'(x)e^{f(x)}
  \]
  puisque par définition la dérivée de la fonction exponentielle est la fonction exponentielle. Dans le cas qui nous intéresse, il vient~:
  \[
    \Bigl( \log x \Bigr)' e^{\log x} = 1
  \]
  et donc~:
  \[
    x \Bigl( \log x \Bigr)' = 1
  \]
  d'où~:
  \[
    \Bigl( \log x \Bigr)' = \frac{1}{x}
  \]

  \bigskip

  \textbf{Preuve du théorème \hyperlink{slide_derivee_exp_log_1}{\ref{thm:log_derivee_2}}.} direct avec le théorème \hyperlink{slide_derivee_composition_1}{\ref{thm:composition}}.

\end{notes}


\begin{frame}
  \frametitle{Fonctions exponentielle et logarithme, II}
  \hypertarget{slide_derivee_exp_log_1}{}

  \bigskip

  \begin{corollary}
    Si $f(x)$ est une fonction positive, alors~:
    \[
      f'(x) = f(x) \Bigl(\log f(x)\Bigr)'
    \]
  \end{corollary}

  \bigskip

  \begin{example}
    Soit la fonction $f(x) = x^x$ définie sur $\mathbb R_+^{\star}$. On a~:
    \[
      \begin{split}
        f'(x) &= x^x \Bigl(\log x^x\Bigr)'\\
        &= x^x \Bigl(x\log x\Bigr)'\\
        &= x^x \Bigl(\frac{x}{x}+\log x\Bigr)\\
        &= x^x \Bigl(1+\log x\Bigr)
      \end{split}
    \]
  \end{example}

\end{frame}


\begin{frame}
  \frametitle{Dérivées des fonctions usuelles}
  \hypertarget{slide_derivee_usuelles}{}

  \bigskip
  \renewcommand{\arraystretch}{1.8}
  \begin{table}[H]
    \centering
    {\small
      \begin{tabular}{l|c|l|c|l}
        \hline
        $D_f$ & $f(x)$ & $D_{f'}$ & $f'(x)$ & Remarques\\ \hline
        $\R$ & $k$ & $\R$ & 0 & $k\in\R$\\
        $\R$ & $kx$ & $\R$ & $k$ & $k\in\R$\\
        $\R$ & $x^n$ & $\R$ & $nx^{n-1}$ & $n\in\N$\\
        $\R^{\star}$ & $\frac{1}{x^n}$ & $\R^{\star}$ & $-\frac{n}{x^{n+1}}$ & $n\in\N$\\
        $\R_+^\star$ & $x^{\alpha}$  & $\R_+^{\star}$ &  $\alpha x^{\alpha - 1}$ & $\alpha\in\R$\\
        % </math> constante réelle. Fonction prolongeable par continuité en 0 si {{math|''α'' ≥ 0}}, et de prolongée dérivable en 0 si {{math|''α'' ≥ 1}}.
        $\R^\star$ & $\log |x|$ & $\R^\star$ & $\frac{1}{x}$ & \\
        $\R^\star$ & $\log_a |x|$ & $\R^\star$ & $\frac{1}{x \log a}$ & $a>0$ et $a\neq 1$\\
        $\R$ & $e^x$  & $\R$  $e^x$ & \\
        $\R$ & $a^x$ & $\R$ & $a^x \log a$ & $a > 0$ \\ \hline\hline
      \end{tabular}}
  \end{table}

\end{frame}


\begin{frame}
  \frametitle{Fonction puissance: $x^n$, avec $n\in\mathbb N$}

  \begin{itemize}

  \item Montrons que $\left(x^n\right)' = nx^{n-1}$ par récurrence.\newline

  \item Au rang 1, nous avons $x' = 1\times x^0 = 1$.\newline

  \item Supposons que $\left(x^n\right)' = nx^{n-1}$ et montrons que l'on doit alors avoir $\left(x^{n+1}\right)' = (n+1)x^{n}$~:
    \[
      \begin{split}
        \left(x^{n+1}\right)' &= \left(x^{n+1}\right)'\\
        &=\left(xx^{n}\right)'\\
        &=x^n + x n x^{n-1}\\
        &=x^n + n x^{n} = (n+1)x^{n}\qed
      \end{split}
    \]

  \end{itemize}
\end{frame}




\end{document}

% Local Variables:
% ispell-check-comments: exclusive
% ispell-local-dictionary: "francais"
% TeX-master: t
% End: