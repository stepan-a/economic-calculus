% Adapted from https://tex.stackexchange.com/questions/70081/improved-unit-circle-animation

\documentclass[pstricks,border=0pt]{standalone}
\usepackage{ifthen}
\usepackage{pst-plot,pst-node}
\usepackage[nomessages]{fp}

\usepackage{pst-node}
\usepackage{pst-eucl}
\usepackage{pstricks-add}

% User Defined Constants
\FPeval\NumberOfPoints{18}

% Internally Defined Data
\FPeval\DeltaX{round(2*pi/(NumberOfPoints-1):3)}

\def\sine#1{%
  \begin{pspicture}(-0.25,-1.25)(\dimexpr\psPiTwo\psxunit+0.6\psxunit\relax,1.60)
    \psplot[algebraic,linecolor=blue]{0}{#1}{sin(x)}
    \pnode(!#1 dup RadtoDeg sin){B}
    \psaxes
    [
    linecolor=gray,
    ticksize=2pt -2pt,
    labelFontSize=\scriptscriptstyle,
    trigLabels=true,trigLabelBase=2,dx=\psPiH
    ]{->}(0,0)(-0.10,-1.10)(\dimexpr\psPiTwo\psxunit+0.3\psxunit\relax,1.30)[$x$,0][$y$,90]
  \end{pspicture}}

\def\circle#1{%
  \begin{pspicture}(-1.25,-1.25)(1.6,1.6)
    \pscircle[linecolor=gray](0,0){1}
    \pnode(!#1 RadtoDeg dup cos exch sin){A}
    \psline{->}(A)
    \psaxes
    [
    linecolor=gray,
    ticksize=2pt -2pt,
    labelFontSize=\scriptscriptstyle,
    ]{->}(0,0)(-1.1,-1.1)(1.3,1.3)
  \end{pspicture}}

\begin{document}

\newcounter{loopindex}
\setcounter{loopindex}{0}
\multido{\nx=0.000+\DeltaX}{\NumberOfPoints}
{
  \addtocounter{loopindex}{1}
  \begin{pspicture}[showgrid=false](0,-1.6)(10.2,1.6)
    \rput(1.375,0){\circle\nx}
    \rput(6.515,0){\sine\nx}
    \ifnum \value{loopindex} > 1 %
    \ifnum \value{loopindex} < 18
    \pcline[linestyle=dotted,linewidth=.8pt, linecolor=red](A)(B)
    \fi
    \fi
  \end{pspicture}
}
\end{document}

%%% Local Variables:
%%% mode: latex
%%% TeX-master: t
%%% End:
