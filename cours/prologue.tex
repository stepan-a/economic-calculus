\synctex=1

\documentclass[10pt,notheorems]{beamer}

\usepackage{etex}
\usepackage{fourier-orns}
\usepackage{ccicons}
\usepackage{amssymb}
\usepackage{amstext}
\usepackage{amsbsy}
\usepackage{amsopn}
\usepackage{amscd}
\usepackage{amsxtra}
\usepackage{amsthm}
\usepackage{float}
\usepackage{color, colortbl}
\usepackage{mathrsfs}
\usepackage{bm}
\usepackage{lastpage}
\usepackage[nice]{nicefrac}
\usepackage{setspace}
\usepackage{ragged2e}
\usepackage{listings}
\usepackage{algorithms/algorithm}
\usepackage{algorithms/algorithmic}
\usepackage[frenchb]{babel}
\usepackage{tikz,pgfplots}
\pgfplotsset{compat=newest}
\usetikzlibrary{patterns, arrows, decorations.pathreplacing, decorations.markings, calc}
\pgfplotsset{plot coordinates/math parser=false}
\newlength\figureheight
\newlength\figurewidth
\usepackage[utf8x]{inputenc}
\usepackage{cancel}
\usepackage{tikz-qtree}
\usepackage{dcolumn}
\usepackage{adjustbox}
\usepackage{environ}

\newcommand{\trace}{\mathrm{tr}}
\newcommand{\vect}{\mathrm{vec}}
\newcommand{\tracarg}[1]{\mathrm{tr}\left\{#1\right\}}
\newcommand{\vectarg}[1]{\mathrm{vec}\left(#1\right)}
\newcommand{\vecth}[1]{\mathrm{vech}\left(#1\right)}
\newcommand{\iid}[2]{\mathrm{iid}\left(#1,#2\right)}
\newcommand{\normal}[2]{\mathcal N\left(#1,#2\right)}
\newcommand{\dynare}{\href{http://www.dynare.org}{\color{blue}Dynare}}
\newcommand{\sample}{\mathcal Y_T}
\newcommand{\samplet}[1]{\mathcal Y_{#1}}
\newcommand{\slidetitle}[1]{\fancyhead[L]{\textsc{#1}}}

\newcommand{\R}{{\mathbb R}}
\newcommand{\C}{{\mathbb C}}
\newcommand{\N}{{\mathbb N}}
\newcommand{\Z}{{\mathbb Z}}
\newcommand{\binomial}[2]{\begin{pmatrix} #1 \\ #2 \end{pmatrix}}
\newcommand{\bigO}[1]{\mathcal O \left(#1\right)}
\newcommand{\red}{\color{red}}
\newcommand{\blue}{\color{blue}}

\newcolumntype{d}{D{.}{.}{-1}}
\definecolor{gray}{gray}{0.9}
\newcolumntype{g}{>{\columncolor{gray}}c}

\setbeamertemplate{theorems}[numbered]

\theoremstyle{plain}
\newtheorem{theorem}{Théorème}

\theoremstyle{definition} % insert bellow all blocks you want in normal text
\newtheorem{definition}{Définition}
\newtheorem{properties}{Propriétés}
\newtheorem{example}{Exemple}
\newtheorem*{idea}{Éléments de preuve} % no numbered block

\setbeamertemplate{footline}{
{\hfill\vspace*{1pt}\href{http://creativecommons.org/licenses/by-sa/3.0/legalcode}{\ccbysa}\hspace{.1cm}
\raisebox{-.075cm}{\href{https://git.adjemian.eu/University/economic-calculus}{\includegraphics[scale=.1]{../img/gitlab.png}}}
}\hspace{1cm}}

\setbeamertemplate{navigation symbols}{}
\setbeamertemplate{blocks}[rounded][shadow=true]
\setbeamertemplate{caption}[numbered]

\NewEnviron{notes}{\justifying\tiny\begin{spacing}{1.0}\BODY\vfill\pagebreak\end{spacing}}

\newenvironment{exercise}[1]
{\bgroup \small\begin{block}{Ex. #1}}
{\end{block}\egroup}

\newenvironment{defn}[1]
{\bgroup \small\begin{block}{Définition. #1}}
{\end{block}\egroup}

\newenvironment{exemple}[1]
{\bgroup \small\begin{block}{Exemple. #1}}
{\end{block}\egroup}

\begin{document}

\title{Calcul Économique\\\small{I. Prologue}}
\author[S. Adjemian]{Stéphane Adjemian}
\institute{\texttt{stepan@adjemian.eu}} \date{Septembre 2020}

\begin{frame}
  \titlepage{}
\end{frame}

\begin{frame}
  \frametitle{Plan}
  \tableofcontents
\end{frame}

\section{Vocabulaire}

\begin{frame}
  \frametitle{Un peu de vocabulaire}
  \hypertarget{slide_vocabulaire}{}

  \begin{itemize}

  \item \textbf{Axiome}. Un énoncé supposé vrai a priori, sans preuve,
    et qu'on ne cherche pas à démontrer.\newline

  \item \textbf{Proposition}. Un énoncé qui peut être vrai (V) ou faux
    (F).\newline

  \item \textbf{Théorème}. Une proposition (démontrée) vraie.\newline

  \item \textbf{Corollaire}. Un « petit » théorème conséquence
    d'un autre théorème.\newline

  \item \textbf{Lemme}. Un « petit » théorème préparatoire d'un
    autre théorème.\newline

  \item \textbf{Conjecture}. Une proposition supposée vraie, sans
    preuve, tant que l'on n'exhibe pas un contre exemple.\newline

  \end{itemize}

\end{frame}

\section{Calcul propositionnel}

\subsection{Table de vérité}

\begin{frame}
  \frametitle{Calcul propositionnel}
  \framesubtitle{Proposition et table de vérité}
  \hypertarget{slide_proposition_et_table_de_verite}{}

  \begin{itemize}

  \item Une proposition est un énoncé pouvant être vrai ou faux. On
    dit que vrai (V) ou faux (F) sont les valeurs de vérité d'une
    proposition.\newline

  \item Dans la suite on représentera souvent les valeurs de vérité
    d'une proposition, notée P par exemple, dans une table~:

  \end{itemize}

  \begin{table}[H]

    \centering
    \begin{tabular}[H]{|c|}
      \hline
      P \\ \hline
      V \\
      F \\
      \hline\hline
    \end{tabular}
    \caption{Table de vérité}
    \label{tab:verite}
  \end{table}

\end{frame}

\subsection{Équivalence logique}

\begin{frame}
  \frametitle{Calcul propositionnel}
  \framesubtitle{Équivalence logique, I}
  \hypertarget{slide_equivalence_logique_1}{}

  \begin{definition}\label{def:equivalence}
    Deux propositions $P$ et $Q$ sont équivalentes si elles sont
    simultanément vraies et simultanément fausses. On note
    $P \Leftrightarrow Q$.
  \end{definition}


  \begin{table}[H]

    \centering
    \begin{tabular}[H]{|cc|c|}
      \hline
      $P$ & $Q$ & $P \Leftrightarrow Q$\\ \hline
      V & V & V \\
      V & F & F \\
      F & V & F \\
      F & F & V \\
      \hline\hline
    \end{tabular}
    \caption{Équivalence logique}
    \label{tab:equivalence}
  \end{table}

  \bigskip

  L'équivalence logique entre propositions joue un rôle analogue à
  l'égalité entre des nombres (ou des ensembles comme nous le verrons plus loin).\newline

  Nous venons de créer une nouvelle proposition
  ($P \Leftrightarrow Q$) à partir de deux propositions ($P$ et $Q$)

\end{frame}

\begin{frame}
  \frametitle{Calcul propositionnel}
  \framesubtitle{Équivalence logique, II}
  \hypertarget{slide_equivalence_logique_2}{}

  On note que la table de vérité contient $4 = 2^{2}$ lignes.\newline

  Plus généralement si nous devons construire une nouvelle proposition
  à partir de $n$ propositions, la table de vérité devra contenir
  $2^{n}$ lignes.\newline

  On note aussi la méthode utilisée pour définir les valeurs de vérité
  des propositions $P$ et $Q$. On commence par poser que la
  proposition $P$ est vraie, puis on considère les deux valeurs de
  vérité possibles pour la seconde proposition $Q$. Après on pose que
  la proposition $P$ est fausse\ldots

    \begin{example}
      Les expressions 3+2 et 4+1, même si elles ne sont pas
      identiques, ont la même valeur, on dit que ces expressions sont
      égales ($3+2=4+1$). Les propositions $(x^{2} = 1)$ et
      $(x=1\text{ ou }x=-1)$ sont distinctes mais équivalentes, on
      écrit~:
      \[
        (x^{2} = 1) \Leftrightarrow (x=1\text{ ou }x=-1)
      \]
    \end{example}

  \end{frame}

  \subsection{Négation d'une proposition}

  \begin{frame}
    \frametitle{Calcul propositionnel}
    \framesubtitle{Négation}
    \hypertarget{slide_negation}{}

    \begin{definition}\label{definition:negation}

    La négation d'une proposition $P$, notée $\bar P$, change sa
    valeur de vérité.
  \end{definition}

  \bigskip

  \begin{table}[H]

     \centering
     \begin{tabular}[H]{|c|c|}
       \hline
       $P$ & $\bar P$\\ \hline
       V & F \\
       F & V \\
       \hline\hline
     \end{tabular}
     \caption{Négation}
     \label{tab:negation}
   \end{table}

   \begin{theorem}\label{theorem:negation}

      Soit $P$ une proposition, alors
      $\bar{\bar P} \Leftrightarrow P$.
    \end{theorem}

    \medskip

    $\looparrowright$ À montrer avec une table de vérité.
  \end{frame}


  \subsection{Connecteurs logiques « et » et « ou »}

  \begin{frame}
    \frametitle{Connecteurs logiques}
    \framesubtitle{La conjonction $\land$, I}
    \hypertarget{slide_conjonction_1}{}

    \begin{definition}\label{def:conjonction}

    La conjonction de deux propositions $P$ et $Q$, on note $P\land Q$
    et on lit « $P$ et $Q$ », est vraie si et seulement si les
    deux propositions sont vraies.
  \end{definition}

  \bigskip

  \begin{table}[H]

    \centering
    \begin{tabular}[H]{|cc|c|}
      \hline
      $P$ & $Q$ & $P \land Q$\\ \hline
      V & V & V \\
      V & F & F \\
      F & V & F \\
      F & F & F \\
      \hline\hline
    \end{tabular}
    \caption{Conjonction logique}
    \label{tab:conjonction}
  \end{table}

  \bigskip

  On retient que la conjonction de deux propositions est fausse dès
  lors qu'au moins une proposition est fausse.

  \end{frame}

  \begin{frame}
    \frametitle{Connecteurs logiques}
    \framesubtitle{La conjonction $\land$, II}
    \hypertarget{slide_conjonction_2}{}

    \begin{properties}\label{properties:conjonction}

    Soient $P$, $Q$ et $R$ des propositions. La conjonction satisfait
    les propriétés suivantes~:
    \begin{enumerate}
    \item \textbf{Idempotence:} $(P \land P) \Leftrightarrow P$.
    \item \textbf{Commutativité:}
      $(P \land Q) \Leftrightarrow (Q \land P)$.
    \item \textbf{Associativité:}
      $((P \land Q)\land R) \Leftrightarrow (P \land (Q\land R)$.
    \item \textbf{Non contradiction:} La proposition $P \land \bar P$
      est fausse.
    \end{enumerate}
  \end{properties}

  \bigskip

  On démontre facilement ces propriétés en utilisant des tables de
  vérité.\newline

  On note que dans la dernière propriété (4) nous venons de construire
  une proposition (disons $Q$) dont la valeur de vérité est certaine,
  alors que nous ne connaissons pas la valeur de vérité de la
  proposition de départ ($P$).

  \end{frame}


  \begin{notes}
    \begin{itemize}

    \item On montre la première propriété à l'aide d'une table de
      vérité~:
      \begin{table}
        \centering
        \begin{tabular}[H]{|c|c|}
          \hline
          $P$ & $P \land P$ \\ \hline
          V & V \\
          F & F \\
          \hline\hline
        \end{tabular}
      \end{table}
      Puisque les colonnes ont toujours les mêmes valeurs sur chaque
      ligne, les deux propositions $P\land P$ et $P$ sont équivalentes
      (voir la définition \hyperlink{slide_equivalence_logique_1}{\ref{def:equivalence}}).

    \item On procède de la même façon pour la deuxième propriété. Les
      propositions à droite et à gauche du symbole d'équivalence
      $\Leftrightarrow$ font intervenir deux propositions $P$ et
      $Q$. La table de vérité doit donc contenir quatre lignes qui
      correspondent aux couples ordonnés possibles pour les
      valeurs de $P$ et $Q$~:
      \begin{table}[H]
        \centering
        \begin{tabular}[H]{|cc|cc|}
          \hline
          $P$ & $Q$ & $P \land Q$ & $Q \land P$\\ \hline
          V & V & V & V\\
          V & F & F & F\\
          F & V & F & F\\
          F & F & F & F \\
          \hline\hline
        \end{tabular}
      \end{table}
      On observe que les troisième et quatrième colonnes ont toujours
      la même valeur sur chaque ligne, les deux propositions associées
      $P \land Q$ et $Q \land P$ sont donc
      équivalentes.

    \item Nous suivons la même démarche pour la troisième
      propriété. Cette fois nous construisons des propositions à
      partir de trois propositions de base $P$, $Q$ et $R$. La table
      de vérité doit donc contenir huit lignes (c'est-à-dire $2^3$ lignes)~:
      \begin{table}[H]
        \centering
        \begin{tabular}[H]{|ccc|cgcg|}
          \hline
          $P$ & $Q$ & $R$ & $P\land Q$ & $(P\land Q)\land R$ & $Q\land R$ & $P \land (Q \land R)$ \\ \hline
          V & V & V & V & V & V & V\\
          V & V & F & V & F & F & F\\
          V & F & V & F & F & F & F\\
          V & F & F & F & F & F & F\\
          F & V & V & F & F & V & F\\
          F & V & F & F & F & F & F\\
          F & F & V & F & F & F & F\\
          F & F & F & F & F & F & F\\
          \hline\hline
        \end{tabular}
      \end{table}
      On note que la cinquième et la septième colonnes sont identiques, ce qui démontre la troisième propriété.

    \item Pour la dernière propriété la table de vérité contient seulement deux lignes~:
      \begin{table}
        \centering
        \begin{tabular}[H]{|c|cc|}
          \hline
          $P$ & $\bar P$ & $P \land \bar P$\\ \hline
          V & F & F\\
          F & V & F\\
          \hline\hline
        \end{tabular}
      \end{table}
      On remarque que la dernière colonne est toujours fausse.

    \end{itemize}

  \end{notes}


  \begin{frame}
    \frametitle{Connecteurs logiques}
    \framesubtitle{La disjonction $\lor$, I}
    \hypertarget{slide_disjonction_1}{}

    \begin{definition}\label{def:disjonction}

    La disjonction de deux propositions $P$ et $Q$, on note $P\lor Q$
    et on lit « $P$ ou $Q$ », est fausse si et seulement si les
    deux propositions sont fausses.
  \end{definition}

  \bigskip

  \begin{table}[H]

    \centering
    \begin{tabular}[H]{|cc|c|}
      \hline
      $P$ & $Q$ & $P \lor Q$\\ \hline
      V & V & V \\
      V & F & V \\
      F & V & V \\
      F & F & F \\
      \hline\hline
    \end{tabular}
    \caption{Disjonction logique}
    \label{tab:disjonction}
  \end{table}

  \bigskip

  On retient que la disjonction de deux propositions est vraie dès
  lors qu'au moins une proposition est vraie.

  \bigskip

  On considère ici une définition \textit{inclusive} de la disjonction (par opposition à une définition \textit{exclusive}).

  \end{frame}


  \begin{frame}
    \frametitle{Connecteurs logiques}
    \framesubtitle{La disjonction $\lor$, II}
    \hypertarget{slide_disjonction_2}{}

    \begin{properties}\label{properties:disjonction}
    Soient $P$, $Q$ et $R$ des propositions. La disjonction satisfait
    les propriétés suivantes~:
    \begin{enumerate}
    \item \textbf{Idempotence:} $(P \lor P) \Leftrightarrow P$.
    \item \textbf{Commutativité:}
      $(P \lor Q) \Leftrightarrow (Q \lor P)$.
    \item \textbf{Associativité:}
      $((P \lor Q)\lor R) \Leftrightarrow (P \lor (Q\lor R)$.
    \item La proposition $P \lor \bar P$
      est vraie.
    \end{enumerate}
  \end{properties}

  \bigskip

  On démontre facilement ces propriétés en utilisant des tables de
  vérité.\newline

  Nous retrouvons les mêmes propriétés que pour la conjonction (à part pour la dernière).

  \end{frame}


  \begin{notes}
    \begin{itemize}
    \item On montre la première propriété à l'aide d'une table de
      vérité~:
      \begin{table}
        \centering
        \begin{tabular}[H]{|c|c|}
          \hline
          $P$ & $P \lor P$ \\ \hline
          V & V \\
          F & F \\
          \hline\hline
        \end{tabular}
      \end{table}
      Puisque les colonnes ont toujours les mêmes valeurs sur chaque
      ligne, les deux propositions $P\lor P$ et $P$ sont équivalentes
      (voir la définition \hyperlink{slide_equivalence_logique_1}{\ref{def:equivalence}}).
    \item On procède de la même façon pour la deuxième propriété. Les
      propositions à droite et à gauche du symbole d'équivalence
      $\Leftrightarrow$ font intervenir deux propositions $P$ et
      $Q$. La table de vérité doit donc contenir quatre lignes qui
      correspondent aux couples ordonnés possibles pour les
      valeurs de $P$ et $Q$~:
      \begin{table}[H]
        \centering
        \begin{tabular}[H]{|cc|cc|}
          \hline
          $P$ & $Q$ & $P \lor Q$ & $Q \lor P$\\ \hline
          V & V & V & V\\
          V & F & V & V\\
          F & V & V & V\\
          F & F & F & F \\
          \hline\hline
        \end{tabular}
      \end{table}
      On observe que les troisième et quatrième colonnes ont toujours
      la même valeur sur chaque ligne, les deux propositions associées
      $P \lor Q$ et $Q \lor P$ sont donc
      équivalentes.
    \item Nous suivons la même démarche pour la troisième
      propriété. Cette fois nous construisons des propositions à
      partir de trois propositions de base $P$, $Q$ et $R$. La table
      de vérité doit donc contenir huit lignes~:
      \begin{table}[H]
        \centering
        \begin{tabular}[H]{|ccc|cgcg|}
          \hline
          $P$ & $Q$ & $R$ & $P\lor Q$ & $(P\lor Q)\lor R$ & $Q\lor R$ & $P \lor (Q \lor R)$ \\ \hline
          V & V & V & V & V & V & V\\
          V & V & F & V & V & V & V\\
          V & F & V & V & V & V & V\\
          V & F & F & V & V & F & V\\
          F & V & V & V & V & V & V\\
          F & V & F & V & V & V & V\\
          F & F & V & F & V & V & V\\
          F & F & F & F & F & F & F\\
          \hline\hline
        \end{tabular}
      \end{table}
      On note que les cinquième et septième colonnes sont identiques,
      ce qui démontre la troisième propriété.
    \item Pour la dernière propriété la table de vérité contient
      seulement deux lignes~:
      \begin{table}
        \centering
        \begin{tabular}[H]{|c|cc|}
          \hline
          $P$ & $\bar P$ & $P \lor \bar P$\\ \hline
          V & F & V\\
          F & V & V\\
          \hline\hline
        \end{tabular}
      \end{table}
      On remarque que la dernière colonne est toujours vraie.

    \end{itemize}

  \end{notes}


  \begin{frame}
    \frametitle{Connecteurs logiques}
    \framesubtitle{Disjonction, conjonction et négation}
    \hypertarget{slide_disjonction_conjonction_et_negation}{}

    \begin{properties}[Distributivité]\label{properties:distributivite}
    Soient trois propositions $P$, $Q$ et $R$, on a~:
    \begin{enumerate}
    \item $(P\land Q)\lor R \Leftrightarrow (P\lor R) \land (Q\lor R)$
    \item $(P\lor Q)\land R \Leftrightarrow (P\land R) \lor (Q\land R)$
    \end{enumerate}
  \end{properties}

  \bigskip

  \begin{theorem}[Loi de Morgan]\label{theorem:morgan}

    Soient $P$ et $Q$ deux propositions, on a~:
    \begin{enumerate}
    \item $\overline{P \land Q} \Leftrightarrow \bar P \lor \bar Q$.
    \item $\overline{P \lor Q} \Leftrightarrow \bar P \land \bar Q$.
    \end{enumerate}
  \end{theorem}

  \bigskip

  La loi de Morgan souligne le lien entre la conjonction et la disjoinction, que nous pouvions déjà anticiper en comparant les tables de vérité respectives.\newline

  La négation d'une conjonction est la disjonction des négations.

  \end{frame}


  \begin{notes}
    \mbox{}
    \textbf{Preuve de la propriété \hyperlink{slide_disjonction_conjonction_et_negation}{\ref{properties:distributivite}}.}
    On montre seulement la première propriété, pour la seconde on suit
    exactement la même approche, toujours en utilisant une table de
    vérité. On dénombre huit triplets de valeurs de vérité pour les
    propriétés $P$, $Q$ et $R$. La table est donc formée de huit
    lignes~:
    \begin{table}[H]
      \centering
      \begin{tabular}[H]{|ccc|cgccg|}
        \hline
        $P$ & $Q$ & $R$ & $P\land Q$ & $(P\land Q)\lor R$ & $P\lor R$ & $Q \lor R$ & $(P\lor R)\land (Q \lor R)$ \\ \hline
        V & V & V & V & V & V & V & V \\
        V & V & F & V & V & V & V & V \\
        V & F & V & F & V & V & V & V \\
        V & F & F & F & F & V & F & F \\
        F & V & V & F & V & V & V & V \\
        F & V & F & F & F & F & V & F \\
        F & F & V & F & V & V & V & V \\
        F & F & F & F & F & F & F & F \\
        \hline\hline
      \end{tabular}
    \end{table}

    \bigskip\bigskip

    \textbf{Preuve du théorème \hyperlink{slide_disjonction_conjonction_et_negation}{\ref{theorem:morgan}}.} On démontre seulement le premier point, on peut suivre la même approche pour le second. On utilise une table de vérité de quatre lignes~:
    \begin{table}[H]
      \centering
      \begin{tabular}[H]{|cc|cgccg|}
        \hline
        $P$ & $Q$ & $P\land Q$ & $\overline{P\land Q}$ & $\bar P$ & $\bar Q$ & $\bar P \lor \bar Q$ \\ \hline
        V & V & V & F & F & F & F\\
        V & F & F & V & F & V & V\\
        F & V & F & V & V & F & V\\
        F & F & F & V & V & V & V\\
        \hline\hline
      \end{tabular}
    \end{table}

    \vspace*{\fill}

  \end{notes}
\subsection{Implication logique}

\begin{frame}
  \frametitle{Implication logique}
  \framesubtitle{Le connecteur $\Rightarrow$, I}
  \hypertarget{slide_implication_1}{}
  \begin{definition}

    Soient deux propositions $P$ et $Q$. La proposition
    $P\Rightarrow Q$, on dit « $P$ implique $Q$ », est fausse si $P$
    est vraie \textbf{et} $Q$ est fausse, la proposition
    $P\Rightarrow Q$ est vraie sinon.
  \end{definition}


  \begin{table}[H]

    \centering
    \begin{tabular}[H]{|cc|c|}
      \hline
      $P$ & $Q$ & $P \Rightarrow Q$\\ \hline
      V & V & V \\
      V & F & F \\
      F & V & V \\
      F & F & V \\
      \hline\hline
    \end{tabular}
    \caption{L'implication logique}
    \label{tab:implication}
  \end{table}

  Ce connecteur logique peut paraître peu intuitif\ldots\newline

  Si l'implication est vraie, $Q$ vraie peut être déduite de $P$ vraie.\newline

  Si l'implication est vraie, on ne peut rien inférer sur la vérité de $Q$ lorsque $P$ est fausse.\newline

\end{frame}


\begin{frame}
  \frametitle{Implication logique}
  \framesubtitle{Le connecteur $\Rightarrow$, II}
  \hypertarget{slide_implication_2}{}

  \begin{example}\label{ex:implication:1}

    Considérons une proposition qui ne devrait pas vous causer le
    moindre doute~:

    \medskip

    \begin{quotation}
      {Pour tout entier relatif $n$, si $n>2$ alors $n^2>4$}
    \end{quotation}

    \medskip

    On note $R(n)$ la proposition précédente, et on pose $P(n)$~: «
    $n>2$ » et $Q(n)$~: « $n^2>4$ ». On peut vérifier que pour
    différentes valeurs de $n$ on retrouve trois des lignes de
    la table de vérité de l'implication logique~:
    \begin{table}[H]
      \centering
      \begin{tabular}[H]{|r|cc|c|}
        \hline
        $n$ & $P(n)$ & $Q(n)$ & $R(n)$\\ \hline
        3 & V & V & V\\
        -3 & F & V & V\\
        1 & F & F & V\\
        \hline\hline
      \end{tabular}
    \end{table}
    où l'implication est vraie, mais pas le cas où l'implication est
    fausse (puisque $R(n)$ est vraie pour tout $n$).
  \end{example}

\end{frame}


\begin{frame}
  \frametitle{Implication logique}
  \framesubtitle{Le connecteur $\Rightarrow$, III}
  \hypertarget{slide_implication_3}{}

  \begin{example}[$F\Rightarrow F$ est vraie]\label{ex:implication:2}

    Montrons que si $10^n+1$ est divisible par 9, alors $10^{n+1}+1$
    est divisible par 9, pour tout entier $n$.\newline

    La première proposition, $10^n+1$ est divisible par 9, exige
    l'existence d'un entier $k$ tel que $10^n+1 = 9k$. Nous avons~:
    \[
      10^{n+1}+1 = 10\times 10^n+1 =
      10\times(10^{n}+1)-9=9\times(10k-1)
    \]
    et donc $10^{n+1}+1$ est divisible par 9.\newline

    Clairement la proposition « $10^n+1$ divisible par 9 implique
    $10^{n+1}+1$ divisible par 9 » est vraie. Il est tout aussi
    évident que la proposition « $10^n+1$ divisible par 9 » est
    fausse. Il suffit de considérer le cas n=0 pour s'en convaincre.
  \end{example}

\end{frame}


\begin{frame}
  \frametitle{Implication logique}
  \framesubtitle{Le connecteur $\Rightarrow$, IV}
  \hypertarget{slide_implication_4}{}

  \begin{example}[$F\Rightarrow V$ est vraie]\label{ex:implication:3}

    Soit la proposition $P~:$ « $2=3$ et $2=1$ ». Cette proposition est
    clairement fausse, néanmoins en sommant les deux égalités on
    obtient~:
    \[
      \begin{split}
        2+2 &= 3+1\\
        4 &= 4
      \end{split}
    \]
    la proposition $Q$~: « $4=4$ » est évidemment vraie (forfuitement). Si la
    proposition $P\Rightarrow Q$ est vraie, alors $Q$ peut être vraie
    même si $P$ est fausse. Autrement dit un raisonnement correcte
    peut (par chance) amener à un résultat correct même si le prémisse
    est faux.
  \end{example}

\end{frame}


\begin{frame}
  \frametitle{Implication logique}
  \framesubtitle{$\Rightarrow$, $\land$, $\Rightarrow$}
  \hypertarget{slide_implication_transitivite}{}

  \begin{theorem}[Transitivité]\label{theorem:implication:transitivite}

    Soient $P$, $Q$ et $R$ trois propositions. On a~:
    \[
      ((P\Rightarrow Q) \land (Q\Rightarrow R)) \Rightarrow
      (P\Rightarrow R)
    \]
  \end{theorem}

  \bigskip

  Si $P$ est vraie et si $P\Rightarrow Q$ est vraie, alors $Q$ est
  vraie (Cf. la première ligne de la table de vérité
  \hyperlink{slide_implication_1}{\ref{tab:implication}}). Si $Q\Rightarrow R$ est vraie, alors puisque
  $Q$ est vraie on en déduit que $R$ est vraie.\newline

  Cette propriété de transitivité sera souvent exploitée.\newline

  Pour démontrer le théorème on procède toujours en construisant une
  table de vérité (à huit lignes).\newline

\end{frame}


\begin{notes}
  \textbf{Preuve du théorème \hyperlink{slide_implication_transitivite}{\ref{theorem:implication:transitivite}}.}
  La table de vérité contient huit lignes psuique nous travaillons
  avec trois propositions~: $P$, $Q$ et $R$~:

  \begin{table}[H]
    \begin{tabular}[H]{|ccc|ccccg|}
      \hline
      $P$ & $Q$ & $R$ & $P\Rightarrow Q$ & $Q\Rightarrow R$ & $(P\Rightarrow Q) \land (Q\Rightarrow R)$ & $P\Rightarrow R$ & $((P\Rightarrow Q) \land (Q\Rightarrow R))\Rightarrow (P\Rightarrow R)$\\ \hline
      V & V & V & V & V & V & V & V \\
      V & V & F & V & F & F & F & V \\
      V & F & V & F & V & F & V & V \\
      V & F & F & F & V & F & F & V \\
      F & V & V & V & V & V & V & V \\
      F & V & F & V & F & F & V & V \\
      F & F & V & V & V & V & V & V \\
      F & F & F & V & V & V & V & V \\
      \hline\hline
    \end{tabular}
  \end{table}

  Comme la dernière colonne est vraie sur toutes les lignes,
  c'est-à-dire pour tout triplet de valeurs de vérité des propositions
  $P$, $Q$ et $R$, la proposition relative à la transitivité de
  l'implication logique est vraie.
\end{notes}


\begin{frame}
  \frametitle{Implication logique}
  \framesubtitle{$\Rightarrow$, $\Leftarrow$ et $\Leftrightarrow$, I}
  \hypertarget{slide_implication_et_equivalence_1}{}

  \begin{theorem}[Équivalence]\label{theorem:implication:equivalence}

    Soient $P$ et $Q$ deux propositions. On a~:
    \[
      (P\Leftrightarrow Q) \Leftrightarrow ((P\Rightarrow Q) \land
      (Q\Rightarrow P))
    \]
  \end{theorem}

  \bigskip

  Cette expression de l'équivalence en termes d'implications est très
  importante. Quand on vous demande d'établir une équivalence, il faut
  garder à l'esprit que la preuve se décomposera en deux parties (une
  pour chaque implication)\newline

  On retouvera la même idée quand on cherchera (voir la section
  suivante) à établir que deux ensembles $A$ et $B$ sont
  identiques. Il faut montrer que l'ensemble $A$ est contenu dans
  l'ensemble $B$ et que l'ensemble $B$ est contenu dans l'ensemble
  $A$.\newline

\end{frame}


\begin{notes}
  \textbf{Preuve du théorème \hyperlink{slide_implication_et_equivalence_1}{\ref{theorem:implication:equivalence}}.}
  \begin{table}[H]
    \begin{tabular}[H]{|cc|gccg|}
      \hline
      $P$ & $Q$ & $P\Leftrightarrow Q$ & $P\Rightarrow Q$ & $Q \Rightarrow P$ & $(P\Rightarrow Q) \land (Q \Rightarrow P)$ \\ \hline
      V & V & V & V & V & V \\
      V & F & F & F & V & F \\
      F & V & F & V & F & F \\
      F & F & V & V & V & V \\
      \hline\hline
    \end{tabular}
  \end{table}
  Puisque les colonnes 3 et 6 ont les mêmes valeurs de vérité sur
  chaque ligne les propositions $P\Leftrightarrow Q$ et
  $(P\Rightarrow Q) \land (Q \Rightarrow P)$ sont équivalentes, comme annoncée dans le théorème.
\end{notes}

\begin{frame}
  \frametitle{Implication logique}
  \framesubtitle{$\Rightarrow$, $\Leftarrow$ et $\Leftrightarrow$, II}
  \hypertarget{slide_implication_et_equivalence_2}{}

  Les expressions « condition nécessaire et suffisante » (CNS), « si et seulement si » (ssi) ou encore « il faut et il suffit », font toutes référence à l'équivalence logique.\newline

  Ainsi pour établir qu'une condition est nécessaire et suffisante il faudra décomposer la preuve en deux parties~:
  \begin{itemize}
  \item montrer que la condition $P$ est nécessaire ($Q \Rightarrow P$)
  \item montrer que la condition $P$ est suffisante ($Q \Leftarrow P$)
  \end{itemize}

  \begin{example}\label{ex:implication:4}

    \begin{itemize}
    \item La proposition « $(x+1=3) \Rightarrow \left((x+1)^2=9\right)$ » est vraie. Il est \emph{suffisant} que $x+1$ soit égal à 3 pour que $(x+1)^2$ soit égal à 9.
    \item La proposition « $(x+1=3) \Leftarrow \left((x+1)^2=9\right)$ » n'est pas vraie. Il n'est pas \emph{nécessaire} que $x+1$ soit égal à 3 pour que $(x+1)^2$ soit égal à 9.
      \item Les deux propositions ne sont pas équivalentes.
    \end{itemize}

  \end{example}

\end{frame}


\begin{frame}
  \frametitle{Implication logique}
  \framesubtitle{Réciproque et contreaposée}
  \hypertarget{slide_reciproque_et_contraposee}{}

  \begin{definition}{Réciproque d'une implication}\label{def:implication:equivalence}

    Soient deux propositions $P$ et $Q$. L'implication $Q\Rightarrow P$ est la réciproque de l'implication $P\Rightarrow Q$.
  \end{definition}

  \bigskip

  \textbf{Remarque:} Si deux propositions $P$ et $Q$ sont équivalentes alors par le théorème \hyperlink{slide_implication_et_equivalence_1}{\ref{theorem:implication:equivalence}} l'implication et sa réciproque sont vraies.

  \bigskip

  \begin{definition}{Contraposée d'une implication}\label{def:implication:contraposee}

    Soient deux propositions $P$ et $Q$. L'implication $\bar Q\Rightarrow \bar P$ est la contraposée de l'implication $P\Rightarrow Q$.
  \end{definition}

  \bigskip

  \begin{theorem}[Contraposée]\label{theorem:implication:contraposee}

    Soient $P$ et $Q$ deux propositions. On a~: $(\bar Q \Rightarrow \bar P) \Leftrightarrow (P\Rightarrow Q)$
  \end{theorem}
\end{frame}

\begin{notes}

  \textbf{Preuve du théorème \hyperlink{slide_reciproque_et_contraposee}{\ref{theorem:implication:contraposee}}.} La proposition
  $\bar Q \Rightarrow \bar P$ est fausse si et seulement si $\bar Q$
  est vraie et $\bar P$ est fausse, c'est-à-dire si et seulement si $Q$
  est fausse et $P$ est vraie. Ainsi $\bar Q \Rightarrow \bar P$ a les
  mêmes valeurs de vérité que $P\Rightarrow Q$ les deux propositions
  sont donc équivalentes.

\end{notes}


\begin{frame}
  \frametitle{Implication logique}
  \framesubtitle{Connecteurs logiques}
  \hypertarget{slide_implication_et_connecteurs}{}

  \begin{theorem}\label{theorem:implication:connecteurs}

    Soient $P$ et $Q$ deux propositions. On a~:
    \[
      (P \Rightarrow Q) \Leftrightarrow (\bar P \lor Q)
      \]
  \end{theorem}

  \bigskip

  \textbf{Remarque:} Par la loi de Morgan (théorème \hyperlink{slide_disjonction_conjonction_et_negation}{\ref{theorem:morgan}}) on a aussi~:
  \[
    (P \Rightarrow Q) \Leftrightarrow \overline{P \land \bar Q}
  \]

  Ce résultat est très important, on peut exprimer l'implication à l'aide d'un connecteur logique et d'une (ou deux) négation(s).\newline

  \begin{theorem}\label{theorem:equivalence:connecteurs}

    Soient $P$ et $Q$ deux propositions. On a~:
    \[
      (P \Leftrightarrow Q) \Leftrightarrow \left((\bar P \lor Q) \land (P \lor \bar Q)\right)
    \]
  \end{theorem}

\end{frame}

\section{Les ensembles}

\begin{frame}
  \frametitle{Les ensembles}
  \hypertarget{slide_ensembles_notations_definitions_1}{}

  \begin{definition}
    Un ensemble est une collection d'objets.
  \end{definition}

  \medskip

  \begin{itemize}
  \item Les éléments d'un ensemble peuvent être de types variés (nombres, lettres, mots, phrases, ensembles, \ldots).\newline
  \item Un ensemble peut être défini par une liste exhaustive des éléments qui le compose. Par exemple~:
    \[
      A = \{1,3\}
    \]
  \item Un ensemble peut être défini à partir d'une règle (indispensable si l'ensemble contient un nombre infini d'éléments). Par exemple~:
    \[
      B = \left\{ x \text{ entier naturel } | \text{ } x \text{ est un nombre impair }\right\}
    \]
  \end{itemize}

\end{frame}


\begin{frame}
  \frametitle{Les ensembles}
  \framesubtitle{Notations}
  \hypertarget{slide_ensembles_notations_definitions_2}{}

  \begin{itemize}

    \item $\in$ note l'appartenance d'un élément à un ensemble~: $5\in B$.\newline

    \item $\notin$ note la non appartenance d'un élément à un ensemble~: $6\notin B$.\newline

    \item $\emptyset$ note l'ensemble vide~: $\emptyset = \{\}$.\newline

    \item $\subseteq$ note l'inclusion d'un ensemble dans un autre~: $ A \subseteq B $ si tout élément de $A$ est aussi un élément de $B$.\newline

    \item $\subset$ note l'inclusion stricte d'un ensemble dans un autre~: $ A \subset B $ si tout élément de $A$ est aussi un élément de $B$ \emph{et} s'il existe au moins un élément de $B$ qui n'appartient pas à $A$.\newline

    \item Deux ensembles $A$ et $B$ sont égaux si et seulement si $A\subseteq B$ et $B\subseteq A$.\newline

    \item On note $\Omega$ l'ensemble universel qui contient tout les ensembles.

  \end{itemize}

\end{frame}

\begin{frame}
  \frametitle{Les ensembles}
  \framesubtitle{Exemples}
  \hypertarget{slide_ensembles_exemples}{}

  \begin{itemize}

  \item $\mathbb N$ l'ensemble des entiers naturels, $\mathbb Z$
    l'ensemble des entiers relatifs, $\mathbb Q$ l'ensemble des
    rationnels, $\mathbb R$ l'ensemble des réels.
    \[
      \mathbb Q = \left\{ \frac{a}{b}\text{ }\Bigl|\text{ }a\in\mathbb Z\text{, }b\in\mathbb Z\text{, et }b\neq 0\right\}
    \]
    Ces ensembles vérifient $\mathbb N \subset \mathbb Z \subset \mathbb Q \subset \mathbb R$.\newline

  \item Les intervalles sur la droite des réels~:
    \[
      \begin{split}
        ]a,b[ &= \{x\in\mathbb R\text{ }|\text{ }a<x<b\}\\
        ]a,b] &= \{x\in\mathbb R\text{ }|\text{ }a<x\leq b\}\\
        [a,b] &= \{x\in\mathbb R\text{ }|\text{ }a\leq x\leq b\}\\
        [a,b[ &= \{x\in\mathbb R\text{ }|\text{ }a\leq x< b\}
      \end{split}
    \]

  \end{itemize}

\end{frame}

\subsection{Opérations sur les ensembles}


\begin{frame}
  \frametitle{Les ensembles}
  \framesubtitle{Union}
  \hypertarget{slide_ensembles_union_1}{}

  \begin{definition} L'union de deux ensembles $A$ et $B$, notée $A$ et $B$

  \end{definition}


\end{frame}



\end{document}
% Local Variables:
% ispell-check-comments: exclusive
% ispell-local-dictionary: "francais"
% TeX-master: t
% End: