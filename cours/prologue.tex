\synctex=1

\documentclass[10pt,notheorems]{beamer}

\usepackage{etex}
\usepackage{fourier-orns}
\usepackage{ccicons}
\usepackage{amssymb}
\usepackage{amstext}
\usepackage{amsbsy}
\usepackage{amsopn}
\usepackage{amscd}
\usepackage{amsxtra}
\usepackage{amsthm}
\usepackage{float}
\usepackage{color, colortbl}
\usepackage{mathrsfs}
\usepackage{bm}
\usepackage{lastpage}
\usepackage[nice]{nicefrac}
\usepackage{setspace}
\usepackage{ragged2e}
\usepackage{listings}
\usepackage{algorithms/algorithm}
\usepackage{algorithms/algorithmic}
\usepackage[frenchb]{babel}
\usepackage{tikz,pgfplots}
\pgfplotsset{compat=newest}
\usetikzlibrary{patterns, arrows, decorations.pathreplacing, decorations.markings, calc}
\pgfplotsset{plot coordinates/math parser=false}
\newlength\figureheight
\newlength\figurewidth
\usepackage[utf8x]{inputenc}
\usepackage{cancel}
\usepackage{tikz-qtree}
\usepackage{dcolumn}
\usepackage{adjustbox}
\usepackage{environ}
\usepackage[cal=boondox]{mathalfa}
\usepackage{manfnt}
\usepackage{hyperref}
\hypersetup{
  colorlinks=true,
  linkcolor=blue,
  filecolor=black,
  urlcolor=black,
}
\usepackage{venndiagram}

% Git hash
\usepackage{xstring}
\usepackage{catchfile}
\immediate\write18{git rev-parse HEAD > git.hash}
\CatchFileDef{\HEAD}{git.hash}{\endlinechar=-1}
\newcommand{\gitrevision}{\StrLeft{\HEAD}{7}}

\newcommand{\trace}{\mathrm{tr}}
\newcommand{\vect}{\mathrm{vec}}
\newcommand{\tracarg}[1]{\mathrm{tr}\left\{#1\right\}}
\newcommand{\vectarg}[1]{\mathrm{vec}\left(#1\right)}
\newcommand{\vecth}[1]{\mathrm{vech}\left(#1\right)}
\newcommand{\iid}[2]{\mathrm{iid}\left(#1,#2\right)}
\newcommand{\normal}[2]{\mathcal N\left(#1,#2\right)}
\newcommand{\dynare}{\href{http://www.dynare.org}{\color{blue}Dynare}}
\newcommand{\sample}{\mathcal Y_T}
\newcommand{\samplet}[1]{\mathcal Y_{#1}}
\newcommand{\slidetitle}[1]{\fancyhead[L]{\textsc{#1}}}

\newcommand{\R}{{\mathbb R}}
\newcommand{\C}{{\mathbb C}}
\newcommand{\N}{{\mathbb N}}
\newcommand{\Z}{{\mathbb Z}}
\newcommand{\binomial}[2]{\begin{pmatrix} #1 \\ #2 \end{pmatrix}}
\newcommand{\bigO}[1]{\mathcal O \left(#1\right)}
\newcommand{\red}{\color{red}}
\newcommand{\blue}{\color{blue}}

\renewcommand{\qedsymbol}{C.Q.F.D.}

\newcolumntype{d}{D{.}{.}{-1}}
\definecolor{gray}{gray}{0.9}
\newcolumntype{g}{>{\columncolor{gray}}c}

\setbeamertemplate{theorems}[numbered]

\theoremstyle{plain}
\newtheorem{theorem}{Théorème}

\theoremstyle{definition} % insert bellow all blocks you want in normal text
\newtheorem{definition}{Définition}
\newtheorem{properties}{Propriétés}
\newtheorem{lemma}{Lemme}
\newtheorem{property}[properties]{Propriété}
\newtheorem{example}{Exemple}
\newtheorem*{idea}{Éléments de preuve} % no numbered block



\setbeamertemplate{footline}{
  {\hfill\vspace*{1pt}\href{http://creativecommons.org/licenses/by-sa/3.0/legalcode}{\ccbysa}\hspace{.1cm}
    \raisebox{-.075cm}{\href{https://git.adjemian.eu/stepan/economic-calculus}{\includegraphics[scale=.1]{../img/gitlab.png}}}\enspace
    \href{https://git.adjemian.eu/stepan/economic-calculus/-/blob/\HEAD/cours/prologue.tex}{\gitrevision}\enspace\today
  }\hspace{1cm}}

\setbeamertemplate{navigation symbols}{}
\setbeamertemplate{blocks}[rounded][shadow=true]
\setbeamertemplate{caption}[numbered]

\NewEnviron{notes}{\justifying\tiny\begin{spacing}{1.0}\BODY\vfill\pagebreak\end{spacing}}

\newenvironment{exercise}[1]
{\bgroup \small\begin{block}{Ex. #1}}
  {\end{block}\egroup}

\newenvironment{defn}[1]
{\bgroup \small\begin{block}{Définition. #1}}
  {\end{block}\egroup}

\newenvironment{exemple}[1]
{\bgroup \small\begin{block}{Exemple. #1}}
  {\end{block}\egroup}

\begin{document}

\title{Calcul Économique\\\small{I. Prologue}}
\author[S. Adjemian]{Stéphane Adjemian}
\institute{\texttt{stephane.adjemian@univ-lemans.fr}} \date{Septembre 2020}

\begin{frame}
  \titlepage{}
\end{frame}

\begin{frame}
  \frametitle{Plan}
  \tableofcontents
\end{frame}


\begin{frame}
  \frametitle{Un peu de vocabulaire}
  \hypertarget{slide_vocabulaire}{}

  \begin{itemize}

  \item \textbf{Axiome}. Un énoncé supposé vrai a priori, sans preuve,
    et qu'on ne cherche pas à démontrer.\newline

  \item \textbf{Proposition}. Un énoncé qui peut être vrai (V) ou faux
    (F).\newline

  \item \textbf{Théorème}. Une proposition (démontrée) vraie.\newline

  \item \textbf{Corollaire}. Un « petit » théorème conséquence
    d'un autre théorème.\newline

  \item \textbf{Lemme}. Un « petit » théorème préparatoire d'un
    autre théorème.\newline

  \item \textbf{Conjecture}. Une proposition supposée vraie, sans
    preuve, tant que l'on n'exhibe pas un contre exemple.\newline

  \end{itemize}

\end{frame}

\section{Calcul propositionnel}

% \subsection{Table de vérité}

\begin{frame}
  \frametitle{Calcul propositionnel}
  \framesubtitle{Proposition et table de vérité}
  \hypertarget{slide_proposition_et_table_de_verite}{}

  \begin{itemize}

  \item Une proposition est un énoncé pouvant être vrai ou faux. On
    dit que vrai (V) ou faux (F) sont les valeurs de vérité d'une
    proposition.\newline

  \item Dans la suite on représentera souvent les valeurs de vérité
    d'une proposition, notée P par exemple, dans une table~:

  \end{itemize}

  \begin{table}[H]

    \centering
    \begin{tabular}[H]{|c|}
      \hline
      P \\ \hline
      V \\
      F \\
      \hline\hline
    \end{tabular}
    \caption{Table de vérité}
    \label{tab:verite}
  \end{table}

\end{frame}

% \subsection{Équivalence logique}

\begin{frame}
  \frametitle{Calcul propositionnel}
  \framesubtitle{Équivalence logique, I}
  \hypertarget{slide_equivalence_logique_1}{}

  \begin{definition}\label{def:equivalence}
    Deux propositions $P$ et $Q$ sont équivalentes si elles sont
    simultanément vraies et simultanément fausses. On note
    $P \Leftrightarrow Q$.
  \end{definition}


  \begin{table}[H]

    \centering
    \begin{tabular}[H]{|cc|c|}
      \hline
      $P$ & $Q$ & $P \Leftrightarrow Q$\\ \hline
      V & V & V \\
      V & F & F \\
      F & V & F \\
      F & F & V \\
      \hline\hline
    \end{tabular}
    \caption{Équivalence logique}
    \label{tab:equivalence}
  \end{table}

  \bigskip

  L'équivalence logique entre propositions joue un rôle analogue à
  l'égalité entre des nombres (ou des ensembles comme nous le verrons plus loin).\newline

  Nous venons de créer une nouvelle proposition
  ($P \Leftrightarrow Q$) à partir de deux propositions ($P$ et $Q$)

\end{frame}

\begin{frame}
  \frametitle{Calcul propositionnel}
  \framesubtitle{Équivalence logique, II}
  \hypertarget{slide_equivalence_logique_2}{}

  On note que la table de vérité contient $4 = 2^{2}$ lignes.\newline

  Plus généralement si nous devons construire une nouvelle proposition
  à partir de $n$ propositions, la table de vérité devra contenir
  $2^{n}$ lignes.\newline

  On note aussi la méthode utilisée pour définir les valeurs de vérité
  des propositions $P$ et $Q$. On commence par poser que la
  proposition $P$ est vraie, puis on considère les deux valeurs de
  vérité possibles pour la seconde proposition $Q$. Après on pose que
  la proposition $P$ est fausse\ldots

  \begin{example}
    Les expressions 3+2 et 4+1, même si elles ne sont pas
    identiques, ont la même valeur, on dit que ces expressions sont
    égales ($3+2=4+1$). Les propositions $(x^{2} = 1)$ et
    $(x=1\text{ ou }x=-1)$ sont distinctes mais équivalentes, on
    écrit~:
    \[
      (x^{2} = 1) \Leftrightarrow (x=1\text{ ou }x=-1)
    \]
  \end{example}

\end{frame}

% \subsection{Négation d'une proposition}

\begin{frame}
  \frametitle{Calcul propositionnel}
  \framesubtitle{Négation}
  \hypertarget{slide_negation}{}

  \begin{definition}\label{definition:negation}

    La négation d'une proposition $P$, notée $\bar P$, change sa
    valeur de vérité.
  \end{definition}

  \bigskip

  \begin{table}[H]

    \centering
    \begin{tabular}[H]{|c|c|}
      \hline
      $P$ & $\bar P$\\ \hline
      V & F \\
      F & V \\
      \hline\hline
    \end{tabular}
    \caption{Négation}
    \label{tab:negation}
  \end{table}

  \begin{theorem}\label{theorem:negation}

    Soit $P$ une proposition, alors
    $\bar{\bar P} \Leftrightarrow P$.
  \end{theorem}

  \medskip

  $\looparrowright$ À montrer avec une table de vérité.
\end{frame}


% \subsection{Connecteurs logiques $\land$ et $\lor$}

\begin{frame}
  \frametitle{Connecteurs logiques}
  \framesubtitle{La conjonction $\land$, I}
  \hypertarget{slide_conjonction_1}{}

  \begin{definition}\label{def:conjonction}

    La conjonction de deux propositions $P$ et $Q$, on note $P\land Q$
    et on lit « $P$ et $Q$ », est vraie si et seulement si les
    deux propositions sont vraies.
  \end{definition}

  \bigskip

  \begin{table}[H]

    \centering
    \begin{tabular}[H]{|cc|c|}
      \hline
      $P$ & $Q$ & $P \land Q$\\ \hline
      V & V & V \\
      V & F & F \\
      F & V & F \\
      F & F & F \\
      \hline\hline
    \end{tabular}
    \caption{Conjonction logique}
    \label{tab:conjonction}
  \end{table}

  \bigskip

  On retient que la conjonction de deux propositions est fausse dès
  lors qu'au moins une proposition est fausse.

\end{frame}

\begin{frame}
  \frametitle{Connecteurs logiques}
  \framesubtitle{La conjonction $\land$, II}
  \hypertarget{slide_conjonction_2}{}

  \begin{properties}\label{properties:conjonction}

    Soient $P$, $Q$ et $R$ des propositions. La conjonction satisfait
    les propriétés suivantes~:
    \begin{enumerate}
    \item \textbf{Idempotence:} $(P \land P) \Leftrightarrow P$.
    \item \textbf{Commutativité:}
      $(P \land Q) \Leftrightarrow (Q \land P)$.
    \item \textbf{Associativité:}
      $((P \land Q)\land R) \Leftrightarrow (P \land (Q\land R)$.
    \item \textbf{Non contradiction:} La proposition $P \land \bar P$
      est fausse.
    \end{enumerate}
  \end{properties}

  \bigskip

  On démontre facilement ces propriétés en utilisant des tables de
  vérité.\newline

  On note que dans la dernière propriété (4) nous venons de construire
  une proposition (disons $Q$) dont la valeur de vérité est certaine,
  alors que nous ne connaissons pas la valeur de vérité de la
  proposition de départ ($P$).

\end{frame}


\begin{notes}
  \begin{itemize}

  \item On montre la première propriété à l'aide d'une table de
    vérité~:
    \begin{table}
      \centering
      \begin{tabular}[H]{|c|c|}
        \hline
        $P$ & $P \land P$ \\ \hline
        V & V \\
        F & F \\
        \hline\hline
      \end{tabular}
    \end{table}
    Puisque les colonnes ont toujours les mêmes valeurs sur chaque
    ligne, les deux propositions $P\land P$ et $P$ sont équivalentes
    (voir la définition \hyperlink{slide_equivalence_logique_1}{\ref{def:equivalence}}).

  \item On procède de la même façon pour la deuxième propriété. Les
    propositions à droite et à gauche du symbole d'équivalence
    $\Leftrightarrow$ font intervenir deux propositions $P$ et
    $Q$. La table de vérité doit donc contenir quatre lignes qui
    correspondent aux couples ordonnés possibles pour les
    valeurs de $P$ et $Q$~:
    \begin{table}[H]
      \centering
      \begin{tabular}[H]{|cc|cc|}
        \hline
        $P$ & $Q$ & $P \land Q$ & $Q \land P$\\ \hline
        V & V & V & V\\
        V & F & F & F\\
        F & V & F & F\\
        F & F & F & F \\
        \hline\hline
      \end{tabular}
    \end{table}
    On observe que les troisième et quatrième colonnes ont toujours
    la même valeur sur chaque ligne, les deux propositions associées
    $P \land Q$ et $Q \land P$ sont donc
    équivalentes.

  \item Nous suivons la même démarche pour la troisième
    propriété. Cette fois nous construisons des propositions à
    partir de trois propositions de base $P$, $Q$ et $R$. La table
    de vérité doit donc contenir huit lignes (c'est-à-dire $2^3$ lignes)~:
    \begin{table}[H]
      \centering
      \begin{tabular}[H]{|ccc|cgcg|}
        \hline
        $P$ & $Q$ & $R$ & $P\land Q$ & $(P\land Q)\land R$ & $Q\land R$ & $P \land (Q \land R)$ \\ \hline
        V & V & V & V & V & V & V\\
        V & V & F & V & F & F & F\\
        V & F & V & F & F & F & F\\
        V & F & F & F & F & F & F\\
        F & V & V & F & F & V & F\\
        F & V & F & F & F & F & F\\
        F & F & V & F & F & F & F\\
        F & F & F & F & F & F & F\\
        \hline\hline
      \end{tabular}
    \end{table}
    On note que la cinquième et la septième colonnes sont identiques, ce qui démontre la troisième propriété.

  \item Pour la dernière propriété la table de vérité contient seulement deux lignes~:
    \begin{table}
      \centering
      \begin{tabular}[H]{|c|cc|}
        \hline
        $P$ & $\bar P$ & $P \land \bar P$\\ \hline
        V & F & F\\
        F & V & F\\
        \hline\hline
      \end{tabular}
    \end{table}
    On remarque que la dernière colonne est toujours fausse.

  \end{itemize}

\end{notes}


\begin{frame}
  \frametitle{Connecteurs logiques}
  \framesubtitle{La disjonction $\lor$, I}
  \hypertarget{slide_disjonction_1}{}

  \begin{definition}\label{def:disjonction}

    La disjonction de deux propositions $P$ et $Q$, on note $P\lor Q$
    et on lit « $P$ ou $Q$ », est fausse si et seulement si les
    deux propositions sont fausses.
  \end{definition}

  \bigskip

  \begin{table}[H]

    \centering
    \begin{tabular}[H]{|cc|c|}
      \hline
      $P$ & $Q$ & $P \lor Q$\\ \hline
      V & V & V \\
      V & F & V \\
      F & V & V \\
      F & F & F \\
      \hline\hline
    \end{tabular}
    \caption{Disjonction logique}
    \label{tab:disjonction}
  \end{table}

  \bigskip

  On retient que la disjonction de deux propositions est vraie dès
  lors qu'au moins une proposition est vraie.

  \bigskip

  On considère ici une définition \textit{inclusive} de la disjonction (par opposition à une définition \textit{exclusive}).

\end{frame}


\begin{frame}
  \frametitle{Connecteurs logiques}
  \framesubtitle{La disjonction $\lor$, II}
  \hypertarget{slide_disjonction_2}{}

  \begin{properties}\label{properties:disjonction}
    Soient $P$, $Q$ et $R$ des propositions. La disjonction satisfait
    les propriétés suivantes~:
    \begin{enumerate}
    \item \textbf{Idempotence:} $(P \lor P) \Leftrightarrow P$.
    \item \textbf{Commutativité:}
      $(P \lor Q) \Leftrightarrow (Q \lor P)$.
    \item \textbf{Associativité:}
      $((P \lor Q)\lor R) \Leftrightarrow (P \lor (Q\lor R)$.
    \item La proposition $P \lor \bar P$
      est vraie.
    \end{enumerate}
  \end{properties}

  \bigskip

  On démontre facilement ces propriétés en utilisant des tables de
  vérité.\newline

  Nous retrouvons les mêmes propriétés que pour la conjonction (à part pour la dernière).

\end{frame}


\begin{notes}
  \begin{itemize}
  \item On montre la première propriété à l'aide d'une table de
    vérité~:
    \begin{table}
      \centering
      \begin{tabular}[H]{|c|c|}
        \hline
        $P$ & $P \lor P$ \\ \hline
        V & V \\
        F & F \\
        \hline\hline
      \end{tabular}
    \end{table}
    Puisque les colonnes ont toujours les mêmes valeurs sur chaque
    ligne, les deux propositions $P\lor P$ et $P$ sont équivalentes
    (voir la définition \hyperlink{slide_equivalence_logique_1}{\ref{def:equivalence}}).
  \item On procède de la même façon pour la deuxième propriété. Les
    propositions à droite et à gauche du symbole d'équivalence
    $\Leftrightarrow$ font intervenir deux propositions $P$ et
    $Q$. La table de vérité doit donc contenir quatre lignes qui
    correspondent aux couples ordonnés possibles pour les
    valeurs de $P$ et $Q$~:
    \begin{table}[H]
      \centering
      \begin{tabular}[H]{|cc|cc|}
        \hline
        $P$ & $Q$ & $P \lor Q$ & $Q \lor P$\\ \hline
        V & V & V & V\\
        V & F & V & V\\
        F & V & V & V\\
        F & F & F & F \\
        \hline\hline
      \end{tabular}
    \end{table}
    On observe que les troisième et quatrième colonnes ont toujours
    la même valeur sur chaque ligne, les deux propositions associées
    $P \lor Q$ et $Q \lor P$ sont donc
    équivalentes.
  \item Nous suivons la même démarche pour la troisième
    propriété. Cette fois nous construisons des propositions à
    partir de trois propositions de base $P$, $Q$ et $R$. La table
    de vérité doit donc contenir huit lignes~:
    \begin{table}[H]
      \centering
      \begin{tabular}[H]{|ccc|cgcg|}
        \hline
        $P$ & $Q$ & $R$ & $P\lor Q$ & $(P\lor Q)\lor R$ & $Q\lor R$ & $P \lor (Q \lor R)$ \\ \hline
        V & V & V & V & V & V & V\\
        V & V & F & V & V & V & V\\
        V & F & V & V & V & V & V\\
        V & F & F & V & V & F & V\\
        F & V & V & V & V & V & V\\
        F & V & F & V & V & V & V\\
        F & F & V & F & V & V & V\\
        F & F & F & F & F & F & F\\
        \hline\hline
      \end{tabular}
    \end{table}
    On note que les cinquième et septième colonnes sont identiques,
    ce qui démontre la troisième propriété.
  \item Pour la dernière propriété la table de vérité contient
    seulement deux lignes~:
    \begin{table}
      \centering
      \begin{tabular}[H]{|c|cc|}
        \hline
        $P$ & $\bar P$ & $P \lor \bar P$\\ \hline
        V & F & V\\
        F & V & V\\
        \hline\hline
      \end{tabular}
    \end{table}
    On remarque que la dernière colonne est toujours vraie.

  \end{itemize}

\end{notes}


\begin{frame}
  \frametitle{Connecteurs logiques}
  \framesubtitle{Disjonction, conjonction et négation}
  \hypertarget{slide_disjonction_conjonction_et_negation}{}

  \begin{properties}[Distributivité]\label{properties:distributivite}
    Soient trois propositions $P$, $Q$ et $R$, on a~:
    \begin{enumerate}
    \item $(P\land Q)\lor R \Leftrightarrow (P\lor R) \land (Q\lor R)$
    \item $(P\lor Q)\land R \Leftrightarrow (P\land R) \lor (Q\land R)$
    \end{enumerate}
  \end{properties}

  \bigskip

  \begin{theorem}[Loi de Morgan]\label{theorem:morgan}

    Soient $P$ et $Q$ deux propositions, on a~:
    \begin{enumerate}
    \item $\overline{P \land Q} \Leftrightarrow \bar P \lor \bar Q$.
    \item $\overline{P \lor Q} \Leftrightarrow \bar P \land \bar Q$.
    \end{enumerate}
  \end{theorem}

  \bigskip

  La loi de Morgan souligne le lien entre la conjonction et la disjoinction, que nous pouvions déjà anticiper en comparant les tables de vérité respectives.\newline

  La négation d'une conjonction est la disjonction des négations.

\end{frame}


\begin{notes}
  \mbox{}
  \textbf{Preuve de la propriété \hyperlink{slide_disjonction_conjonction_et_negation}{\ref{properties:distributivite}}.}
  On montre seulement la première propriété, pour la seconde on suit
  exactement la même approche, toujours en utilisant une table de
  vérité. On dénombre huit triplets de valeurs de vérité pour les
  propriétés $P$, $Q$ et $R$. La table est donc formée de huit
  lignes~:
  \begin{table}[H]
    \centering
    \begin{tabular}[H]{|ccc|cgccg|}
      \hline
      $P$ & $Q$ & $R$ & $P\land Q$ & $(P\land Q)\lor R$ & $P\lor R$ & $Q \lor R$ & $(P\lor R)\land (Q \lor R)$ \\ \hline
      V & V & V & V & V & V & V & V \\
      V & V & F & V & V & V & V & V \\
      V & F & V & F & V & V & V & V \\
      V & F & F & F & F & V & F & F \\
      F & V & V & F & V & V & V & V \\
      F & V & F & F & F & F & V & F \\
      F & F & V & F & V & V & V & V \\
      F & F & F & F & F & F & F & F \\
      \hline\hline
    \end{tabular}
  \end{table}

  \bigskip\bigskip

  \textbf{Preuve du théorème \hyperlink{slide_disjonction_conjonction_et_negation}{\ref{theorem:morgan}}.} On démontre seulement le premier point, on peut suivre la même approche pour le second. On utilise une table de vérité de quatre lignes~:
  \begin{table}[H]
    \centering
    \begin{tabular}[H]{|cc|cgccg|}
      \hline
      $P$ & $Q$ & $P\land Q$ & $\overline{P\land Q}$ & $\bar P$ & $\bar Q$ & $\bar P \lor \bar Q$ \\ \hline
      V & V & V & F & F & F & F\\
      V & F & F & V & F & V & V\\
      F & V & F & V & V & F & V\\
      F & F & F & V & V & V & V\\
      \hline\hline
    \end{tabular}
  \end{table}

  \vspace*{\fill}

\end{notes}

\section{Implication logique}

\begin{frame}
  \frametitle{Implication logique}
  \framesubtitle{Le connecteur $\Rightarrow$, I}
  \hypertarget{slide_implication_1}{}
  \begin{definition}

    Soient deux propositions $P$ et $Q$. La proposition
    $P\Rightarrow Q$, on dit « $P$ implique $Q$ », est fausse si $P$
    est vraie \textbf{et} $Q$ est fausse, la proposition
    $P\Rightarrow Q$ est vraie sinon.
  \end{definition}


  \begin{table}[H]

    \centering
    \begin{tabular}[H]{|cc|c|}
      \hline
      $P$ & $Q$ & $P \Rightarrow Q$\\ \hline
      V & V & V \\
      V & F & F \\
      F & V & V \\
      F & F & V \\
      \hline\hline
    \end{tabular}
    \caption{L'implication logique}
    \label{tab:implication}
  \end{table}

  Ce connecteur logique peut paraître peu intuitif\ldots\newline

  Si l'implication est vraie, $Q$ vraie peut être déduite de $P$ vraie.\newline

  Si l'implication est vraie, on ne peut rien inférer sur la vérité de $Q$ lorsque $P$ est fausse.\newline

\end{frame}


\begin{frame}
  \frametitle{Implication logique}
  \framesubtitle{Le connecteur $\Rightarrow$, II}
  \hypertarget{slide_implication_2}{}

  \begin{example}\label{ex:implication:1}

    Considérons une proposition qui ne devrait pas vous causer le
    moindre doute~:

    \medskip

    \begin{quotation}
      {Pour tout entier relatif $n$, si $n>2$ alors $n^2>4$}
    \end{quotation}

    \medskip

    On note $R(n)$ la proposition précédente, et on pose $P(n)$~: «
    $n>2$ » et $Q(n)$~: « $n^2>4$ ». On peut vérifier que pour
    différentes valeurs de $n$ on retrouve trois des lignes de
    la table de vérité de l'implication logique~:
    \begin{table}[H]
      \centering
      \begin{tabular}[H]{|r|cc|c|}
        \hline
        $n$ & $P(n)$ & $Q(n)$ & $R(n)$\\ \hline
        3 & V & V & V\\
        -3 & F & V & V\\
        1 & F & F & V\\
        \hline\hline
      \end{tabular}
    \end{table}
    où l'implication est vraie, mais pas le cas où l'implication est
    fausse (puisque $R(n)$ est vraie pour tout $n$).
  \end{example}

\end{frame}


\begin{frame}
  \frametitle{Implication logique}
  \framesubtitle{Le connecteur $\Rightarrow$, III}
  \hypertarget{slide_implication_3}{}

  \begin{example}[$F\Rightarrow F$ est vraie]\label{ex:implication:2}

    Montrons que si $10^n+1$ est divisible par 9, alors $10^{n+1}+1$
    est divisible par 9, pour tout entier $n$.\newline

    La première proposition, $10^n+1$ est divisible par 9, exige
    l'existence d'un entier $k$ tel que $10^n+1 = 9k$. Nous avons~:
    \[
      10^{n+1}+1 = 10\times 10^n+1 =
      10\times(10^{n}+1)-9=9\times(10k-1)
    \]
    et donc $10^{n+1}+1$ est divisible par 9.\newline

    Clairement la proposition « $10^n+1$ divisible par 9 implique
    $10^{n+1}+1$ divisible par 9 » est vraie. Il est tout aussi
    évident que la proposition « $10^n+1$ divisible par 9 » est
    fausse. Il suffit de considérer le cas n=0 pour s'en convaincre.
  \end{example}

\end{frame}


\begin{frame}
  \frametitle{Implication logique}
  \framesubtitle{Le connecteur $\Rightarrow$, IV}
  \hypertarget{slide_implication_4}{}

  \begin{example}[$F\Rightarrow V$ est vraie]\label{ex:implication:3}

    Soit la proposition $P~:$ « $2=3$ et $2=1$ ». Cette proposition est
    clairement fausse, néanmoins en sommant les deux égalités on
    obtient~:
    \[
      \begin{split}
        2+2 &= 3+1\\
        4 &= 4
      \end{split}
    \]
    la proposition $Q$~: « $4=4$ » est évidemment vraie (forfuitement). Si la
    proposition $P\Rightarrow Q$ est vraie, alors $Q$ peut être vraie
    même si $P$ est fausse. Autrement dit un raisonnement correcte
    peut (par chance) amener à un résultat correct même si le prémisse
    est faux.
  \end{example}

\end{frame}


\begin{frame}
  \frametitle{Implication logique}
  \framesubtitle{$\Rightarrow$, $\land$, $\Rightarrow$}
  \hypertarget{slide_implication_transitivite}{}

  \begin{theorem}[Transitivité]\label{theorem:implication:transitivite}

    Soient $P$, $Q$ et $R$ trois propositions. On a~:
    \[
      ((P\Rightarrow Q) \land (Q\Rightarrow R)) \Rightarrow
      (P\Rightarrow R)
    \]
  \end{theorem}

  \bigskip

  Si $P$ est vraie et si $P\Rightarrow Q$ est vraie, alors $Q$ est
  vraie (Cf. la première ligne de la table de vérité
  \hyperlink{slide_implication_1}{\ref{tab:implication}}). Si $Q\Rightarrow R$ est vraie, alors puisque
  $Q$ est vraie on en déduit que $R$ est vraie.\newline

  Cette propriété de transitivité sera souvent exploitée.\newline

  Pour démontrer le théorème on procède toujours en construisant une
  table de vérité (à huit lignes).\newline

\end{frame}


\begin{notes}
  \textbf{Preuve du théorème \hyperlink{slide_implication_transitivite}{\ref{theorem:implication:transitivite}}.}
  La table de vérité contient huit lignes psuique nous travaillons
  avec trois propositions~: $P$, $Q$ et $R$~:

  \begin{table}[H]
    \begin{tabular}[H]{|ccc|ccccg|}
      \hline
      $P$ & $Q$ & $R$ & $P\Rightarrow Q$ & $Q\Rightarrow R$ & $(P\Rightarrow Q) \land (Q\Rightarrow R)$ & $P\Rightarrow R$ & $((P\Rightarrow Q) \land (Q\Rightarrow R))\Rightarrow (P\Rightarrow R)$\\ \hline
      V & V & V & V & V & V & V & V \\
      V & V & F & V & F & F & F & V \\
      V & F & V & F & V & F & V & V \\
      V & F & F & F & V & F & F & V \\
      F & V & V & V & V & V & V & V \\
      F & V & F & V & F & F & V & V \\
      F & F & V & V & V & V & V & V \\
      F & F & F & V & V & V & V & V \\
      \hline\hline
    \end{tabular}
  \end{table}

  Comme la dernière colonne est vraie sur toutes les lignes,
  c'est-à-dire pour tout triplet de valeurs de vérité des propositions
  $P$, $Q$ et $R$, la proposition relative à la transitivité de
  l'implication logique est vraie.
\end{notes}


\begin{frame}
  \frametitle{Implication logique}
  \framesubtitle{$\Rightarrow$, $\Leftarrow$ et $\Leftrightarrow$, I}
  \hypertarget{slide_implication_et_equivalence_1}{}

  \begin{theorem}[Équivalence]\label{theorem:implication:equivalence}

    Soient $P$ et $Q$ deux propositions. On a~:
    \[
      (P\Leftrightarrow Q) \Leftrightarrow ((P\Rightarrow Q) \land
      (Q\Rightarrow P))
    \]
  \end{theorem}

  \bigskip

  Cette expression de l'équivalence en termes d'implications est très
  importante. Quand on vous demande d'établir une équivalence, il faut
  garder à l'esprit que la preuve se décomposera en deux parties (une
  pour chaque implication)\newline

  On retouvera la même idée quand on cherchera (voir la section
  suivante) à établir que deux ensembles $A$ et $B$ sont
  identiques. Il faut montrer que l'ensemble $A$ est contenu dans
  l'ensemble $B$ et que l'ensemble $B$ est contenu dans l'ensemble
  $A$.\newline

\end{frame}


\begin{notes}
  \textbf{Preuve du théorème \hyperlink{slide_implication_et_equivalence_1}{\ref{theorem:implication:equivalence}}.}
  \begin{table}[H]
    \begin{tabular}[H]{|cc|gccg|}
      \hline
      $P$ & $Q$ & $P\Leftrightarrow Q$ & $P\Rightarrow Q$ & $Q \Rightarrow P$ & $(P\Rightarrow Q) \land (Q \Rightarrow P)$ \\ \hline
      V & V & V & V & V & V \\
      V & F & F & F & V & F \\
      F & V & F & V & F & F \\
      F & F & V & V & V & V \\
      \hline\hline
    \end{tabular}
  \end{table}
  Puisque les colonnes 3 et 6 ont les mêmes valeurs de vérité sur
  chaque ligne les propositions $P\Leftrightarrow Q$ et
  $(P\Rightarrow Q) \land (Q \Rightarrow P)$ sont équivalentes, comme annoncée dans le théorème.
\end{notes}

\begin{frame}
  \frametitle{Implication logique}
  \framesubtitle{$\Rightarrow$, $\Leftarrow$ et $\Leftrightarrow$, II}
  \hypertarget{slide_implication_et_equivalence_2}{}

  Les expressions « condition nécessaire et suffisante » (CNS), « si et seulement si » (ssi) ou encore « il faut et il suffit », font toutes référence à l'équivalence logique.\newline

  Ainsi pour établir qu'une condition est nécessaire et suffisante il faudra décomposer la preuve en deux parties~:
  \begin{itemize}
  \item montrer que la condition $P$ est nécessaire ($Q \Rightarrow P$)
  \item montrer que la condition $P$ est suffisante ($Q \Leftarrow P$)
  \end{itemize}

  \begin{example}\label{ex:implication:4}

    \begin{itemize}
    \item La proposition « $(x+1=3) \Rightarrow \left((x+1)^2=9\right)$ » est vraie. Il est \emph{suffisant} que $x+1$ soit égal à 3 pour que $(x+1)^2$ soit égal à 9.
    \item La proposition « $(x+1=3) \Leftarrow \left((x+1)^2=9\right)$ » n'est pas vraie. Il n'est pas \emph{nécessaire} que $x+1$ soit égal à 3 pour que $(x+1)^2$ soit égal à 9.
    \item Les deux propositions ne sont pas équivalentes.
    \end{itemize}

  \end{example}

\end{frame}


\begin{frame}
  \frametitle{Implication logique}
  \framesubtitle{Réciproque et contreaposée}
  \hypertarget{slide_reciproque_et_contraposee}{}

  \begin{definition}[Réciproque d'une implication]\label{def:implication:equivalence}

    Soient deux propositions $P$ et $Q$. L'implication $Q\Rightarrow P$ est la réciproque de l'implication $P\Rightarrow Q$.
  \end{definition}

  \bigskip

  \textbf{Remarque:} Si deux propositions $P$ et $Q$ sont équivalentes alors par le théorème \hyperlink{slide_implication_et_equivalence_1}{\ref{theorem:implication:equivalence}} l'implication et sa réciproque sont vraies.

  \bigskip

  \begin{definition}[Contraposée d'une implication]\label{def:implication:contraposee}

    Soient deux propositions $P$ et $Q$. L'implication $\bar Q\Rightarrow \bar P$ est la contraposée de l'implication $P\Rightarrow Q$.
  \end{definition}

  \bigskip

  \begin{theorem}[Contraposée]\label{theorem:implication:contraposee}

    Soient $P$ et $Q$ deux propositions. On a~: $(\bar Q \Rightarrow \bar P) \Leftrightarrow (P\Rightarrow Q)$
  \end{theorem}
\end{frame}

\begin{notes}

  \textbf{Preuve du théorème \hyperlink{slide_reciproque_et_contraposee}{\ref{theorem:implication:contraposee}}.} La proposition
  $\bar Q \Rightarrow \bar P$ est fausse si et seulement si $\bar Q$
  est vraie et $\bar P$ est fausse, c'est-à-dire si et seulement si $Q$
  est fausse et $P$ est vraie. Ainsi $\bar Q \Rightarrow \bar P$ a les
  mêmes valeurs de vérité que $P\Rightarrow Q$ les deux propositions
  sont donc équivalentes.

\end{notes}


\begin{frame}
  \frametitle{Implication logique}
  \framesubtitle{Connecteurs logiques}
  \hypertarget{slide_implication_et_connecteurs}{}

  \begin{theorem}\label{theorem:implication:connecteurs}

    Soient $P$ et $Q$ deux propositions. On a~:
    \[
      (P \Rightarrow Q) \Leftrightarrow (\bar P \lor Q)
    \]
  \end{theorem}

  \bigskip

  \textbf{Remarque:} Par la loi de Morgan (théorème \hyperlink{slide_disjonction_conjonction_et_negation}{\ref{theorem:morgan}}) on a aussi~:
  \[
    (P \Rightarrow Q) \Leftrightarrow \overline{P \land \bar Q}
  \]

  Ce résultat est très important, on peut exprimer l'implication à l'aide d'un connecteur logique et d'une (ou deux) négation(s).\newline

  \begin{theorem}\label{theorem:equivalence:connecteurs}

    Soient $P$ et $Q$ deux propositions. On a~:
    \[
      (P \Leftrightarrow Q) \Leftrightarrow \left((\bar P \lor Q) \land (P \lor \bar Q)\right)
    \]
  \end{theorem}

\end{frame}

\section{Les grands types de raisonnements}

\begin{frame}
  \frametitle{Les grands types de raisonnements}

  \centering
  \includegraphics[scale=.8]{../img/le-chat.png}

\end{frame}


% \subsection{Le raisonnement déductif}

\begin{frame}
  \frametitle{Les grands types de raisonnements}
  \framesubtitle{Le raisonnement déductif}
  \hypertarget{slide_raisonnement_deductif}{}

  \begin{definition}[Modus ponens]
    Si $P\Rightarrow Q$ est une proposition vraie, et si $P$ est une
    proposition vraie alors $Q$ est une proposition vraie.
  \end{definition}

  \begin{itemize}

    \bigskip

  \item Le raisonnement déductif correspond à la première ligne de la
    table de vérité \hyperlink{slide_implication_1}{\ref{tab:implication}}.\newline

  \item On peut généraliser avec un nombre arbitraire d'implications,
    en utilisant la transitivité de l'implication
    (voir le théorème \hyperlink{slide_implication_transitivite}{\ref{theorem:implication:transitivite}}). Soient
    $(P_i, i=1\dots,n)$ des propositions. Si $P_1\Rightarrow P_2 \Rightarrow \dots \Rightarrow P_n$ est une proposition vraie, et si $P_1$ est une proposition vraie alors $P_n$ est une proposition vraie.\newline

  \item On peut généraliser à un ensemble dénombrable de propositions, voir plus loin le raisonnement par récurrence.

  \end{itemize}
\end{frame}

% \subsection{Le raisonnement par l'absurde}

\begin{frame}
  \frametitle{Les grands types de raisonnements}
  \framesubtitle{Le raisonnement par l'absurde}
  \hypertarget{slide_raisonnement_par_l_absurde_1}{}

  \begin{definition}[Reductio ad absurdum]
    Si $\overline P\Rightarrow Q$ est une proposition vraie, et si $Q$ est une
    proposition fausse alors $P$ est une proposition vraie.
  \end{definition}

  \bigskip

  \begin{itemize}

  \item On montre que la proposition $P$ est vraie en montrant que $P$ fausse aboutit à une contradiction logique.\newline

  \item Le raisonnement par l'absurde correspond à la dernière ligne de la
    table de vérité \hyperlink{slide_implication_1}{\ref{tab:implication}}. Si $Q$ est une proposition fausse, alors la proposition $\overline P\Rightarrow Q$ ne peut être vraie que si $\overline P$ est fausse, c'est-à-dire $P$ vraie.\newline

  \item Contrairement au raisonnement déductif, ici on ne comprend pas vraiment pourquoi $P$ est vraie. C'est une limite du raisonnement par l'absurde.

  \end{itemize}
\end{frame}


\begin{frame}
  \frametitle{Les grands types de raisonnements}
  \framesubtitle{Exemple de raisonnement par l'absurde}
  \hypertarget{slide_raisonnement_par_l_absurde_2}{}

  \bigskip

  Montrons qu'il existe une infinité de nombres premiers.\newline

  \medskip

  \begin{itemize}

  \item Supposons qu'il existe seulement $n$ nombres premiers: $p_1<p_2<\dots<p_n$ (par exemple 2, 3, 5, 7 et 11).\newline

  \item Posons $P = p_1 \times p_2 \times \dots \times p_n + 1$.\newline

  \item Clairement $P$ est plus grand que $p_n$.\newline

  \item $P$ est-il un nombre premier~?\newline

    \begin{itemize}

    \item Si $P$ est premier, alors il y a plus de $n$ nombres premiers. Cela contredit l'hypothèse de départ.\newline

    \item Si $P$ n'est pas un nombre premier, alors il est divisible
      par un nombre premier. Par construction $P$ n'est pas divisible
      par $p_1, \dots, p_n$. Il doit donc exister au moins un autre
      nombre premier plus grand que $p_n$. À nouveau, cela contredit
      l'hypothèse de départ.

    \end{itemize}

  \end{itemize}

\end{frame}

% \subsection{Le raisonnement par contraposition}

\begin{frame}
  \frametitle{Les grands types de raisonnements}
  \framesubtitle{Le raisonnement par contraposition}
  \hypertarget{slide_raisonnement_par_contraposition}{}

  \begin{definition}[Modus tollens]
    Pour montrer que $P\Rightarrow Q$ il faut et il suffit de montrer que $\overline Q\Rightarrow \overline P$.
  \end{definition}

  \begin{itemize}

    \bigskip

  \item Voir le théorème \hyperlink{slide_reciproque_et_contraposee}{\ref{theorem:implication:contraposee}}.\newline

  \item Montrer que la proposition « $n^2$ pair implique $n$ pair » est équivalent à montrer que la proposition « $n$ impair implique $n^2$ impair». Or si $n$ est impair alors il peut s'écrire sous la forme $n = 2k+1$ avec $k$ un entier. En prenant le carré~:
    \[
      n^2 = (2k+1)^2 \Leftrightarrow n^2 = 2(2n+2k) + 1 \Leftrightarrow n^2 = 2\mathcal{k} + 1
    \]
    et donc $n^2$ est impair. On conclut que la proposition de départ « $n^2$ pair implique $n$ pair » est vraie.

  \end{itemize}

\end{frame}


% \subsection{Le raisonnement par récurrence}

\begin{frame}
  \frametitle{Les grands types de raisonnements}
  \framesubtitle{Le raisonnement par récurrence}
  \hypertarget{slide_raisonnement_par_reccurence_1}{}

  \begin{definition}
    Soit $P_n$ une proposition dépendant d'un entier $n$. Pour montrer
    que cette proposition est vraie pour tout entier $n \geq n_0$, on procède en deux étapes~:
    \begin{enumerate}
    \item On montre que la proposition $P_{n_0}$ est vraie,
    \item On montre que, pour un entier $n\geq n_0$ quelconque, si $P_n$ est vraie alors $P_{n+1}$ est vraie (ou la proposition $P_n \Rightarrow P_{n+1}$ est vraie).
    \end{enumerate}
  \end{definition}

  \bigskip

  \begin{itemize}

  \item La première étape est l'\textit{initialisation} de la récurrence, la seconde étape est l'\textit{hérédité}.\newline
  \item En répétant indéfiniment l'étape 2, on obtient en partant de $P_{n_0}$ vraie~: $P_{n_0+1}$ vraie, $P_{n_0+2}$ vraie, $P_{n_0+3}$ vraie, ...

  \end{itemize}

\end{frame}



\begin{frame}
  \frametitle{Les grands types de raisonnements}
  \framesubtitle{Exemple de raisonnement par récurrence, I}
  \hypertarget{slide_raisonnement_par_reccurence_2}{}

  \bigskip

  \begin{itemize}

  \item Montrons qu'un polygône convexe à $n$ côtés possède $d_n = \frac{n(n-3)}{2}$ diagonales.\newline

  \item On vérifie facilement que la prédiction est correcte pour un triangle ($n=3$)~: un triangle n'a pas de diagonales~!\newline

  \item Soit un polygône convexe avec $n$ sommets ou côtés, voir \hyperlink{slide_raisonnement_par_reccurence_3}{l'illustration}. Supposons qu'il ait $d_n = \frac{n(n-3)}{2}$ diagonales et montrons qu'un polygône avec $n+1$ sommets a nécessairement $d_{n+1} = \frac{(n+1)(n-2)}{2}$ diagonales.\newline

  \item Créons un polygône à $n+1$ sommets, $\mathcal P_{n+1}$, à partir de d'un polygône à $n$ sommets, $\mathcal P_{n}$. Notons $A_1$, $A_2$, ..., $A_n$ les sommets de $\mathcal P_{n}$.\newline

  \item On construit $\mathcal P_{n+1}$ en créant un nouveau sommet $A_{n+1}$ entre $A_1$ et $A_2$ à l'extérieur du polygône (de sorte que le nouveau polygône soit convexe, autrement certaines diagonales sortiraient du polygône).

  \end{itemize}

\end{frame}


\begin{frame}
  \frametitle{Les grands types de raisonnements}
  \framesubtitle{Exemple de raisonnement par récurrence, II}
  \hypertarget{slide_raisonnement_par_reccurence_2}{}

  \bigskip

  \begin{itemize}

  \item Les diagonales de $\mathcal P_n$ sont des diagonales de $\mathcal P_{n+1}$.\newline
  \item En reliant $A_{n+1}$ à $A_3$, $A_4$, ..., $A_n$ on obtient $n-2$ nouvelles diagonales dans $\mathcal P_{n+1}$.\newline
  \item La dernière diagonale de $\mathcal P_{n+1}$ est le segment $A_1A_2$ (un côté de $\mathcal P_n$).\newline
  \item Au total, nous avons~:
    \[
      d_{n+1} = \frac{n(n-3)}{2} + n-2 + 1
    \]
    \[
      \Leftrightarrow d_{n+1} = \frac{n^2-n-2}{2} \Leftrightarrow d_{n+1} = \frac{(n+1)(n-2)}{2} \qed
    \]
  \end{itemize}

\end{frame}


\begin{frame}
  \frametitle{Les grands types de raisonnements}
  \framesubtitle{Exemple de raisonnement par récurrence, III (héxagone)}
  \hypertarget{slide_raisonnement_par_reccurence_3}{}

  \begin{center}
    \begin{tikzpicture}[scale=0.7]
      \coordinate (1) at (4,9);
      \coordinate (2) at (10,6);
      \coordinate (3) at (8,2);
      \coordinate (4) at (4,0);
      \coordinate (5) at (1,2);
      \coordinate (6) at (0,6);
      \path[draw, thick] (4) node[below] {$A_4$}
      -- (5) node[left] {$A_5$}
      -- (6) node[left] {$A_6$}
      -- (1) node[above] {$A_1$}
      -- (2) node[right] {$A_2$}
      -- (3) node[below] {$A_3$}
      -- cycle;
      \draw[dashed] (4) -- (6);
      \draw[dashed] (4) -- (1);
      \draw[dashed] (4) -- (2);
      \draw[dashed] (3) -- (5);
      \draw[dashed] (3) -- (6);
      \draw[dashed] (3) -- (1);
      \draw[dashed] (2) -- (5);
      \draw[dashed] (2) -- (6);
      \draw[dashed] (1) -- (5);
      \foreach \x in {(1), (2), (3), (4), (5), (6)}{
        \fill \x circle[radius=0.1cm];
      };
    \end{tikzpicture}
  \end{center}

\end{frame}

\begin{frame}
  \frametitle{Les grands types de raisonnements}
  \framesubtitle{Exemple de raisonnement par récurrence, III (héptagone)}

  \begin{center}
    \begin{tikzpicture}[scale=0.7]
      \coordinate (1) at (4,9);
      \coordinate (2) at (10,6);
      \coordinate (3) at (8,2);
      \coordinate (4) at (4,0);
      \coordinate (5) at (1,2);
      \coordinate (6) at (0,6);
      \coordinate (7) at (8,8);
      \path[draw, thick] (4) node[below] {$A_4$}
      -- (5) node[left] {$A_5$}
      -- (6) node[left] {$A_6$}
      -- (1) node[above] {$A_1$}
      -- (2) node[right] {$A_2$}
      -- (3) node[below] {$A_3$}
      -- cycle;
      \draw[dashed] (4) -- (6);
      \draw[dashed] (4) -- (1);
      \draw[dashed] (4) -- (2);
      \draw[dashed] (3) -- (5);
      \draw[dashed] (3) -- (6);
      \draw[dashed] (3) -- (1);
      \draw[dashed] (2) -- (5);
      \draw[dashed] (2) -- (6);
      \draw[dashed] (1) -- (5);
      \path[draw, thick, red] (1) -- (7) node [right] {$A_7$} -- (2);
      \foreach \x in {(3), (4), (5), (6)}{
        \draw[dashed, red] (7) -- \x;
      };
      \draw[dashed, red, thick] (1) -- (2);
      \draw[red, fill=red] (7) circle [radius=0.1cm];
      \foreach \x in {(1), (2), (3), (4), (5), (6)}{
        \fill \x circle[radius=0.1cm];
      };
    \end{tikzpicture}
  \end{center}

\end{frame}


\section{Les ensembles}

% \subsection{Définitions}

\begin{frame}
  \frametitle{Les ensembles}
  \hypertarget{slide_ensembles_notations_definitions_1}{}

  \begin{definition}
    Un ensemble est une collection d'objets.
  \end{definition}

  \medskip

  \begin{itemize}
  \item Les éléments d'un ensemble peuvent être de types variés (nombres, lettres, mots, phrases, ensembles, \ldots).\newline
  \item Un ensemble peut être défini par une liste exhaustive des éléments qui le compose. Par exemple~:
    \[
      A = \{1,3\}
    \]
  \item Un ensemble peut être défini à partir d'une règle (indispensable si l'ensemble contient un nombre infini d'éléments). Par exemple~:
    \[
      B = \left\{ x \text{ entier naturel } | \text{ } x \text{ est un nombre impair }\right\}
    \]
  \end{itemize}

\end{frame}


\begin{frame}
  \frametitle{Les ensembles}
  \framesubtitle{Notations}
  \hypertarget{slide_ensembles_notations_definitions_2}{}

  \begin{itemize}

  \item $\in$ note l'appartenance d'un élément à un ensemble~: $5\in B$.\newline

  \item $\notin$ note la non appartenance d'un élément à un ensemble~: $6\notin B$.\newline

  \item $\emptyset$ note l'ensemble vide~: $\emptyset = \{\}$.\newline

  \item $\subseteq$ note l'inclusion d'un ensemble dans un autre~: $ A \subseteq B $ si tout élément de $A$ est aussi un élément de $B$.\newline

  \item $\subset$ note l'inclusion stricte d'un ensemble dans un autre~: $ A \subset B $ si tout élément de $A$ est aussi un élément de $B$ \emph{et} s'il existe au moins un élément de $B$ qui n'appartient pas à $A$.\newline

  \item Deux ensembles $A$ et $B$ sont égaux si et seulement si $A\subseteq B$ et $B\subseteq A$.\newline

  \item On note $\Omega$ l'ensemble universel qui contient tout les ensembles.

  \end{itemize}

\end{frame}

\begin{frame}
  \frametitle{Les ensembles}
  \framesubtitle{Exemples}
  \hypertarget{slide_ensembles_exemples}{}

  \begin{itemize}

  \item $\mathbb N$ l'ensemble des entiers naturels, $\mathbb Z$
    l'ensemble des entiers relatifs, $\mathbb Q$ l'ensemble des
    rationnels, $\mathbb R$ l'ensemble des réels.
    \[
      \mathbb Q = \left\{ \frac{a}{b}\text{ }\Bigl|\text{ }a\in\mathbb Z\text{, }b\in\mathbb Z\text{, et }b\neq 0\right\}
    \]
    Ces ensembles vérifient $\mathbb N \subset \mathbb Z \subset \mathbb Q \subset \mathbb R$.\newline

  \item Les intervalles sur la droite des réels~:
    \[
      \begin{split}
        ]a,b[ &= \{x\in\mathbb R\text{ }|\text{ }a<x<b\}\\
        ]a,b] &= \{x\in\mathbb R\text{ }|\text{ }a<x\leq b\}\\
        [a,b] &= \{x\in\mathbb R\text{ }|\text{ }a\leq x\leq b\}\\
        [a,b[ &= \{x\in\mathbb R\text{ }|\text{ }a\leq x< b\}
      \end{split}
    \]

  \end{itemize}

\end{frame}


\begin{frame}
  \frametitle{Les ensembles}
  \framesubtitle{Remarques}
  \hypertarget{slide_ensembles_remarques}{}

  \begin{itemize}

  \item[\dbend] Ne pas confondre $\emptyset$ et $\{\emptyset\}$, qui n'est pas vide puisqu'il contient l'ensemble vide.\newline

  \item[\dbend] Ne pas confondre $\in$ et $\subset$. La proposition $\pi\in\mathbb R$ dit que le nombre $\pi$ appartient à l'ensemble des réels, \textbf{mais} la proposition $\pi\subset\mathbb R$ n'a pas de sens car $\pi$ n'est pas un ensemble. La proposition $[0,1]\subset\mathbb R$ dit que l'intervalle $[0,1]$ est un sous ensemble de l'ensemble des réels, \textbf{mais} la proposition $[0,1]\in\mathbb R$ n'a pas de sens car cet interval n'est pas un élément de l'ensemble des réels.\newline

  \item Le symbole d'appartenance, $\in$, relie un élément à un ensemble ; le symbole d'inclusion, $\subset$, relie deux ensembles.\newline
  \end{itemize}

\end{frame}


\begin{frame}
  \frametitle{Les ensembles}
  \framesubtitle{L'insolite ensemble vide}
  \hypertarget{slide_ensemble_vide_1}{}


  \begin{property}
    Pour tout ensemble $A$, l'ensemble vide $\emptyset$ est un sous ensemble de $A$.
  \end{property}

  \bigskip\bigskip

  Si $\emptyset$ n'est pas inclus dans $A$, alors il existe au moins
  un élément de l'ensemble vide qui n'appartienne pas à $A$. Or,
  l'ensemble vide ne contient aucun élément, \textit{a fortiori} aucun
  élément qui n'appartienne pas à $A$. Donc tout élément de
  $\emptyset$ appartient à $A$, ainsi $\emptyset$ est un sous ensemble
  de $A$.
\end{frame}


\begin{frame}
  \frametitle{Les ensembles}
  \framesubtitle{Quantificateurs}
  \hypertarget{quantificateurs_notations}{}


  Pour définir un ensemble ou décrire ses propriétés, on utilise souvent des phrases comme~: « il existe », « il n'existe pas », « il existe un unique », « pour tout x », ... Afin  d'abréger on utilise les notations suivantes :\newline

  \begin{itemize}

  \item[$\forall$] $\rightarrow$ pour tout (par exemple $\forall x \in A$ se lit « pour tout $x$ dans l'ensemble $A$ »).\newline

  \item[$\exists$] $\rightarrow$ il existe (par exemple $\exists x\in A$ tel que $x\notin B$, se lit « il existe un élément x dans $A$ qui n'appartient pas à $B$ »).\newline

  \item[$\exists!$] $\rightarrow$ il existe un unique.\newline

  \item[$\nexists$] $\rightarrow$ il n'existe pas.\newline

  \end{itemize}

  Nous reviendrons plus loin sur les propriétés de ces quantificateurs.

\end{frame}

% \subsection{Opérations sur les ensembles}


\begin{frame}
  \frametitle{Les ensembles}
  \framesubtitle{Union}
  \hypertarget{slide_ensembles_union}{}

  \begin{definition}\label{def:union} L'union de deux ensembles $A$ et $B$, notée $A\cup B$, est un ensemble contenant les éléments des ensembles $A$ et $B$. Plus formellement, $A \cup B = \{x | x \in A \lor x\in B\}$.
  \end{definition}

  \bigskip

  \begin{properties}\label{properties:union}
    \begin{enumerate}
    \item L'union est associative. Soient $A$, $B$ et $C$ trois ensembles, alors $(A\cup B)\cup C = A\cup (B\cup C)$.
    \item L'union est commutative. Soient $A$ et $B$ deux ensembles, alors $A \cup B = B \cup A$.
    \item L'union est idempotente. Soit $A$ un ensemble, alors $A\cup A = A$.
    \item $\emptyset$ est l'élément neutre de l'union. Soit $A$ un ensemble, alors $A\cup\emptyset=A$
    \end{enumerate}
  \end{properties}

\end{frame}


\begin{notes}
  \begin{itemize}
  \item Quand on écrit $A\cup B = B\cup A$, ou plus généralement quand on considère l'égalité de deux ensembles, il faut comprendre que $A\cup B \subseteq B\cup A$ et $B\cup A \subseteq A\cup B$. Pour montrer l'égalité de deux ensembles il faut donc montrer deux inclusions.\newline

  \item On peut généraliser en considérant un ensemble dénombrable d'ensembles. Par exemple, soit $E = \{E_i, i\in \mathbb N\}$ où chaque $E_i$ est un ensemble. On peut définir l'union de ces ensembles~: $\mathcal E = \bigcup_{i\in\mathbb N}E_i$. La proposition  $x\in\mathcal E$ est équivalente à la proposition $\exists i\in\mathbb N \text{ tel que } x\in E_i$.
  \end{itemize}

  \bigskip

  \textbf{Preuve de Propriétés \hyperlink{slide_ensembles_union}{\ref{properties:union}}.} (2) Pour établir la commutativité nous devrions montrer que si $x\in A\cup B$ alors $x\in B\cup A$ et que si $x\in B\cup A$ alors $x\in A\cup B$ (voir la première remarque plus haut). La preuve est très simple, car l'opérateur union hérite des propriétés de l'opérateur logique $\lor$ (la disjonction). En effet,
  \[
    \begin{split}
      x \in A \cup B &\Leftrightarrow x \in A \lor x\in B \text{, par définition de l'union}\\
      &\Leftrightarrow x \in B \lor x\in A \text{, car la disjonction est commutative}\\
      &\Leftrightarrow x \in B\cup A
    \end{split}
  \]
  (1) On montre tout aussi simplement que l'union est associative~:
  \[
    \begin{split}
      x \in (A\cup B)\cup C &\Leftrightarrow x \in (A \cup \in B) \lor x\in C\\
      &\Leftrightarrow (x\in A \lor x\in B) \lor x\in C\\
      &\Leftrightarrow x\in A \lor (x\in B \lor x\in C) \text{, car la disjonction est associative}\\
      &\Leftrightarrow x\in A\cup (B\cup C)
    \end{split}
  \]
  (3) On procède de la même façon en rappelant que la disjonction est idempotente, voir Propriétés \hyperlink{slide_disjonction_2}{\ref{properties:disjonction}}. (4) Car $\emptyset$ est un élément de $A$.

\end{notes}


\begin{frame}
  \frametitle{Les ensembles}
  \framesubtitle{Intersection}
  \hypertarget{slide_ensembles_intersection}{}

  \begin{definition}\label{def:intersection} L'intersection de deux ensembles $A$ et $B$, notée $A\cap B$, est un ensemble contenant les éléments communs des ensembles $A$ et $B$. Plus formellement, $A \cap B = \{x | x \in A \land x\in B\}$.
  \end{definition}

  \bigskip

  \begin{properties}\label{properties:intersection}
    \begin{enumerate}
    \item L'intersection est associative. Soient $A$, $B$ et $C$ trois ensembles, alors $(A\cap B)\cup C = A\cap (B\cap C)$.
    \item L'intersection est commutative. Soient $A$ et $B$ deux ensembles, alors $A \cap B = B \cap A$.
    \item L'intersection est idempotente. Soit $A$ un ensemble, alors $A \cap A = A$.
    \item $\Omega$ est l'élément neutre de l'intersection. Soit $A$ un ensemble, alors $A\cap\Omega=A$
    \end{enumerate}
  \end{properties}

\end{frame}


\begin{notes}
  \begin{itemize}
  \item On peut généraliser en considérant un ensemble dénombrable d'ensembles. Par exemple, soit $E = \{E_i, i\in \mathbb N\}$ où chaque $E_i$ est un ensemble. On peut définir l'intersection de ces ensembles~: $\mathcal E = \bigcap_{i\in\mathbb N}E_i$. La proposition  $x\in\mathcal E$ est équivalente à la proposition $\forall i\in\mathbb N, x\in E_i$.
  \end{itemize}

  \bigskip

  \textbf{Preuve de Propriétés \hyperlink{slide_ensembles_intersection}{\ref{properties:intersection}}.} On suit la même démarche que pour la preuve de Propriétés \hyperlink{slide_ensembles_union}{\ref{properties:union}}, l'intersection hérite des propriétés de la conjonction. Par exemple, pour la commutativité, on a :
  \[
    \begin{split}
      x \in A \cap B &\Leftrightarrow x \in A \land x\in B \text{, par définition de l'intersection}\\
      &\Leftrightarrow x \in B \land x\in A \text{, car la conjonction est commutative}\\
      &\Leftrightarrow x \in B\cap A
    \end{split}
  \]

\end{notes}


\begin{frame}
  \frametitle{Les ensembles}
  \framesubtitle{Différence}
  \hypertarget{slide_ensembles_difference}{}

  \begin{definition}\label{def:difference} La différence de deux ensembles $A$ et $B$, notée $A\setminus B$, est un ensemble contenant les éléments de $A$ qui n'appartiennent pas à $B$. Plus formellement, $A \setminus B = \{x\in A | x \notin B\}$.
  \end{definition}

  \bigskip

  \begin{properties}\label{properties:difference}
    \begin{enumerate}
    \item Soit $A$ un ensemble, alors $A\setminus A = \emptyset$.
    \item Soit $A$ un ensemble, alors $A\setminus \emptyset = A$.
    \item Soit $A$ un ensemble, alors $\emptyset \setminus A = \emptyset$.
    \item Soit $A$ un ensemble, alors $A\setminus\Omega = \emptyset$.
    \end{enumerate}
  \end{properties}

\end{frame}


\begin{frame}
  \frametitle{Les ensembles}
  \framesubtitle{Complémentaire}
  \hypertarget{slide_ensembles_complementaire}{}

  \begin{definition}\label{def:complementaire} Le complémentaire d'un ensemble $A$, noté $\bar A$, est l'ensemble des éléments qui n'appartiennent pas à $A$. Plus formellement, $\bar A = \{x\in\Omega|x\notin A\}$.
  \end{definition}

  \bigskip

  \begin{properties}\label{properties:complementaire}
    \begin{enumerate}
    \item Soit $A$ un ensemble, alors $A \cup \bar A = \Omega$.
    \item Soit $A$ un ensemble, alors $A \cap \bar A = \emptyset$.
    \item $\bar\emptyset = \Omega$ et $\bar\Omega = \emptyset$.
    \item Soient $A$ et $B$ deux ensembles tels que $A\subseteq B$, alors $\bar B \subseteq \bar A$.
    \item Soit $A$ un ensemble, alors $\bar{\bar A} = A$.
    \end{enumerate}
  \end{properties}

\end{frame}


\begin{frame}
  \frametitle{Les ensembles}
  \framesubtitle{Distributivité}
  \hypertarget{slide_ensembles_distributivite}{}

  \begin{properties}\label{properties:distributivite} Soient $A$, $B$ et $C$ trois ensembles.
    \begin{enumerate}
    \item $A\cap (B\cup C) = (A\cap B) \cup (A\cap C)$.
    \item $A\cup (B\cap C) = (A\cup B) \cap (A\cup C)$.
    \item $C \setminus (A\cap B) = (C\setminus A) \cup (C\setminus B)$.
    \item $C \setminus (A\cup B) = (C\setminus A) \cap (C\setminus B)$.
    \end{enumerate}
  \end{properties}

\end{frame}

\begin{notes}
  \textbf{Preuve de Propriétés \hyperlink{slide_ensembles_distributivite}{\ref{properties:distributivite}}.} On se contente de montrer la première propriété, le reste est laissé comme exercice. On a~:
  \[
    \begin{split}
      x\in A\cap (B\cup C) &\Leftrightarrow x\in A \land x\in (B\cup C)\\
      & \Leftrightarrow x\in A \land (x\in B \lor x\in C)\\
      & \Leftrightarrow (x\in A \land x\in B) \lor (x\in A \land x\in C) \text{, en exploitant la distributivité de la disjonction et conjonction}\\
      & \Leftrightarrow x\in (A\cap B) \cup (A\cap C)
    \end{split}
  \]
\end{notes}


\begin{frame}
  \frametitle{Les ensembles}
  \framesubtitle{Diagramme de Venn, union}
  \hypertarget{slide_ensembles_venn_union}{}

  \begin{center}
    \begin{venndiagram3sets}
      \fillA \fillB \fillC
    \end{venndiagram3sets}\\
    $A\cup B \cup C$
  \end{center}
\end{frame}


\begin{frame}
  \frametitle{Les ensembles}
  \framesubtitle{Diagramme de Venn, intersection}
  \hypertarget{slide_ensembles_venn_intersection}{}

  \begin{center}
    \begin{venndiagram3sets}
      \fillACapBCapC
    \end{venndiagram3sets}\\
    $A \cap B \cap C$
  \end{center}
\end{frame}


\begin{frame}
  \frametitle{Les ensembles}
  \framesubtitle{Diagramme de Venn, différence}
  \hypertarget{slide_ensembles_venn_difference}{}

  \begin{center}
    \begin{venndiagram2sets}
      \fillANotB
    \end{venndiagram2sets}\\
    $A \setminus B$
  \end{center}
\end{frame}

\begin{frame}
  \frametitle{Les ensembles}
  \framesubtitle{Diagramme de Venn, complémentaire}
  \hypertarget{slide_ensembles_venn_complementaire}{}

  \begin{center}
    \begin{venndiagram2sets}
      \fillNotA
    \end{venndiagram2sets}\\
    $\bar A$
  \end{center}
\end{frame}


\begin{frame}
  \frametitle{Les ensembles}
  \framesubtitle{Opérateurs binaires et complémentaire}
  \hypertarget{slide_ensembles_morgan}{}

  \begin{theorem}[Loi de Morgan]\label{theorem:ensemble_morgan}
    Soient $A$ et $B$ deux ensembles. On a~:
    \begin{enumerate}
    \item $\overline{A\cup B} = \bar A \cap \bar B$.
    \item $\overline{A\cap B} = \bar A \cup \bar B$.
    \end{enumerate}
  \end{theorem}

  \bigskip

  \begin{theorem}\label{theorem:ensemble_diff_comp}
    Soient $A$ et $B$ deux ensembles. On a~:
    \begin{enumerate}
    \item $\overline{A\setminus B} = \bar A \cup B$.
    \item $\bar A\setminus \bar B = B\setminus A$.
    \end{enumerate}
  \end{theorem}

\end{frame}


\begin{notes}

  \textbf{Preuve du théorème \hyperlink{slide_ensembles_morgan}{\ref{theorem:ensemble_morgan}}.} Montrons que $\overline{A\cup B} = \bar A \cap \bar B$, le second point est laissé en exercice.
  \begin{itemize}
  \item[$\bullet$] Montrons d'abord que $\overline{A\cup B} \subseteq \bar A \cap \bar B$. C'est à dire que si $x$ est dans $\overline{A\cup B}$ alors $x$ est nécessairement dans $\bar A \cap \bar B$. Soit $x\in\overline{A\cup B}$, alors $x\notin A\cup B$. Or
    \[
      A\cup B = \{z | z\in A \lor z\in B \}
    \]
    Ainsi il faut que $x\notin A \land x\notin B$, c'est-à-dire $x\in\bar A \land x\in\bar B$ (par définition du complémentaire), et donc $x\in \bar A \cap \bar B$ (par définition de l'intersection). Nous avons bien montré que $\overline{A\cup B} \subseteq \bar A \cap \bar B$.
  \item[$\bullet$] Montrons que $\bar A \cap \bar B \subseteq \overline{A\cup B}$. Soit $x\in \bar A \cap \bar B$, par définition de l'intersection, cela se traduit par $x\in \bar A \land x \in \bar B$ ou encore, par définition du complémentaire, $x\notin A \land x\notin B$. Finalement, en rappelant que la négation d'une disjonction est une conjonction des négations, $\overline{x\in A \lor x\in B}$. Par définition de l'union nous avons donc~: $x\in \overline{A\cup B}$. Ainsi nous avons bien montré que $\bar A \cap \bar B \subseteq \overline{A\cup B}$
  \end{itemize}
  Au total, nous avons l'égalité des deux ensembles (puisque chacun contient l'autre). Comme dans les preuves précédentes, on voit que nous héritons directement des propriétés des opérateurs logiques.

  \bigskip

  Afin de montrer le théorème \hyperlink{slide_ensembles_morgan}{\ref{theorem:ensemble_diff_comp}} nous allons d'abord établir le lemme suivant~:\newline

  \textbf{Lemme} Soient $A$ et $B$ deux ensembles. On a $A\setminus B = A \cap\bar B$.\newline

  \textbf{Preuve} Nous avons~:
  \[
    \begin{split}
      x\in A\setminus B &\Leftrightarrow x\in A \land x\notin B\\
      &\Leftrightarrow x\in A \land x\in \bar B\\
      &\Leftrightarrow x\in A \ x\cap \bar B \text{, par définition de l'intersection}\qed
    \end{split}
  \]

  \textbf{Preuve du théorème \hyperlink{slide_ensembles_morgan}{\ref{theorem:ensemble_diff_comp}}.} (1) Direct en utilisant le lemme et la loi de Morgan. (2) Direct en utilisant le lemme et la commutativité de l'intersection.

\end{notes}


\begin{frame}
  \frametitle{Les ensembles}
  \framesubtitle{Différence symétrique}
  \hypertarget{slide_ensembles_difference_symetrique}{}

  \begin{definition}\label{def:difference_symetrique} La différence symétrique de deux ensembles $A$ et $B$, notée $A\Delta B$, est un ensemble contenant les éléments de $A$ qui n'appartiennent pas à $B$ et les éléments de $B$ qui n'appartiennent pas à $A$. Plus formellement, $A \Delta B = (A\cup B)\setminus (A\cap B)$.
  \end{definition}

  \bigskip

  \begin{properties}\label{properties:difference}
    \begin{enumerate}
    \item La différence symétrique est associative. Si $A$, $B$ et $C$ sont des ensembles, alors $A\Delta (B\Delta C)=(A\Delta B)\Delta C$.
    \item La différence symétrique est commutative. Si $A$ et $B$ sont deux ensembles, alors $A\Delta B = B\Delta A$.
    \item L'ensemble vide est l'élément neutre de la différence symétrique. Si $A$ est un ensemble, alors $A\Delta \emptyset = A$.
    \item Si $A$ est un ensemble, alors $A\Delta A = \emptyset$.
    \end{enumerate}
  \end{properties}

\end{frame}

\begin{frame}
  \frametitle{Les ensembles}
  \framesubtitle{Diagramme de Venn, différence symétrique}
  \hypertarget{slide_ensembles_venn_difference}{}

  \begin{center}
    \begin{venndiagram3sets}
      \fillOnlyA \fillOnlyB \fillOnlyC \fillACapBCapC
    \end{venndiagram3sets}\\
    $A \Delta B \Delta C$
  \end{center}
\end{frame}


\begin{frame}
  \frametitle{Les ensembles}
  \framesubtitle{Le produit cartésien}
  \hypertarget{slide_ensembles_produit}{}

  \begin{definition}\label{def:produit_cartesien} Le produit cartésien de deux ensembles $E$ et $F$ est l'ensemble des couples ordonnés $(x,y)$ où $x \in E$ et $y \in F$. On note~:
    \[
      E\times F = \{(x, y) | x\in E \land y\in F\}
    \]
  \end{definition}

  \bigskip

  \begin{itemize}

  \item[\dbend] Les couples sont ordonnés $\Rightarrow$ $(x,y) \neq (y, x)$.\newline\newline

  \item On peut généraliser en considérant le produit cartésien d'un nombre arbitraire d'ensembles. Soit $E_1, E_2, \ldots E_n$ une collection d'ensembles. Alors le produit cartésien des $n$ ensembles $E_i$ est~:
    \[
      E_1\times E_2 \times \ldots \times E_n = \{(x_1,x_2, \ldots, x_n)| x_1\in E_1 \land x_2\in E_2 \land \ldots \land E_n\}
    \]
    où les $(x_1,x_2, \ldots, x_n)$ sont des n-uplets (ordonnés).

  \end{itemize}

\end{frame}


\begin{frame}
  \frametitle{Les ensembles}
  \framesubtitle{Exemple de produit cartésien}
  \hypertarget{slide_exemple_ensembles_produit}{}


  \begin{itemize}

  \item Soit deux ensembles $A$ et $B$~:
    \[
      A = \{0,1,2,3,4\}
    \]
    \[
      B = \{1,3,5\}
    \]


  \item Le produit cartésien de $A$ et $B$ est~:
    \[
      \begin{split}
        A\times B = \{&(0,1), (0,3), (0,5)\\
        &(1,1), (1,3), (1,5)\\
        &(2,1), (2,3), (2,5)\\
        &(3,1), (3,3), (3,5)\\
        &(4,1), (4,3), (4,5)\}
      \end{split}
    \]

  \item $A$ has 5 elements, $B$ has 3 elements and $A\times B$ has $15=5\times 3$ elements.

  \end{itemize}

\end{frame}


\begin{frame}
  \frametitle{Les ensembles}
  \framesubtitle{Cardinal d'un ensemble}
  \hypertarget{slide_ensembles_cardinal}{}


  \begin{definition}\label{def:cardinal} Le cardinal d'un ensemble est le nombre d'éléments d'un ensemble. On note $\textrm{Card}(E)$ le cardinal d'un ensemble $E$. On dit qu'un ensemble est fini si son cardinal est fini.
  \end{definition}

  \bigskip

  \begin{theorem}\label{thm:cardinal} Soient $E$ et $F$ deux ensembles finis~: $\textrm{Card}(E\times F) = \textrm{Card}(E)\times \textrm{Card}(F)$.
  \end{theorem}

\end{frame}


\begin{frame}
  \frametitle{Les ensembles}
  \framesubtitle{Partitions d'un ensemble}
  \hypertarget{slide_ensembles_partitions}{}


  \begin{definition}\label{def:partitions} Tous les sous ensembles d'un ensemble $E$ peuvent être considérés comme les éléments d'un nouvel ensemble que l'on appelle \textbf{ensemble des parties} de l'ensemble $E$, noté $\mathcal P(E)$.
  \end{definition}

  \bigskip

  \begin{theorem}\label{thm:partitions} Soit $E$ un ensemble. Le nombre de sous ensemble est donné par~:
    \[
      \textrm{Card}(\mathcal P(E)) = 2^{\textrm{Card(E)}}
    \]
  \end{theorem}

\end{frame}


\begin{frame}
  \frametitle{Les ensembles}
  \framesubtitle{Exemple de Partitions}
  \hypertarget{slide_ensembles_exemple_partitions}{}

  \begin{itemize}

  \item Soit l'ensemble $E = \{a,b,c\}$.\newline

  \item Les sous ensembles sont $\{a\}$, $\{b\}$, $\{c\}$, $\{a,b\}$, $\{a,c\}$, $\{b,c\}$, $\{a,b,c\}$ et $\emptyset$ (n'oublions pas que l'ensemble vide appartient à tout ensemble).\newline

  \item Nous avons donc~:
    \[
      \textrm{Card}(\mathcal P(E)) = \left\{\emptyset, \{a\}, \{b\}, \{c\}, \{a,b\}, \{a,c\}, \{b,c\}, \{a,b,c\}\right\}
    \]

  \item On vérifie que $\textrm{Card}(\mathcal P(E)) = 8 = 2^3$.

  \end{itemize}

\end{frame}


\begin{frame}
  \frametitle{Les ensembles}
  \framesubtitle{Union et cardinal}
  \hypertarget{slide_ensembles_union_cardnal}{}

  \begin{theorem}\label{thm:union_cardinal} Soient $A$ et $B$ deux ensembles finis, alors
    \[
      \textrm{Card}(A \cup B) = \textrm{Card}(A) + \textrm{Card}(B) - \textrm{Card}(A \cap B)
    \]
  \end{theorem}

  \bigskip

  \begin{itemize}
  \item Si l'intersection est vide le cardinal de l'union est la somme des cardinaux.\newline
  \item On doit retrancher le cardinal de l'intersection, autrement les éléments de l'intersection sont comptés deux fois.\newline
  \item Il est possible de généraliser cette formule pour un nombre arbitraire d'ensembles finis.
  \end{itemize}
\end{frame}

\section{Les fonctions}

\begin{frame}
  \frametitle{Les fonctions}
  \framesubtitle{Relations et fonctions}
  \hypertarget{slide_relations_et_fonctions_1}{}

  \begin{definition}\label{def:relation} Un ensemble de paires ordonnées de nombres est appelé une relation binaire.
  \end{definition}

  \begin{itemize}
  \item L'ensemble des premiers nombres d'une relation binaire est l'\textbf{ensemble de départ}, ou le \textbf{domaine} de la relation.\newline
  \item L'ensemble des seconds nombres d'une relation binaire est l'\textbf{ensemble d'arrivée}.\newline
  \item Les ensembles suivants sont des relations binaires:
    \[
      A = \{(x,y)| x\in\mathbb N \land y\in\mathbb N \land x\leq y\}
    \]
    \[
      B = \{(x,y)| x\in\mathbb N \land y = 2x-1\}
    \]
    On note que dans le premier cas, on peut associer plus d'une valeur $y$ à chaque valeur de $x$.
  \end{itemize}

\end{frame}


\begin{frame}
  \frametitle{Les fonctions}
  \framesubtitle{Relations et fonctions}
  \hypertarget{slide_relations_et_fonctions_2}{}

  \begin{definition}\label{def:fonction} Si une relation binaire est telle qu'à chaque élément de l'ensemble de départ est associé \textbf{un et un seul élément} de l'ensemble d'arrivée, on dit que la relation binaire est une \textbf{fonction}.
  \end{definition}

  \bigskip

  \begin{itemize}
  \item Une fonction est un ensemble de paires ordonnées.\newline
  \item Toutes les fonctions sont des relations binaires, mais l'inverse n'est pas vrai.\newline
  \item Dans l'exemple de la page précédente $B$ est une fonction, mais pas l'ensemble $A$.
  \end{itemize}

\end{frame}


\begin{frame}
  \frametitle{Les fonctions}
  \framesubtitle{Notations}
  \hypertarget{slide_fonctions_notations}{}

  \begin{itemize}
  \item On représente habituellement une fonction par une lettre minuscule. Par exemple $f$ (fonction), ou encore $g$, ou $h$, \ldots\newline
  \item Si une fonction $f$ associe à chaque élément $x$ de l'ensemble de départ $E$ l'élément $y$ de l'ensemble d'arrivée $F$, on note~:
    \[
      \begin{split}
        f: \quad &E \longrightarrow F\\
        &x \longmapsto y = f(x)
      \end{split}
    \]

  \item On dit que $y$ est l'image de $x$ par la fonction $f$.
  \end{itemize}

\end{frame}


\begin{frame}
  \frametitle{Les fonctions}
  \framesubtitle{Exemple}
  \hypertarget{slide_fonctions_exemple}{}

  \begin{itemize}
  \item Soit la fonction~:
    \[
      \begin{split}
        f: \quad &\mathbb R \longrightarrow \mathbb R\\
        &x \longmapsto y = x^2+x-2
      \end{split}
    \]

  \item On peut alors évaluer cette fonction, c'est-à-dire déterminer les images associées à différentes valeurs de $x$~:

    \medskip

    \begin{itemize}
    \item $f(0) = -2$\newline
    \item $f(a) = a^2+a-2$ pour tout $a\in\mathbb R$\newline
    \item $f(x+h)-f(x) = 2xh+h^2+h$
    \end{itemize}
  \end{itemize}

\end{frame}


\begin{frame}
  \frametitle{Les fonctions}
  \framesubtitle{Somme, différence, produit et quotient}
  \hypertarget{slide_fonctions_opérations}{}

  Soient deux fonctions~:
  \begin{eqnarray*}
    \begin{split}
      f: \quad &\mathbb R \longrightarrow \mathbb R\\
      &x \longmapsto y = f(x)
    \end{split}
      &
        \quad \begin{split}
          g: \quad &\mathbb R \longrightarrow \mathbb R\\
          &x \longmapsto y = g(x)
        \end{split}
  \end{eqnarray*}
  \bigskip
  On définit les opérations suivantes~:
  \begin{description}
  \item[\textbf{Somme}] $(f+g)(x) = f(x)+g(x)$\newline
  \item[\textbf{Différence}] $(f-g)(x) = f(x)-g(x)$\newline
  \item[\textbf{Produit}] $(f \cdot g)(x) = f(x) \cdot g(x)$\newline
  \item[\textbf{Quotient}] $\left(\frac{f}{g}\right)(x) = \frac{f(x)}{g(x)}$ avec $g(x)\neq 0$\newline
  \end{description}

\end{frame}


\begin{frame}
  \frametitle{Les fonctions}
  \framesubtitle{Composition}
  \hypertarget{slide_fonctions_composition}{}

  Soient deux fonctions~:
  \begin{eqnarray*}
    \begin{split}
      f: \quad & E \longrightarrow F\\
      &x \longmapsto y = f(x)
    \end{split}
      &
        \quad \begin{split}
          g: \quad & F \longrightarrow G\\
          &x \longmapsto y = g(x)
        \end{split}
  \end{eqnarray*}
  \bigskip
  La composition de ces deux fonctions est la fonction $h(x) = (g \circ f)(x)$~:
  \[
    \begin{split}
      g \circ f: \quad & E \longrightarrow G\\
      &x \longmapsto y = g(f(x))
    \end{split}
  \]

  \bigskip

  \begin{itemize}
  \item[\dbend] En général $g(f(x))\neq f(g(x))$.
  \end{itemize}

\end{frame}


\begin{frame}
  \frametitle{Les fonctions}
  \framesubtitle{Exemple de composition}
  \hypertarget{slide_fonctions_composition_exemple}{}

  Soient deux fonctions~:
  \[
    f(x) = x^2+x+1 \quad\text{ et }\quad g(x) = x+1
  \]
  Alors on a~:
  \[
    \begin{split}
      f(g(x)) &= f(x+1)\\
      &= (x+1)^2+(x+1)+1\\
      &= x^2+3x+3
    \end{split}
  \]
  et
  \[
    \begin{split}
      g(f(x)) &= g(x^2+x+1)\\
      &= (x^2+x+1)+1\\
      &= x^2+x+2
    \end{split}
  \]
  Ces deux compositions de $f$ et $g$ sont bien différentes.

\end{frame}


\begin{frame}
  \frametitle{Les fonctions}
  \framesubtitle{Surjection}
  \hypertarget{slide_fonctions_surjection}{}

  \begin{definition}\label{def:surjection} Une fonction $f: E\longrightarrow F$ \textbf{surjective} est une fonction telle que tout élément $y$ de l'ensemble d'arrivée $F$ soit l'image d'au moins un élément $x$ de l'ensemble de départ $E$. Plus formellement une fonction $f: E\longrightarrow F$ est surjective si et seulement si $\forall y\in F$, $\exists x\in E$ tel que $y=f(x)$.
  \end{definition}

  \begin{example}\label{ex:surjection_1}
    Soient deux ensembles $E = \{x\in [-1,1]\}$ et $F = {y\in[0,1]}$, la fonction $f: E\longrightarrow F$ définie par $f(x) = |x|$ est surjective. $y$ admet un unique antécédant si $y=0$ ($x=0$), dans les autres cas $y$ admet deux antécédants ($y$ et $-y$) dans $[-1,1]$.
  \end{example}

  \begin{example}\label{ex:surjection_2}
    La fonction $f: \mathbb N \longrightarrow \mathbb N$ qui à $x$ associe $2x$ n'est pas surjective ($y$ impair n'a pas d'antécédant dans $\mathbb N$).
  \end{example}

\end{frame}


\begin{frame}
  \frametitle{Les fonctions}
  \framesubtitle{Injection}
  \hypertarget{slide_fonctions_injection}{}

  \begin{definition}\label{def:injection} Une fonction $f: E\longrightarrow F$ est \textbf{injective} si et seulement si deux éléments distincts de l'ensemble de départ ont deux images par $f$ distinctes, ou de façon équivalente si $f(x_1)= f(x_2) \Rightarrow x_1=x_2$.
  \end{definition}

  \begin{example}\label{ex:injection_1}
    La fonction $f: \mathbb R\longrightarrow \mathbb R$ définie par $f(x) = x+1$ est injective. En effet, $f(x_1)=f(x_2)\Leftrightarrow x_1+1 = x_2+1 \Rightarrow x_1 = x_2$.
  \end{example}

  \begin{example}\label{ex:injection_2}
    Soient deux ensembles $E = \{x\in [-1,1]\}$ et $F = {y\in[0,1]}$, la fonction $f: E\longrightarrow F$ définie par $f(x) = |x|$ n'est pas injective. Par exemple $x=1$ et $x=-1$ ont la même image par $f$ ($y=1)$.
  \end{example}

\end{frame}


\begin{frame}
  \frametitle{Les fonctions}
  \framesubtitle{Bijection}
  \hypertarget{slide_fonctions_bijection}{}

  \begin{definition}\label{def:bijection} Une fonction $f: E\longrightarrow F$ est \textbf{bijective} si et seulement si elle est injective et surjective. Dans ce cas chaque élément de l'ensemble d'arrivée est l'image d'un unique élément $x$ de l'ensemble de départ.
  \end{definition}

  \bigskip

  \begin{example}\label{ex:bijection_1}
    La fonction $f: \mathbb R\longrightarrow \mathbb R$ définie par $f(x) = x+1$ est bijective. Nous savons déjà qu'elle est injective, de plus il est évident que pour chaque élément $y\in\mathbb R$ on peut trouver un antécédant dans $\mathbb R$ ($x = y-1$).
  \end{example}

\end{frame}


\begin{frame}
  \frametitle{Les fonctions}
  \framesubtitle{Patates et relation binaire}
  \hypertarget{slide_fonctions_fig_1}{}

  \begin{center}
    \begin{tikzpicture}[scale=1.2]
      \filldraw[fill=blue!20, draw=blue!60] (-1.5,0) circle (1cm);
      \filldraw[fill=red!20, draw=red!60] (1.5,0) circle (1cm);
      \node at (-1.5,1.5) {$E$};
      \node at (1.5,1.5) {$F$};
      \node (x1) at (-1.3,0.7) {$x_1$};
      \node (x2) at (-1.7,0.3) {$x_2$};
      \node (x3) at (-1.2,-0.2) {$x_3$};
      \node (x4) at (-1.5,-0.7) {$x_4$};
      \node (y1) at (1.5,0.8) {$y_1$};
      \node (y2) at (1.7,0.2) {$y_2$};
      \node (y3) at (1.9,-0.3) {$y_3$};
      \node (y4) at (1.6,-0.85) {$y_4$};
      \draw[->] (x1) -- (y4);
      \draw[->] (x1) -- (y1);
      \draw[->] (x2) -- (y2);
      \draw[->] (x3) -- (y1);
      \draw[->] (x4) -- (y3);

    \end{tikzpicture}

    \bigskip

    \textbf{Ceci n'est pas une fonction,} car $x_1$ a deux images ($y_1$ et $y_4$)

  \end{center}

\end{frame}


\begin{frame}
  \frametitle{Les fonctions}
  \framesubtitle{Patates et fonction}
  \hypertarget{slide_fonctions_fig_2}{}

  \begin{center}
    \begin{tikzpicture}[scale=1.2]
      \filldraw[fill=blue!20, draw=blue!60] (-1.5,0) circle (1cm);
      \filldraw[fill=red!20, draw=red!60] (1.5,0) circle (1cm);
      \node at (-1.5,1.5) {$E$};
      \node at (1.5,1.5) {$F$};
      \node (x1) at (-1.3,0.7) {$x_1$};
      \node (x2) at (-1.7,0.3) {$x_2$};
      \node (x3) at (-1.2,-0.2) {$x_3$};
      \node (x4) at (-1.5,-0.7) {$x_4$};
      \node (y1) at (1.5,0.8) {$y_1$};
      \node (y2) at (1.7,0.2) {$y_2$};
      \node (y3) at (1.9,-0.3) {$y_3$};
      \node (y4) at (1.6,-0.85) {$y_4$};
      \draw[->] (x1) -- (y1);
      \draw[->] (x2) -- (y2);
      \draw[->] (x3) -- (y1);
      \draw[->] (x4) -- (y3);

    \end{tikzpicture}

    \bigskip

    \textbf{Ceci est une fonction,} tout élément de $E$ a une unique image dans $F$

  \end{center}

\end{frame}


\begin{frame}
  \frametitle{Les fonctions}
  \framesubtitle{Patates et fonction surjective}
  \hypertarget{slide_fonctions_fig_3}{}

  \begin{center}
    \begin{tikzpicture}[scale=1.2]
      \filldraw[fill=blue!20, draw=blue!60] (-1.5,0) circle (1cm);
      \filldraw[fill=red!20, draw=red!60] (1.5,0) circle (1cm);
      \node at (-1.5,1.5) {$E$};
      \node at (1.5,1.5) {$F$};
      \node (x1) at (-1.3,0.7) {$x_1$};
      \node (x2) at (-1.7,0.3) {$x_2$};
      \node (x3) at (-1.2,-0.2) {$x_3$};
      \node (x4) at (-1.5,-0.7) {$x_4$};
      \node (y1) at (1.5,0.8) {$y_1$};
      \node (y2) at (1.7,0.2) {$y_2$};
      \node (y3) at (1.9,-0.3) {$y_3$};
      \draw[->] (x1) -- (y1);
      \draw[->] (x2) -- (y2);
      \draw[->] (x3) -- (y1);
      \draw[->] (x4) -- (y3);

    \end{tikzpicture}

    \bigskip

    \textbf{Ceci est une surjection,} tout élément de $F$ a un antécédant dans $E$

  \end{center}

\end{frame}


\begin{frame}
  \frametitle{Les fonctions}
  \framesubtitle{Patates et fonction injective}
  \hypertarget{slide_fonctions_fig_4}{}

  \begin{center}
    \begin{tikzpicture}[scale=1.2]
      \filldraw[fill=blue!20, draw=blue!60] (-1.5,0) circle (1cm);
      \filldraw[fill=red!20, draw=red!60] (1.5,0) circle (1cm);
      \node at (-1.5,1.5) {$E$};
      \node at (1.5,1.5) {$F$};
      \node (x1) at (-1.3,0.7) {$x_1$};
      \node (x2) at (-1.7,0.3) {$x_2$};
      \node (x4) at (-1.5,-0.7) {$x_4$};
      \node (y1) at (1.5,0.8) {$y_1$};
      \node (y2) at (1.7,0.2) {$y_2$};
      \node (y3) at (1.9,-0.3) {$y_3$};
      \node (y4) at (1.6,-0.85) {$y_4$};
      \draw[->] (x1) -- (y1);
      \draw[->] (x2) -- (y2);
      \draw[->] (x4) -- (y3);

    \end{tikzpicture}

    \bigskip

    \textbf{Ceci est une injection,} $f(y_i)=f(y_j)\Rightarrow x_i=x_j$.

  \end{center}

\end{frame}


\begin{frame}
  \frametitle{Les fonctions}
  \framesubtitle{Patates et fonction bijective}
  \hypertarget{slide_fonctions_fig_5}{}

  \begin{center}
    \begin{tikzpicture}[scale=1.2]
      \filldraw[fill=blue!20, draw=blue!60] (-1.5,0) circle (1cm);
      \filldraw[fill=red!20, draw=red!60] (1.5,0) circle (1cm);
      \node at (-1.5,1.5) {$E$};
      \node at (1.5,1.5) {$F$};
      \node (x1) at (-1.3,0.7) {$x_1$};
      \node (x2) at (-1.7,0.3) {$x_2$};
      \node (x3) at (-1.2,-0.2) {$x_3$};
      \node (x4) at (-1.5,-0.7) {$x_4$};
      \node (y1) at (1.5,0.8) {$y_1$};
      \node (y2) at (1.7,0.2) {$y_2$};
      \node (y3) at (1.9,-0.3) {$y_3$};
      \node (y4) at (1.6,-0.85) {$y_4$};
      \draw[->] (x1) -- (y1);
      \draw[->] (x2) -- (y2);
      \draw[->] (x3) -- (y3);
      \draw[->] (x4) -- (y4);

    \end{tikzpicture}

    \bigskip

    \textbf{Ceci est une bijection.} Notons qu'il s'agit du seul cas où on peut imaginer sans équivoque une fonction qui nous amène de $F$ vers $E$.

  \end{center}

\end{frame}


\begin{frame}
  \frametitle{Les fonctions}
  \framesubtitle{Inverse ou réciproque, I}
  \hypertarget{slide_fonctions_reciproque_1}{}

  \begin{definition}\label{def:fonction_reciproque}
    Soit $f : E\rightarrow F$ une fonction bijective. Par définition des bijections, tout élément $y\in F$ possède un antécédent et un seul par $f$. On peut donc définir une application $f^{-1}: F\rightarrow E$, qui à $y\in F$ associe son unique antécédent $x\in E$. $f^{-1}$ est la fonction inverse ou réciproque associée à $f$.
  \end{definition}

  \bigskip

  \begin{itemize}
  \item $f^{-1}$ est aussi une bijection.\newline
  \item[\dbend] Ne pas confondre $f^{-1}$ et $\frac{1}{f}$.\newline
  \item Si $f$ est une bijection alors $(f \circ f^{-1})(x) = (f^{-1} \circ f)(x) = x$ pour tout $x$.
  \end{itemize}

\end{frame}


\begin{frame}
  \frametitle{Les fonctions}
  \framesubtitle{Inverse ou réciproque, II}
  \hypertarget{slide_fonctions_reciproque_2}{}

  \begin{example}\label{def:fonction_reciproque}
    Soit $f(x) = 2x+4$ une fonction de $\mathbb R$ dans $\mathbb R$. Il s'agit d'une bijection. La fonction inverse $f^{-1}(x)$ s'obtient de la façon suivante~:
    \[
      \begin{split}
        y &= 2x+4\\
        \Leftrightarrow y-4 &=2x\\
        \Leftrightarrow x &=\frac{y-4}{2}\\
      \end{split}
    \]
    Nous avons donc obtenu une expression de $x$ comme une fonction de $y$, $g(y)=\frac{y-4}{2}$. En inversant $x$ et $y$, puisqu'il est d'usage de noter $x$ un élément de l'ensemble de départ et $y$ son image, on obtient la fonction inverse $f^{-1}(x) = \frac{x-4}{2}$. On vérifie facilement que $f(f^{-1}(x)) = x = f^{-1}(f(x))$. Par exemple~:
    \[
      f(f^{-1}(x)) = 2 \left(\frac{x-4}{2}\right)+4 = x-4+4 = x
    \]
  \end{example}

\end{frame}

\section{Les quantificateurs}

\begin{frame}
  \frametitle{Les quantificateurs}
  \framesubtitle{}
  \hypertarget{slide_quantificateur_def}{}

  \begin{definition}\label{def:quantificateurs_universel_existentiel}
    $\forall$ s'appelle le quantificateur universel et $\exists$ s'appelle le quantificateur existentiel.
  \end{definition}

  \bigskip

  \begin{itemize}
  \item La proposition « $\forall x\in E, P(x)$ » se lit « Pour tous les éléments $x$ de $E$ la proposition $P(x)$ est vraie »\newline
  \item La proposition « $\exists x\in E | P(x) $ » se lit « Il existe au moins un élément $x$ dans $E$ tel que la proposition $P(x)$ est vraie »\newline
  \item La proposition « $\exists! x\in E | P(x) $ » se lit « Il existe un et un seul élément $x$ dans $E$ tel que la proposition $P(x)$ est vraie »\newline
  \item La proposition « $\nexists x\in E | P(x) $ » se lit « Il n'existe pas $x$ dans $E$ tel que la proposition $P(x)$ est vraie », on pourrait aussi écrire « $\forall x\in E | \overline{P(x)} $ »
  \end{itemize}

\end{frame}


\begin{frame}
  \frametitle{Les quantificateurs}
  \framesubtitle{Propriétés, I}
  \hypertarget{slide_quantificateur_proprietes_1}{}

  \begin{theorem}\label{thm:quantificateurs_negation}
    Soient $E$ un ensemble et $P(x)$ une proposition dont la valeur de vérité dépend des éléments $x$ dans $E$. On a~:
    \begin{enumerate}
    \item $\overline{\forall x\in E, P(x)} \quad \Leftrightarrow \quad \exists x\in E| \overline{P(x)}$
    \item $\overline{\exists x\in E| P(x)} \quad \Leftrightarrow \quad \forall x\in E, \overline{P(x)}$
    \end{enumerate}
  \end{theorem}

  \begin{theorem}\label{thm:quantificateurs_arithmetic_1}
    Soient $E$ un ensemble, $P(x)$ et $Q(x)$ deux propositions dont les valeurs de vérité dépendent des éléments $x$ dans $E$. On a~:
    \begin{enumerate}
    \item $(\forall x\in E, P(x) \land Q(x)) \quad \Leftrightarrow \quad (\forall x\in E, P(x)) \land (\forall x\in E, Q(x))$
    \item $(\exists x\in E| P(x) \lor Q(x)) \quad \Leftrightarrow \quad (\exists x\in E| P(x)) \lor (\exists x\in E| Q(x))$
    \item $(\exists x\in E| P(x) \land Q(x)) \quad \Rightarrow \quad (\exists x\in E| P(x)) \land (\exists x\in E| Q(x))$
    \item $(\forall x\in E, P(x) \lor Q(x)) \quad \Leftarrow \quad (\forall x\in E, P(x)) \lor (\forall x\in E, Q(x))$

    \end{enumerate}

  \end{theorem}

\end{frame}


\begin{notes}
  \begin{itemize}

  \item Pour comprendre pourquoi nous n'avons pas l'équivalence dans le point 3 du théorème \hyperlink{slide_quantificateurs_proprietes_1}{\ref{thm:quantificateurs_arithmetic_1}}, il faut noter que s'il est possible de trouver un $x$ tel que $P(x)$ est vraie et un (autre) $x$ tel que $Q(x)$ est vraie, rien n'assure qu'il soit possible de trouver un $x$ tel que les deux propositions $P(x)$ et $Q(x)$ soient simultanément vraies.\newline

  \item De même pour comprendre l'absence d'équivalence dans point 4 du théorème \hyperlink{slide_quantificateurs_proprietes_1}{\ref{thm:quantificateurs_arithmetic_1}}, remarquons que la proposition « dans la classe, tout élève est un garçon ou une fille » est vraie mais que la proposition  « dans la classe, tout élève est un garçon ou tout élève est une fille » est fausse (en général).\newline

  \item[\dbend] On peut distribuer $\forall$ sur $\land$ et $\exists$ sur $\lor$, \textbf{mais} on ne peut pas distribuer $\forall$ sur $\lor$ ou $\exists$ sur $\land$.

  \end{itemize}
\end{notes}


\begin{frame}
  \frametitle{Les quantificateurs}
  \framesubtitle{Propriétés, II}
  \hypertarget{slide_quantificateur_proprietes_2}{}

  \begin{theorem}\label{thm:quantificateurs_perm_1}
    Soient $E$ et $F$ deux ensembles et $P(x,y)$ une proposition dont la valeur de vérité dépend des éléments $x$ dans $E$ et $y$ dans $F$. On a~:
    \begin{enumerate}
    \item $(\forall x\in E, \forall y\in F, P(x,y)) \quad \Leftrightarrow \quad (\forall y\in F, \forall x\in E, P(x,y))$
    \item $(\exists x\in E, \exists y\in F, P(x,y)) \quad \Leftrightarrow \quad (\exists y\in F, \exists x\in E, P(x,y))$
    \end{enumerate}
  \end{theorem}

  \bigskip

  \begin{itemize}
  \item On peut permuter deux quantificateurs universels, ou deux quantificateurs existentiels\ldots Mais
  \end{itemize}

  \begin{theorem}\label{thm:quantificateurs_perm_2}
    Soient $E$ et $F$ deux ensembles et $P(x,y)$ une proposition dont la valeur de vérité dépend des éléments $x$ dans $E$ et $y$ dans $F$. On a~:
    \[
      (\exists x\in E | \forall y\in F, P(x,y)) \quad \Rightarrow (\forall y\in F, \exists x\in E| P(x,y))
    \]
  \end{theorem}

\end{frame}


\end{document}

% Local Variables:
% ispell-check-comments: exclusive
% ispell-local-dictionary: "francais"
% TeX-master: t
% End: