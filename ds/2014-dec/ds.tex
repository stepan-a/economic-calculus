\documentclass[10pt,a4paper,notitlepage]{article}
\usepackage{amsmath}
\usepackage{amssymb}
\usepackage{amsbsy}
\usepackage{float}
\usepackage[french]{babel}
\usepackage{graphicx}
\usepackage{enumerate}

\usepackage[utf8x]{inputenc}
\usepackage[T1]{fontenc}
\usepackage{palatino}

 \usepackage[active]{srcltx}
\usepackage{scrtime}

\newcommand{\exercice}[1]{\textsc{\textbf{Exercice}} #1}
\newcommand{\question}[1]{\textbf{(#1)}}
\setlength{\parindent}{0cm}

\begin{document}

\title{\textsc{Calcul Économique\\ \small{(Devoir surveillé n°1)}}}
\author{Stéphane Adjemian\thanks{Université du Maine, Gains. \texttt{stephane DOT adjemian AT univ DASH lemans DOT fr}}}
\date{Le \today\ à \thistime}

\maketitle


\exercice{1} Montrer la transitivité de l'implication logique.

\bigskip

\exercice{2} Montrer par récurrence :
\begin{enumerate}[(a)]
\item $\sum_{i=1}^n 2i-1 = 1+3+5+\dots+(2n-1)= n^2$
\item $\sum_{i=1}^n i(i+1) = 1\times2 + 2\times 3 + \dots +
  n\times(n+1) = \frac{n(n+1)(n+2)}{3}$ 
\end{enumerate}

\bigskip

\exercice{3} Soient les fonctions d'offre et de demande :
\[
\begin{split}
  D(p):& q = a - p\\
  S(p):& q = b + 2p
\end{split}
\] 
où $a$ et $b$ sont des paramètres réels positifs. \textbf{(1)} Interpréter les
paramètres $a$ et $b$. \textbf{(2)} Représenter graphiquement ces
fonctions. \textbf{(3)} Déterminer sous quelle condition un prix d'équilibre
$p^{\star}$ existe. Déterminer ce prix.

\bigskip

\exercice{4} Déterminer les solutions de l'équation suivante :
\[
x^2 - 4x\sqrt{2} + 6
\]

\bigskip

\exercice{5} 
Soit la fonction $f(x) = x^2+2x+2$ définie pour toutes valeurs de $x$
dans $\mathbb R$. Identifier $x^{\star}$ qui minimise $f$, calculer $f(x^{\star})$.

\bigskip

\exercice{6} Soient quatre ensemble $A$, $B$, $C$ et $D$. Déterminer :
\begin{enumerate}
\item $\mathrm{card}(A \cup B \cup C)$
\item $\mathrm{card}(A \cup B \cup C \cup D)$
\end{enumerate}


\end{document}


% Exercice (1)
Vu dans la fiche de TD n°1. Il faut montrer que pour trois
propositions P, Q et R on a P=>Q et Q=>R implique P=>R.

% Exercice 2(1)
$\sum_{i=1}^{n+1} 2i-1 = \sum_{i=1}^{n} 2i-1 + 2(n+1)-1 = n^2+
2(n+1)-1 = n^2+2n+1 = (n+1)^2$

% Exercice 2(2)
$\sum_{i=1}^{n+1} i(i+1) = \sum_{i=1}^{n} i(i+1) + (n+1)(n+2) =
\frac{n(n+1)(n+2)}{3} + (n+1)(n+2) = 
\frac{n(n+1)(n+2)+3(n+1)(n+2)}{3} =
\frac{(n+1)(n+2)(n+3)}{3}$

% Exercice 3(1)
Le prix d'équilibre doit vérifier :
$ a-p = b+2p$
soit de façon équivalente
$3p = a-b$
ou
$p = (a-b)/3$

% Exercice 4
$(x-\sqrt{2})(x-3\sqrt{2}) = x^2 - 4x\sqrt{2} + 6$

$\Delta = 16*2 - 24 = 32 - 24 = 8$
$x_1 = 2\sqrt(2) + .5*\sqrt{4*2} = 2\sqrt(2) + \sqrt{2} = 3\sqrt{2}$
$x_2 = 2\sqrt(2) - .5*\sqrt{4*2} = \sqrt{2}$

% Exercice 5
1/x = x