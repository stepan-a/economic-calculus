\documentclass[10pt,a4paper,notitlepage]{article}


\usepackage{amsmath}
\usepackage{amssymb}
\usepackage{amsbsy}
\usepackage{float}
\usepackage[french]{babel}
\usepackage{graphicx}
\usepackage{enumerate}
\usepackage{color, colortbl}

\usepackage{palatino}

\usepackage[active]{srcltx}
\usepackage{scrtime}

\newcommand{\exercice}[1]{\textsc{\textbf{Exercice}} #1}
\newcommand{\question}[1]{\textbf{(#1)}}
\providecommand{\lxor}{\oplus}

\setlength{\parindent}{0cm}

\definecolor{gray}{gray}{0.9}
\newcolumntype{g}{>{\columncolor{gray}}c}


\begin{document}

\title{\textsc{Calcul Économique\\ \small{(Devoir surveillé, éléments de correction)}}}
\author{Stéphane Adjemian\thanks{Université du Mans. \texttt{stephane DOT adjemian AT univ DASH lemans DOT fr}}}
\date{Le \today\ à \thistime}

\maketitle

\exercice{1} \textbf{(1)} La disjonction exclusive si et seulement si exactement une des deux propositions
est vraie. On peut définir ce connecteur logique à l'aide de la table
de vérité suivante :

\begin{table}[H]
  \centering
  \begin{tabular}[H]{ccc}
    \hline
    $P$ & $Q$ & $P \lxor Q$ \\ \hline
    V & V & F \\
    V & F & V \\
    F & V & V \\
    F & F & F \\ \hline\hline
  \end{tabular}
\end{table}

\textbf{(2)} On peut montrer, par exemple à l'aide d'une table de vérité, que $(P \lxor Q) \Leftrightarrow ((P \lor Q) \land \overline{P\land Q})$. \textbf{(3)} On a :
\[
  \begin{split}
    \overline{P \lxor Q} &\Leftrightarrow \overline{(P \lor Q) \land \overline{P\land Q}}\\
    &\Leftrightarrow \overline{P \lor Q} \lor \overline{\overline{P\land Q}}\\
    &\Leftrightarrow (\overline{P} \land \overline{Q}) \lor (P\land Q) 
  \end{split}
\]

\bigskip

\exercice{2} Soient 3 propositions $P$, $Q$ et $R$, nous devons montrer que
\[
  (P\Rightarrow Q) \land (Q\Rightarrow R) \Rightarrow (P\Rightarrow R)
\]
  Nous utilisons une table de vérité contenant huit lignes psuique nous travaillons
  avec trois propositions: $P$, $Q$ et $R$ :
  \begin{table}[H]
    \begin{center}
    \scalebox{.7}{
    \begin{tabular}[H]{|ccc|ccccg|}
      \hline
      $P$ & $Q$ & $R$ & $P\Rightarrow Q$ & $Q\Rightarrow R$ & $(P\Rightarrow Q) \land (Q\Rightarrow R)$ & $P\Rightarrow R$ & $((P\Rightarrow Q) \land (Q\Rightarrow R))\Rightarrow (P\Rightarrow R)$\\ \hline
      V & V & V & V & V & V & V & V \\
      V & V & F & V & F & F & F & V \\
      V & F & V & F & V & F & V & V \\
      V & F & F & F & V & F & F & V \\
      F & V & V & V & V & V & V & V \\
      F & V & F & V & F & F & V & V \\
      F & F & V & V & V & V & V & V \\
      F & F & F & V & V & V & V & V \\
      \hline\hline
    \end{tabular}}
  \end{center}
  \end{table}
  Comme la dernière colonne est vraie sur toutes les lignes,
  c'est-à-dire pour tout triplet de valeurs de vérité des propositions
  $P$, $Q$ et $R$, la proposition relative à la transitivité de
  l'implication logique est vraie.

\bigskip
  
\exercice{3} Montrer que :
\[
  \sum_{i=1}^n i(i+1) = \frac{n(n+1)(n+2)}{3}
\]
Vérifions que la formule est correcte au rang $n=1$. Dans ce cas le
membre de gauche de l'égalité est égale à 2, et le membre de droite donne :
\[
\frac{1\times 2\times 3}{3} = \frac{6}{3} = 2 
\]
La formule est donc correcte au rang 1. Montrons que si la formule est  vraie à un rang $n$ quelconque, alors elle est nécessairement vraiment vraie au rang suivant $n+1$. Nous supposons donc que :
\[
\sum_{i=1}^n i(i+1) = \frac{n(n+1)(n+2)}{3}
\]
est vraie, et cherchons à montrer que :
\[
\sum_{i=1}^{n+1} i(i+1) = \frac{(n+1)(n+2)(n+3)}{3}
\]
Nous avons :
\[
  \begin{split}
    \sum_{i=1}^{n+1} i(i+1) &= \sum_{i=1}^n i(i+1) + (n+1)(n+2)\\
    &= \frac{n(n+1)(n+2)}{3} + (n+1)(n+2) \\
    &= \frac{n(n+1)(n+2)+3(n+1)(n+2)}{3} \\
    &= \frac{(n+1)(n+2)(n+3)}{3}
  \end{split}
\]
ce qu'il fallait démontrer. La formule est donc toujours vraie,
puisque nous savons qu'elle est vraie au rang $n=1$ et puisque nous
venons de montrer que si elle est vraie au rang $n$ alors elle est vraie au rang $n+1$.

\bigskip

\exercice{4} Soient les fonctions d'offre et de demande :
\[
\begin{split}
  D(p):& q = a - p\\
  S(p):& q = b + 2p
\end{split}
\] 
où $a$ et $b$ sont des paramètres réels positifs. \textbf{(1)} Les
paramètres $a$ et $b$ représentent respectivement la demande et
l'offre de bien quand le prix est nul. \textbf{(2)} Dans le plan prix
quantité, la demande est une droite décroissante de pente -1 reliant
les points $(0,a)$ et $(a,0)$. Dans le même plan, l'offre est une
droite croissante de pente 2 reliant les points $(0,b)$ et $(-b/2, 0)$
\textbf{(3)} Pour qu'un prix d'équilibre existe il faut les deux
droites aient un point d'intersection, ce qui est le cas ici puisque
pentes sont différentes, et que ce point d'intersection se situe dans
l'orthan positif, puisqu'un prix doit être positif. Cela n'est
possible que si $a>b$. Sous cette condition le prix d'équilibre est:
\[
p^{\star} = \frac{a-b}{3}
\]

\bigskip

\exercice{5} Le discriminant est $\Delta = 1 - 4 = -3$. Puisque le déterminant est négatif, nous savons que l'équation admet deux solutions complexes conjuguées :
\[
x^{\star} = \frac{-1\pm i\sqrt{3}}{2}
\]

\bigskip

\exercice{6} En utilisant une identité remarquable on a directement :
\[
f(x) = (x+1)^{2} + 1
\]
Cette fonction est supérieure ou égale à 1 pour tout $x\in\mathbb R$
puisque le premier terme est le carré dn nombre réel (et donc positif). Pour minimiser $f$, il faut minimiser le premier terme. Comme celui-ci est positif ou nul, il faut trouver $x$ tel que le premier terme est nul. Le minimum est atteint lorsque $x$ est égal à -1.


\end{document}

%%% Local Variables:
%%% mode: latex
%%% TeX-master: t
%%% End:
