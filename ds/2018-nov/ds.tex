\documentclass[10pt,a4paper,notitlepage]{article}
\synctex=1
\usepackage{amsmath}
\usepackage{amssymb}
\usepackage{amsbsy}
\usepackage{float}
\usepackage[french]{babel}
\usepackage{graphicx}
\usepackage{enumerate}

\usepackage[utf8x]{inputenc}
\usepackage[T1]{fontenc}
\usepackage{palatino}

 \usepackage[active]{srcltx}
\usepackage{scrtime}

\newcommand{\exercice}[1]{\textsc{\textbf{Exercice}} #1}
\newcommand{\question}[1]{\textbf{(#1)}}
\providecommand{\lxor}{\oplus}

\setlength{\parindent}{0cm}

\begin{document}

\title{\textsc{Calcul Économique\\ \small{(Devoir surveillé)}}}
\author{Stéphane Adjemian\thanks{Université du Maine, Gains. \texttt{stephane DOT adjemian AT univ DASH lemans DOT fr}}}
\date{Le \today\ à \thistime}

\maketitle

\exercice{1} Nous avons vu en cours la définition de la disjonction
entre deux propostions $P$ et $Q$, notée $P \lor Q$, à l'aide d'une
table de vérité. Cette définition est inclusive dans le sens où la
disjonction est vraie dès lors qu'au moins une des deux propositions
est vraie (les deux peuvent être simultanément vraies). \textbf{(1)}
Définir, à l'aide d'une table de vérité, la disjonction exclusive
de deux propositions $P$ et $Q$. On notera $P \lxor Q$ la
disjonction exclusive. \textbf{(2)} Exprimer la disjonction exclusive à
l'aide de disjonction (inclusive, celle que nous avons étudiée en
cours) $lor$, conjonction(s) $\land$ et négation(s). \textbf{(3)}
Comment faut-il réécrire la loi de Morgan (c-à-d calculez la négation de
$P\lxor Q$) ?

\bigskip

\exercice{2} Montrer la transitivité de l'implication logique.

\bigskip

\exercice{3} Montrer que :
\[
  \sum_{i=1}^n i(i+1) = \frac{n(n+1)(n+2)}{3}
\]

\bigskip

\exercice{4} Soient les fonctions d'offre et de demande :
\[
\begin{split}
  D(p):& q = a - p\\
  S(p):& q = b + 2p
\end{split}
\] 
où $a$ et $b$ sont des paramètres réels positifs. \textbf{(1)} Interpréter les
paramètres $a$ et $b$. \textbf{(2)} Représenter graphiquement ces
fonctions. \textbf{(3)} Déterminer sous quelle condition un prix d'équilibre
$p^{\star}$ existe. Déterminer ce prix.

\bigskip

\exercice{5} Déterminer les solutions de l'équation suivante :
\[
x^2 + x + 1 = 0
\]

\bigskip

\exercice{6} 
Soit la fonction $f(x) = x^2+2x+2$ définie pour toutes valeurs de $x$
dans $\mathbb R$. Identifier $x^{\star}$ qui minimise $f$, calculer $f(x^{\star})$.


\end{document}


% Exercice (1)
Vu dans la fiche de TD n°1. Il faut montrer que pour trois
propositions P, Q et R on a P=>Q et Q=>R implique P=>R.

% Exercice 2(1)
$\sum_{i=1}^{n+1} 2i-1 = \sum_{i=1}^{n} 2i-1 + 2(n+1)-1 = n^2+
2(n+1)-1 = n^2+2n+1 = (n+1)^2$

% Exercice 2(2)
$\sum_{i=1}^{n+1} i(i+1) = \sum_{i=1}^{n} i(i+1) + (n+1)(n+2) =
\frac{n(n+1)(n+2)}{3} + (n+1)(n+2) = 
\frac{n(n+1)(n+2)+3(n+1)(n+2)}{3} =
\frac{(n+1)(n+2)(n+3)}{3}$

% Exercice 3(1)
Le prix d'équilibre doit vérifier :
$ a-p = b+2p$
soit de façon équivalente
$3p = a-b$
ou
$p = (a-b)/3$

% Exercice 4
$(x-\sqrt{2})(x-3\sqrt{2}) = x^2 - 4x\sqrt{2} + 6$

$\Delta = 16*2 - 24 = 32 - 24 = 8$
$x_1 = 2\sqrt(2) + .5*\sqrt{4*2} = 2\sqrt(2) + \sqrt{2} = 3\sqrt{2}$
$x_2 = 2\sqrt(2) - .5*\sqrt{4*2} = \sqrt{2}$

% Exercice 5
1/x = x

%%% Local Variables:
%%% mode: latex
%%% TeX-master: t
%%% End:
