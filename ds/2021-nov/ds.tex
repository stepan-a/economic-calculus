\documentclass[10pt,a4paper,notitlepage]{article}
\synctex=1
\usepackage{amsmath}
\usepackage{amssymb}
\usepackage{amsbsy}
\usepackage{float}
\usepackage[french]{babel}
\usepackage{graphicx}
\usepackage{enumerate}

\usepackage{palatino}

 \usepackage[active]{srcltx}
\usepackage{scrtime}

\newcommand{\exercice}[1]{\textsc{\textbf{Exercice}} #1}
\newcommand{\question}[1]{\textbf{(#1)}}
\providecommand{\lxor}{\oplus}

\setlength{\parindent}{0cm}

\begin{document}

\title{\textsc{Calcul Économique\\ \small{(Devoir surveillé)}}}
\author{Stéphane Adjemian\thanks{Université du Mans. \texttt{stephane DOT adjemian AT univ DASH lemans DOT fr}}}
\date{Le \today\ à \thistime}

\maketitle

\exercice{1} Nous avons vu en cours la définition de la disjonction
entre deux propostions $P$ et $Q$, notée $P \lor Q$, à l'aide d'une
table de vérité. Cette définition est inclusive dans le sens où la
disjonction est vraie dès lors qu'au moins une des deux propositions
est vraie (les deux peuvent être simultanément vraies). \textbf{(1)}
Définir, à l'aide d'une table de vérité, la disjonction exclusive
de deux propositions $P$ et $Q$. On notera $P \lxor Q$ la
disjonction exclusive. \textbf{(2)} Exprimer la disjonction exclusive à
l'aide de disjonction (inclusive, celle que nous avons étudiée en
cours) $\lor$, conjonction(s) $\land$ et négation(s). \textbf{(3)}
Comment faut-il réécrire la loi de Morgan (c-à-d calculez la négation de
$P\lxor Q$) ?

\bigskip

\exercice{2} Soient deux propositions $P$ et $Q$. Montrez que ces deux
propositions sont équivalentes si et seulement si $P$ implique $Q$ et $Q$
implique $P$.

\bigskip

\exercice{3} Montrez par récurrence que~:
\[
  \sum_{i=1}^n i(i+1) = \frac{n(n+1)(n+2)}{3}
\]

\bigskip

\exercice{4} Soient les fonctions de demande et d'offre~:
\[
\begin{split}
  D(p):& q = -p^2+2\\
  S(p):& q = p^2
\end{split}
\] 
\textbf{(1)} Représentez graphiquement ces fonctions. \textbf{(2)} Déterminez,
s'il existe, le prix d'équilibre sur ce marché. Quelles sont les quantités
offertes et demandées à l'équilibre (s'il existe)~?

\bigskip

\exercice{5} Déterminer les solutions de l'équation suivante :
\[
x^3+x^2-\frac{1}{4}x-\frac{1}{4} = 0
\]

\exercice{6} Soit la suite de terme général :
\[
u_n = \sqrt{n+1}-\sqrt{n}\quad\forall n\geq 1
\]
En montrant qu'il est possible de réécrire cette suite de façon équivalente sous
la forme $u_n = \frac{v_n}{\sqrt{n+1}+\sqrt{n}}$ (vous déterminerez la suite
$v_b$), calculez la limite de cette suite.

\end{document}

(√(n+1)-√n)(√(n+1)+√n) = n+1 + √(n+1)√n - √(n+1)√n - n = n+1-n = 1 (la suite vₙ est constante) 



%%% Local Variables:
%%% mode: latex
%%% TeX-master: t
%%% End:
