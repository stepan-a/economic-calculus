\documentclass[10pt,a4paper,notitlepage]{article}
\usepackage{amsmath}
\usepackage{amssymb}
\usepackage{amsbsy}
\usepackage{float}
\usepackage[french]{babel}
\usepackage{graphicx}
\usepackage{enumerate}

\usepackage{palatino}

 \usepackage[active]{srcltx}
\usepackage{scrtime}

\newcommand{\exercice}[1]{\textsc{\textbf{Exercice}} #1}
\newcommand{\question}[1]{\textbf{(#1)}}
\setlength{\parindent}{0cm}

\begin{document}

\title{\textsc{Calcul Économique\\ \small{(Devoir surveillé)}}}
\author{Stéphane Adjemian\thanks{Université du Mans. \texttt{stephane DOT adjemian AT univ DASH lemans DOT fr}}}
\date{Le \today\ à \thistime}

\maketitle


\exercice{1} Soient trois propositions $P$, $Q$ et $R$. On doit montrer que~:
\[
(P\Rightarrow Q) \land (Q\Rightarrow R) \Longrightarrow (P\Rightarrow R)
\]
On utilise une table de vérité~:


\begin{table}[H]
    \hspace*{-2cm}\begin{tabular}[H]{|ccc|ccccc|}
      \hline
      $P$ & $Q$ & $R$ & $P\Rightarrow Q$ & $Q\Rightarrow R$ & $(P\Rightarrow Q) \land (Q\Rightarrow R)$ & $P\Rightarrow R$ & $((P\Rightarrow Q) \land (Q\Rightarrow R))\Rightarrow (P\Rightarrow R)$\\ \hline
      V & V & V & V & V & V & V & V \\
      V & V & F & V & F & F & F & V \\
      V & F & V & F & V & F & V & V \\
      V & F & F & F & V & F & F & V \\
      F & V & V & V & V & V & V & V \\
      F & V & F & V & F & F & V & V \\
      F & F & V & V & V & V & V & V \\
      F & F & F & V & V & V & V & V \\
      \hline\hline
    \end{tabular}\hspace*{-2cm}
\end{table}
  
  Comme la dernière colonne est vraie sur toutes les lignes, c'est-à-dire pour
  tout triplet de valeurs de vérité des propositions $P$, $Q$ et $R$, la
  proposition relative à la transitivité de l'implication logique est vraie.

  \bigskip
  
  \exercice{2} On vérifie d'abord que la formule est correcte pour $n=1$. Celle-ci nous dit que la somme devrait être égale à 0. C'est bien le cas puisque nous avons~:
  \[
    \sum_{i=1}^1  i(i-1) = 1 \times 0 = 0
  \]
  Supposons que la formule est vraie au rang $n$ et montrons alors qu'elle doit être vraie au rang $n+1$, c'est-à-dire que nous avons~:
  \[
    \sum_{i=1}^n  i(i-1) = \frac{n(n+1)(n+2)}{3}
  \]
  Il suffit de découper la somme pour faire apparaître la formule au rang $n$. On a~:
  \[
    \begin{split}
      \sum_{i=1}^{n+1}  i(i-1) &= \sum_{i=1}^{n}  i(i-1) + (n+1)n\\
                               &= \frac{(n-1)n(n+1)}{3} + (n+1)n\\
                               &= \frac{(n-1)n(n+1)+3n(n+1)}{3}\\
                               &= \frac{n(n+1)(n+2)}{3}
    \end{split}
  \]
  Nous trouvons bien l'expression attendue. La formule est donc vraie pour tout $n\geq 1$.

  \bigskip

  \exercice{3}\textbf{(1)} Les paramètres $a$ et $b$ sont les ordonnées à
  l'origine des deux droites. Le paramètre $a$ est le niveau de la demande quand
  le prix est nul. Le parmètre $b$ est le niveau de l'offre quand le prix est
  nul. \textbf{(2)}-\textbf{(3)} L'offre est une fonction croissante du prix, la
  demande est une fonction décroissante du prix, pour que les deux droites se
  coupent dans l'orthan positif il faut et il suffit que $b<a$. Le prix d'équilibre est tel que~:
  \[
    a - p^{\star} = b + 2p^{\star}
  \]
  \[
    \Leftrightarrow a-b = 3 p^{\star}
  \]
  \[
    \Leftrightarrow a-b = 3 p^{\star}
  \]
  \[
    p^{\star}=\frac{a-b}{3} > 0
  \]
  Notons que le prix est positif si et seulement si la contrainte sur les paramètres $a$ et $b$ est satisfaite.

  \bigskip

  \exercice{4} Le discriminant est~:
  \[
    \Delta = \left(\frac{11}{6}\right)^2-4\frac{3}{6} = \frac{121}{36}-\frac{72}{36} = \frac{49}{36}>0
  \]
  Le polynôme d'ordre deux possède donc deux racines distinctes ~:
  \[
    x^{\star} = \frac{\frac{11}{6}\pm \frac{7}{6}}{2} = \frac{11\pm 7}{12}
  \]
  Les racines sont $1/3$ et $3/2$.

  \bigskip

  \exercice{5} Le discriminant est~:
  
\end{document}


\exercice{4} Déterminer les solutions de l'équation suivante~:
\[
x^2 - \frac{11}{6}x + \frac{3}{6} = 0
\]

\bigskip

\exercice{5} Soient trois ensembles $A$, $B$ et $C$. L'égalité~:
\[
  \mathrm{card}(A \cup B \cup C) = \mathrm{card}(A) + \mathrm{card}(B) + \mathrm{card}(C)
\]
est-elle vraie~? Pourquoi~? Justifiez votre réponse à l'aide d'une démonstration.

%%% Local Variables:
%%% mode: latex
%%% TeX-master: t
%%% End:
