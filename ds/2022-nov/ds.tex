\documentclass[10pt,a4paper,notitlepage]{article}
\usepackage{amsmath}
\usepackage{amssymb}
\usepackage{amsbsy}
\usepackage{float}
\usepackage[french]{babel}
\usepackage{graphicx}
\usepackage{enumerate}

\usepackage{palatino}

 \usepackage[active]{srcltx}
\usepackage{scrtime}

\newcommand{\exercice}[1]{\textsc{\textbf{Exercice}} #1}
\newcommand{\question}[1]{\textbf{(#1)}}
\setlength{\parindent}{0cm}

\begin{document}

\title{\textsc{Calcul Économique\\ \small{(Devoir surveillé)}}}
\author{Stéphane Adjemian\thanks{Université du Mans. \texttt{stephane DOT adjemian AT univ DASH lemans DOT fr}}}
\date{Le \today\ à \thistime}

\maketitle


\exercice{1} Montrer la transitivité de l'implication logique.

\bigskip

\exercice{2} Montrer par récurrence que~:
\[
  \sum_{i=1}^n  i(i-1) = 0 + 2\times 1 + 3\times 2 + \dots + n\times(n-1) = \frac{(n-1)n(n+1)}{3}
\]

\bigskip

\exercice{3} Soient les fonctions d'offre et de demande~:
\[
\begin{split}
  D(p):& q = a - p\\
  S(p):& q = b + 2p
\end{split}
\] 
où $a$ et $b$ sont des paramètres réels positifs. \textbf{(1)} Interpréter les
paramètres $a$ et $b$. \textbf{(2)} Représenter graphiquement ces
fonctions. \textbf{(3)} Déterminer sous quelle condition un prix d'équilibre
$p^{\star}$ existe. Déterminer ce prix.

\bigskip

\exercice{4} Déterminer les solutions de l'équation suivante~:
\[
x^2 - \frac{11}{6}x + \frac{3}{6} = 0
\]

\bigskip

\exercice{5} Soient trois ensembles $A$, $B$ et $C$. L'égalité~:
\[
  \mathrm{card}(A \cup B \cup C) = \mathrm{card}(A) + \mathrm{card}(B) + \mathrm{card}(C)
\]
est-elle vraie~? Pourquoi~? Justifiez votre réponse à l'aide d'une démonstration.
\end{document}

%%% Local Variables:
%%% mode: latex
%%% TeX-master: t
%%% End:
