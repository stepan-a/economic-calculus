\documentclass[10pt,a4paper,notitlepage]{article}
\usepackage{amsmath}
\usepackage{amssymb}
\usepackage{amsbsy}
\usepackage{float}
\usepackage[french]{babel}
\usepackage{graphicx}
\usepackage{enumerate}

\usepackage{palatino}

\usepackage[active]{srcltx}
\usepackage{scrtime}
\usepackage{nicefrac}

\newcommand{\exercice}[1]{\textsc{\textbf{Exercice}} #1}
\newcommand{\question}[1]{\textbf{(#1)}}
\setlength{\parindent}{0cm}

\begin{document}

\title{\textsc{Calcul Économique}\\\textbf{(Éléments de correction)}}
\author{Stéphane Adjemian\thanks{Université du Maine, Gains. \texttt{stephane DOT adjemian AT univ DASH lemans DOT fr}}}
\date{Le \today\ à \thistime}

\maketitle

\exercice{1} Il faut faire une table de vérité.
\begin{table}[H]
  \centering
  \begin{tabular}{cc|ccccc}
    $P$ & $Q$ & $P \land Q$ & $\overline{P \land Q}$ & $\overline{P}$
    & $\overline{Q}$ & $\overline{P} \lor \overline{Q}$\\ \hline
    V & V & V & F & F & F & F \\
    V & F & F & V & F & V & V \\
    F & V & F & V & V & F & V \\
    F & F & F & V & V & V & V \\
  \end{tabular}
\end{table}
Les valeurs dans la quatrième et la septième colonnes sont identiques,
nous avons donc bien :  
\[
\overline{P \land Q} \Leftrightarrow \overline{P} \lor \overline{Q}
\]

\bigskip

\exercice{2} \textbf{(1)} La dérivée d'une fonction $f$ est définie
par la limite suivante :
\[
f'(x) = \lim_{h\rightarrow 0} \frac{f(x+h)-f(x)}{h}
\]
\textbf{(2)} Pour $n=1$ on a $f(x) = x$. La formule nous dit alors que
$f'(x) = 1\times x^0 = 1$ pour tout $x$. Montrons que c'est bien le
cas, en utilisant la définition de la dérivée :
\[
f'(x) = \lim_{h\rightarrow 0} \frac{x+h-x}{h} = \lim_{h\rightarrow 0}
\frac{h}{h} = 1 
\]
La formule est donc correcte au rang $n=1$. On suppose maintenant que
la formule est correcte à un rang $n$ quelconque, montrons qu'elle est
alors nécessairement vraie au rang $n+1$, c'est-à-dire que nous avons
bien :
\[
\frac{\mathrm d}{\mathrm dx}x^{n+1} = (n+1)x^n
\]
Nous avons :
\[
\frac{\mathrm d}{\mathrm dx} x^{n+1} = \frac{\mathrm d}{\mathrm dx} x
\times x^n
\]
En appliquant la règle $(uv)'=u'v+uv'$ il vient :
\[
\frac{\mathrm d}{\mathrm dx} x^{n+1} = 1 \times x^n + x \frac{\mathrm
  d}{\mathrm dx} x^n
\]
En substituant la formule supposée vraie au rang $n$ :
\[
\frac{\mathrm d}{\mathrm dx} x^{n+1} = x^n + x \times n \times x^{n-1}
\]
soit de façon équivalente :
\[
\frac{\mathrm d}{\mathrm dx} x^{n+1} = x^n + n \times x^{n}
\]
ou encore :
\[
\frac{\mathrm d}{\mathrm dx} x^{n+1} = (n+1) x^{n}
\]
ce qu'il fallait démontrer.

\bigskip

\exercice{3} On note $y_t$ le PIB de la France à la date $t$ et on
suppose que cette variable est déterminée par  l'équation :
\[
y_t = 1,02 y_{t-1}
\]
avec une condition initiale $y_0 = 1$. \textbf{(1)} Notons $g_{y,t}$
le taux de croissance du PIB entre les dates $t-1$ et $t$. Par
définition du taux de croissance on a :
\[
g_{y,t} = \frac{y_{t}-y_{t-1}}{y_{t-1}} = \frac{y_{t}}{y_{t-1}}-1 
\]
En appliquant cette formule on trouve directement que le taux de
croissance du PIB est constant :
\[
g_{y,t} = 1,02-1 = 0,02
\]
soit 2\%. \textbf{(2)} À la date 1, nous avons :
\[
y_1 = 1,02 y_0 = 1,02
\]
À la date 2, nous avons :
\[
y_2 = 1,02 y_1 = 1,02^2
\]
À la date 3 :
\[
y_3 = 1,02 y_2 = 1,02^3
\]
Plus généralement, on peut montrer par récurrence que l'on a :
\[
y_t = 1,02^t
\]
\textbf{(3)} Plaçons nous à la date $t$, on cherche $s$ tel que :
\[
\frac{y_{t+s}}{y_t} = 2 
\]
En substituant le résultat à la question précédente :
\[
\frac{1,02^{t+s}}{1,02^t} = 2 
\]
\[
\Leftrightarrow 1,02^s = 2 
\]
En appliquant le logarithme népérien sur les deux membres de l'égalité
il vient :
\[
s \log 1,02 = \log 2
\]
soit de façon équivalente :
\[
s = \frac{\log 2}{\log 1.02}
\]
Le logarithme de 2 est approximativement égal à 0.69. Le logarithme de
$\log (1+x)$ peut être approximer par $1+x$ pour des petites valeurs
de $x$ (approximation de Taylor à l'ordre un). On a donc
\[
s \approx \frac{0,69}{0,02} = 34,5
\]
Il faut donc de l'ordre de 35 périodes pour doubler le niveau du PIB
si le taux de croissance est égale à 2\%. \textbf{(4)} On peut montrer
que lorsque $t$ tend vers l'infini, le niveau du PIB diverge vers
l'infini. Il suffit de remarquer que la suite est monotone croissante
(en effet $y_t-y_{t-1} = 0,02\times y_{t-1} > 0 $ pour tout
$t$). Ainsi pour tout $t$ les termes suivants (en $t+1$, $t+2$, ...)
sont toujours plus grands. On peut remarquer aussi que $y_t$ est une
suite géométrique de raison (1,02) supérieure à 1 en valeur absolue.

\bigskip

\exercice{4} Sur un marché la demande pour un bien à la date $t$ est linéaire par
rapport au prix du bien :
\[
q_t = a - b p_t
\]
où $a$ et $b$ sont deux paramètres réels strictement positifs. Sur le même marché,
la quantité offerte à la date $t$ dépend du prix anticipé (à la date
$t-1$) pour la date $t$ :
\[
q_t = c + d \hat p_t
\]
où $c$ et $d$ sont deux paramètres réels positifs, $\hat p_t$ est le
prix anticipé pour la période $t$. On supposera que les anticipations
sont naïves dans le sens où :
\[
\hat p_t = p_{t-1}
\]
Les offreurs anticipent que le prix à la date $t$ sera le prix observé
à la date $t-1$. \textbf{(1)} Si l'offre est égale à la demande, alors
on doit avoir :
\[
a - b p_t = c + d \hat p_t
\]
en substituant l'anticipation naïve, il vient :
\[
a - b p_t = c + d p_{t-1}
\]
soit de façon équivalente :
\[
p_t = \frac{a-c}{b} - \frac{d}{b}p_{t-1}
\]
\textbf{(2)} $p^{\star}$ doit satisfaire :
\[
a - b p^{\star} = c + d p^{\star}
\]
\[
\Leftrightarrow a - c = (b+d) p^{\star}
\]
\[
\Leftrightarrow  p^{\star} = \frac{a-c}{b+d}
\]
Si $p_t = p^{\star}$ alors $p_{t+s} = p^{\star}$ pour tout $s\in
\mathbb N$. Pour que ce point fixe puisse être interprété comme un
prix il faut qu'il soit positif, c'est-à-dire que $a>c$. \textbf{(3)}
En substituant le point fixe $p^{\star}$ dans les fonctions d'offre et
de demande, on remarque que :
\[
D(p^{\star}) = S(p^{\star}) = \frac{da+bc}{b+d}
\]
% a - b (a-c)/(b+d) = (ab + da - ba + bc)(b+d) = 
% c + d (a-c)/(b+d) = (bc+dc+da-cd)/(b+d) = 
La quantité offerte égalise la quantité demandée. $p^{\star}$ est
donc bien un prix d'équilibre. \textbf{(4)}
Supposons que  le prix à la date 0 soit $p_0\neq p^{\star}$. On omet
le cas où le prix est initialement à l'équilibre car on sait que le
prix est alors à l'équilibre pour toute date $t>0$. À la date 1, nous
avons :
\[
p_1 = \frac{a-c}{b} - \frac{d}{b}p_0
\]
À la date 2 :
\[
p_2 = \frac{a-c}{b} - \frac{d}{b}p_1 =
\frac{a-c}{b}\left(1-\frac{d}{b}\right) + \left(\frac{d}{b}\right)^2p_0
\]
À la date 3 :
\[
p_3 = \frac{a-c}{b} - \frac{d}{b}p_2 =
\frac{a-c}{b}\left(1-\frac{d}{b}+\left(\frac{d}{b}\right)^2\right) - \left(\frac{d}{b}\right)^3p_0
\]
Plus généralement nous devrions avoir :
\[
p_t =
\frac{a-c}{b}\left(1-\frac{d}{b}+\left(\frac{d}{b}\right)^2+\dots+\left(-\frac{d}{b}\right)^{t-1}\right)
+ \left(-\frac{d}{b}\right)^{t}p_0
\]
ou de façon équivalente :
\[
p_t = \frac{a-c}{b}\sum_{\tau=0}^{t-1}
\left(-\frac{d}{b}\right)^{\tau} + \left(-\frac{d}{b}\right)^{t}p_0
\]
ou encore :
\[
p_t = \frac{a-c}{b}\frac{1-\left(-\frac{d}{b}\right)^{t}}{1+\frac{d}{b}} + \left(-\frac{d}{b}\right)^{t}p_0
\]
Admettons que cette formule soit correcte au rang $t$ et montrons que
nous retrouvons alors nécessairement la même au rang $t+1$. Nous avons
\[
p_{t+1} = \frac{a-c}{b} - \frac{d}{b}p_t
\]
En substituant l'expression pour $p_t$, il vient :
\[
p_{t+1} = \frac{a-c}{b} - \frac{d}{b}\left(\frac{a-c}{b}\frac{1-\left(-\frac{d}{b}\right)^{t}}{1+\frac{d}{b}} + \left(-\frac{d}{b}\right)^{t}p_0\right)
\]
\[
\Leftrightarrow p_{t+1} = \frac{a-c}{b} \left(1 - \frac{d}{b}\frac{1-\left(-\frac{d}{b}\right)^{t}}{1+\frac{d}{b}}\right)+ \left(-\frac{d}{b}\right)^{t+1}p_0
\]
\[
\Leftrightarrow p_{t+1} = \frac{a-c}{b} \left(1 - \frac{\frac{d}{b}-\left(-\frac{d}{b}\right)^{t+1}}{1+\frac{d}{b}}\right)+ \left(-\frac{d}{b}\right)^{t+1}p_0
\]
\[
\Leftrightarrow p_{t+1} = \frac{a-c}{b} \frac{1-\left(-\frac{d}{b}\right)^{t+1}}{1+\frac{d}{b}}+ \left(-\frac{d}{b}\right)^{t+1}p_0
\]
ce qu'il fallait démontrer. Nous avons donc bien obtenu l'expression
du prix à la date $t$ comme une fonction du prix initial ($p_0$) et
des paramètres des fonctions de demande et d'offre. \textbf{(5)}
Clairement le prix ne diverge pas que si $b<d$ de sorte que
$\lim_{t\rightarrow\infty}(-\nicefrac{b}{d})^t=0$. Dans ce modèle où
les anticipations sont naïves, il faut que la demande soit moins \og
pentue \fg\ que l'offre pour que prix ne diverge pas. Sous cette
condition, on voit directement que :
\[
\lim_{t\rightarrow\infty}p_t = p^{\star}
\]
Le prix converge vers le prix d'équilibre du marché dès lors que
$b<d$. Cette convergence n'est pas monotone. Pour le montrer, notons
que nous pouvons écrire l'écart entre $p_t$ et $p^{\star}$ comme une
fonction de l'écart entre $p_0$ et $p^{\star}$ :
\[
p_t - p^{\star} = \left(p_0-p^{\star}\right)\left(-\frac{b}{d}\right)^t
\]
Clairement le signe de $p_t-p^{\star}$ change à chaque itération, la
suite $p_t$ oscille donc autour de $p^{\star}$.

\bigskip

\exercice{5} Il suffit de remarquer que $f(-1)=f(1)=1$.

\bigskip

\exercice{6} \textbf{(i)} Une fonction $f$ est convexe si
sa courbe représentative se situe au dessus de ses
tangentes. \textbf{(ii)} Une fonction $f$ est convexe si
pour tout $x, y$ dans le domaine de définition de $f$ et tout
$\lambda\in[0,1]$ on a $f(\lambda x + (1-\lambda)x) \leq \lambda f(x)
+ (1-\lambda)f(y)$. \textbf{(iii)} Une fonction $f$ est convexe si
$f''(x)\geq 0$ pour tout $x$ dans le domaine de $f$. 

\bigskip

\exercice{7} L'équilibre, s'il existe, doit égaliser l'offre et la
demande sur le marché. On doit donc chercher un prix tel que :
\[
p^2=6-p
\]
\[
\Leftrightarrow p^2+p-6 = 0
\]
Le prix d'équilibre est une racine d'un polynôme d'ordre 2. Le
discriminant de ce polynôme est $\Delta = 1+4\times 6 = 25$. Il existe
donc deux racines. Il n'y a cependant pas d'incertitude sur le prix
d'équilibre puisque les racines sont de signes opposés. L'unique prix
d'équilibre correspond à l'unique racine positive du polynôme
d'ordre : $p^{\star}=2$ (l'autre racine n'est pas économiquement pertinente). 

\bigskip

\exercice{8} \textbf{(1)} On a d'après le cours :
\[
\left(f(g(x))\right)' = f'(g(x))\times g'(x)
\]
\textbf{(2)} On obtient la dérivée d'ordre deux en dérivant à nouveau
la dérivée d'ordre un donnée plus haut :
\[
\left(f(g(x))\right)'' = \left(f'(g(x))\times g'(x)\right)'
\]
\[
\Leftrightarrow \left(f(g(x))\right)'' = \left(f'(g(x))\right)'g'(x) + f'(g(x))g''(x)
\]
\[
\Leftrightarrow \left(f(g(x))\right)'' = f''(g(x))\left(g'(x)\right)^2 + f'(g(x))g''(x)
\]
\textbf{(3)} $h''(x) = 4 \times 3 x^2$. \textbf{(4)} Notons que
$h(x)=\left(x^2\right)^2$, en appliquant la formule pour la dérivée
d'ordre deux d'une composition de fonction, on obtient :
\[
h''(x) = 2 \times (2x)^2 + 2 (x^2)^3 \times 2
\]
\[
h''(x) = 8 x^2 +  4 x^2
\]
\[
h''(x) = 12 x^2
\]
Nous retrouvons bien le même résultat qu'au point (3).

\end{document}