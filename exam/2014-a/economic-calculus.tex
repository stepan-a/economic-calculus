\documentclass[10pt,a4paper,notitlepage]{article}
\usepackage{amsmath}
\usepackage{amssymb}
\usepackage{amsbsy}
\usepackage{float}
\usepackage[french]{babel}
\usepackage{graphicx}
\usepackage{enumerate}

\usepackage[utf8x]{inputenc}
\usepackage[T1]{fontenc}
\usepackage{palatino}

 \usepackage[active]{srcltx}
\usepackage{scrtime}

\newcommand{\exercice}[1]{\textsc{\textbf{Exercice}} #1}
\newcommand{\question}[1]{\textbf{(#1)}}
\setlength{\parindent}{0cm}

\begin{document}

\title{\textsc{Calcul Économique}}
\author{Stéphane Adjemian\thanks{Université du Maine, Gains. \texttt{stephane DOT adjemian AT univ DASH lemans DOT fr}}}
\date{Le \today\ à \thistime}

\maketitle

\exercice{1} Soit deux propositions $P$ et $Q$. Montrer que $\overline
{P \land Q} \Leftrightarrow \overline{P} \lor \overline{Q}$ 

\bigskip

\exercice{2} \textbf{(1)} Donner la définition de la dérivée d'une
fonction. \textbf{(2)} Soit la fonction $f(x) = x^n$ avec
$n\in\mathbb N$. Le but de l'exercice est de montrer que la dérivée de
cette fonction est $f'(x) = nx^{n-1}$. Montrer que cette formule est
correcte pour $n=1$ (en utilisant la définition de la
dérivée). \textbf{(3)} Montrer que si la formule est vraie au rang $n$
alors elle est nécessairement vraie au rang $n+1$. Conclure.

\bigskip

\exercice{3} On note $y_t$ le PIB de la France à la date $t$ et on
suppose que cette variable est déterminée par  l'équation :
\[
y_t = 1,02 y_{t-1}
\]
avec une condition initiale $y_0 = 1$. \textbf{(1)} Quel est le taux
de croissance du PIB ? \textbf{(2)} Donner le niveau du PIB à la date
$t$. \textbf{(3)} Calculer le nombre de périodes nécessaires pour
doubler le niveau du PIB. \textbf{(4)} Montrer que le
niveau du PIB tend vers l'infini lorsque $t$ tend vers l'infini.

\bigskip

\exercice{4} Sur un marché la demande pour un bien à la date $t$ est linéaire par
rapport au prix du bien :
\[
q_t = a - b p_t
\]
où $a$ et $b$ sont deux paramètres réels strictement positifs. Sur le même marché,
la quantité offerte à la date $t$ dépend du prix anticipé (à la date
$t-1$) pour la date $t$ :
\[
q_t = c + d \hat p_t
\]
où $c$ et $d$ sont deux paramètres réels positifs, $\hat p_t$ est le
prix anticipé pour la période $t$. On supposera que les anticipations
sont naïves dans le sens où :
\[
\hat p_t = p_{t-1}
\]
Les offreurs anticipent que le prix à la date $t$ sera le prix observé
à la date $t-1$. \textbf{(1)} Montrer que la quantité offerte est
égale à la quantité demandée si et seulement si le prix à la date $t$
est donné par :
\[
p_t = \frac{a-c}{b} - \frac{d}{b} p_{t-1} \equiv f(p_{t-1})
\]
\textbf{(2)} Calculer le point fixe (on dit aussi état stationnaire)
de cette équation récurrente pour le prix, c'est-à-dire calculer
$p^{\star}$ tel que $p^{\star} = f(p^{\star})$. Quelle hypothèse
faut-il poser sur les paramètres pour que ce prix ait un sens ? \textbf{(3)}
Montrer que $p^{\star}$ est le prix d'équilibre sur ce
marché. Calculer la quantité échangée à l'équilibre. \textbf{(4)}
Calculer le prix à la $t$. \textbf{(5)} Donner la condition sous
laquelle le prix converge vers $p^{\star}$. Commenter. La convergence
est-elle monotone ?

\bigskip

\exercice{5} Montrer que la fonction $f(x) = x^2$, $x\in\mathbb R$, n'est pas bijective.

\bigskip

\exercice{6} Donner trois conditions assurant la convexité d'une
fonction de $\mathbb R$ dans $\mathbb R$. 

\bigskip

\exercice{7} Sur un marché la quantité offerte ($q$) est donnée comme
une fonction du prix par :
\[
q = p^2
\]
et la demande est déterminée par :
\[
q = 6-p
\]
\textbf{(1)} Montrer qu'il existe un unique prix d'équilibre
$p^{\star}>0$. \textbf{(2)} Donner une représentation graphique des
fonctions de demande et d'offre, et de l'équilibre.     

\bigskip

\exercice{8} Soient $f$ et $g$ deux fonctions pour lesquelles les
dérivées d'ordre 1 et deux sont définies. \textbf{(1)} Quelle est la
dérivée de la composition $f(g(x))$ ? \textbf{(2)} Calculer la dérivée
d'ordre deux de $f(g(x))$. \textbf{(3)} Quelle est la dérivée d'ordre
deux de $h(x) = x^4$. \textbf{(4)} Vérifier que la formule
obtenue pour la dérivée d'ordre deux donne un résultat correcte si
$f(g(x)) = \left(x^2\right)^2$. 
\end{document}