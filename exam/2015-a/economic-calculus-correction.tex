\documentclass[10pt,a4paper,notitlepage]{article}
\usepackage{amsmath}
\usepackage{amssymb}
\usepackage{amsbsy}
\usepackage{float}
\usepackage[french]{babel}
\usepackage{graphicx}
\usepackage{enumerate}

\usepackage{palatino}

\usepackage[active]{srcltx}
\usepackage{scrtime}
\usepackage{nicefrac}
\usepackage{adjustbox}

\newcommand{\exercice}[1]{\textsc{\textbf{Exercice}} #1}
\newcommand{\question}[1]{\textbf{(#1)}}
\setlength{\parindent}{0cm}

\begin{document}

\title{\textsc{Calcul Économique}\\\textbf{(Éléments de correction)}}
\author{Stéphane Adjemian\thanks{Université du Mans. \texttt{stephane DOT adjemian AT univ DASH lemans DOT fr}}}
\date{Le \today\ à \thistime}

\maketitle

\exercice{1} Il faut faire une table de vérité. Puisque nous devons
établir une proposition faisant intervenir trois propositions ($P$,
$Q$ et $R$) et que chacune de ces propositions peut prendre deux
valeurs ($V$ et $F$), la table de vérité doit contenir $2^3=8$ lignes.

\begin{table}[H]
  \centering
\adjustbox{max width=\textwidth}{
  \begin{tabular}{ccc|ccccc}
    $P$ & $Q$ & $R$ & $P\Rightarrow Q$ & $Q\Rightarrow R$ &
    $P\Rightarrow Q \land Q\Rightarrow R$ & $P\Rightarrow R$ & 
    $(P\Rightarrow Q \land Q\Rightarrow R)\Rightarrow (P\Rightarrow R)$\\ 
    \hline
    V & V & V & V & V & V & V & V\\
    V & V & F & V & F & F & F & V\\
    V & F & V & F & V & F & V & V\\
    V & F & F & F & V & F & F & V\\
    F & V & V & V & V & V & V & V\\
    F & V & F & V & F & F & V & V\\
    F & F & V & V & V & V & V & V\\
    F & F & F & V & V & V & V & V\\
  \end{tabular}}
\end{table}

La proposition est donc vraie puisque la dernière colonne contient la
valeur $V$ (vraie) sur toutes les lignes.

\bigskip

\exercice{2} \textbf{(1)} La dérivée d'une fonction $f$ est définie
par la limite suivante :
\[
f'(x) = \lim_{h\rightarrow 0} \frac{f(x+h)-f(x)}{h}
\]
\textbf{(2)} Pour $n=1$ on a $f(x) = x$. La formule nous dit alors que
$f'(x) = 1\times x^0 = 1$ pour tout $x$. Montrons que c'est bien le
cas, en utilisant la définition de la dérivée :
\[
f'(x) = \lim_{h\rightarrow 0} \frac{x+h-x}{h} = \lim_{h\rightarrow 0}
\frac{h}{h} = 1 
\]
La formule est donc correcte au rang $n=1$. On suppose maintenant que
la formule est correcte à un rang $n$ quelconque, montrons qu'elle est
alors nécessairement vraie au rang $n+1$, c'est-à-dire que nous avons
bien :
\[
\frac{\mathrm d}{\mathrm dx}x^{n+1} = (n+1)x^n
\]
Nous avons :
\[
\frac{\mathrm d}{\mathrm dx} x^{n+1} = \frac{\mathrm d}{\mathrm dx} x
\times x^n
\]
En appliquant la règle $(uv)'=u'v+uv'$ il vient :
\[
\frac{\mathrm d}{\mathrm dx} x^{n+1} = 1 \times x^n + x \frac{\mathrm
  d}{\mathrm dx} x^n
\]
En substituant la formule supposée vraie au rang $n$ :
\[
\frac{\mathrm d}{\mathrm dx} x^{n+1} = x^n + x \times n \times x^{n-1}
\]
soit de façon équivalente :
\[
\frac{\mathrm d}{\mathrm dx} x^{n+1} = x^n + n \times x^{n}
\]
ou encore :
\[
\frac{\mathrm d}{\mathrm dx} x^{n+1} = (n+1) x^{n}
\]
ce qu'il fallait démontrer.

\bigskip

\exercice{3} Sur un marché la demande pour un bien à la date $t$ est linéaire par
rapport au prix du bien :
\[
q_t = a - b p_t
\]
où $a$ et $b$ sont deux paramètres réels strictement positifs. Sur le même marché,
la quantité offerte à la date $t$ dépend du prix anticipé (à la date
$t-1$) pour la date $t$ :
\[
q_t = c + d \hat p_t
\]
où $c$ et $d$ sont deux paramètres réels positifs, $\hat p_t$ est le
prix anticipé pour la période $t$. On supposera que les anticipations
sont naïves dans le sens où :
\[
\hat p_t = p_{t-1}
\]
Les offreurs anticipent que le prix à la date $t$ sera le prix observé
à la date $t-1$. \textbf{(1)} Si l'offre est égale à la demande, alors
on doit avoir :
\[
a - b p_t = c + d \hat p_t
\]
en substituant l'anticipation naïve, il vient :
\[
a - b p_t = c + d p_{t-1}
\]
soit de façon équivalente :
\[
p_t = \frac{a-c}{b} - \frac{d}{b}p_{t-1}
\]
\textbf{(2)} $p^{\star}$ doit satisfaire :
\[
a - b p^{\star} = c + d p^{\star}
\]
\[
\Leftrightarrow a - c = (b+d) p^{\star}
\]
\[
\Leftrightarrow  p^{\star} = \frac{a-c}{b+d}
\]
Si $p_t = p^{\star}$ alors $p_{t+s} = p^{\star}$ pour tout $s\in
\mathbb N$. Pour que ce point fixe puisse être interprété comme un
prix il faut qu'il soit positif, c'est-à-dire que $a>c$. \textbf{(3)}
En substituant le point fixe $p^{\star}$ dans les fonctions d'offre et
de demande, on remarque que :
\[
D(p^{\star}) = S(p^{\star}) = \frac{da+bc}{b+d}
\]
% a - b (a-c)/(b+d) = (ab + da - ba + bc)(b+d) = 
% c + d (a-c)/(b+d) = (bc+dc+da-cd)/(b+d) = 
La quantité offerte égalise la quantité demandée. $p^{\star}$ est
donc bien un prix d'équilibre. \textbf{(4)}
Supposons que  le prix à la date 0 soit $p_0\neq p^{\star}$. On omet
le cas où le prix est initialement à l'équilibre car on sait que le
prix est alors à l'équilibre pour toute date $t>0$. À la date 1, nous
avons :
\[
p_1 = \frac{a-c}{b} - \frac{d}{b}p_0
\]
À la date 2 :
\[
p_2 = \frac{a-c}{b} - \frac{d}{b}p_1 =
\frac{a-c}{b}\left(1-\frac{d}{b}\right) + \left(\frac{d}{b}\right)^2p_0
\]
À la date 3 :
\[
p_3 = \frac{a-c}{b} - \frac{d}{b}p_2 =
\frac{a-c}{b}\left(1-\frac{d}{b}+\left(\frac{d}{b}\right)^2\right) - \left(\frac{d}{b}\right)^3p_0
\]
Plus généralement nous devrions avoir :
\[
p_t =
\frac{a-c}{b}\left(1-\frac{d}{b}+\left(\frac{d}{b}\right)^2+\dots+\left(-\frac{d}{b}\right)^{t-1}\right)
+ \left(-\frac{d}{b}\right)^{t}p_0
\]
ou de façon équivalente :
\[
p_t = \frac{a-c}{b}\sum_{\tau=0}^{t-1}
\left(-\frac{d}{b}\right)^{\tau} + \left(-\frac{d}{b}\right)^{t}p_0
\]
ou encore :
\[
p_t = \frac{a-c}{b}\frac{1-\left(-\frac{d}{b}\right)^{t}}{1+\frac{d}{b}} + \left(-\frac{d}{b}\right)^{t}p_0
\]
Admettons que cette formule soit correcte au rang $t$ et montrons que
nous retrouvons alors nécessairement la même au rang $t+1$. Nous avons
\[
p_{t+1} = \frac{a-c}{b} - \frac{d}{b}p_t
\]
En substituant l'expression pour $p_t$, il vient :
\[
p_{t+1} = \frac{a-c}{b} - \frac{d}{b}\left(\frac{a-c}{b}\frac{1-\left(-\frac{d}{b}\right)^{t}}{1+\frac{d}{b}} + \left(-\frac{d}{b}\right)^{t}p_0\right)
\]
\[
\Leftrightarrow p_{t+1} = \frac{a-c}{b} \left(1 - \frac{d}{b}\frac{1-\left(-\frac{d}{b}\right)^{t}}{1+\frac{d}{b}}\right)+ \left(-\frac{d}{b}\right)^{t+1}p_0
\]
\[
\Leftrightarrow p_{t+1} = \frac{a-c}{b} \left(1 - \frac{\frac{d}{b}-\left(-\frac{d}{b}\right)^{t+1}}{1+\frac{d}{b}}\right)+ \left(-\frac{d}{b}\right)^{t+1}p_0
\]
\[
\Leftrightarrow p_{t+1} = \frac{a-c}{b} \frac{1-\left(-\frac{d}{b}\right)^{t+1}}{1+\frac{d}{b}}+ \left(-\frac{d}{b}\right)^{t+1}p_0
\]
ce qu'il fallait démontrer. Nous avons donc bien obtenu l'expression
du prix à la date $t$ comme une fonction du prix initial ($p_0$) et
des paramètres des fonctions de demande et d'offre. \textbf{(5)}
Clairement le prix ne diverge pas que si $b<d$ de sorte que
$\lim_{t\rightarrow\infty}(-\nicefrac{b}{d})^t=0$. Dans ce modèle où
les anticipations sont naïves, il faut que la demande soit moins \og
pentue \fg\ que l'offre pour que prix ne diverge pas. Sous cette
condition, on voit directement que :
\[
\lim_{t\rightarrow\infty}p_t = p^{\star}
\]
Le prix converge vers le prix d'équilibre du marché dès lors que
$b<d$. Cette convergence n'est pas monotone. Pour le montrer, notons
que nous pouvons écrire l'écart entre $p_t$ et $p^{\star}$ comme une
fonction de l'écart entre $p_0$ et $p^{\star}$ :
\[
p_t - p^{\star} = \left(p_0-p^{\star}\right)\left(-\frac{b}{d}\right)^t
\]
Clairement le signe de $p_t-p^{\star}$ change à chaque itération, la
suite $p_t$ oscille donc autour de $p^{\star}$.

\bigskip

\exercice{4} On note que $f(-1)=f(1)=1$, la fonction $f$ n'est donc
pas bijective puisque deux antécédents ($1$ et $-1$) ont la même image
($1$). On peut redéfinir la fonction valeur absolue par morceaux, de
la façon suivante :
\[
f(x) = 
\begin{cases}
  -x& \quad\forall x\leq 0,\\
   x& \quad \text{sinon}.
\end{cases}
\]
Nous pouvons donc réécrire la fonction $f$ sous la forme de deux
droites. Sur chaque sous intervalles (\emph{ie} pour les valeurs
négatives puis positives de $x$) la fonction est donc continue (les
droites sont des fonctions continues). En zéro on vérifie que la
limite de $f$ est définie est égale à zéro. La fonction $f$ est donc
continue sur $\mathbb R$. Elle est aussi dérivable sur $\mathbb R^*$,
mais pas en zéro, car en ce point les dérivées à droite ($1$) et à
gauche ($-1$) sont différentes. 

\bigskip

\exercice{5} \textbf{(1)} Pour la représentation graphique il faut
remarquer que cette fonction est monotone croissante, que sa dérivée
est monotone décroissante et qu'elle passe par les points $(0, 0)$ et
$(1, 1)$. La représentation graphique doit ressembler à celle de
$\sqrt{x}$. \textbf{(2)} Cette fonction est concave puisque les
tangentes sont toutes au dessus de la courbe
représentative. \textbf{(3)} Les dérivées d'ordre 1 et 2 sont :
\[
f'(x) = \alpha x^{\alpha-1}
\]
et
\[
f''(x) = -\alpha(1-\alpha) x^{\alpha-2}
\]
Clairement, la dérivée d'ordre deux est négative pour tout $x$ puisque
$0<\alpha<1$. La fonction de production $f$ est donc bien
concave. \textbf{(4)} $\Pi(x)$ représente le profit de la firme comme
une fonction de la quantité de facteur de production. Ce profit est
exprimé en termes de quantité produite. \textbf{(5)} La fonction est
définie pour les valeurs positive de $x$. Elle passe pas le point
$(0,0)$, est croissante pour des petites valeurs de $x$ puis devient
décroissante, lorsque la variation du coût (c'est-à-dire le coût
marginal, $p$) domine la variation de la production (c'est-à-dire la
productivité marginal, $f'(x)$). Cette fonction est concave, puisque
toutes les tangentes sont au dessus de la courbe représentative, on aussi
peut vérifier que la dérivée d'ordre 2 est négative :
\[
\Pi''(x) = f''(x)
\]
\textbf{(6)} La dérivée d'ordre un de la fonction de profit est donnée
par :
\[
\Pi''(x) = f'(x)-p
\]
\textbf{(7)} Nous savons que $f'(x)$ est une fonction continue (pour $x>0$) positive et monotone
décroissante, on peut aussi vérifier que $\lim_{x\rightarrow 0} f'(x)
= \infty$. Par construction, la fonction $\Pi'(x)$ hérite de ces
propriétés. Ainsi nous savons que la fonction $\Pi'(x)$ ne s'annule
qu'en un unique point que nous noterons $x^{\star}$ (\emph{ie} il
s'agit du point ou la courbe représentative de $\Pi'$ croise l'axe des
abscisses). Nous avons :
\[
  \begin{split}
    \Pi'(x^{\star}) = 0\\
    \Leftrightarrow \alpha \left. x^{\star} \right. ^{\alpha-1} - p = 0\\
    \Leftrightarrow \left. x^{\star} \right. ^{\alpha-1} = \frac{p}{\alpha} \\
    \Leftrightarrow \left. x^{\star} \right. ^{1-\alpha} = \frac{\alpha}{p} \\
    \Leftrightarrow x^{\star} = \left(\frac{\alpha}{p}\right)^{\frac{1}{1-\alpha}} \\
  \end{split}
\] 
Il vient :
\[
y^{\star} = \left(\frac{\alpha}{p}\right)^{\frac{\alpha}{1-\alpha}}
\]
et
\[
y^{\star} = \left(\frac{\alpha}{p}\right)^{\frac{\alpha}{1-\alpha}}-p\left(\frac{\alpha}{p}\right)^{\frac{1}{1-\alpha}}
\]
\textbf{(8)} Puisque la fonction $\Pi$ est concave, la valeur
$x^{\star}$ qui annule la dérivée première de $\Pi$, est la quantité
de facteur de production qui maximise le profit. Pour tout $x\neq
x^{\star}$, on a $\Pi(x)<\Pi^{\star}$. Ainsi $Pi^{\star}$ est
la valeur maximale du profit. La condition d'optimalité du profit est
l'égalisation du coût marginal ($p$) et de la productivité marginale
($\alpha x^{alpha-1}$). Pour comprendre cette condition, supposons que
$x$ soit tel que la productivité marginale est supérieur au coût
marginal. Dans ce cas la firme a intérêt à augmenter la quantité de
facteur de production (et donc à augmenter la quantité produite)
puisque le gain lié à une augmentation de $x$ est supérieur au coût
induit. Cette augmentation contribue à réduire l'excès de la
productivité marginale relativement au coût marginal, puisque la
productivité marginal décroît lorsque $x$ augmente. Dans une situation
où la productivité marginale est différente du coût marginal, la firme
a toujours intérêt à dévier (en augmentant ou diminuant la quantité de
facteur de production, $x$).

\bigskip

\exercice{(7)} On note que $1$ est une racine évidente, puisque $P(1)=1-\frac{1}{6}-\frac{4}{6}-\frac{1}{6}=0$. On peut réécrire
le polynôme sous la forme\footnote{Il est évident que le coefficient
  associé à $x^2$ doit être égal à 1.} :
\[
P(x) = (x-1)(x^2+bx+c)
\]
En développant, on a aussi :
\[
P(x) = x^3 + (b-1)x^2 + (c-b)x - c
\]
Par identification, nous devons donc avoir :
\[
  \begin{cases}
    -\frac{1}{6} &= b-1\\
    -\frac{4}{6} &= c-b\\
    \frac{1}{6} &= c
  \end{cases}
\]
Nous avons donc $c =\nicefrac{1}{6}$ et $b=\nicefrac{5}{6}$, et
donc :
\[
P(x) = (x-1)\left(x^2+\frac{5}{6}x+\frac{1}{6}\right)
\]
Pour calculer les racines du polynôme d'ordre deux on calcule le
discriminant associé :
\[
\Delta = \frac{25}{36}-\frac{2}{3} = \frac{1}{36}
\]
Ainsi les racines du polynôme d'ordre deux sont :
\[
x_{1,2} = \frac{-\frac{5}{6} \pm\sqrt{\Delta}}{2} =
\frac{-\frac{5}{6}\pm \frac{1}{6}}{2} = 
\frac{-5 \pm 1}{12}
\]
Au total, les racines de $P(x)$ sont donc $1$, $-\nicefrac{1}{2}$ et
$-\nicefrac{1}{3}$.

\bigskip

\exercice{8} Soit $f$ une fonction bijective. D'après la définition,
la fonction réciproque $f^{-1}$ est définie implicitement par :
\[
f^{-1}(f(x)) = x
\]
En appliquant la formule de dérivation en chaîne sur les deux membres
de cette égalité il vient :
\[
\left. f^{-1} \right.' (f(x)) f'(x) = 1
\]
soit de façon équivalente :
\[
\left. f^{-1} \right.' (f(x)) = \frac{1}{f'(x)}
\]
La dérivée de la fonction réciproque est donnée par l'inverse de la
fonction dérivée.
\end{document}