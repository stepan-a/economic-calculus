\documentclass[10pt,a4paper,notitlepage]{article}
\usepackage{amsmath}
\usepackage{amssymb}
\usepackage{amsbsy}
\usepackage{float}
\usepackage[french]{babel}
\usepackage{graphicx}
\usepackage{enumerate}

\usepackage{palatino}

 \usepackage[active]{srcltx}
\usepackage{scrtime}

\newcommand{\exercice}[1]{\textsc{\textbf{Exercice}} #1}
\newcommand{\question}[1]{\textbf{(#1)}}
\setlength{\parindent}{0cm}

\begin{document}

\title{\textsc{Calcul Économique}}
\author{Stéphane Adjemian\thanks{Université du Maine, Gains. \texttt{stephane DOT adjemian AT univ DASH lemans DOT fr}}}
\date{Le \today\ à \thistime}

\maketitle

\exercice{1} Soient $P$, $Q$ et $R$ trois propositions. Montrer que :
	\[
		(P \Rightarrow Q) \land (Q \Rightarrow R) \Rightarrow (P \Rightarrow R)
	\]
	Interpréter ce résultat.
	
\bigskip

\exercice{2} \textbf{(1)} Donner la définition de la dérivée d'une
fonction. \textbf{(2)} Soit la fonction $f(x) = x^n$ avec
$n\in\mathbb N$. Le but de l'exercice est de montrer que la dérivée de
cette fonction est $f'(x) = nx^{n-1}$. Montrer que cette formule est
correcte pour $n=1$ (en utilisant la définition de la
dérivée). \textbf{(3)} Montrer que si la formule est vraie au rang $n$
alors elle est nécessairement vraie au rang $n+1$. Conclure.

\bigskip

\exercice{3} Sur un marché la demande pour un bien à la date $t$ est linéaire par
rapport au prix du bien :
\[
q_t = a - b p_t
\]
où $a$ et $b$ sont deux paramètres réels strictement positifs. Sur le même marché,
la quantité offerte à la date $t$ dépend du prix anticipé (à la date
$t-1$) pour la date $t$ :
\[
q_t = c + d \hat p_t
\]
où $c$ et $d$ sont deux paramètres réels positifs, $\hat p_t$ est le
prix anticipé pour la période $t$. On supposera que les anticipations
sont naïves dans le sens où :
\[
\hat p_t = p_{t-1}
\]
À la date $t-1$, lorsqu'ils décident la quantité offerte en $t$, les offreurs anticipent que le prix à la date $t$ sera le prix observé
à la date $t-1$. \textbf{(1)} Montrer que la quantité offerte est
égale à la quantité demandée si et seulement si le prix à la date $t$
est donné par :
\[
p_t = \frac{a-c}{b} - \frac{d}{b} p_{t-1} \equiv f(p_{t-1})
\]
\textbf{(2)} Calculer le point fixe (on dit aussi état stationnaire)
de cette équation récurrente pour le prix, c'est-à-dire calculer le
prix invariant $p^{\star}$ tel que $p^{\star} = f(p^{\star})$. Quelle hypothèse
faut-il poser sur les paramètres pour que ce prix ait un sens ? \textbf{(3)}
Montrer que $p^{\star}$ est le prix d'équilibre sur ce
marché. Calculer la quantité échangée à l'équilibre. \textbf{(4)}
Calculer le prix à la $t$. \textbf{(5)} Donner la condition sous
laquelle le prix converge vers $p^{\star}$. Commenter. La convergence
est-elle monotone ?

\pagebreak

\exercice{4} Soit la fonction $f(x) = |x|$, définie pour tout
$x\in\mathbb R$.
\begin{enumerate}
\item La fonction $f$ est-elle bijective ? Pourquoi ?
\item La fonction $f$ est-elle continue sur $\mathbb R$ ? Pourquoi ?
\item La fonction $f$ est-elle dérivable sur $\mathbb R$ ? Pourquoi ?
\end{enumerate}
 
\bigskip

\exercice{5} Soit la fonction $y = f(x) = x^{\alpha}$ une fonction définie
sur $\mathbb R^+$, avec $\alpha$ un paramètre réel positif inférieur à
un. Cette fonction représente le niveau de production
obtenu par une firme lorsqu'elle utilise une quantité $x$ du facteur
de production. \textbf{(1)} Représenter graphiquement cette
fonction. \textbf{(2)} Discuter la concavité/convexité de cette
fonction à partir d'un argument graphique. \textbf{(3)} Conclure sur
la concavité/convexité à partir de la dérivée d'ordre 2 de
$f$. \textbf{(4)} Cette firme doit payer la location du facteur de
production. On note $p$ le prix d'une unité du facteur de
production en termes de bien produit. Posons la fonction :
\[
\Pi(x) = x^{\alpha} - p x
\]
Interpréter cette fonction. Quel est le concept représenté par cette
fonction ? \textbf{(5)}. Représenter graphiquement cette fonction. Cette fonction est-elle concave ou convexe ?
\textbf{(6)} Calculer la dérivée d'ordre un de la fonction
$\Pi$. \textbf{(7)} Identifier la quantité $x^{\star}$ telle que
$\Pi'(x^{\star}) = 0$. Calculer $y^{\star}=f(x^{\star})$ et
$\Pi^{\star} = \Pi(x^{\star})$. \textbf{(8)} Interpréter $\Pi^{\star}$
et la condition qui détermine $x^{\star}$.

\bigskip

\exercice{7} Calculer les racines du polynôme :
\[
P(x) = x^3 - \frac{1}{6}x^2 - \frac{4}{6}x - \frac{1}{6}
\]    

\bigskip

\exercice{8} Déterminer la dérivée de la réciproque d'une fonction (en supposant que celle-ci existe).

\bigskip

\end{document}