\documentclass[10pt,a4paper,notitlepage]{article}
\usepackage{amsmath}
\usepackage{amssymb}
\usepackage{amsbsy}
\usepackage{float}
\usepackage[french]{babel}
\usepackage{graphicx}
\usepackage{enumerate}

\usepackage{palatino}

 \usepackage[active]{srcltx}
\usepackage{scrtime}

\newcommand{\exercice}[1]{\textsc{\textbf{Exercice}} #1}
\newcommand{\question}[1]{\textbf{(#1)}}
\setlength{\parindent}{0cm}

\begin{document}

\title{\textsc{Calcul Économique}}
\author{Stéphane Adjemian\thanks{Université du Maine, Gains. \texttt{stephane DOT adjemian AT univ DASH lemans DOT fr}}}
\date{Le \today\ à \thistime}

\maketitle

\exercice{1} Soient $P$ et $Q$ deux propositions. Montrer  que :
\[
(P \Leftrightarrow Q) \Leftrightarrow (P \Rightarrow Q) \land (Q
\Rightarrow P)
\]
Interpréter ce résultat.

\bigskip

\exercice{2} Établir le résultat suivant :
\[
\sum_{i=1}^n i^2 = \frac{n(n+1)(2n+1)}{6} 
\]

\bigskip

\exercice{3} Soit une entreprise de production d'épingles. On suppose
que le taux de croissance de sa production annuelle est $g>0$ (ce taux
est supposé constant). La première année elle produit 1000
unités. Calculer le nombre d'années $T$ tel que sa production
cumulée (ie, en sommant les productions des $T$ années) atteigne
100000 d'unités ?

\bigskip

\exercice{4} Montrer par récurrence que si $f(x) = x^n$, avec $n\in\mathbb N$, alors 
$f'(x) = nx^{n-1}$.

\bigskip

\exercice{5} Soit la fonction de $\mathbb R$ dans $\mathbb R$ $f(x) =
x^4$. Justifier et donner les valeurs des propositions suivantes :
\begin{enumerate}
\item $f$ est une fonction continue.
\item $f$ est une fonction monotone croissante.
\item $f$ est une fonction bijective.
\item $f$ est une fonction concave.
\end{enumerate}

\bigskip

\exercice{6} Calculer les racines du polynôme :
\[
P(x) = 2x^3 - 5x^2 - x - 6
\]    

\exercice{7} Soient $f$, $g$ et $h$ trois fonctions.
\begin{enumerate}
\item Calculer la dérivée, par rapport à $x$, de $\frac{1}{f(g(x))}$.
\item Calculer la derivée, par rapport à $x$, de $h(f(g(x)))$.
\item Calculer la dérivée d'ordre deux, par rapport à $x$, de $f(g(x))$.  
\end{enumerate}

\exercice{8} Soit la fonction $y = f(x) = x^{\alpha}$ une fonction définie
sur $\mathbb R^+$, avec $\alpha$ un paramètre réel positif inférieur à
un. Cette fonction représente le niveau de production
obtenu par une firme lorsqu'elle utilise une quantité $x$ du facteur
de production. \textbf{(1)} Représenter graphiquement cette
fonction. \textbf{(2)} Discuter la concavité/convexité de cette
fonction à partir d'un argument graphique. \textbf{(3)} Conclure sur
la concavité/convexité à partir de la dérivée d'ordre 2 de
$f$. \textbf{(4)} Cette firme doit payer la location du facteur de
production. On note $p$ le prix d'une unité du facteur de
production en termes de bien produit. Posons la fonction :
\[
\Pi(x) = x^{\alpha} - p x
\]
Interpréter cette fonction. Quel est le concept représenté par cette
fonction ? \textbf{(5)}. Représenter graphiquement cette fonction. Cette fonction est-elle concave ou convexe ?
\textbf{(6)} Calculer la dérivée d'ordre un de la fonction
$\Pi$. \textbf{(7)} Identifier la quantité $x^{\star}$ telle que
$\Pi'(x^{\star}) = 0$. Calculer $y^{\star}=f(x^{\star})$ et
$\Pi^{\star} = \Pi(x^{\star})$. \textbf{(8)} Interpréter $\Pi^{\star}$
et la condition qui détermine $x^{\star}$.

\end{document}