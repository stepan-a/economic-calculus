\documentclass[10pt,a4paper,notitlepage]{article}
\usepackage{amsmath}
\usepackage{amssymb}
\usepackage{amsbsy}
\usepackage{float}
\usepackage[french]{babel}
\usepackage{graphicx}
\usepackage{enumerate}

\usepackage{palatino}

\usepackage[active]{srcltx}
\usepackage{scrtime}
\usepackage{nicefrac}
\usepackage{adjustbox}

\newcommand{\exercice}[1]{\textsc{\textbf{Exercice}} #1}
\newcommand{\question}[1]{\textbf{(#1)}}
\setlength{\parindent}{0cm}

\begin{document}

\title{\textsc{Calcul Économique}\\\textbf{(Éléments de correction)}}
\author{Stéphane Adjemian\thanks{Université du Maine, Gains. \texttt{stephane DOT adjemian AT univ DASH lemans DOT fr}}}
\date{Le \today\ à \thistime}

\maketitle

\exercice{1} Il faut faire une table de vérité. Puisque nous devons
établir une proposition faisant intervenir trois propositions ($P$,
$Q$ et $R$) et que chacune de ces propositions peut prendre deux
valeurs ($V$ et $F$), la table de vérité doit contenir $2^3=8$ lignes.

\begin{table}[H]
  \centering
\adjustbox{max width=\textwidth}{
  \begin{tabular}{ccc|ccccc}
    $P$ & $Q$ & $R$ & $P\Rightarrow Q$ & $Q\Rightarrow R$ &
    $P\Rightarrow Q \land Q\Rightarrow R$ & $P\Rightarrow R$ & 
    $(P\Rightarrow Q \land Q\Rightarrow R)\Rightarrow (P\Rightarrow R)$\\ 
    \hline
    V & V & V & V & V & V & V & V\\
    V & V & F & V & F & F & F & V\\
    V & F & V & F & V & F & V & V\\
    V & F & F & F & V & F & F & V\\
    F & V & V & V & V & V & V & V\\
    F & V & F & V & F & F & V & V\\
    F & F & V & V & V & V & V & V\\
    F & F & F & V & V & V & V & V\\
  \end{tabular}}
\end{table}

La proposition est donc vraie puisque la dernière colonne contient la
valeur $V$ (vraie) sur toutes les lignes.

\bigskip

\exercice{2} \textbf{(a)} Nous utilisons une table de vérité avec $2^2=4$ lignes.

\begin{table}[H]
  \centering
\adjustbox{max width=\textwidth}{
  \begin{tabular}{cc|ccccc}
    $P$ & $Q$ & $P\Rightarrow Q$ & $\overline{P}$ &
    $\overline{P}\lor Q$\\ 
    \hline
    V & V & V & F & V \\
    V & F & F & F & F \\
    F & V & V & V & V \\
    F & F & V & V & V \\
  \end{tabular}}
\end{table}

Puisque la troisième colonne et la cinquième colonne ont les mêmes valeurs de vérité sur chaque ligne, on a bien l'équivalence entre $P\Rightarrow Q$ et $\overline{P}\lor Q$. Autrement dit, il est possible d'exprimer l'implication logique avec une négation et le connecteur logique «ou». \textbf{(b)} On sait que l'équivalence logique est une double implication, c'est-à-dire que $P \Leftrightarrow Q$ est équivalent à $(P\Rightarrow Q) \land (Q\Rightarrow P)$. On vient de montrer que l'implication peut s'exprimer à l'aide d'une négation et du connecteur logique «ou». Par ailleurs nous savons que pour deux propositions $A$ et $B$ nous avons $\overline{A \land B} = \overline{A}\lor\overline{B}$. Nous pouvons donc réécrire le résultat précédant sous la forme :
\[
  (P \Rightarrow Q) \Leftrightarrow \overline{P \land \overline{Q}}
\]
En faisant de même pour $Q\Rightarrow P$ on obtient le résultat désiré.

\bigskip

\exercice{3} Pour $n=1$, nous avons bien $1+x\geq 1+x$ pour tout $x\in \mathbb R$. Pour $n=2$, nous avons  $(1+x)^2 = 1+2x+x^2 \geq 1+2x$, puisque le carré est non négatif, pour tout $x\in\mathbb R$. Pour $n=3$, si $x$ est positif nous avons : $(1+x)^3 = 1 + 3x + 3x^2 + x^3\geq 1+3x$ car la somme des deux derniers termes est nécessairement non négative. Supposons que l'inégalité soit vraie au rang $n$, c'est-à-dire que $(1+x)^n \geq 1+nx$ pour tout $x\geq 0$, et montrons que l'inégalité est nécessairement vérifiée au rang $n+1$, c'est-à-dire que l'on a $(1+x)^{n+1} \geq 1+(n+1)x$ pour tout $x$ positif ou nul. Nous avons :
\[
  \begin{split}
    (1+x)^{n+1} &= (1+x)(1+x)^n\\
    &\geq (1+x)(1+xn)\\
    &=1+nx+x+nx^2\\
    &\geq 1+(n+1)x
  \end{split}
\]
où le passage de la première ligne à la deuxième est obtenu en substituant l'inégalité au rang $n$ et en notant que le facteur $(1+x)$ est nécessairement positif. Nous avons donc bien retrouvé l'inégalité au rang $n+1$. Cette inégalité est donc vraie pour tout $n$. 

\bigskip

\exercice{4} Cette proposition se lit comme suit : « Pour tout $\epsilon$ positif, il existe $\delta(\epsilon)$ positif, tel que pour tout $x$ dans l'intervalle  $I$ $|f(x)-f(a)|<\epsilon$ dès lors que $|x-a|<\delta(\epsilon)$ ». Il s'agit de la définition d'une fonction continue sur l'intervalle $I$. 

\bigskip

\exercice{5} Cette fonction est continue sur $\mathbb R$. Calculons sa dérivée :
\[
  \begin{split}
    f'(x) &= \frac{2}{3}(1+x)^{-\frac{1}{3}}(2-x)^{\frac{1}{3}} -
    \frac{1}{3}(1+x)^{\frac{2}{3}}(2-x)^{-\frac{2}{3}}\\
    &= \frac{1}{3}(1+x)^{-\frac{1}{3}}(2-x)^{-\frac{2}{3}}\left(2(2-x)-(1-x)\right)\\
    &=\frac{1-x}{(1+x)^{\frac{1}{3}}(2-x)^{\frac{2}{3}}}
  \end{split}
\]
La dérivée de $f$ est une fonction continue sur $\mathbb R$ sauf en $x=-1$ et $x=2$, c'est-à-dire lorsque le dénominateur est nul. Nous avons des asymptotes en ces deux points :
\[
\lim_{x\rightarrow -1^-}f'(x) = -\infty  \quad\text{ et }\quad  \lim_{x\rightarrow -1^+}f'(x) = \infty
\]
\[
\lim_{x\rightarrow 2^-}f'(x) = -\infty  \quad\text{ et }\quad  \lim_{x\rightarrow 2^+}f'(x) = -\infty
\]
On remarque aussi que la dérivée est nulle en $x=1$. Ces trois points sont des optima potentiels, il reste à vérifier qu'il y a bien un changement de signe des dérivées autour de ces points.

\medskip

\begin{itemize}

\item Autour de $x=-1$ nous avons : $f'(x)<0\quad \forall x<-1$ et $f'(x)>0\quad\forall x\in[-1,1]$. Il y a donc bien un signe autour de la singularité en $x=1$. Il s'agit d'un minimum local.

\item Autour de $x=1$ nous avons : $f'(x)>0\quad\forall x\in[-1,1]$ et $f'(x)<0\quad\forall x>1$. Il y a un changement de signe de la pente autour du point où la dérivée est nulle. Il s'agit donc d'un maximum local.

\item Autour de $x=2$ il n'y a pas de changement de signe de la pente puisque  $f'(x)<0\quad\forall x>1$. Il ne s'agit donc pas d'un optimum local.
\end{itemize} 

\medskip

Pour résumer, la fonction possède deux optima locaux : un minimum local au point $x=-1$ et un maximum local au point $x=1$.

\bigskip

\exercice{6} \textbf{(a)} $f \circ g (x)$ est une fonction de $I$ dans $K$. \textbf{(b)} Une composition de fonctions continues est continue. \textbf{(c)} La dérivée de $f \circ g (x)$ est donnée (voir le cours) par :
\[
\frac{\mathrm d}{\mathrm dx}f \circ g (x) = f'(g(x))g'(x)
\]
\textbf{(d)} Il faut dérivée la dérivée pour obtenir la dérivée d'ordre deux :
\[
  \begin{split}
    \frac{\mathrm d^2}{\mathrm dx^2}f \circ g (x) &= \left(\frac{\mathrm d}{\mathrm dx}f'(g(x))\right)g'(x)+f'(g(x))g''(x)\\
    &= f''(g(x))g'(x)^2+f'(g(x))g''(x)
  \end{split}
\]

\bigskip

\exercice{7} On note que $x=1$ est une racine évidente puisque $P(1)=0$. Le polynôme peut donc s'écrire sous la forme factorisée :
\[
P(x) = (x-1)(ax^2+bx+c)
\]
Il reste à déterminer les paramètres $a$, $b$ et $c$. On procède par identification en développant l'expression précédente :
\[
  \begin{split}
    P(x) &= ax^3+bx^2+cx-ax^2-bx-c\\
    &= ax^3 +(b-a)x^2+(c-b)x-c
  \end{split}
\]
On doit donc avoir $a=1$, $c=\frac{5}{4}$ et $b=1$. Nous pouvons donc réécrire le polynôme sous la forme :
\[
P(x) = (x-1)(x^2+x+\frac{5}{4})
\]
Calculons les racines du polynôme d'ordre deux. Le discriminant est :
\[
\Delta = 1-4\frac{5}{4}=-4
\]
Puisque le discriminant est complexe, nous savons que les deux racines sont complexes :
\[
  x = \frac{1\pm 2i}{2} =
  \begin{cases}
    \frac{1}{2}-i\\
    \frac{1}{2}+i
  \end{cases}
\]
Au final le polynôme $P(x)$ possède trois racines : $x=1$, $x=\frac{1}{2}-i$ et $\frac{1}{2}+i$.

\bigskip

\exercice{8} Soit $f(x)$ une fonction continue en $x_0$. La dérivée de $f$ en $x_0$ est définie par la limite suivante :
\[
\lim_{h\rightarrow 0}\frac{f(x_0+h)-f(x_0)}{h}
\]
Calculons la dérivée de la fonction exponentielle en utilisant cette définition :
\[
  \begin{split}
    \left(e^x\right)' &= \lim_{h \rightarrow 0}\frac{e^{x+h}-e^x}{h}\\
    &= \lim_{h \rightarrow 0}\frac{e^xe^h-e^x}{h}\\
    &= \lim_{h \rightarrow 0}e^x\frac{e^h-1}{h}\\
    &= e^x\lim_{h \rightarrow 0}\frac{e^h-1}{h}\\
  \end{split}
\]
En utilisant le résultat donné dans l'énoncé, on obtient :
\[
\left(e^x\right)' = e^x
\]
\end{document}