\documentclass[10pt,a4paper,notitlepage]{article}
\usepackage{amsmath}
\usepackage{amssymb}
\usepackage{amsbsy}
\usepackage{float}
\usepackage[french]{babel}
\usepackage{graphicx}
\usepackage{enumerate}

\usepackage{palatino}

 \usepackage[active]{srcltx}
\usepackage{scrtime}

\newcommand{\exercice}[1]{\textsc{\textbf{Exercice}} #1}
\newcommand{\question}[1]{\textbf{(#1)}}
\setlength{\parindent}{0cm}

\begin{document}

\title{\textsc{Calcul Économique}}
\author{Stéphane Adjemian\thanks{Université du Mans. \texttt{stephane DOT adjemian AT univ DASH lemans DOT fr}}}
\date{Le \today\ à \thistime}

\maketitle

\exercice{1} Soient $P$, $Q$ et $R$ trois propositions. Montrer que :
	\[
		(P \Rightarrow Q) \land (Q \Rightarrow R) \Rightarrow (P \Rightarrow R)
	\]
	Interpréter ce résultat.
	
\bigskip

\exercice{2} \textbf{(a)} Montrer qu'il est possible d'exprimer l'implication
logique à l'aide d'un connecteur logique et d'une négation, c'est-à
dire que pour deux propositions $P$ et $Q$, on a :
\[
  (P \Rightarrow Q) \Leftrightarrow (\overline{P} \lor Q)
\]
\textbf{(b)} Montrer que l'équivalence logique entre deux propositions
$P$ et $Q$ peut s'écrire de façon équivalente sous la forme :
\[
  \overline{P \land \overline{Q}} \land \overline{\overline{P} \land Q}
\]

\bigskip

\exercice{3} Montrer par récurrence que pour tout $x\in \mathbb R^+$ on a $(1+x)^n \geq 1+nx$, où $n\in \mathbb N$.

\bigskip

\exercice{4} Traduire avec des mots la proposition suivante :
\[
\forall \epsilon>0, \exists \delta(\epsilon)>0 \text{ tel que } \forall x \in I,  |x-a|<\delta(\epsilon) \Rightarrow |f(x)-f(a)|<\epsilon   
\]
où $I$ est un intervalle réel, et $f$ une fonction de $I$ dans
$\mathbb R$. Que pouvez-vous dire de la fonction $f$ si cette
proposition est vraie ?

\bigskip

\exercice{5} Soit la fonction de $\mathbb R$ dans $\mathbb R$ :
$f(x) = (1+x)^{\frac{2}{3}}(2-x)^{\frac{1}{3}}$. Identifier les minima
et maxima de cette fonction.
 
\bigskip

\exercice{6} Soient les fonctions continues et dérivables
$f: I\rightarrow J$ et $g: J\rightarrow K$. \textbf{(a)} Quel est
l'ensemble de départ de la fonction composée $f \circ g (x)$ ? Quel
est l'ensemble d'arrivée de la la fonction $f \circ g (x)$ ?
\textbf{(b)} La fonction composée $f \circ g (x)$ est-elle continue ?
\textbf{(c)} Quelle est la dérivée d'ordre un de la fonction
$f \circ g (x)$ ? \textbf{(d)} Calculer la dérivée d'ordre deux de la
fonction $f \circ g (x)$ par rapport à $x$.

\bigskip

\exercice{7} Calculer les racines du polynôme :
\[
P(x) = x^3 + \frac{1}{4}x - \frac{5}{4}
\]    

\bigskip

\exercice{8} Donner la définition de la dérivée d'une fonction. En admettant que $\lim_{x \rightarrow 0}\frac{e^x-1}{x} = 1$, montrer que la dérivée de $e^x$ est $e^x$ pour tout $x$.


\end{document}