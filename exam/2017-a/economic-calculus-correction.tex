\documentclass[10pt,a4paper,notitlepage]{article}
\usepackage{amsmath}
\usepackage{amssymb}
\usepackage{amsbsy}
\usepackage{float}
\usepackage[french]{babel}
\usepackage{graphicx}
\usepackage{enumerate}

\usepackage{palatino}
\usepackage{mathrsfs}

\usepackage[active]{srcltx}
\usepackage{scrtime}
\usepackage{nicefrac}
\usepackage{adjustbox}

\newcommand{\exercice}[1]{\textsc{\textbf{Exercice}} #1}
\newcommand{\question}[1]{\textbf{(#1)}}
\setlength{\parindent}{0cm}

\begin{document}

\title{\textsc{Calcul Économique}\\\textbf{(Éléments de correction)}}
\author{Stéphane Adjemian\thanks{Université du Mans. \texttt{stephane DOT adjemian AT univ DASH lemans DOT fr}}}
\date{Le \today\ à \thistime}

\maketitle

\exercice{1} \textbf{(a)} L'implication logique est définie de la façon suivante, à l'aide d'une table de vérité :
\begin{table}[H]
  \centering
\adjustbox{max width=\textwidth}{
  \begin{tabular}{cc|c}
    $P$ & $Q$ & $P\Rightarrow Q$ \\ 
    \hline
    V & V & V\\
    V & F & F\\
    F & V & V\\
    F & F & V\\
  \end{tabular}}
\end{table}
où $P$ et $Q$ sont deux propositions. Il est possible d'exprimer de
façon équivalente la proposition $P\Rightarrow Q$ sous la forme
$\overline P \lor Q$. Nous utilisons une table de vérité:
\begin{table}[H]
  \centering
\adjustbox{max width=\textwidth}{
  \begin{tabular}{cc|ccc}
    $P$ & $Q$ & $\overline P$ & $\overline P \lor Q$ & $P\Rightarrow Q$ \\ 
    \hline
    V & V & F & V & V\\
    V & F & F & F & F\\
    F & V & V & V & V\\
    F & F & V & V & V\\
  \end{tabular}}
\end{table}
On note que sur chaque ligne des deux dernières colonnes nous avons
les mêmes valeurs de vérité. Les deux propositions sont donc bien
équivalentes. \textbf{(b)} Deux propositions $P$ et $Q$ sont
équivalentes si et seulement si elles ont les mêmes valeurs.
\begin{table}[H]
  \centering
\adjustbox{max width=\textwidth}{
  \begin{tabular}{cc|c}
    $P$ & $Q$ & $P\Leftrightarrow Q$ \\ 
    \hline
    V & V & V\\
    V & F & F\\
    F & V & F\\
    F & F & V\\
  \end{tabular}}
\end{table}
On peut montrer que l'équivalence entre $P$ et $Q$ peut s'exprimer
comme $ (P\Rightarrow Q) \land (Q\Rightarrow P)$. On utilise encore
une table de vérité :
\begin{table}[H]
  \centering
\adjustbox{max width=\textwidth}{
  \begin{tabular}{cc|cccc}
    $P$ & $Q$ & $P\Rightarrow Q$ & $Q \Rightarrow P$ & $(P\Rightarrow Q) \land (Q \Rightarrow P)$  & $P\Leftrightarrow Q$ \\ 
    \hline
    V & V & V & V & V & V\\
    V & F & F & V & F & F\\
    F & V & V & F & F & F\\
    F & F & V & V & V & F\\
  \end{tabular}}
\end{table}
Les valeurs des deux dernières colonnes sont identiques, nous avons
donc bien montré l'équivalence entre ces deux propositions.

\bigskip

\exercice{2} Il s'agit ici de justifier (en logique) le \emph{reductio
  ad absurdum}. Le raisonnement par l'absurde consiste à établir la
vérité d'une proposition en montrant que la proposition complémentaire
(\emph{i.e.} la négation de la proposition d'intérêt) aboutit à une
proposition absurde. Évidemment, nous n'attachons aucun jugement moral
à la notion d'absurdité, une proposition est absurde si elle est
fausse. Soient deux propositions $P$ et $Q$, il s'agit de
montrer que si la proposition $\overline{P} \Rightarrow Q$ est vraie
\textbf{et} si la proposition $Q$ est fausse, alors la proposition $P$
est forcément vraie. À cette fin nous utilisons une table de vérité.

\begin{table}[H]
  \centering
\adjustbox{max width=\textwidth}{
  \begin{tabular}{cc|cc}
    $P$ & $Q$ & $\overline{P}$ & $\overline{P}\Rightarrow Q$ \\ 
    \hline
    V & V & F & V \\
    V & F & F & V \\
    F & V & V & V \\
    F & F & V & F \\
  \end{tabular}}
\end{table}

La proposition $\overline{P}\Rightarrow Q$ est vraie sur les trois premières lignes du tableau. La proposition
$Q$ est fausse sur les lignes 2 et 4 du même tableau. L'intersection
de ces deux ensembles de lignes (les lignes où
$\overline{P}\Rightarrow Q$ est vraie et où
$Q$ est fausse) se réduit à une ligne du tableau : la deuxième. Sur
cette ligne on observe que la valeur de la proposition $P$ est
$V$. Nous n'avons donc aucun doute sur la valeur de cette proposition
et cela suffit à montrer la validité du raisonnement par l'absurde.

\bigskip

\exercice{3} \textbf{(a)} En cours nous avons vu que la dérivée première est définie comme une limite, de la façon suivante :
\[
f'(x) = \lim_{h\rightarrow 0} \frac{f(x+h)-f(x)}{h}
\]

\textbf{(b)} Dans le cas $n=1$, nous avons :
\[
f'(x) = \lim_{h\rightarrow 0} \frac{x+h-x}{h} = \lim_{h\rightarrow 0} \frac{h}{h} = 1\quad \forall x
\]

\textbf{(c)} Dans le cas $n=2$, nous avons :
\[
f'(x) = \lim_{h\rightarrow 0} \frac{(x+h)^2-x}{h} = \lim_{h\rightarrow 0} \frac{2xh+h^2}{h} = \lim_{h\rightarrow 0} 2x + \lim_{h\rightarrow 0} h = 2x \quad \forall x
\]

\textbf{(d)} En notant que $x^{n+1} = x x^n$ et en appliquant la règle de dérivation d'un produit de fonction il vient :
\[
\left(x^{n+1}\right)' = x^n + x \left(x^n\right)'
\]
En utilisant la proposition $P_n$, on obtient :
\[
\left(x^{n+1}\right)' = x^n + n x x^{n-1}
\]
soit encore, en factorisant :
\[
\left(x^{n+1}\right)' = (n+1)x^n
\]
Ainsi la proposition $P_{n+1}$ est vraie, dès lors que $P_n$ est vraie. \textbf{(e)} Nous pouvons conclure que la dérivée de la fonction puissance est $\left(x^n\right)' = n x^{n-1}$ pour tout $n\in\mathbb N$.

\bigskip

\exercice{4} Pour $n=0$, nous avons $2^{2 \times 0}+2 = 3$ qui est bien un entier divisible par trois. Ainsi nous savons que la proposition $P_0$ est vraie. Montrons que si la proposition $P_n$ est vraie alors la proposition est nécessairement vraie au rang $n+1$. Nous avons :
\[
  \begin{split}
    2^{2(n+1)}+2 &= 2^{2n+2} + 2\\
    &= 4 \times 2^{2n} +2\\
    &= 3 \times 2^{2n} + 2^{2n} + 2\\
  \end{split}
\]
Puisque $P_n$ est vraie, nous pouvons écrire $2^{2n}+2=3\kappa$ avec $\kappa\in\mathbb N$. Nous avons donc :
\[
  2^{2(n+1)}+2 = 3 \left(2^{2n} + \kappa \right)
\]
Puisque $2^{2n}+\kappa$ est un entier, $2^{2(n+1)}+2$ est divisible par trois et donc la proposition $P_{n+1}$ est vraie. Nous avons ainsi montré par récurrence que $2^{2n}+2$ est un multiple de trois pour tout $n\in\mathbb N$. 

\bigskip

\exercice{5} On note que $x=1$ est une racine évidente puisque $P(1)=0$. Le polynôme peut donc s'écrire sous la forme factorisée :
\[
P(x) = (x-1)(ax^2+bx+c)
\]
Il reste à déterminer les paramètres $a$, $b$ et $c$. On procède par identification en développant l'expression précédente :
\[
  \begin{split}
    P(x) &= ax^3+bx^2+cx-ax^2-bx-c\\
    &= ax^3 +(b-a)x^2+(c-b)x-c
  \end{split}
\]
On doit donc avoir $a=1$, $c=\frac{5}{4}$ et $b=1$. Nous pouvons donc réécrire le polynôme sous la forme :
\[
P(x) = (x-1)(x^2+x+\frac{5}{4})
\]
Calculons les racines du polynôme d'ordre deux. Le discriminant est :
\[
\Delta = 1-4\frac{5}{4}=-4
\]
Puisque le discriminant est complexe, nous savons que les deux racines sont complexes :
\[
  x = \frac{1\pm 2i}{2} =
  \begin{cases}
    -\frac{1}{2}-i\\
    -\frac{1}{2}+i
  \end{cases}
\]
Au final le polynôme $P(x)$ possède trois racines : $x=1$, $x=\frac{1}{2}-i$ et $\frac{1}{2}+i$.

\bigskip

\exercice{6} \textbf{(1)} La dérivée de la fonction de demande est :
\[
D'(p) = -\epsilon p^{-\epsilon-1}
\]
comme $\epsilon$ est positif par hypothèse, la dérivée est forcément négative. La demande est donc bien une fonction décroissante du prix. \textbf{(2)} En substituant l'expression de la dérivée dans la définition de l'élasticité nous obtenons :
\[
\varepsilon (p) = -\frac{p \epsilon p^{-\epsilon-1}}{p^{-\epsilon}}
\]
\[
\Leftrightarrow \varepsilon (p) = -\epsilon\quad\forall p
\]
L'élasticité ne dépend pas du prix. Ce résultat nous dit que la
sensibilité de la demande aux variations du prix ne dépend pas du
niveau du prix. Si le prix est de 10, une augmentation de 1\% du prix
induira une diminution de $\epsilon$\% de la demande, si le prix est
de 100, une augmentation de 1\% du prix induira aussi une diminution
de $\epsilon$\% de la demande. \textbf{(3)} On a $C'(q) = c$ pour tout
$q$. Cette dérivée est ce que les économistes appellent le coût
marginal. Le coût subit par la firme pour produire une unité
supplémentaire de bien est donnée par le coût marginal. Ici le coût
marginal est constant, ce que la firme doit débourser pour produire
une unité de bien supplémentaire ne dépend pas de la quantité de bien
déjà produite. \textbf{(4)} Le profit de la firme est la différence
entre ses recettes et ses dépenses. Les recettes de la firme sont
données par la quantité de bien produite multipliée par le prix de
chaque unité de bien, c'est-à-dire
$R(p) = p\times p^{-\epsilon} = p^{1-\epsilon}$. Les dépenses de la
firme sont données par les coûts de production :
$\mathcal C(p) = C(D(p)) = c p^{-\epsilon}$. Ainsi le profit est:
\[
\Pi(p) = R(p)-\mathcal C(p) = p^{1-\epsilon} - c p^{-\epsilon} 
\]
\textbf{(5)} Les dérivées d'ordre un et deux de la fonction de profit sont :
\[
\Pi'(p) = (1-\epsilon)p^{-\epsilon} + c \epsilon p^{-\epsilon-1}
\]
et
\[
\Pi''(p) = -\epsilon(1-\epsilon)p^{-\epsilon-1} - c(1+\epsilon) \epsilon p^{-\epsilon-2}
\]
On note que la dérivée d'ordre deux est négative pour toutes les
valeurs du prix $p$, car $\epsilon$ est strictement compris entre 0 et
1 par hypothèse. \textbf{(6)} La dérivée première du profit est une
fonction continue pour tout $p>0$. Le prix qui annule la dérivée
première du profit est le prix qui maximise le profit car la fonction
de profit est concave. Cela revient à égaliser la recette marginale
(\emph{i.e.} la dérivée de la recette par rapport au prix) et le coût
marginal (plus précisément la dérivée du coût par rapport au prix). Si
on note $p^{\star}$ le prix qui maximise le profit, il doit
satisfaire :
\[
  (1-\epsilon){p^{\star}}^{-\epsilon} = - c \epsilon {p^{\star}}^{-\epsilon-1}
\]
\[
p^{\star} = \frac{\epsilon}{\epsilon-1} c
\]
Ici on doit se rendre compte qu'il y a un problème. Il est supposé
plus haut que $\epsilon$ est plus petit que 1. Sous cette hypothèse le
prix optimal est négatif, ce qui n'a évidemment aucun sens ici. Nous
avons trouvé une solution à un problème d'optimisation qui n'a aucun
sens économique. Pour que la solution ait un sens il faudrait que
$\epsilon$ soit plus grand que 1, de sorte que la recette diminue
lorsque le prix augmente (l'augmentation du prix ne compense pas la
baisse de la demande). Mais dans ce cas, nous devons vérifier que le
prix qui annule la dérivée première du profit est bien un
optimum. Notons $\mathscr S (x)$ le signe de $x$. Nous avons :
\[
  \begin{split}
    \mathscr{S}\left(\Pi'(p)\right) &= \mathscr{S}\left((1-\epsilon)+c\epsilon p^{-1}\right)\\
    &= \mathscr{S}\left( p(1-\epsilon)+c\epsilon \right)\\
    &= -\mathscr{S}\left( p+c\frac{\epsilon}{1-\epsilon} \right)\\
    &= -\mathscr{S}\left( p-c\frac{\epsilon}{\epsilon-1} \right)\\
    &= \mathscr{S}\left(p^{\star}-p \right)
  \end{split}
\]
La dérivée première du profit est positive si le prix est inférieur à
$p^{\star}$ et négative si le prix est supérieur à $p^{\star}$. Le
profit diminue donc dès lors que le prix s'éloigne de $p^{\star}$, qui
est donc bien le prix qui maximise le profit.\newline

Notons que si le prix optimal positif (en supposant que $\epsilon>1$)
est bien donné par l'expression de $p^{\star}$, alors ce prix est
supérieur à $c$ (le coût marginal par rapport aux quantités). Le prix
sera d'autant plus grand, relativement au coût marginal, que la
demande est peu élastique au prix. Ce résultat est intuitif dans la
mesure où la firme peut plus librement fixer son prix si elle sait que la
demande réagira peu à des augmentations de prix.  

\end{document}
%%% Local Variables:
%%% mode: latex
%%% TeX-master: t
%%% End:
