\documentclass[10pt,a4paper,notitlepage,twocolumn]{article}
\usepackage{amsmath}
\usepackage{amssymb}
\usepackage{amsbsy}
\usepackage{float}
\usepackage[french]{babel}
\usepackage{graphicx}
\usepackage{enumerate}

\usepackage{palatino}

 \usepackage[active]{srcltx}
\usepackage{scrtime}

\newcommand{\exercice}[1]{\textsc{\textbf{Exercice}} #1}
\newcommand{\question}[1]{\textbf{(#1)}}
\setlength{\parindent}{0cm}

\begin{document}

\title{\textsc{Calcul Économique}}
\author{Stéphane Adjemian\thanks{Université du Maine, Gains. \texttt{stephane DOT adjemian AT univ DASH lemans DOT fr}}}
\date{Le \today\ à \thistime}

\maketitle
\thispagestyle{empty}

\exercice{1} \textbf{(a)} Rappeler à l'aide d'une table de vérité la définition de l'implication logique entre deux propositions $P$ et $Q$. Montrer qu'il est possible de l'exprimer à l'aide d'un connecteur logique ($\land$ ou $\lor$) et d'une (ou plusieurs) négation(s). \textbf{(b)} Rappeler à l'aide d'une table de vérité la définition de l'équivalence logique entre deux propositions $P$ et $Q$. Montrer qu'il est possible de l'exprimer à l'aide de deux implications logiques et du connecteur logique $\land$.

\bigskip

\exercice{2} Soient deux propositions $P$ et $Q$. Montrer que si $\bar P \Rightarrow Q$ est une proposition vraie et si $Q$ est une proposition fausse, alors la proposition $P$ est nécessairement vraie. Dans quel type de raisonnement ce résultat est-il utilisé ?
	
\bigskip

\exercice{3} Soit la fonction de $\mathbb R$ dans $\mathbb R$ $f(x) = x^n$ avec $n\in\mathbb N$. \textbf{(a)} Rappeler la définition de la dérivée première d'une fonction. \textbf{(b)} Dans le cas $n=1$, montrer que $f'(x) = 1$ pour tout $x$. \textbf{(c)} Dans le cas $n=2$, montrer que $f'(x) = 2x$ pour tout $x$. \textbf{(d)} On suppose que la propriété :
\begin{center}
$P_{n}$ : \textit{Si $f(x)=x^{n}$ alors $f'(x) = n x^{n-1}$}  
\end{center}
est vraie. Montrer que la propriété :
\begin{center}
$P_{n+1}$ : \textit{Si $f(x)=x^{n+1}$ alors $f'(x) = (n+1) x^{n}$}  
\end{center}
 est nécessairement vraie. \textbf{(e)} Conclure sur la dérivée première de la fonction $f(x)$.

\bigskip

\exercice{4} Montrer que la proposition:
\begin{center}
  $P_n$: $2^{2\times n}+2$ est un entier divisible par trois
\end{center}
est vraie pour tout $n\in\mathbb N$.

\bigskip

\exercice{5} Calculer les racines du polynôme suivant :
\[
P(x) = x^3 + \frac{1}{4}x - \frac{5}{4}
\]    

\bigskip

\exercice{6} On suppose que la demande, qui dépend du prix $p$, adressée à une entreprise est caractérisée par la fonction suivante :
\[
D(p) = p^{-\epsilon}
\]
avec $\epsilon>0$. Dans la suite \textit{nous supposerons que $0<\epsilon<1$}. \textbf{(1)} Montrer que la demande est bien une fonction monotone décroissante du prix. \textbf{(2)} On définit
l'élasticité de la demande au prix de la façon suivante :
\[
\varepsilon(p) = D'(p)\frac{p}{D(p)}
\]
Calculer cette élasticité et montrer qu'elle est constante. Comment s'interprète cette quantité ? \textbf{(3)} Pour répondre à la demande l'entreprise doit supporter un coût (de production). On suppose que l'entreprise doit payer $c>0$ pour chaque unité de bien produite, c'est à dire que la fonction de coût est de la forme :
\[
C(q) = cq
\]
où $q$ est la quantité de bien produite. Calculer $C'(q)$. Comment s'interprète cette quantité ? \textbf{(4)} Justifier la forme de la fonction de profit de l'entreprise :
\[
\Pi(q) = p^{1-\epsilon} - c p^{-\epsilon}
\]
\textbf{(5)} Calculer les dérivées d'ordre 1 et 2 de la fonction de profit. Montrer que la fonction de profit est concave. \textbf{(6)} Calculer le prix qui maximise le profit. Interpréter le résultat.

\end{document}

%%% Local Variables:
%%% mode: latex
%%% TeX-master: t
%%% End:
