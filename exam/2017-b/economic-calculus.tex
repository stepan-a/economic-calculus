\documentclass[10pt,a4paper,notitlepage,onecolumn]{article}
\usepackage{amsmath}
\usepackage{amssymb}
\usepackage{amsbsy}
\usepackage{float}
\usepackage[french]{babel}
\usepackage{graphicx}
\usepackage{enumerate}


\usepackage{palatino}

\usepackage[active]{srcltx}
\usepackage{scrtime}

\newcommand{\exercice}[1]{\textsc{\textbf{Exercice}} #1}
\newcommand{\question}[1]{\textbf{(#1)}}
\setlength{\parindent}{0cm}

\begin{document}

\title{\textsc{Calcul Économique}}
\author{Stéphane Adjemian\thanks{Université du Maine, Gains.}}
\date{Le 11 janvier 2018}

\maketitle
\thispagestyle{empty}

\exercice{1} Calculer les racines du polynôme :
\[
P(x) = x^3 - \frac{1}{6}x^2 - \frac{4}{6}x - \frac{1}{6}
\]    

\bigskip

\exercice{2} Déterminer la dérivée de la réciproque d'une fonction (en supposant que celle-ci existe).

\bigskip

\exercice{3} Montrer par récurrence l'inégalité suivante (pour tout $x\geq 0$ et $n\in\mathbb N$) :
\[
(1+x)^n \geq 1+nx
\]

\bigskip

\exercice{4} Montrer que l'implication logique est transitive.

\bigskip

\exercice{5} Montrer qu'il est possible d'exprimer de façon équivalente la proposition $P \lor Q$, où $P$ et $Q$
sont des propositions, à l'aide du connecteur logique $\land$ et de la négation.

\bigskip

\exercice{6} Montrer que la proposition :
\[
P_n: \quad 2^{2\times n}+2 \quad \text{est divisible par trois }\forall n\in\mathbb N 
\]
est vraie.

\bigskip

\exercice{7} Soit la fonction $y = f(x) = x^{\alpha}$ une fonction définie
sur $\mathbb R^+$, avec $\alpha$ un paramètre réel positif inférieur à
un. Cette fonction représente le niveau de production
obtenu par une firme lorsqu'elle utilise une quantité $x$ du facteur
de production. \textbf{(1)} Représenter graphiquement cette
fonction. \textbf{(2)} Discuter la concavité/convexité de cette
fonction à partir d'un argument graphique. \textbf{(3)} Conclure sur
la concavité/convexité à partir de la dérivée d'ordre 2 de
$f$. \textbf{(4)} Cette firme doit payer la location du facteur de
production. On note $p$ le prix d'une unité du facteur de
production en termes de bien produit. Posons la fonction :
\[
\Pi(x) = x^{\alpha} - p x
\]
Interpréter cette fonction. Quel est le concept représenté par cette
fonction ? \textbf{(5)}. Représenter graphiquement cette fonction. Cette fonction est-elle concave ou convexe ?
\textbf{(6)} Calculer la dérivée d'ordre un de la fonction
$\Pi$. \textbf{(7)} Identifier la quantité $x^{\star}$ telle que
$\Pi'(x^{\star}) = 0$. Calculer $y^{\star}=f(x^{\star})$ et
$\Pi^{\star} = \Pi(x^{\star})$. \textbf{(8)} Interpréter $\Pi^{\star}$
et la condition qui détermine $x^{\star}$.



\end{document} 

%%% Local Variables:
%%% mode: latex
%%% TeX-master: t
%%% End:
