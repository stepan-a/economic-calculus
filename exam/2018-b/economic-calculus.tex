\documentclass[10pt,a4paper,notitlepage,twocolumn]{article}
\synctex=1
\usepackage{amsmath}
\usepackage{amssymb}
\usepackage{amsbsy}
\usepackage{float}
\usepackage[french]{babel}
\usepackage{graphicx}
\usepackage{enumerate}

\usepackage{palatino}

 \usepackage[active]{srcltx}
 \usepackage{scrtime}
 \usepackage{exercise}

\newcommand{\exercise}[1]{\textsc{\textbf{Exercice}} #1}
\newcommand{\question}[1]{\textbf{(#1)}}
\setlength{\parindent}{0cm}

\begin{document}

\title{\textsc{Calcul Économique}}
\author{Stéphane Adjemian\thanks{Université du Maine, Gains. \texttt{stephane DOT adjemian AT univ DASH lemans DOT fr}}}
\date{Mardi 4 juin 2019}

\maketitle

\thispagestyle{empty}

\exercise{1} \question{a} Rappeler à l'aide d'une table de vérité la
définition de l'implication logique entre deux propositions $P$ et
$Q$. Montrer qu'il est possible de l'exprimer à l'aide d'un connecteur
logique ($\land$ ou $\lor$) et d'une (ou plusieurs)
négation(s). \question{b} Rappeler à l'aide d'une table de vérité la
définition de l'équivalence logique entre deux propositions $P$ et
$Q$. Montrer qu'il est possible de l'exprimer à l'aide de deux
implications logiques et d'un connecteur logique. \question{c} Expliquer
en quoi ce résultat est important.

\bigskip
  
\exercise{2} Montrer la transitivité de l'implication logique.

\bigskip

\exercise{3} Soit la fonction de $\mathbb R$ dans $\mathbb R$ $f(x) = x^n$ avec $n\in\mathbb N$. \question{a} Rappeler la définition de la dérivée première d'une fonction. \question{b} Dans le cas $n=1$, montrer que $f'(x) = 1$ pour tout $x$. \question{c} Dans le cas $n=2$, montrer que $f'(x) = 2x$ pour tout $x$. \question{d} On suppose que la proposition :
\begin{center}
$P_{n}$ : \textit{Si $f(x)=x^{n}$ alors $f'(x) = n x^{n-1}$}  
\end{center}
est vraie. Montrer que la proposition :
\begin{center}
$P_{n+1}$ : \textit{Si $f(x)=x^{n+1}$ alors $f'(x) = (n+1) x^{n}$}  
\end{center}
 est nécessairement vraie. \question{e} Conclure sur la dérivée première de la fonction $f(x)$.

\bigskip

\exercise{4} Soit $f$, $g$ et $h$ trois fonctions continues deux fois dérivables et dont les dérivées sont continues. \question{a} Quelle est la dérivée de $f(g(x))$ par rapport à $x$. \question{b} Calculer la dérivée d'ordre deux de $f(g(x))$ par rapport à $x$. \question{c} Calculer la dérivée d'ordre un de $f(g(h(x)))$. \question{d} Sous quelle(s) condition(s) la fonction réciproque $f^{-1}$ existe-t-elle ? \question{e} Calculer la dérivée d'ordre un de $f^{-1}(x)$.

\bigskip

\exercise{5} Notons $S_n$ la somme des $n$ premiers entiers impairs. Nous avons donc $S_1 = 1$, $S_2 = 1 + 3 = 4$ et plus généralement $S_n = 1+3+\dots+2n-1$. Montrer que pour tout $n\in\mathbb N$ on a $S_n = n^2$, c'est-à-dire que la somme des $n$ premiers entiers impairs est égale au carré de $n$.  

\bigskip

\exercise{6} \question{a} Représenter graphiquement une fonction convexe monotone croissante $f$. \question{b} Tracer une tangente à cette fonction en un point positif quelconque, disons $x_{0}>0$. \question{c} Donner l'équation de cette tangente.

\bigskip

\exercise{7} Sur un marché la quantité offerte ($q$) est donnée comme
une fonction du prix par :
\[
q = p^2
\]
et la demande est déterminée par :
\[
q = 6-p
\]
\question{a} Montrer qu'il existe un unique prix d'équilibre
$p^{\star}$. \question{b} Donner une représentation graphique des
fonctions de demande et d'offre, et de l'équilibre.

\bigskip

\exercise{8} Calculer les racines du polynôme suivant : 
\[
P(x) = 4x^3+4x^2-3x-5
\]



\end{document}


%%% Local Variables:
%%% mode: latex
%%% TeX-master: t
%%% End:
