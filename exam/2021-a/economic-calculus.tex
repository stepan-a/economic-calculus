\documentclass[10pt,a4paper,notitlepage,twocolumn]{article}
\usepackage{amsmath}
\usepackage{amssymb}
\usepackage{amsbsy}
\usepackage{float}
\usepackage[french]{babel}
\usepackage{graphicx}
\usepackage{enumerate}

\usepackage{palatino}

 \usepackage[active]{srcltx}
\usepackage{scrtime}

\newcommand{\exercice}[1]{\textsc{\textbf{Exercice}} #1}
\newcommand{\question}[1]{\textbf{(#1)}}
\setlength{\parindent}{0cm}

\begin{document}

\title{\textsc{Calcul Économique}}
\author{Stéphane Adjemian\thanks{Université du Maine, Gains. \texttt{stephane DOT adjemian AT univ DASH lemans DOT fr}}}
\date{Le \today\ à \thistime}

\maketitle
\thispagestyle{empty}

\exercice{1} Soient deux propositions $P$ et $Q$. Montrer que les
propositions $\bar Q \Rightarrow \bar P$ et $P \Rightarrow Q$ sont
équivalentes. Donner un exemple de raisonnement utilisant ce résultat.

\bigskip

\exercice{2} Soient deux propositions $P$ et $Q$. Montrer que si
$\bar P \Rightarrow Q$ est une proposition vraie et si $Q$ est une proposition
fausse, alors la proposition $P$ est nécessairement vraie. Dans quel type de
raisonnement ce résultat est-il utilisé ?
	
\bigskip

\exercice{3} \textbf{(a)} Donnez la définition de la dérivée d'une fonction en
un point. \textbf{(b)} En utilisant cette définition calculez la dérivée de la
fonction~:
\[
f(x) = \frac{1}{1-2x}
\]
en précisant auparavant le domaine de définition de cette fonction. \textbf{(c)}
Quelle formule usuelle de dérivation auriez-vous pu utiliser pour arriver au
même résultat~?

\bigskip

\exercice{4} Montrer par récurrence que~:
\begin{equation}
  \sum_{i=1}^n i^3  = \frac{n^2(n+1)^2}{4}
\end{equation}

\bigskip

\exercice{5} Soit le polynôme suivant~:
\[
P(X) = X^3 + \frac{1}{4}X - \frac{5}{4}
\]    
\textbf{(a)} Calculez les racines de ce polynôme. \textbf{(b)} Montrez que la
fonction $f(x)=x^3+\frac{1}{4}x-\frac{5}{4}$ est monotone croissante.
\textbf{(c)} Montrez que la fonction est concave pour les valeurs négatives de
$x$ puis convexe pour les valeurs positives de $x$. \textbf{(d)} Que pouvez vous
dire du point $(0,-\frac{5}{4})$~? \textbf{(e)} Représentez graphiquement la fonction $f(x)$.

\bigskip

\exercice{6} Soient $f$ et $g$ deux fonctions de $\mathbb R$ dans $\mathbb R$ de
classe $\mathcal C^2$. Calculez la dérivée d'ordre deux de la fonction~:
$h(x) = \bigl(f \circ g\bigr) (x)$.

\bigskip

\exercice{6} On suppose que la demande, adressée à une entreprise en situation
de monopole sur un marché, dépend du prix $p$ de la façon suivante~:
\[
D(p) = p^{-\epsilon}\quad\forall p>0
\]
avec $\epsilon>1$. \textbf{(a)} Montrer que la demande est bien une fonction
monotone décroissante du prix. \textbf{(b)} On définit l'élasticité de la
demande par rapport au prix de la façon suivante :
\[
\sigma(p) = -D'(p)\frac{p}{D(p)}
\]
Calculez cette élasticité et montrez que celle-ci est constante, c'est-à-dire
qu'elle ne dépend pas du diveau du prix. L'élasticité prix de la demande
quantifie la variation de la demande induite par une variation du prix. Si
l'élasticité est constante, cela veut dire qu'une augmentation de $1\%$ du prix
induira toujours une baisse de $\sigma\%$ de la demande quelle que soit le
niveau du prix. \textbf{(c)} Pour répondre à la demande l'entreprise doit
supporter un coût de production. On suppose que l'entreprise doit payer $c>0$
pour chaque unité de bien produite, c'est à dire que la fonction de coût est de
la forme~:
\[
C(q) = cq
\]
où $q$ est la quantité de bien produite. Calculez $C'(q)$, c'est à dire le coût
marginal. \textbf{(d)} Le profit de la firme est la différence entre ses
recettes et ses dépenses. Les recettes de la firme sont données par la quantité
de bien produite multipliée par le prix de chaque unité de bien, c'est-à-dire
$R(p) = p\times p^{-\epsilon} = p^{1-\epsilon}$. Les dépenses de la firme sont
données par ses coûts de production :
$\mathcal C(p) = C(D(p)) = c p^{-\epsilon}$. Ainsi, au total, le profit de la
firme comme une fonction du prix $p$ est:
\[
\Pi(p) = R(p)-\mathcal C(p) = p^{1-\epsilon} - c p^{-\epsilon} 
\]
Puisque la firme est en situation de monopole, elle peut choisir le prix du
bien. Cependant, sa liberté est limitée par la demande~: elle n'a pas intérêt à
choisir un prix trop élevé car elle sait qu'une augmentation du prix (qui
devrait augmenter sa recette) induira aussi une baisse de la demande (ce qui
contribuera, à l'inverse, à diminuer sa recette). La firme va choisir le prix
qui maximise son profit, c'est ce prix optimal que nous allons chercher dans les
questions suivantes, mais auparavant montrez que~:
\begin{enumerate}[(i)]
\item La recette du monopole est une fonction décroissante du prix
\item Le coût du monopole est une fonction décroissante du prix
\end{enumerate}
en interprétant ces résultats et en discutant les arbitrages de l'entreprise (par
rapport à sa décision sur le prix du bien). \textbf{(e)} Calculez la dérivée de la fonction de
profit. \textbf{(f)} On notera $p^{\star}$ le prix qui annule la dérivée du
profit, calculez $p^{^\star}$ en l'exprimant en fonction de $c$ et $
\epsilon$.
\textbf{(g)} Montrez que $\Pi'(p)>0 \Leftrightarrow p<p^{\star}$. Le prix $p^{\star}$
est-il le prix qui maximise le profit du monopole~?

\end{document}

%%% Local Variables:
%%% mode: latex
%%% TeX-master: t
%%% End:
