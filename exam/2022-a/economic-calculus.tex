\documentclass[12pt,a4paper,notitlepage]{article}
\usepackage{amsmath}
\usepackage{amssymb}
\usepackage{amsbsy}
\usepackage{float}
\usepackage[french]{babel}
\usepackage{graphicx}
\usepackage{enumerate}

\usepackage{palatino}

 \usepackage[active]{srcltx}
\usepackage{scrtime}

\newcounter{xnumber}
\setcounter{xnumber}{0}
\newcounter{qnumber}


\newcommand{\exercice}{\setcounter{qnumber}{0}\textsc{\textbf{Exercice}} \textbf{\addtocounter{xnumber}{1}\thexnumber}\,\,}
\newcommand{\question}{\textbf{(\addtocounter{qnumber}{1}\theqnumber)}\,}
\setlength{\parindent}{0cm}


\begin{document}

\title{\textsc{Calcul Économique}}
\author{Stéphane Adjemian\thanks{Université du Mans. \texttt{stephane DOT adjemian AT univ DASH lemans DOT fr}}}
\date{Le \today\ à \thistime}

\maketitle

\exercice Soient $P$, $Q$ et $R$ trois propositions. \question Montrer que la
proposition $(P\Leftrightarrow Q)$ est équivalente à la proposition
$(P\Rightarrow Q)\land (Q\Rightarrow P)$. \question Montrer que la proposition
$(P\lor Q)\land R$ est équivalente à la proposition $(P\land R)\lor (Q\land R)$.
Comment se nomme cette propriété~?

\bigskip

\exercice Calculer les racines du polynôme suivant~:
\[
  P(X) = X^3 - 4X^2 + 6X - 4
\]

\bigskip

\exercice Résoudre l'équation suivante~:
\[
4^x-2^{x+1}-3 = 0
\]
en montrant que cette équation n'admet qu'une seule solution.

\bigskip

\exercice Soit la fonction de $\mathbb R$ dans $\mathbb R$ $f(x) = (1-x)^2$.
\question Calculer la dérivée de cette fonction avec les formules usuelles.
\question Vérifier que l'on obtient un résultat identique en utilisant la
définition de la dérivée (avec une limite).

\bigskip

\exercice Soit la fonction $f(x) = \frac{4-x^2}{e^x}$. \question Quel est le
domaine de définition de la fonction~? \question Calculer les limites en
$-\infty$ et $+\infty$. \question Calculer $f'(x)$. \question Cette fonction
admet-elle un maximum et un minimum ? Justifier la réponse. \question Le maximum
est-il global~? \question Le minimum est-il global~?

\bigskip

\exercice \question Montrer que la fonction $f(x) = 1-|x|$ est continue en 0
mais n'admet pas de dérivée en 0. \question Représenter graphiquement la
fonction $f(x)$.

\end{document}

%%% Local Variables:
%%% mode: latex
%%% TeX-master: t
%%% End:
