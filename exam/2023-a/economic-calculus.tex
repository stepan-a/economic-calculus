\synctex=1
\documentclass[11pt,a4paper,notitlepage,twocolumn]{article}
\usepackage{amsmath}
\usepackage{amssymb}
\usepackage{amsbsy}
\usepackage{float}
\usepackage[french]{babel}
\usepackage{graphicx}
\usepackage{enumerate}

\usepackage{palatino}

 \usepackage[active]{srcltx}
\usepackage{scrtime}

\newcounter{xnumber}
\setcounter{xnumber}{0}
\newcounter{qnumber}


\newcommand{\exercice}{\setcounter{qnumber}{0}\textsc{\textbf{Exercice}} \textbf{\addtocounter{xnumber}{1}\thexnumber}\,\,}
\newcommand{\question}{\textbf{(\addtocounter{qnumber}{1}\theqnumber)}\,}
\setlength{\parindent}{0cm}


\begin{document}

\title{\textsc{Calcul Économique}}
\date{Mercredi 13 décembre 2023}

\pagenumbering{gobble}

\maketitle

\begin{quote}
  \textit{Les réponses non commentées ou insuffisamment détaillées ne seront pas considérées. Prenez le temps de faire des phrases.}
\end{quote}

\bigskip

\exercice Soient $P$, $Q$ et $R$ trois propositions. Montrer que la
proposition $(P\Leftrightarrow Q)$ est équivalente à la proposition
$(P\Rightarrow Q)\land (Q\Rightarrow P)$.\newline

\bigskip

\exercice Soient $P$ et $Q$ deux propositions. Le connecteur de
Sheffer (nous ne l'avons pas vu en cours, il est très utilisé en
informatique où il est généralement appelé \verb+nand+) est noté et
défini comme~: $P \barwedge Q = \overline{P\land Q}$. \question
Construire une table logique pour donner les valeurs de vérité de
$P \barwedge Q$ et définir en français (c'est-à-dire avec des mots) ce
connecteur. \question Montrer que
$(P\lor Q) \Leftrightarrow (P \barwedge P)\barwedge (Q\barwedge
Q)$. \question Montrer que
$(P\Rightarrow Q) \Leftrightarrow P\barwedge (Q\barwedge Q)$.\newline

\bigskip

\exercice Calculer les racines du polynôme suivant~:
\[
  P(X) = X^3 - 3X^2 + \frac{13}{4}X - \frac{5}{4}
\]

\bigskip

\exercice Résoudre l'équation suivante~:
\[
4^x-2^{x+1}-3 = 0
\]
en montrant que cette équation n'admet qu'une seule solution.\newline

\bigskip

\exercice Soit la fonction à valeurs réelles $f(x) =
\frac{1}{1-x}$. \question Préciser son domaine de
définition. \question Cette fonction est-elle continue~? Pourquoi~?
\question Calculer, lorsqu'elle existe, la dérivée de cette fonction,
$f'(x)$, en utilisant la définition de la dérivée (avec une
limite).\newline

\bigskip

\exercice Soient $f: E\rightarrow F$ et $g: F\rightarrow G$ deux
fonctions continues et dérivables deux fois. Calculer la dérivée
seconde de $\left(f\circ g \right)(x) = f(g(x))$.\newline

\bigskip

\exercice Soit la fonction à valeurs réelles
$f(x) = \frac{2x-\sqrt{x}}{2+\sqrt{x}}$. \question Quel est le domaine
de définition de la fonction~? \question Quelle est la limite de
$f(x)$ quand $x$ tend vers l'infini. \question Montrer que cette
fonction admet un minimum global.\newline

\textbf{\textsc{Indice }:} Pour étudier le signe de la dérivée de $f$
(du numérateur car vous devriez trouver que le dénominateur est
toujours positif) vous pouvez réécrire le numérateur comme un polynôme
d'ordre deux à l'aide du changement de variable $X=\sqrt{x}$.

\end{document}

%%% Local Variables:
%%% mode: latex
%%% TeX-master: t
%%% End:
