\documentclass[10pt,a4paper,notitlepage]{article}

\synctex=1

\usepackage{amsmath}
\usepackage{amssymb}
\usepackage{amsbsy}
\usepackage{float}
\usepackage[french]{babel}
\usepackage{graphicx}
\usepackage{enumerate}

\usepackage{palatino}
\usepackage{mathrsfs}

\usepackage[active]{srcltx}
\usepackage{scrtime}
\usepackage{nicefrac}
\usepackage{adjustbox}

\newcommand{\exercice}[1]{\textsc{\textbf{Exercice}} #1}
\newcommand{\question}[1]{\textbf{(#1)}}
\setlength{\parindent}{0cm}

\begin{document}

\title{\textsc{Calcul Économique}\\\textbf{(Éléments de correction)}}
\author{Stéphane Adjemian\thanks{Université du Mans. \texttt{stephane DOT adjemian AT univ DASH lemans DOT fr}}}
\date{Le \today\ à \thistime}

\maketitle

\exercice{1} \textbf{(a)} L'implication logique est définie de la façon suivante, à l'aide d'une table de vérité :
\begin{table}[H]
  \centering
\adjustbox{max width=\textwidth}{
  \begin{tabular}{cc|c}
    $P$ & $Q$ & $P\Rightarrow Q$ \\ 
    \hline
    V & V & V\\
    V & F & F\\
    F & V & V\\
    F & F & V\\
  \end{tabular}}
\end{table}
où $P$ et $Q$ sont deux propositions. Il est possible d'exprimer de
façon équivalente la proposition $P\Rightarrow Q$ sous la forme
$\overline P \lor Q$. Nous utilisons une table de vérité:
\begin{table}[H]
  \centering
\adjustbox{max width=\textwidth}{
  \begin{tabular}{cc|ccc}
    $P$ & $Q$ & $\overline P$ & $\overline P \lor Q$ & $P\Rightarrow Q$ \\ 
    \hline
    V & V & F & V & V\\
    V & F & F & F & F\\
    F & V & V & V & V\\
    F & F & V & V & V\\
  \end{tabular}}
\end{table}
On note que sur chaque ligne des deux dernières colonnes nous avons
les mêmes valeurs de vérité. Les deux propositions sont donc bien
équivalentes. \textbf{(b)} Deux propositions $P$ et $Q$ sont
équivalentes si et seulement si elles ont les mêmes valeurs.
\begin{table}[H]
  \centering
\adjustbox{max width=\textwidth}{
  \begin{tabular}{cc|c}
    $P$ & $Q$ & $P\Leftrightarrow Q$ \\ 
    \hline
    V & V & V\\
    V & F & F\\
    F & V & F\\
    F & F & V\\
  \end{tabular}}
\end{table}
On peut montrer que l'équivalence entre $P$ et $Q$ peut s'exprimer
comme $ (P\Rightarrow Q) \land (Q\Rightarrow P)$. On utilise encore
une table de vérité :
\begin{table}[H]
  \centering
\adjustbox{max width=\textwidth}{
  \begin{tabular}{cc|cccc}
    $P$ & $Q$ & $P\Rightarrow Q$ & $Q \Rightarrow P$ & $(P\Rightarrow Q) \land (Q \Rightarrow P)$  & $P\Leftrightarrow Q$ \\ 
    \hline
    V & V & V & V & V & V\\
    V & F & F & V & F & F\\
    F & V & V & F & F & F\\
    F & F & V & V & V & V\\
  \end{tabular}}
\end{table}
Les valeurs des deux dernières colonnes sont identiques, nous avons
donc bien montré l'équivalence entre ces deux
propositions. \textbf{(c)} Ce résultat nous dit que pour établir
l'équivalence entre deux propositions $P$ et $Q$, on peut décomposer
la preuve en deux parties en montrant d'abord que $P\Rightarrow Q$
puis en montrant la réciproque $Q\Rightarrow P$.

\bigskip

\exercice{2} On utilise une table de vérité.

\begin{table}[H]
  \centering
 \begin{tabular}{cc|cccccc}
    $P$ & $Q$ & $P\Rightarrow Q$ & $\overline{Q}$ & $\overline{P}$ & $\overline{Q}\Rightarrow\overline{P}$   \\ 
    \hline
    V & V & V & F & F & V\\
    V & F & F & V & F & F\\
    F & V & V & F & V & V\\
    F & F & V & V & V & V\\
  \end{tabular}
\end{table}

La dernière colonne et la troisième colonne ont les mêmes valeurs de
vérité sur chaque ligne, les deux propositions sont donc bien
équivalentes. La proposition $\overline{Q}\Rightarrow\overline{P}$ est
la contraposée de la proposition $P\Rightarrow Q$. Un raisonnement par
contraposition consiste à montrer que
$\overline{Q}\Rightarrow\overline{P}$ pour établir que
$P\Rightarrow Q$. Supposons que la proposition $P$ soit <<~Il pleut~>>
et la proposition $Q$ soit <<~Il y a des nuages~>> . On a bien
$P\Rightarrow Q$ (la présence de nuages dans le ciel est bien une
condition nécessaire pour qu'il pleuve). Il est tout aussi évident que
<<~Il n'y a pas de nuages~>> implique <<~Il ne pleut pas~>>.

\bigskip

\exercice{3} \textbf{(a)} En cours nous avons vu que la dérivée première est définie comme une limite, de la façon suivante :
\[
f'(x) = \lim_{h\rightarrow 0} \frac{f(x+h)-f(x)}{h}
\]
\textbf{(b)} Nous avons:
\[
f(x+h) = \frac{1}{1-(x+h)^2}
\]
et donc
\[
  \begin{split}
    f(x+h)-f(x) &= \frac{1}{1-(x+h)^2} - \frac{1}{1-x^2}\\
    &= \frac{1-x^{2}-1+(x+h)^{2}}{\left(1-(x+h)^2\right)\left(1-x^{2}\right)}\\
    &= \frac{(2x+h)h}{\left(1-(x+h)^2\right)\left(1-x^{2}\right)}\\
  \end{split}
\]
puis :
\[
\frac{f(x+h)-f(x)}{h} = \frac{2x+h}{\left(1-(x+h)^2\right)\left(1-x^{2}\right)}
\]
Finalement la dérivée est la limite de la dernière expression quand $h$ tend vers zéro, c'est-à-dire le ratio des limites au numérateur et au dénominateur (il n'y a pas d'indétermination dans ce cas) :
\[
f'(x) = \frac{2x}{\left(1-x^{2}\right)^{2}}
\]


\bigskip

\exercice{4} \textbf{(a)} On a $g'(x) = \frac{\mathrm d}{\mathrm dx}f(\theta x)$, par la formule de dérivation en chaîne, on a donc :
\[
g'(x) = \theta g(x)
\]
c'est-à-dire la dérivée de $\theta x$ par rapport à $x$, fois la dérivée de $f$ évaluée en $\theta x$ (sachant que la dérivée de $f$ est $f$). La formule est donc correcte au premier rang ($n=1$). Supposons que la formule soit vraie au rang $n$ et montrons alors qu'elle est alors nécessairement vraie au rang $n+1$, c'est-à-dire que :
\[
g^{(n+1)}(x) = \theta^{n+1}g(x)
\]
Nous avons :
\[
  \begin{split}
    g^{(n+1)} &= \left(g^{n}(x)\right)'\\
    &= \left(\theta^{n}g(x)\right)'\\
    &= \theta^{n}g'(x)\\
    &= \theta^{n}\theta g(x)\\
    &= \theta^{n+1}g(x)
  \end{split}
\]
où la deuxième égalité vient de la formule supposée vraie au rang $n$, la troisième vient des régles de dérivation (la dérivée d'une constante fois une fonction est égale à la constante fois la dérivée de la fonction), la quatrième égalité vient de la formule démontrée vraie au rang 1. Nous retrouvons donc bien la formule au rang $n+1$, lorsqu'elle est vraie au rang $n$. La formule est donc vraie pour toutes les valeurs de $n$. \textbf{(b)} La fonction exponentielle satisfait la même propriété que la fonction $f$.

\bigskip

\exercice{5} On note que $x=1$ est une racine évidente puisque $P(1)=0$. Le polynôme peut donc s'écrire sous la forme factorisée :
\[
P(x) = (x-1)(ax^2+bx+c)
\]
Il reste à déterminer les paramètres $a$, $b$ et $c$. On procède par identification en développant l'expression précédente :
\[
  \begin{split}
    P(x) &= ax^3+bx^2+cx-ax^2-bx-c\\
    &= ax^3 +(b-a)x^2+(c-b)x-c
  \end{split}
\]
On doit donc avoir $a=4$, $b-a=4$ (donc $b=8$), $c-b=-3$ (donc $c=5$). Nous pouvons donc réécrire le polynôme sous la forme :
\[
P(x) = (x-1)(4x^2+8x+5)
\]
Calculons les racines du polynôme d'ordre deux. Le discriminant est :
\[
\Delta = 64-4 \times 4 \times 5 = -16
\]
Puisque le discriminant est complexe, nous savons que les deux racines sont complexes :
\[
  x = \frac{8 \pm 4i}{8} =
  \begin{cases}
    -1-\frac{i}{2}\\
    -1+\frac{i}{2}
  \end{cases}
\]
Au final le polynôme $P(x)$ possède trois racines : $x=1$, $x=-1-\frac{i}{2}$ et $x=-1+\frac{i}{2}$.

\bigskip

\exercice{6} \textbf{(a)} Dans ce cas, $f'(x)=a$ et on aurait donc :
\[
\frac{a}{ax+b} = \alpha
\]
Clairement il n'est pas possible de trouver $a$ et $b$ satisfaisant
cette équation (notons que si $\alpha=0$, alors l'équation est
satisfaite pour $a=0$, mais nous chercons une fonction $f$ qui puisse
vérifier la contrainte pour toute valeur de $\alpha$). \textbf{(b)}
Dans ce cas, on a : $f'(x) = a e^{ax+b}$ c'est-à-dire
$f'(x) = a f(x)$. Ainsi la contrainte est satisfaite si $a=\alpha$ et
pour toute valeur de $b$. Puisque n'importe quelle valeur de $b$ est
acceptable, il existe une infinité (c'est-à-dire autant que de valeurs
de $b$ possibles) de fonctions $f$ vérifiant la contrainte.

\bigskip

\exercice{7} Il suffit de noter (identité remarquable) qu'il est possible de réécrire la fonction sous la forme :
\[
f(x) = (x+1)^{4}+1
\]
Puisque le premier terme est élevé à une puissance paire, il est
forcément non négatif. Minimiser cette fonction revient à minimiser le
premier terme, puisque le second terme (1) ne dépend pas de la
variable de contrôle $x$. Le premier terme est nul (il ne peut pas
être plus petit) lorsque $x=-1$. La fonction atteint donc sa valeur
minimale en $x=-1$ et on a $f(-1)=1$.

\bigskip

\exercice{8} Puisque la courbe est convexe, la tangente doit se situer
sous la courbe représentative de la fonction. Pour trouver l'équation
de la tangente à la courbe en $x_{0}$ il suffit de remarquer que la
tangente passe par le point $(x_{0}, f(x_{0}))$ et que sa pente est
donnée par la dérivée de la fonction évaluée en $x_{0}$, $f'(x_{0})$. La tangente, que nous noterons $T(x)$ est donc de la forme :
\[
T(x) = f'(x_{0}) x + b
\]
où $b$ est l'ordonée à l'origine qu'il nous reste à identifier. La
tangente est une droite qui passe par les points $(0,b)$ et
$(x_{0}, f'(x_{0})$, comme sa pente est $f'(x_{0})$, nous devons
avoir :
\[
\frac{f(x_{0})-f(0)}{x_{0}-b} = f'(x_{0})
\]
soit de façon équivalente :
\[
b = f(x_{0}) - x_{0}f'(x_{0})
\]
Nous avons finalement :
\[
T(x) = f'(x_{0}) x + f(x_{0}) - x_{0}f'(x_{0})
\]
ou encore :
\[
T(x) = f(x_{0}) + \left(x-x_{0}\right) f'(x_{0})
\]



\end{document}
%%% Local Variables:
%%% mode: latex
%%% TeX-master: t
%%% End:
