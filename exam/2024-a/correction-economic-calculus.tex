\synctex=1
\documentclass[11pt,a4paper,notitlepage]{article}
\usepackage{amsmath}
\usepackage{amssymb}
\usepackage{amsbsy}
\usepackage{float}
\usepackage[french]{babel}
\usepackage{graphicx}
\usepackage{enumerate}

\usepackage{palatino}

 \usepackage[active]{srcltx}
\usepackage{scrtime}

\newcounter{xnumber}
\setcounter{xnumber}{0}
\newcounter{qnumber}


\newcommand{\exercice}{\setcounter{qnumber}{0}\textsc{\textbf{Exercice}} \textbf{\addtocounter{xnumber}{1}\thexnumber}\,\,}
\newcommand{\question}{\textbf{(\addtocounter{qnumber}{1}\theqnumber)}\,}
\setlength{\parindent}{0cm}


\begin{document}

\title{\textsc{Calcul Économique}\\(Éléments de correction)}
\date{Mercredi 18 décembre 2024}

\pagenumbering{gobble}

\maketitle

\exercice Soient $P$ et $Q$ deux propositions. Le connecteur de
Sheffer (nous ne l'avons pas vu en cours, il est très utilisé en
informatique où il est généralement appelé \verb+nand+) est noté et
défini comme~: $P \barwedge Q = \overline{P\land Q}$. \question Définissons le connecteur de Sheffer à l'aide d'une table de vérité~:
\begin{table}[H]
      \centering
      \begin{tabular}[H]{|cc|cc|}
        \hline
        $P$ & $Q$ & $P \land Q$ & $P \barwedge Q$\\ \hline
        V & V & V & F\\
        V & F & F & V\\
        F & V & F & V\\
        F & F & F & V \\
        \hline\hline
      \end{tabular}
    \end{table}
Le connecteur de Scheffer est la négation d'une conjonction. \question On utilise une table de vérité pour montrer que 
$(P\lor Q) \Leftrightarrow (P \barwedge P)\barwedge (Q\barwedge Q)$
\begin{table}[H]
      \centering
      \begin{tabular}[H]{|cc|c|ccc|}
        \hline
        $P$ & $Q$ & $P \lor Q$ & $P \barwedge P$ & $Q \barwedge Q$ & $(P \barwedge P)\barwedge (Q \barwedge Q)$\\ \hline
        V & V & V & F & F & V \\
        V & F & V & F & V & V \\
        F & V & V & V & F & V \\
        F & F & F & V & V & F \\
        \hline\hline
      \end{tabular}
    \end{table}
    La troisième colonne et la dernière ont les mêmes valeurs de
    vérité, on a donc bien montré l'équivalence. \question Pour montrer l'équivalence entre les propositions $P\Rightarrow Q$ et 
    $P\barwedge (Q\barwedge Q)$, on utilise à nouveau une table de vérité~:
\begin{table}[H]
      \centering
      \begin{tabular}[H]{|cc|c|cc|}
        \hline
        $P$ & $Q$ & $P \Rightarrow Q$ & $Q \barwedge Q$ & $P \barwedge (Q \barwedge Q)$\\ \hline
        V & V & V & F & V \\
        V & F & F & V & F \\
        F & V & V & F & V \\
        F & F & V & V & V \\
        \hline\hline
      \end{tabular}
    \end{table}
    La troisième colonne et la dernière ont les mêmes valeurs de
    vérité, on a donc bien montré l'équivalence.\newline

\bigskip

\exercice Les racines du polynôme suivant~:
\[
  P(X) = X^3 - 4X^2 + \frac{21}{4}X - \frac{5}{2}
\]
sont 2 (la racine évidente), $1+\frac{i}{2}$ et $1-\frac{i}{2}$ (les racines complexes conjuguées).\newline

\bigskip

\exercice On veut résoudre l'équation suivante~:
\[
4^x-2^{x+1}-3 = 0
\]
On peut réécrire cette équation sous la forme~:
\[
\left(2^{x}\right)^2-2\times 2^x - 3 = 0
\]
Posons $X=2^x$, on cherche $X>0$ (car $2^x$ doit être positif) tel que~:
\[
X^2-2X-3 = 0
\]
Cette équation admet deux solutions réelles distinctes $-1$ et $3$. Seule la seconde solution est pertinente (positive). On a donc~:
\[
2^x = 3
\]
en prenant le logarithme népérien, il vient~:
\[
x\log 2 = \log 3
\]
\[
\Leftrightarrow x = \frac{\log 3}{\log 2}
\]
L'unique solution de l'équation $4^x-2^{x+1}-3 = 0$.\newline

\bigskip

\exercice Soit la fonction à valeurs
réelles $f(x) = \frac{1}{1-x^2}$. \question Cette fonction est
discontinue en $x=-1$ et $x=1$ car en ces points le dénominateur est
nul. La fonction est continue sur $\mathbb R\setminus \{-1,1\}$.\question Par définition on a~:
\[
  \begin{split}
    f'(x) &= \lim_{h\to 0}\frac{f(x+h)-f(x)}{h}\\
          &= \lim_{h\to 0}\frac{\frac{1}{1-(x+h)^2}-\frac{1}{1-x^2}}{h}\\
          &= \lim_{h\to 0}\frac{\frac{1-x^2-1+(x+h)^2}{\left(1-(x+h)^2\right)\left(1-x^2\right)}}{h}\\
          &= \lim_{h\to 0}\frac{\frac{2xh+h^2}{\left(1-(x+h)^2\right)\left(1-x^2\right)}}{h}\\
          &= \lim_{h\to 0}\frac{2x+h}{\left(1-(x+h)^2\right)\left(1-x^2\right)}\\
          &= \frac{2x}{\left(1-x^2\right)^2}
  \end{split}
\]

\bigskip

\exercice Soit la fonction à valeurs
réelles $f(x) = \frac{2+x}{x^3+x^2-x+2}$. \question Cette fonction
n'est pas continue sur $\mathbb R$ car elle n'est pas définie
en $x=-2$ où le numérateur et le dénominateur sont nuls. \question
Calculons la limite de la fonction quand $x$ tend vers -2. Pour cela,
notons que l'on peut factoriser le polynôme au dénominateur sous la
forme $(2+x)\left(x^2-x+1\right)$, où le polynôme d'ordre deux n'admet
pas de racines réelles (autrement dit $x=-2$ est le seul point de
discontinuité). Pour le montrer vous pouvez utiliser la méthode des
coefficients indéterminés ou la division euclidienne. On a donc~:
\[
  \begin{split}
    \lim_{x\to -2}f(x) &= \lim_{x\to -2} \frac{2+x}{(2+x)(x^2-x+1)}\\
                       &= \lim_{x\to -2} \frac{1}{x^2-x+1}\\
                       &= \frac{1}{7}
  \end{split}
\]
On pose~:
\[
  g(x) =
  \begin{cases}
    f(x) &\text{ si }x\neq -2\\
    \frac{1}{7} &\text{ sinon}
  \end{cases}
\]
La fonction $g$ est continue sur $\mathbb R$, elle est identique à la
fonction $f$ partout sauf en un point.\newline

\bigskip

\exercice Soient $f: E\rightarrow F$ une fonction bijective dérivable deux fois. Pour calculer la dérivée première, on commence par noter que nous devons avoir~:
\[
f\left( f^{-1}(x) \right) = x
\]
En dérivant par rapport à $x$ et en exploitant la règle de dérivation en chaîne~:
\[
  f'\left(f^{-1}(x)\right)\frac{\mathrm d}{\mathrm dx}f^{-1}(x) = 1
\]
\[
  \Leftrightarrow \left(f^{-1}\right)'(x) = \frac{1}{f'\left(f^{-1}(x)\right)}
\]
En dérivant une seconde fois~:
\[
  \left(f^{-1}\right)''(x) = -\frac{f''\left(f^{-1}(x)\right)\frac{1}{f'\left(f^{-1}(x)\right)}}{\left(f'\left(f^{-1}(x)\right)\right)^2}
\]
\[
  \Leftrightarrow \left(f^{-1}\right)''(x) = -\frac{f''\left(f^{-1}(x)\right)}{\left(f'\left(f^{-1}(x)\right)\right)^3}
\]

La dérivée seconde de la fonction réciproque $f^{-1}(x)$ est définie tant que la dérivée première de $f$ est non nulle.\newline

\bigskip

\exercice Soit la fonction à valeurs réelles $f(x) = x^{-\log x}$. \question Cette fonction est définie pour les valeurs (strictement) positives de $x$. \question On a
\[
  \begin{split}
    e^{-\left( \log x \right)^2} &= \left(e^{\log x}\right)^{-\log x}\\
    &= x^{-\log x}
  \end{split}
\]
\question En utilisant la seconde écriture de la fonction $f$, on a~:
\[
  \begin{split}
    f'(x) &= -\left(\left(\log x\right)^2\right)'e^{-\left( \log x \right)^2}\\
          &=-\frac{2\log x}{x}x^{-\log x}\\
          &=-2e^{-\left( \log x \right)^2}\frac{\log x}{x}
  \end{split}
\]
\question On a directement, toujours en utilisant la seconde écriture de la fonction, les limites suivantes~:
\[
\lim_{x\to 0} f(x) = 0
\]
\[
\lim_{x\to \infty} f(x) = 0
\]
\question La dérivée est nulle si $x=1$, elle est positive si $x<1$ et négative si $x$ est plus grand. La fonction atteint donc un maximum global en $x=1$.

\end{document}

%%% Local Variables:
%%% mode: latex
%%% TeX-master: t
%%% End:
