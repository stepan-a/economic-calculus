\synctex=1
\documentclass[11pt,a4paper,notitlepage,twocolumn]{article}
\usepackage{amsmath}
\usepackage{amssymb}
\usepackage{amsbsy}
\usepackage{float}
\usepackage[french]{babel}
\usepackage{graphicx}
\usepackage{enumerate}

\usepackage{palatino}

 \usepackage[active]{srcltx}
\usepackage{scrtime}

\newcounter{xnumber}
\setcounter{xnumber}{0}
\newcounter{qnumber}


\newcommand{\exercice}{\setcounter{qnumber}{0}\textsc{\textbf{Exercice}} \textbf{\addtocounter{xnumber}{1}\thexnumber}\,\,}
\newcommand{\question}{\textbf{(\addtocounter{qnumber}{1}\theqnumber)}\,}
\setlength{\parindent}{0cm}


\begin{document}

\title{\textsc{Calcul Économique}}
\date{Mercredi 18 décembre 2024}

\pagenumbering{gobble}

\maketitle

\begin{quote}
  \textit{Les réponses non commentées ou insuffisamment détaillées ne seront pas considérées. Prenez le temps d'écrire des phrases.}
\end{quote}

\bigskip

\exercice Soient $P$ et $Q$ deux propositions. Le connecteur de
Sheffer (nous ne l'avons pas vu en cours, il est très utilisé en
informatique où il est généralement appelé \verb+nand+) est noté et
défini comme~: $P \barwedge Q = \overline{P\land Q}$. \question
Construire une table logique pour donner les valeurs de vérité de
$P \barwedge Q$ et définir en français (c'est-à-dire avec des mots) ce
connecteur. \question Montrer que
$(P\lor Q) \Leftrightarrow (P \barwedge P)\barwedge (Q\barwedge Q)$. \question Montrer que
$(P\Rightarrow Q) \Leftrightarrow P\barwedge (Q\barwedge Q)$.\newline

\bigskip

\exercice Calculer les racines du polynôme suivant~:
\[
  P(X) = X^3 - 4X^2 + \frac{21}{4}X - \frac{5}{2}
\]
% Les racines sont 2, 1+.5i et 1-.5i

\bigskip

\exercice Résoudre l'équation suivante~:
\[
4^x-2^{x+1}-3 = 0
\]
en montrant que cette équation n'admet qu'une seule solution.\newline

\bigskip

\exercice Soit la fonction à valeurs réelles $f(x) =
\frac{1}{1-x^2}$. \question Cette fonction est-elle continue sur $\mathbb R$~? Pourquoi~?
\question Calculer, lorsqu'elle existe, la dérivée de cette fonction,
$f'(x)$, en utilisant la définition de la dérivée.\newline

\bigskip

\exercice Soit la fonction à valeurs réelles $f(x) =
\frac{2+x}{x^3+x^2-x+2}$. \question Cette fonction est-elle continue sur $\mathbb R$~? Pourquoi~?
\question Construire une fonction $g(x)$ continue sur $\mathbb R$ la plus proche possible de la fonction $f(x)$.\newline

\bigskip

\exercice Soient $f: E\rightarrow F$ une fonction bijective dérivable deux fois. Calculer, quand elle existe, la dérivée seconde de la fonction réciproque $f^{-1}(x)$.\newline

\bigskip

\exercice Soit la fonction à valeurs réelles $f(x) = x^{-\log x}$. \question Quel est le domaine de définition de cette fonction~? \question Montrer qu'il est possible d'écrire la fonction sous la forme~: $f(x) = e^{-\left( \log x \right)^2}$. \question Calculer la dérivée de la fonction $f$ sur son domaine de définition. \question Déterminer les limites de la fonction $f$ sur les bords de son domaine de définition. \question Étudier le signe de $f'(x)$. \question La fonction $f$ admet-elle un maximum~? Pour quelle valeur de $x$~?

\end{document}

%%% Local Variables:
%%% mode: latex
%%% TeX-master: t
%%% End:
