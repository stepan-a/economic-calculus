\documentclass[10pt,a4paper]{article}
\usepackage{amsmath}
\usepackage{amssymb}
\usepackage{amsbsy}
\usepackage{float}
\usepackage[french]{babel}
\usepackage{graphicx}
\usepackage{enumerate}

\usepackage[utf8x]{inputenc}
\usepackage[T1]{fontenc}
\usepackage{palatino}

 \usepackage[active]{srcltx}
\usepackage{scrtime}

\newcommand{\exercice}[1]{\textsc{\textbf{Exercice}} #1}
\newcommand{\question}[1]{\textbf{(#1)}}
\setlength{\parindent}{0cm}

\begin{document}

\title{\textsc{Calcul Économique}}
\date{Le \today\ à \thistime}

\maketitle

\exercice{1} Soient $P$ et $Q$ deux propositions. Montrer qu'il est possible
d'exprimer la proposition $P\lor Q$ à l'aide de négations et du connecteur
logique $\land$.

\bigskip

\exercice{2} Montrer par récurrence que :
\[
\sum_{i=1}^n (2i-1) = n^2
\]
	
\bigskip

\exercice{3} Traduire avec des mots la proposition suivante :
\[
\forall \epsilon>0, \exists \delta(\epsilon)>0\text{ tel que } \forall x \in I, |x-a|<\delta(\epsilon)\Rightarrow |f(x)-f(a)|<\epsilon
\]
où $I$ est un interval réel, $f$ une fonction de $I$ dans $\mathbb R$. Que
pouvez-vous dire de la fonction $f$ si cette proposition est vraie ?

\bigskip

\exercice{4} Soit la fonction définie sur $\mathbb R$ : $f(x) = |x|$.
\begin{enumerate}
\item Cette fonction est-elle bijective ? Pourquoi ?
\item Cette fonction est-elle continue ? Pourquoi ?
\item Cette fonction est-elle dérivable ? Pourquoi ? 
\end{enumerate}
	
\bigskip

\exercice{5} Soit la fonction $f(x) = \frac{1}{1-x}$. Quel est son domaine de définition ? Calculer
$h(x)=\left(f\circ f\circ f \right)(x)$. Quel est le domaine de définiton de la
fonction $h$ ? Cette fonction est-elle continue ? Si la fonction admet des points
de discontinuité, ceux-ci sont-ils réparables ?

\bigskip

\exercice{6} \textbf{(1)} Donner la définition de la dérivée d'une fonction.
\textbf{(2)} Soit la fonction $f(x) = \log x$ (le logarithme népérien). En
utilisant la définition de la dérivée montrer que
$f'(x)=\frac{1}{x}$. On admet la propriété suivante de la fonction
  $\log$ : $\lim_{u\rightarrow 0} \frac{\log(1+u)}{u}=1$.

\pagebreak

\exercice{7} Calculer les racines du polynôme :
\[
P(X) = X^3 - \frac{1}{6}X^2 - \frac{4}{6}X - \frac{1}{6}
\]    

\bigskip

\exercice{8} Le taux de croissance observé d'une variable $X$ sur deux années
est 10\%. On vous demande de calculer le taux de croissance annuel moyen, vous
devez donc évaluer :
\[
g = (1,1^{\frac{1}{2}}-1)\times 100 = \left(\sqrt{1,1}-1\right)\times 100
\]
En supposant que vous ne ne connaissez pas la racine carrée de 1,1 par c\oe ur
(c'est mon cas) et que vous ne disposez pas d'une calculatrice (c'est votre cas
durant l'épreuve), proposez une approximation de $\sqrt{1,1}$ et donc du taux de
croissance annuel moyen.
 
\end{document}
%%% Local Variables:
%%% mode: latex
%%% TeX-master: t
%%% End:
