\synctex=1

\documentclass[10pt,notheorems]{beamer}

\usepackage{etex}
\usepackage{fourier-orns}
\usepackage{ccicons}
\usepackage{amssymb}
\usepackage{amstext}
\usepackage{amsbsy}
\usepackage{amsopn}
\usepackage{amscd}
\usepackage{amsxtra}
\usepackage{amsthm}
\usepackage{float}
\usepackage{color, colortbl}
\usepackage{mathrsfs}
\usepackage{bm}
\usepackage{lastpage}
\usepackage[nice]{nicefrac}
\usepackage{setspace}
\usepackage{ragged2e}
\usepackage{listings}
\usepackage{algorithms/algorithm}
\usepackage{algorithms/algorithmic}
\usepackage[frenchb]{babel}
\usepackage{tikz,pgfplots}
\pgfplotsset{compat=newest}
\usetikzlibrary{patterns, arrows, decorations.pathreplacing, decorations.markings, calc}
\pgfplotsset{plot coordinates/math parser=false}
\newlength\figureheight
\newlength\figurewidth
\usepackage{cancel}
\usepackage{tikz-qtree}
\usepackage{dcolumn}
\usepackage{adjustbox}
\usepackage{environ}
\usepackage[cal=boondox]{mathalfa}
\usepackage{manfnt}
\usepackage{hyperref}
\hypersetup{
  colorlinks=true,
  linkcolor=blue,
  filecolor=black,
  urlcolor=black,
}
\usepackage{venndiagram}
\usepackage{caption}
\usepackage{subcaption}


% Git hash
\usepackage{xstring}
\usepackage{catchfile}
\immediate\write18{git rev-parse HEAD > git.hash}
\CatchFileDef{\HEAD}{git.hash}{\endlinechar=-1}
\newcommand{\gitrevision}{\StrLeft{\HEAD}{7}}

\newcommand{\trace}{\mathrm{tr}}
\newcommand{\vect}{\mathrm{vec}}
\newcommand{\tracarg}[1]{\mathrm{tr}\left\{#1\right\}}
\newcommand{\vectarg}[1]{\mathrm{vec}\left(#1\right)}
\newcommand{\vecth}[1]{\mathrm{vech}\left(#1\right)}
\newcommand{\iid}[2]{\mathrm{iid}\left(#1,#2\right)}
\newcommand{\normal}[2]{\mathcal N\left(#1,#2\right)}
\newcommand{\dynare}{\href{http://www.dynare.org}{\color{blue}Dynare}}
\newcommand{\sample}{\mathcal Y_T}
\newcommand{\samplet}[1]{\mathcal Y_{#1}}
\newcommand{\slidetitle}[1]{\fancyhead[L]{\textsc{#1}}}

\newcommand{\R}{{\mathbb R}}
\newcommand{\C}{{\mathbb C}}
\newcommand{\N}{{\mathbb N}}
\newcommand{\Z}{{\mathbb Z}}
\newcommand{\binomial}[2]{\begin{pmatrix} #1 \\ #2 \end{pmatrix}}
\newcommand{\bigO}[1]{\mathcal O \left(#1\right)}
\newcommand{\red}{\color{red}}
\newcommand{\blue}{\color{blue}}

\renewcommand{\qedsymbol}{C.Q.F.D.}

\newcolumntype{d}{D{.}{.}{-1}}
\definecolor{gray}{gray}{0.9}
\newcolumntype{g}{>{\columncolor{gray}}c}

\setbeamertemplate{theorems}[numbered]

\theoremstyle{plain}
\newtheorem{theorem}{Théorème}

\theoremstyle{definition} % insert bellow all blocks you want in normal text
\newtheorem{definition}{Définition}
\newtheorem{properties}{Propriétés}
\newtheorem{lemma}{Lemme}
\newtheorem{property}[properties]{Propriété}
\newtheorem{example}{Exemple}
\newtheorem*{idea}{Éléments de preuve} % no numbered block



\setbeamertemplate{footline}{
  {\hfill\vspace*{1pt}\href{http://creativecommons.org/licenses/by-sa/3.0/legalcode}{\ccbysa}\hspace{.1cm}
    \raisebox{-.1cm}{\href{https://github.com/stepan-a/economic-calculus}{\includegraphics[scale=.015]{../../img/git.png}}}\enspace
    \href{https://github.com/stepan-a/economic-calculus/blob/\HEAD/td/1/correction.tex}{\gitrevision}\enspace\today
  }\hspace{1cm}}

\setbeamertemplate{navigation symbols}{}
\setbeamertemplate{blocks}[rounded][shadow=true]
\setbeamertemplate{caption}[numbered]

\NewEnviron{notes}{\justifying\tiny\begin{spacing}{1.0}\BODY\vfill\pagebreak\end{spacing}}

\newenvironment{exercise}[1]
{\bgroup \small\begin{block}{Ex. #1}}
  {\end{block}\egroup}

\newenvironment{defn}[1]
{\bgroup \small\begin{block}{Définition. #1}}
  {\end{block}\egroup}

\newenvironment{exemple}[1]
{\bgroup \small\begin{block}{Exemple. #1}}
  {\end{block}\egroup}


\begin{document}

\title{Calcul Économique\\\small{Éléments de correction du TD 1}}
\author[S. Adjemian]{Stéphane Adjemian}
\institute{\texttt{stephane.adjemian@univ-lemans.fr}} \date{Octobre 2021}

\begin{frame}
  \titlepage{}
\end{frame}


\begin{frame}
  \frametitle{Exercice 1}
  \fontsize{8}{10}\selectfont

  \begin{itemize}
    
  \item Montrons que $P\lor P \Leftrightarrow P$, c'est-à-dire que la disjonction est idempotente~:\newline
    \begin{table}[H]
      \centering
      \begin{tabular}[H]{|cc|c|}
        \hline
        $P$ & $P\lor P$ & $P\lor P \Leftrightarrow P$\\ \hline
        V & V & V \\
        F & F & V \\
        \hline\hline
      \end{tabular}
      \label{tab:or:idempotence}
    \end{table}

    \bigskip
    
  Si $P$ est vraie, alors $P\lor P$ est vraie quand $P$ est vraie et fausse
  quand $P$ est fausse. Ainsi la disjonction de $P$ avec lui même a toujours la
  même valeur de vérité que $P$ et les deux propositions sont donc équivalentes.\newline

  \item Montrons que $P\land P \Leftrightarrow P$, c'est-à-dire que la conjonction est idempotente~:\newline
    \begin{table}[H]
      \centering
      \begin{tabular}[H]{|cc|c|}
        \hline
        $P$ & $P\lor P$ & $P\land P \Leftrightarrow P$\\ \hline
        V & V & V \\
        F & F & V \\
        \hline\hline
      \end{tabular}
      \caption{Idempotence de la conjonction}
      \label{tab:and:idempotence}
    \end{table}

    \bigskip
    
  Si $P$ est vraie, alors $P\land P$ est vraie quand $P$ est vraie et fausse
  quand $P$ est fausse. Ainsi la conjonction de $P$ avec lui même a toujours la
  même valeur de vérité que $P$ et les deux propositions sont donc équivalentes.
    
  \end{itemize}
  
\end{frame}


\begin{frame}
  \frametitle{Exercice 2}
  \fontsize{8}{10}\selectfont

  \begin{itemize}
    
  \item Montrons que la conjonction est commutative (on se souvient qu'une conjonction est vraie si et seulement si les deux propositions sont vraies)~:
    \begin{table}[H]
      \centering
      \begin{tabular}[H]{|cc|cc|}
        \hline
        $P$ & $Q$ & $P \land Q$ & $Q \land P$\\ \hline
        V & V & V & V\\
        V & F & F & F\\
        F & V & F & F\\
        F & F & F & F \\
        \hline\hline
      \end{tabular}
    \end{table}
    On observe que les troisième et quatrième colonnes ont toujours
    la même valeur sur chaque ligne, les deux propositions associées
    $P \land Q$ et $Q \land P$ sont donc équivalentes.\qed\newline

    \item Montrons que la disjonction est commutative (on se souvient qu'une disjonction est vraie si et seulement si au moins une des deux propositions est vraie)~:
    \begin{table}[H]
      \centering
      \begin{tabular}[H]{|cc|cc|}
        \hline
        $P$ & $Q$ & $P \lor Q$ & $Q \lor P$\\ \hline
        V & V & V & V\\
        V & F & V & V\\
        F & V & V & V\\
        F & F & F & F \\
        \hline\hline
      \end{tabular}
    \end{table}
    On observe que les troisième et quatrième colonnes ont toujours
    la même valeur sur chaque ligne, les deux propositions associées
    $P \lor Q$ et $Q \lor P$ sont donc équivalentes.\qed\newline
    
  \end{itemize}
  
\end{frame}


\begin{frame}
  \frametitle{Exercice 2 (suite)}
  \fontsize{8}{10}\selectfont

  \begin{itemize}
    
  \item Montrons que la conjonction est associative, c'est-à-dire que $(P\land Q)\land R \Leftrightarrow P\land (Q\land R)$~:\newline
    \begin{table}[H]
      \centering
      \begin{tabular}[H]{|ccc|cgcg|}
        \hline
        $P$ & $Q$ & $R$ & $P\land Q$ & $(P\land Q)\land R$ & $Q\land R$ & $P \land (Q \land R)$ \\ \hline
        V & V & V & V & V & V & V\\
        V & V & F & V & F & F & F\\
        V & F & V & F & F & F & F\\
        V & F & F & F & F & F & F\\
        F & V & V & F & F & V & F\\
        F & V & F & F & F & F & F\\
        F & F & V & F & F & F & F\\
        F & F & F & F & F & F & F\\
        \hline\hline
      \end{tabular}
    \end{table}

    \bigskip
    
    On note que la cinquième et la septième colonnes sont identiques, d'où
    l'équivalence des propositions $(P\land Q)\land R$ et $P\land (Q\land R)$.\qed\newline
    
  \end{itemize}
  
\end{frame}


\begin{frame}
  \frametitle{Exercice 2 (suite)}
  \fontsize{8}{10}\selectfont

  \begin{itemize}
    
  \item Montrons que la disjonction est associative, c'est-à-dire que $(P\lor Q)\lor R \Leftrightarrow P\lor (Q\lor R)$~:\newline
    \begin{table}[H]
      \centering
      \begin{tabular}[H]{|ccc|cgcg|}
        \hline
        $P$ & $Q$ & $R$ & $P\lor Q$ & $(P\lor Q)\lor R$ & $Q\lor R$ & $P \lor (Q \lor R)$ \\ \hline
        V & V & V & V & V & V & V\\
        V & V & F & V & V & V & V\\
        V & F & V & V & V & V & V\\
        V & F & F & V & V & F & V\\
        F & V & V & V & V & V & V\\
        F & V & F & V & V & V & V\\
        F & F & V & F & V & V & V\\
        F & F & F & F & F & F & F\\
        \hline\hline
      \end{tabular}
    \end{table}

    \bigskip
    
    On note que la cinquième et la septième colonnes sont identiques, d'où
    l'équivalence des propositions $(P\lor Q)\lor R$ et $P\lor (Q\lor R)$.\qed\newline
    
  \end{itemize}
  
\end{frame}


\begin{frame}
  \frametitle{Exercice 2 (suite)}
  \fontsize{8}{10}\selectfont

  \begin{itemize}
    
  \item Montrons que la disjonction est distributive par rapport à la conjonction, c'est-à-dire que $(P\land Q)\lor R \Leftrightarrow (P\lor R) \land (Q\lor R)$~:\newline
  \begin{table}[H]
    \centering
    \begin{tabular}[H]{|ccc|cgccg|}
      \hline
      $P$ & $Q$ & $R$ & $P\land Q$ & $(P\land Q)\lor R$ & $P\lor R$ & $Q \lor R$ & $(P\lor R)\land (Q \lor R)$ \\ \hline
      V & V & V & V & V & V & V & V \\
      V & V & F & V & V & V & V & V \\
      V & F & V & F & V & V & V & V \\
      V & F & F & F & F & V & F & F \\
      F & V & V & F & V & V & V & V \\
      F & V & F & F & F & F & V & F \\
      F & F & V & F & V & V & V & V \\
      F & F & F & F & F & F & F & F \\
      \hline\hline
    \end{tabular}
  \end{table}

  \bigskip

  On note que la cinquième et huitième colonnes sont identiques, les propositions $(P\land Q)\lor R$ et $(P\lor R) \land (Q\lor R)$ sont donc équivalentes.\qed 
    
  \end{itemize}
  
\end{frame}


\begin{frame}
  \frametitle{Exercice 2 (suite)}
  \fontsize{8}{10}\selectfont

  \begin{itemize}
    
  \item Montrons que la conjonction est distributive par rapport à la disjonction, c'est-à-dire que $(P\lor Q)\land R \Leftrightarrow (P\land R) \lor (Q\land R)$~:\newline
  \begin{table}[H]
    \centering
    \begin{tabular}[H]{|ccc|cgccg|}
      \hline
      $P$ & $Q$ & $R$ & $P\lor Q$ & $(P\lor Q)\land R$ & $P\land R$ & $Q \land R$ & $(P\land R)\lor (Q \land R)$ \\ \hline
      V & V & V & V & V & V & V & V \\
      V & V & F & V & F & F & F & F \\
      V & F & V & V & V & V & F & V \\
      V & F & F & V & F & F & F & F \\
      F & V & V & V & V & F & V & V \\
      F & V & F & V & F & F & F & F \\
      F & F & V & F & F & F & F & F \\
      F & F & F & F & F & F & F & F \\
      \hline\hline
    \end{tabular}
  \end{table}

  \bigskip

  On note que la cinquième et huitième colonnes sont identiques, les propositions $(P\land Q)\lor R$ et $(P\lor R) \land (Q\lor R)$ sont donc équivalentes.\qed 
    
  \end{itemize}
  
\end{frame}


\begin{frame}
  \frametitle{Exercice 3}
  \fontsize{8}{10}\selectfont
    
  Montrons que l'implication logique est transitive, c'es-à-dire que~:
    \[
      ((P \Rightarrow Q) \land (Q \Rightarrow R)) \Rightarrow (P \Rightarrow R)
    \]
    \fontsize{4}{5}\selectfont
    \begin{table}[H]
    \begin{tabular}[H]{|ccc|ccccg|}
      \hline
      $P$ & $Q$ & $R$ & $P\Rightarrow Q$ & $Q\Rightarrow R$ & $(P\Rightarrow Q) \land (Q\Rightarrow R)$ & $P\Rightarrow R$ & $((P\Rightarrow Q) \land (Q\Rightarrow R))\Rightarrow (P\Rightarrow R)$\\ \hline
      V & V & V & V & V & V & V & V \\
      V & V & F & V & F & F & F & V \\
      V & F & V & F & V & F & V & V \\
      V & F & F & F & V & F & F & V \\
      F & V & V & V & V & V & V & V \\
      F & V & F & V & F & F & V & V \\
      F & F & V & V & V & V & V & V \\
      F & F & F & V & V & V & V & V \\
      \hline\hline
    \end{tabular}
  \end{table}
  \fontsize{8}{10}\selectfont
  \bigskip
  {\tiny \textdbend} Les colonnes 6 et 7 sont différentes, $(P \Rightarrow Q) \land (Q \Rightarrow R)$ et $P \Rightarrow R$ ne sont donc pas des propositions équivalentes.\newline
  
  $(P \Rightarrow Q) \land (Q \Rightarrow R)$ est une condition suffisante pour $P \Rightarrow R$, mais il ne s'agit pas d'une condition nécessaire.\newline

  On obtient la dernière colonne en utilisant la définiton de l'implication logique.
  
\end{frame}


\begin{frame}
  \frametitle{Exercice 4}
  \fontsize{8}{10}\selectfont
    
  Montrons que l'on peut exprimer l'équivalence sous la forme de la conjonction de deux implcations, c'est-à-dire que pour deux propositions $P$ et $Q$ on a~:
  \[
    (P \Leftrightarrow Q) \Leftrightarrow (P \Rightarrow Q) \land (Q
    \Rightarrow P)
  \]

  \begin{table}[H]
    \begin{tabular}[H]{|cc|gccg|}
      \hline
      $P$ & $Q$ & $P\Leftrightarrow Q$ & $P\Rightarrow Q$ & $Q \Rightarrow P$ & $(P\Rightarrow Q) \land (Q \Rightarrow P)$ \\ \hline
      V & V & V & V & V & V \\
      V & F & F & F & V & F \\
      F & V & F & V & F & F \\
      F & F & V & V & V & V \\
      \hline\hline
    \end{tabular}
  \end{table}

  \bigskip

  Puisque les colonnes 3 et 6 ont les mêmes valeurs de vérité sur
  chaque ligne les propositions $P\Leftrightarrow Q$ et
  $(P\Rightarrow Q) \land (Q \Rightarrow P)$ sont équivalentes.\qed
  
\end{frame}


\begin{frame}
  \frametitle{Exercice 5}
  \fontsize{8}{10}\selectfont

  \begin{itemize}

  \item Notons $P_n$ la proposition « $10^n$ est divisible par 9 » (avec $n\in\mathbb N$).\newline

  \item Montrons que la proposition $P_n \Rightarrow P_{n+1}$ est vraie.\newline

  \item Si $P_n$ alors il existe $k\in \mathbb N$ tel que $10^n+1 = 9k$.\newline

  \item Notons qu'il est possible d'écrire $P_{n+1}$ en fonction de $P_n$. En effet, on a~:
    \[
      \begin{split}
        10^{n+1}+1 &= 10\left(10^n+1\right)-10+1\\
        &= 10\left(10^n+1\right)-9
      \end{split}
    \]
  \item Si $P_n$ alors on sait que l'on peut trouver un entier $k$ tel que l'on puisse remplacer $10^n+1$ par $9k$, et donc~:
    \[
      10^{n+1}+1 = 9 (10k-1)
    \]
  
  \item $10^{n+1}+1$ est donc nécessairement divisible par 9.\newline

  \item[\textdbend] $P_n \Rightarrow P_{n+1}$ est une proposition vraie, pourtant $P_n$ et $P_{n+1}$ sont des propositions fausses (essayer avec $n=0$).
    
  \end{itemize}
  
\end{frame}


\begin{frame}
  \frametitle{Exercice 6}
  \framesubtitle{$\sum_{i=1}^n i$}
  \fontsize{8}{10}\selectfont

  \begin{itemize}

  \item Notons $P_n$ la proposition $\sum_{i=1}^n i = \frac{n(n+1)}{2}$.\newline

  \item $P_1$ est vraie, en effet on a bien $\frac{1 \times (1+1)}{2} = 1$\newline

  \item Supposons que $P_n$ est une proposition vraie et montrons que $P_{n+1}$ est vraie, c'est-à-dire  que l'on doit avoir~:
    \[
      \sum_{i=1}^{n+1} i = \frac{(n+1)(n+2)}{2}
    \]

  \item On exprime la somme jusqu'à $n+1$ en fonction de la somme jusqu'à $n$:
    \[
      \sum_{i=1}^{n+1} i = \sum_{i=1}^{n} i + (n+1)
    \]

  \item On utilise $P_n$ (supposée vraie)~:
    \[
        \sum_{i=1}^{n+1} i = \frac{n(n+1)}{2} + (n+1)
        = \frac{n(n+1)+2(n+1)}{2}
        = \frac{(n+1)(n+2)}{2}
    \]

  \item $P_{n+1}$ est donc nécessairement vraie si $P_{n}$ est vraie.\newline

  \item $P_n$ est vraie pour tout $n\in\mathbb N$.
    
  \end{itemize}
  
\end{frame}


\begin{frame}
  \frametitle{Exercice 6}
  \framesubtitle{$\sum_{i=1}^n i^2$}
  \fontsize{8}{10}\selectfont

  \begin{itemize}

  \item Notons $P_n$ la proposition $\sum_{i=1}^n i^2 = \frac{n(n+1)(2n+1)}{6}$.\newline

  \item $P_1$ est vraie, en effet on a bien $\frac{1 \times (1+1) \times (2\times 1 + 1)}{6} = 1$\newline

  \item Supposons que $P_n$ est une proposition vraie et montrons que $P_{n+1}$ est vraie, c'est-à-dire  que l'on doit avoir~:
    \[
      \sum_{i=1}^{n+1} i = \frac{(n+1)(n+2)(2n+3)}{6}
    \]

  \item On exprime la somme jusqu'à $n+1$ en fonction de la somme jusqu'à $n$:
    \[
      \sum_{i=1}^{n+1} i^2 = \sum_{i=1}^{n} i^2 + (n+1)^2
    \]

  \item On utilise $P_n$ (supposée vraie)~:
    \[
      \begin{split}
        \sum_{i=1}^{n+1} i^2 &= \frac{n(n+1)(2n+1)}{6} + (n+1)^2\\
        &= \frac{(n+1)\left[n(2n+1)+6(n+1)\right]}{6}
      \end{split}
    \]
    
  \end{itemize}
  
\end{frame}


\begin{frame}
  \frametitle{Exercice 6}
  \framesubtitle{$\sum_{i=1}^n i^2$ (suite)}
  \fontsize{8}{10}\selectfont

  \begin{itemize}

  \item On fait apparaître $(n+2)$ dans le dernier facteur~:
    \[
      \begin{split}
        \sum_{i=1}^{n+1} i^2 &= \frac{(n+1)\left[(n+2)(2n+1){\red -2(2n+1)}+6(n+2){\red-6}\right]}{6}\\
        &= \frac{(n+1)\left[(n+2)(2n+7){\red -4n-8}\right]}{6}\\
        &= \frac{(n+1)\left[(n+2)(2n+7)-4(n+2)\right]}{6}\\
        &= \frac{(n+1)(n+2)(2n+3)}{6}\\
      \end{split}
    \]

  \item $P_{n+1}$ est donc nécessairement vraie si $P_{n}$ est vraie.\newline

  \item $P_n$ est vraie pour tout $n\in\mathbb N$.
    
  \end{itemize}
  
\end{frame}


\begin{frame}
  \frametitle{Exercice 6}
  \framesubtitle{$\sum_{i=1}^n x^{i-1}$}
  \fontsize{8}{10}\selectfont

  \begin{itemize}

  \item Notons $P_n$ la proposition $\sum_{i=1}^n x^{i-1} = \frac{1-x^n}{1-x}$ pour $x\neq 0$.\newline

  \item $P_1$ est vraie, en effet on a bien $\frac{1-x}{1-x} = 1^0$\newline

  \item Supposons que $P_n$ est une proposition vraie et montrons que $P_{n+1}$ est vraie, c'est-à-dire  que l'on doit avoir~:
    \[
      \sum_{i=1}^{n+1} x^{i-1} = \frac{1-x^{n+1}}{1-x}
    \]

  \item On exprime la somme jusqu'à $n+1$ en fonction de la somme jusqu'à $n$:
    \[
      \sum_{i=1}^{n+1} x^{i-1} = \sum_{i=1}^{n} x^{i-1} + x^n
    \]

  \item On utilise $P_n$ (supposée vraie)~:
    \[
        \sum_{i=1}^{n+1} x^{i-1} = \frac{1-x^n}{1-x} + x^n = \frac{1-x^n+(1-x)x^n}{1-x} = \frac{1-x^{n+1}}{1-x}
    \]

  \item $P_{n+1}$ est donc nécessairement vraie si $P_{n}$ est vraie.\newline

  \item $P_n$ est vraie pour tout $n\in\mathbb N$.
    
  \end{itemize}
  
\end{frame}


\begin{frame}
  \frametitle{Exercice 7}
  \fontsize{8}{10}\selectfont

  \begin{itemize}

  \item $A\cap B$ est l'ensemble des multiples de 2 et 3~:
    \[
      A = \left\{2, 4, {\red 6}, 8, 10, {\red 12}, 14, 16, {\red 18}, 20, \ldots\right\}
    \]
    \[
      B = \left\{3, {\red 6}, 9, {\red 12}, 15, {\red 18}, 21, \ldots\right\}
    \]
    \[
      \begin{split}
        A\cap B &= \left\{6, 12, 18, \ldots\right\}\\
        &= \left\{x\in\mathbb N | x\text{ est un multiple de 6}\right\} = C
      \end{split}
    \]

  \item $A \cap C = C$ car $C \subset A$.\newline

  \item $A \cup C = A$ car $C \subset A$.\newline

  \item $B \cup C = B$ car $C \subset B$.\newline

  \item $C \cap D$  est l'ensemble des multiples de 6 et 8 :
    \[
      C = \left\{6, 12, 18, {\red 24}, 30, 36, 42, {\red 48}, 54, 60, 66, {\red 72}, \ldots\right\}
    \]
    \[
      D = \left\{8, 16, {\red 24}, 32, 40, {\red 48}, 56, 64, {\red 72}, 80, 88, {\red 96}, \ldots\right\}
    \]
    \[
      C\cap D = \left\{24, 48, 72, \ldots\right\}
    \]
     
  \end{itemize}
  
\end{frame}


\begin{frame}
  \frametitle{Exercice 8}
  \framesubtitle{$\overline{A \cap B} = \overline{A} \cup \overline{B}$}
  \fontsize{8}{10}\selectfont

  \begin{itemize}
  \item L'ensemble $A \cap B$ correspond à la surface dans la lentille (en blanc dans la figure \ref{fig:8:1a}), $\overline{A \cap B}$ correspond à tout ce qui n'est pas dans la lentille (en gris dans la figure \ref{fig:8:1a}).
  \item L'ensemble $\overline{A} \cup \overline{B}$ est l'union des surfaces grisées dans les figures \ref{fig:8:1b} et \ref{fig:8:1c}, qui est identique à la surface grisée dans la figure \ref{fig:8:1a}.
  \end{itemize}

  \bigskip

  \begin{figure}
     \centering
     \begin{subfigure}[b]{0.3\textwidth}
       \centering
       \begin{venndiagram2sets}[tikzoptions={scale=.5}]
         \fillNotAorNotB
       \end{venndiagram2sets}
       \caption{$\overline{A \cap B}$}
       \label{fig:8:1a}
     \end{subfigure}
     \hfill
     \begin{subfigure}[b]{0.3\textwidth}
       \centering
       \begin{venndiagram2sets}[tikzoptions={scale=.5}]
         \fillNotA
       \end{venndiagram2sets}
       \caption{$\overline{A}$}
       \label{fig:8:1b}
     \end{subfigure}
     \hfill
     \begin{subfigure}[b]{0.3\textwidth}
       \centering
       \begin{venndiagram2sets}[tikzoptions={scale=.5}]
         \fillNotB
       \end{venndiagram2sets}
       \caption{$\overline{B}$}
       \label{fig:8:1c}
     \end{subfigure}
     \caption{Diagramme de Venn et loi de Morgan}
     \label{fig:8:1}
\end{figure}

\end{frame}


\begin{frame}
  \frametitle{Exercice 8 (suite)}
  \framesubtitle{$\overline{A \cup B} = \overline{A} \cap \overline{B}$}
  \fontsize{8}{10}\selectfont

  \begin{itemize}
  \item L'ensemble $A \cup B$ correspond à la surface dans la « double patate » (en blanc dans la figure \ref{fig:8:2a}), $\overline{A \cup B}$ correspond à tout ce qui n'est pas dans la « double patate » (en gris dans la figure \ref{fig:8:2a}).
  \item L'ensemble $\overline{A} \cap \overline{B}$ est l'intersection des surfaces grisées dans les figures \ref{fig:8:2b} et \ref{fig:8:2c}, qui est identique à la surface grisée dans la figure \ref{fig:8:2a}.
  \end{itemize}

  \bigskip

  \begin{figure}
     \centering
     \begin{subfigure}[b]{0.3\textwidth}
       \centering
       \begin{venndiagram2sets}[tikzoptions={scale=.5}]
         \fillNotAorB
       \end{venndiagram2sets}
       \caption{$\overline{A \cup B}$}
       \label{fig:8:2a}
     \end{subfigure}
     \hfill
     \begin{subfigure}[b]{0.3\textwidth}
       \centering
       \begin{venndiagram2sets}[tikzoptions={scale=.5}]
         \fillNotA
       \end{venndiagram2sets}
       \caption{$\overline{A}$}
       \label{fig:8:2b}
     \end{subfigure}
     \hfill
     \begin{subfigure}[b]{0.3\textwidth}
       \centering
       \begin{venndiagram2sets}[tikzoptions={scale=.5}]
         \fillNotB
       \end{venndiagram2sets}
       \caption{$\overline{B}$}
       \label{fig:8:2c}
     \end{subfigure}
     \caption{Diagramme de Venn et loi de Morgan}
     \label{fig:8:2}
\end{figure}

\end{frame}


\begin{frame}
  \frametitle{Exercice 9}
  \fontsize{8}{10}\selectfont

  \begin{itemize}

  \item $A\times (B\cup C) = \{a,b\}\times\{1,3,4,5\}$
    \[
      \begin{split}
        A\times (B\cup C) = \bigl\{ &(a, 1), (a,3), (a, 4), (a, 5)\\
        &(b, 1), (b,3), (b, 4), (b, 5)\bigr\}
      \end{split}
    \]

  \item On a~:
    \[
      A\times B = \bigl\{ (a, 1), (a,3), (b, 1), (b,3)\bigr\}
    \]
    et
    \[
      A\times C = \bigl\{ (a, 4), (a,5), (b, 4), (b,5)\bigr\}
    \]
    puis~:
    \[
      \begin{split}
        (A\times B)\cup(A\times C) &= \bigl\{(a, 1), (a,3), (a, 4), (a, 5), (b, 1), (b,3), (b, 4), (b, 5)\bigr\}\\
        &= A\times (B\cup C)
      \end{split}
    \]
    $\rightarrow$ Distributivité du produit cartésien par rapport à l'union.  
  \end{itemize}
  
\end{frame}


\begin{frame}
  \frametitle{Exercice 9 (suite)}
  \fontsize{8}{10}\selectfont

  \begin{itemize}

  \item Notons que $B\cap C = \emptyset$ est un ensemble vide, le produit cartésien d'un ensemble avec l'ensemble vide est un ensemble vide, $A\times (B\cap C) = \emptyset$.\newline

  \item Nous avons déjà calculé $A\times B$ et $A\times C$, qui n'ont aucun élément commun. On a donc
    $(A\times B)\cap(A\times C) = \emptyset$.
  \end{itemize}
  
\end{frame}


\begin{frame}
  \frametitle{Exercice 10}
  \fontsize{8}{10}\selectfont


  On utilise~:
    \[
      \textrm{Card}(A \cup B) = \textrm{Card}(A) + \textrm{Card}(B) - \textrm{Card}(A\cap B) 
    \]

    En remplaçant les valeurs fournies dans l'ennoncé, on obtient~:
    \[
      \textrm{Card}(A \cup B) = 20+15-6 = 29
    \]
    
  \end{frame}

\begin{frame}
  \frametitle{Exercice 11}
  \fontsize{8}{10}\selectfont

  \begin{itemize}
  \item On commence par traduire l'énoncé. On note $\mathcal O$ l'ensemble des obèses, $\mathcal H$ l'ensemble des salariés avec de l'hypertension et $\mathcal C$ l'ensemble des salariés avec du cholestérol.\newline
  \item On sait que~:
    \begin{itemize}
    \item $\mathrm{Card}(\mathcal O) = 30$, $\mathrm{Card}(\mathcal H) = 25$, $\mathrm{Card}(\mathcal C) = 20$.
    \item $\mathrm{Card}(\mathcal H \cap \mathcal C) = 12$, $\mathrm{Card}(\mathcal H \cap \mathcal O) = 15$, $\mathrm{Card}(\mathcal C \cap \mathcal O) = 10$
    \item $\mathrm{Card}(\mathcal H \cap \mathcal C \cap \mathcal O) = 5$\newline
    \end{itemize}
  \item On peut déduire~:
    \begin{itemize}
    \item $\mathrm{Card}((\mathcal O \cap \mathcal H)\setminus\mathcal C) = 15-5 = 10$, $\mathrm{Card}((\mathcal O \cap \mathcal C)\setminus\mathcal H) = 10-5 = 5$, $\mathrm{Card}(\mathcal O \setminus\mathcal C \setminus\mathcal H) = 30-10-5-5 = 10$
    \item $\mathrm{Card}((\mathcal H \cap \mathcal C)\setminus\mathcal O) = 12-5 = 7$, $\mathrm{Card}(\mathcal H \setminus\mathcal C \setminus \mathcal O) = 25-10-5-7 = 3$
    \item $\mathrm{Card}(\mathcal C \setminus\mathcal H \setminus \mathcal O) = 20-5-5-7 = 3$\newline
    \end{itemize}
  \item On cherche à déterminer la valeur de $\mathrm{Card}(\overline{\mathcal H \cup \mathcal C \cup \mathcal O})$, c'est-à-dire le nombre de salariés sans aucune de ces pathologies.\newline
  \item Il faut compter le nombre d'éléments dans $\mathcal H \cup \mathcal C \cup \mathcal O$ et retrancher ce total au nombre de salariés dans l'entreprise (50).\newline
  \end{itemize}

\end{frame}

\begin{frame}
  \frametitle{Exercice 11 (suite)}
  \fontsize{8}{10}\selectfont

  \begin{figure}[H]
    \centering
    \begin{venndiagram3sets}[tikzoptions={scale=1}, labelA=$\mathcal O$, labelB=$\mathcal H$, labelC=$\mathcal C$,
      labelABC={5},
      labelOnlyA={10},   % 30-10-5-5
      labelOnlyAC={5},   % 10-5
      labelOnlyAB={10},  % 15-5
      labelOnlyB={3},    % 25-10-5-7
      labelOnlyBC={7},   % 12-5
      labelOnlyC={3},    % 20-5-5-7
      labelNotABC={7}]   % 50-10-10-5-5-3-7-3
       \end{venndiagram3sets}
       \caption{Pathologies des salariés.}
       \label{fig:11}
  \end{figure}

  \bigskip

  Seulement $7$ ($50-10-10-5-5-3-7-3$) salariés ne souffrent d'aucune des trois pathologies.
\end{frame}




\begin{frame}
  \frametitle{Exercice 13}
  \fontsize{8}{10}\selectfont


  \begin{itemize}
  \item L'ensemble (relation binaire) $\mathcal E_1$ n'est pas une fonction car l'élément 2 possède deux images distinctes (8 et 3).\newline
  \item L'ensemble $\mathcal E_2$ n'est pas un ensemble car chaque élément dans l'ensemble de départ possède une infinité d'images.\newline
  \item L'ensemble $\mathcal E_3$ est une fonction (non bijective).\newline
  \item L'ensemble $\mathcal E_4$ est une fonction (non bijective).
    
  \end{itemize}
  
\end{frame}


\begin{frame}
  \frametitle{Exercice 14}
  \fontsize{8}{10}\selectfont


  Nous avons~:
  \[
    \begin{split}
      f(x+h) &= (x+h)^2 + 2(x+h) + 4\\
      &= x^2+2hx+h^2+2x+2h+4\\
      &=x^2+2x+4+2hx+2h
    \end{split}
  \]
  et donc~:
  \[
    f(x+h)-f(x) = 2hx + 2h 
  \]
  puis~:
  \[
    \frac{f(x+h)-f(x)}{h} = 2x + 2
  \]

  \bigskip
  
  Ce ratio représente la pente de la corde de la fonction entre les points $(x, f(x))$ et $(x+h, f(x+h))$. Plus tard on fera tendre $h$ vers zéro et la limite du ratio correspondra à la pente de la tangente de la fonction au point $x$ (la dérivée).
    
\end{frame}


\begin{frame}
  \frametitle{Exercice 15}
  \fontsize{8}{10}\selectfont

  Pour qu'une fonction $f$ soit injective il faut et il suffit que deux éléments distincts dans l'ensemble de départ aient des images distinctes par $f$.\newline

  Ce n'est clairement pas le cas de cette fonction, à cause du terme en $x^2$ (parabole). Par exemple $x=1$ et $x=-2$ ont la même image par $f$ (0). Ces deux valeurs de $x$ sont les racines (évidentes) du polynôme d'ordre 2. 

\end{frame}


\begin{frame}
  \frametitle{Exercice 16}
  \fontsize{8}{10}\selectfont

  \begin{enumerate}
  \item[i] On a~:
    \[
      h(x) = g(f(x)) = 2 ( {\red x + 2} ) + 5 = 2x+9
    \]
    et
    \[
      m(x) = f(g(x)) = ({\red 2x+5}) + 2 = 2x+ 7 
    \]
    on note que ces deux fonctions sont différentes, la composition de fonctions n'est généralement pas commutative.\newline
    
  \item[ii] Les fonctions réciproques sont~:
    \[
      f^{-1}(x) = x - 2 \quad\text{ et }\quad g^{-1}(x) = \frac{1}{2}x - \frac{5}{2}
    \]

  \item[iii] On a~:
    \[
      h^{-1}(x) = \frac{x}{2}-\frac{9}{2} \quad\text{ et }\quad m^{-1}(x) = \frac{x}{2}-\frac{7}{2}
    \]

  \item[iv] On a~:
    \[
      f^{-1}\left(g^{-1}(x)\right) = \frac{x}{2}-\frac{9}{2}  \quad\text{ et }\quad g^{-1}\left(f^{-1}(x)\right) = \frac{x}{2}-\frac{7}{2}
    \]
  \end{enumerate}

\end{frame}


\begin{frame}
  \frametitle{Exercice 17}
  \fontsize{8}{10}\selectfont

  \begin{enumerate}
  \item $\exists x\in\mathbb R$ tel que $f(x)\neq 0$,\newline
  \item $\forall x\in\mathbb R, f(x)\neq 0$,\newline
  \item $\exists x\in\mathbb R$ tel que $f(x)\neq x$,\newline
  \item $\forall (x,y)\in\mathbb R^2$ tel que $x<y$, $f(x)<f(y)$, et\newline
  \item $\exists (x,y)\in\mathbb R^2$ tel que $x<y$ et $f(x)\geq f(y)$.
  \end{enumerate}
  
\end{frame}




\end{document}

%%% Local Variables:
%%% mode: latex
%%% TeX-master: t
%%% ispell-check-comments: exclusive
%%% ispell-local-dictionary: "francais"
%%% TeX-master: t
%%% TeX-master: t
%%% TeX-master: t
%%% End:
