\documentclass[10pt,a4paper,notitlepage]{article}
\usepackage{amsmath}
\usepackage{amssymb}
\usepackage{amsbsy}
\usepackage{float}
\usepackage[french]{babel}
\usepackage{graphicx}
\usepackage{enumerate}

\usepackage[utf8x]{inputenc}
\usepackage[T1]{fontenc}
\usepackage{palatino}

 \usepackage[active]{srcltx}
\usepackage{scrtime}

\newcommand{\exercice}[1]{\textsc{\textbf{Exercice}} #1}
\newcommand{\question}[1]{\textbf{(#1)}}
\setlength{\parindent}{0cm}

\begin{document}

\title{\textsc{Calcul Économique\\ \small{(Fiche de TD n°1)}}}
\author{Stéphane Adjemian\thanks{Université du Maine, Gains. \texttt{stephane DOT adjemian AT univ DASH lemans DOT fr}}}
\date{Le \today\ à \thistime}

\maketitle

\exercice{1} Soit une proposition $P$. Montrer, à l'aide d'un tableau
de vérité, que $P \land P \Leftrightarrow P$ et $P \lor P
\Leftrightarrow P$.

\bigskip

\exercice{2} Soient $P$, $Q$ et $R$ trois propositions. Montrer, à l'aide d'un tableau
de vérité, que :
\begin{enumerate}[(i)]
\item $P \land Q \Leftrightarrow Q \land P$
\item $P \lor Q \Leftrightarrow Q \lor P$
\item $(P \land Q) \land R \Leftrightarrow P \land (Q \land R)$
\item $(P \lor Q) \lor R \Leftrightarrow P \lor (Q \lor R)$
\item $(P \land Q) \lor R \Leftrightarrow (P \lor R) \land (Q \lor R)$
\item $(P \lor Q) \land R \Leftrightarrow (P \land R) \lor (Q \land R)$
\end{enumerate}

\bigskip

\exercice{3} Montrer la transitivité de l'implication logique, c'est-à-dire
que :
\[
((P \Rightarrow Q) \land (Q \Rightarrow R)) \Rightarrow (P \Rightarrow R)
\]
avec $P$, $Q$ et $R$ trois propositions.

\bigskip

\exercice{4} Exprimer l'équivalence logique en termes d'implication logique, en
établissant que :
\[
(P \Leftrightarrow Q) \Leftrightarrow (P \Rightarrow Q) \land (Q
\Rightarrow P)
\]
avec $P$ et $Q$ deux propositions.

\bigskip

\exercice{5} Montrer que l'implication logique suivante :
\[
(10^n+1 \text{ est divisible par } 9) \Rightarrow (10^{n+1}+1 \text{
  est divisible par 9})
\]
est vraie, avec $n\in\mathbb N$. Que pensez vous de ces propositions ?

\bigskip

\exercice{6} Montrer les propositions suivantes :
\begin{enumerate}[(i)]
\item $\sum_{i=1}^n i = \frac{n(n+1)}{2}$.
\item $\sum_{i=1}^n i^2 = \frac{n(n+1)(2n+1)}{6}$.
\item $\sum_{i=1}^nx^{i-1} = \frac{1-x^{n}}{1-x}$, avec $x$ un réel différent de 1.
\end{enumerate}

\bigskip

\exercice{7} Soient les ensembles :
\[
A = \{x \in \mathbb N | x \text{ est un multiple de 2}\}
\]
\[
B = \{x \in \mathbb N | x \text{ est un multiple de 3}\}
\]
\[
C = \{x \in \mathbb N | x \text{ est un multiple de 6}\}
\]
\[
D = \{x \in \mathbb N | x \text{ est un multiple de 8}\}
\]
Déterminer les ensembles $A \cap B$, $A \cap C$, $A \cup C$, $B\cup
C$, $C\cap D$.

\bigskip

\exercice{8} Soient $A$ et $B$ deux sous ensembles de
$\Omega$. Illustrer avec des diagrammes de Venn les deux règles de
Morgan :
\[
\overline{A \cap B} = \overline{A} \cup \overline{B}
\]
\[
\overline{A \cup B} = \overline{A} \cap \overline{B}
\]

\bigskip

\exercice{9} Soient les ensembles $A = \{a, b\}$, $B = \{1, 3\}$ et $C
= \{4, 5\}$. Déterminer les ensembles suivants :
\begin{enumerate}[(i)]
\item $A \times (B\cup C)$
\item $(A\times B) \cup (A\times C)$
\item $A \times (B \cap C)$
\item $(A\times B) \cap (A\times C)$
\end{enumerate}

\bigskip

\exercice{10} Soit $E$ un ensemble tel que $\mathrm{Card}(E)=30$. Si
$A$ et $B$ sont deux sous ensembles de $E$ non disjoints (\emph{ie}
$A\cap B \neq \emptyset$) tels que $\mathrm{Card}(A)=20$,
$\mathrm{Card}(B)=15$ et $\mathrm{Card}(A\cap B)=6$. Déterminer
$\mathrm{Card}(A\cup B)$.

\bigskip

\exercice{11} Les résultats d'une entreprise ont montré que sur 50
employés, 30 sont obèses, 25 souffrent d'hypertension artérielle
tandis que 20 ont un taux de cholestérol trop élevé. Parmi les 25 qui
souffrent d'hypertension, 12 ont un taux de cholestérol trop élevé; 15
obèses souffrent d'hypertension et 10 obèses souffrent d'un taux de
cholestérol trop élevé; finalement, 5 employés souffrent de ces
trois maux à la fois.  Déterminer le nombre d'employés bien portant à
l'aide d'un diagramme de Venn.

\bigskip

\exercice{12}  Sur 100 étudiants, on considère les ensembles $S$ de ceux qui étudient
la sociologie, $E$ de ceux qui étudient l'économie et $G$ de ceux qui étudient la gestion. Sur ces 100 étudiants, 55 étudient la sociologie, 9 la
sociologie et la gestion, 7 la sociologie et l'économie, 8 l'économie et la
gestion, 6 la sociologie et la gestion mais pas l'économie, 80 la sociologie
ou la gestion et 12 l'économie seulement.
\begin{enumerate}[(i)]
\item Combien d'étudiants suivent les trois matières ?
\item Combien sont-ils a étudier la gestion ?
\item Combien sont-ils a étudier l'économie ?
\item Combien n'étudient aucune de ces trois matières ?
\end{enumerate}

\bigskip

\exercice{13} Soient les ensembles :
\[
\mathcal E_1 = \{(1;2), (2;8), (2;3)\}
\]
\[
\mathcal E_2 = \{ (x;y) | x \in \mathbb R \land x\leq y \}
\]
\[
\mathcal E_3 = \{ (x;y) | x \in \mathbb R \land y = x^2 \}
\]
\[
\mathcal E_4 = \{ (x;y) | y = x^2 \text{ si } 0\leq x\leq 2,\quad
y=3-x  \text{ si } 2<x<3, \quad y=3 \text{ si } x=3\}
\]
Déterminez quels ensembles représentent une fonction.

\bigskip

\exercice{14} Soit la fonction :
\[
\begin{split}
  f: &\mathbb R \rightarrow \mathbb R\\
  &x \mapsto f(x) = x^2 + 2x + 4
\end{split}
\]
Calculer:
\[
\frac{f(x+h)-f(x)}{h}
\]
Interpréter cette expression.

\bigskip

\exercice{15} La fonction suivante est-elle injective ?
\[
\begin{split}
  f: &\mathbb R \rightarrow \mathbb R\\
  &x \mapsto f(x) = x^2 + x - 2
\end{split}
\]

\bigskip

\exercice{16} Soient les fonctions $f(x) = x+2$ et $g(x) = 2x + 5$.
\begin{enumerate}[(i)]
\item Calculer $h(x) = (g \circ f)(x) = g\left(f(x)\right)$ et $m(x) = (f \circ g)(x) = f\left(g(x)\right)$.
\item Calculer $f^{-1}(x)$ et $g^{-1}(x)$.
\item Calculer $h^{-1}(x)$ et $m^{-1}(x)$.
\item Calculer $\left(f^{-1} \circ g^{-1}\right)(x)$ et $\left(g^{-1} \circ f^{-1}\right)(x)$
\end{enumerate}
Comparer les résultats des deux dernières questions.

\bigskip

\exercice{17} Exprimer à l'aide de quantificateurs les propositions
suivantes :
\begin{enumerate}
\item La fonction $f: \mathbb R \rightarrow \mathbb R$ n'est pas
  nulle.
\item La fonction $f: \mathbb R \rightarrow \mathbb R$ ne s'annule pas
sur $\mathbb R$.
\item La fonction $f: \mathbb R \rightarrow \mathbb R$ n'est pas
  l'identité de $\mathbb R$.
\item La fonction $f: \mathbb R \rightarrow \mathbb R$ est croissante
  sur $\mathbb R$.
\item La fonction $f: \mathbb R \rightarrow \mathbb R$ n'est pas croissante
  sur $\mathbb R$.
\end{enumerate}


\end{document}
