\documentclass[10pt,a4paper,notitlepage]{article}
\usepackage{amsmath}
\usepackage{amssymb}
\usepackage{amsbsy}
\usepackage{float}
\usepackage[french]{babel}
\usepackage{graphicx}
\usepackage{enumerate}

\usepackage{palatino}

 \usepackage[active]{srcltx}
\usepackage{scrtime}

\newcounter{xnumber}
\setcounter{xnumber}{0}

\newcommand{\exercice}{\textsc{\textbf{Exercice}} \textbf{\addtocounter{xnumber}{1}\thexnumber}\,\,}
\newcommand{\question}[1]{\textbf{(#1)}}
\setlength{\parindent}{0cm}



\begin{document}

\title{\textsc{Calcul Économique\\ \small{(Fiche de TD n°2)}}}
\author{Stéphane Adjemian\thanks{Université du Mans. \texttt{stephane DOT adjemian AT univ DASH lemans DOT fr}}}
\date{Le \today\ à \thistime}

\maketitle

\exercice La demande de montres SLOUK est de 10 unités si le prix est
égal à 160 euros et elle est de 20 unités si le prix est 120
euros. Calculer la fonction de demande supposée linéaire.

\bigskip

\exercice Quand le prix est de 100 euros la quantité d'appareils
photos de marque PISTOL offerte sur le marché est 50 unités. Quand le
prix est 50\% plus élevé le nombre d'unités offertes est de
100. Calculer la fonction d'offre supposée linéaire.

\bigskip

\exercice Sur un marché, la demande et l'offre pour un bien sont
caractérisés par :
\[
\begin{split}
  D(p): q &= -2 p  + 6\\
  S(p): q &= \frac{1}{2} p  + 1
\end{split}
\]
où $p$ est le prix du bien et $q$ sa quantité. Calculer la quantité
d'équilibre et le prix d'équilibre.

\bigskip

\exercice Supposons que la consommation agrégée dans une économie,
notée $C$, soit une fonction linéaire du revenu disponible (hors
taxes), noté $Y$. Supposons qu'il existe un niveau de consommation
incompressible, noté $C_0$. Il s'agit du niveau de consommation
observé même si le revenu disponible est nul. On supposera que lorsque
le revenu augmente de $x$, la consommation en écart à son niveau
incompressible, \emph{ie} $C-C_0$, augmente de $0,8x$. Déterminer la
forme de la fonction de consommation.

\bigskip

\exercice Montrer que la fonction définie sur $\mathbb R$ $f(x) = x^2+2x+1$ admet un unique
minimum en $x=-1$.

\bigskip

\exercice Sur un marché, la demande et l'offre pour un bien sont
caractérisés par :
\[
\begin{split}
  D(p): q &= -2 p^2  + 3\\
  S(p): q &= p^2 + 5p +2
\end{split}
\]
où $p$ est le prix du bien et $q$ sa quantité (on s'intéresse aux
valeurs positives de $p$ et $q$). Calculer la quantité
d'équilibre et le prix d'équilibre.

\bigskip

\exercice Montrer qu'il existe un unique polynôme passant par les points $(0,2)$, $(-2,16)$ et $(1,4)$.

\bigskip

\exercice Calculer les racines de $P(x) = x^2 - 2x -3$ sans utiliser les formules usuelles.

\bigskip

\exercice Sans calculer le discriminant, montrer que le polynôme
$P(x) = x^2-2x+2$ défini sur $\mathbb R$  n'admet pas de solution réelle.

\bigskip

\exercice Soit $P(x)=x^3-8x^2+23x-28$. Déterminer les racines du polynôme $P$
sachant que la somme de deux des racines est égale à la troisième.

\bigskip

\exercice Chercher les solutions des équations suivantes :
\begin{enumerate}[(i)]
\item $x^3-2x^2+2x = 0$
\item $x^3+2x^2-x-2 = 0$
\item $x^4-5x^2+4 = 0$
\item $x^2 - 2\sqrt{2}x + 2 = 0$
\item $x^3-4x+\frac{3}{x} = 0$
\end{enumerate}

\bigskip

\exercice Trouver trois entiers naturels consécutifs tels que la somme de leurs carrés est égale à 50.

\bigskip

\exercice Une fonction $f$ est dite paire si $f(-x)=f(x)$ et impair si $f(-x)=-f(x)$. Par exemple, la fonction $f(x)=x^2$ est paire car $f(-x)=(-x)^2=(-1)^2x ²=x²$, la fonction $f(x)=x^3$ est impaire car $f(-x)=(-x^3)=(-1)^3x^3=-x^3$. Étudier la parité des fonctions suivantes~:
\begin{enumerate}[(i)]
\item $f(x) = e^x-e^{-x}$
\item $g(x) = \frac{e^{2x}-1}{e^{2x}+1}$
\item $h(x) = \frac{e^x}{\left(e^x+1\right)^2}$
\end{enumerate}

\bigskip

\exercice Chercher des solutions réelles pour les équations suivantes~:
\begin{enumerate}[(i)]
\item $e^{2x}-e^x-6 = 0$
\item $3e^x-7e^{-x}-20 = 0$
\end{enumerate}

\bigskip

\exercice Résoudre en $x$ et $y$ les systèmes d'équations suivants~:
\[
  (\mathrm{i})
  \begin{cases}
    e^xe^y &= 10\\ e^{x-y} &= \frac{2}{5}
  \end{cases}
  \quad\quad
  (\mathrm{ii})
  \begin{cases}
    e^x-2e^y &= -5\\ 3e^{x}+e^y &= 13
  \end{cases}
  \quad\quad
  (\mathrm{ii})
  \begin{cases}
    5e^x-e^y &= 19\\ e^{x+y} &= 30
  \end{cases}
\]

\exercice Chercher les solutions réelles pour les équations suivantes~:
\begin{enumerate}[(i)]
\item $\log(x^2-1)-\log(2x-1)+\log 2 = 0$
\item $\log(x+2)-\log(x+1)=\log(x-1)$
\end{enumerate}

\end{document}
