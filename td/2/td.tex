\documentclass[10pt,a4paper,notitlepage]{article}
\usepackage{amsmath}
\usepackage{amssymb}
\usepackage{amsbsy}
\usepackage{float}
\usepackage[french]{babel}
\usepackage{graphicx}
\usepackage{enumerate}

\usepackage{palatino}

 \usepackage[active]{srcltx}
\usepackage{scrtime}

\newcommand{\exercice}[1]{\textsc{\textbf{Exercice}} #1}
\newcommand{\question}[1]{\textbf{(#1)}}
\setlength{\parindent}{0cm}

\begin{document}

\title{\textsc{Calcul Économique\\ \small{(Fiche de TD n°2)}}}
\author{Stéphane Adjemian\thanks{Université du Maine, Gains. \texttt{stephane DOT adjemian AT univ DASH lemans DOT fr}}}
\date{Le \today\ à \thistime}

\maketitle

\exercice{1} La demande de montres SLOUK est de 10 unités si le prix est
égal à 160 euros et elle est de 20 unités si le prix est 120
euros. Calculer la fonction de demande supposée linéaire.

\bigskip

\exercice{2} Quand le prix est de 100 euros la quantité d'appareils
photos de marque PISTOL offerte sur le marché est 50 unités. Quand le
prix est 50\% plus élevé le nombre d'unités offertes est de
100. Calculer la fonction d'offre supposée linéaire.

\bigskip

\exercice{3} Sur un marché, la demande et l'offre pour un bien sont
caractérisés par :
\[
\begin{split}
  D(p): q &= -2 p  + 6\\
  S(p): q &= \frac{1}{2} p  + 1
\end{split}
\]
où $p$ est le prix du bien et $q$ sa quantité. Calculer la quantité
d'équilibre et le prix d'équilibre.

\bigskip

\exercice{4} Supposons que la consommation agrégée dans une économie,
notée $C$, soit une fonction linéaire du revenu disponible (hors
taxes), noté $Y$. Supposons qu'il existe un niveau de consommation
incompressible, noté $C_0$. Il s'agit du niveau de consommation
observé même si le revenu disponible est nul. On supposera que lorsque
le revenu augmente de $x$, la consommation en écart à son niveau
incompressible, \emph{ie} $C-C_0$, augmente de $0,8x$. Déterminer la
forme de la fonction de consommation.

\bigskip

\exercice{5} Montrer que la fonction $f(x) = x^2+2x+1$ admet un unique
minimum en $x=-1$.

\bigskip

\exercice{6} Sur un marché, la demande et l'offre pour un bien sont
caractérisés par :
\[
\begin{split}
  D(p): q &= -2 p^2  + 3\\
  S(p): q &= p^2 + 5p +2
\end{split}
\]
où $p$ est le prix du bien et $q$ sa quantité (on s'intéresse aux
valeurs positives de $p$ et $q$). Calculer la quantité
d'équilibre et le prix d'équilibre.

\bigskip

\exercice{7} Sans calculer le discriminant, montrer que l'équation
$x^2-2x+2 = 0$ n'admet pas de solution réelle.

\bigskip

\exercice{8} Chercher les solutions des équations suivantes :
\begin{enumerate}[(i)]
\item $x^3-2x^2+2x = 0$
\item $x^3+2x^2-x-2 = 0$
\item $x^4-5x^2+4 = 0$
\item $x^2 - 2\sqrt{2}x + 2 = 0$
\end{enumerate}

\bigskip

\exercice{9} Représenter graphiquement à l'aide d'un tableur les
fonctions suivantes :
\begin{enumerate}[(i)]
\item $f(x) = x^5- 5x^3 + 4x$
\item $f(x) = \frac{x-\frac{1}{2}}{x^3-x}$
\end{enumerate}
Et déterminer graphiquement les valeurs de $x$ telles que $f(x)=0$
dans chaque cas.

\bigskip

\exercice{10} Soit un ménage disposant d'un revenu de 100. On suppose
qu'il ne peut acheter que des bananes et des carottes et que les prix
de ces deux biens sont respectivement $p_B= 1$ et $p_C = \frac{1}{2}$
(l'unité dans les deux cas est le kilogramme). \textbf{(1)} Supposons
que le ménage décide de consommer la totalité de son revenu en
achetant ces deux biens (on admet qu'il ne peut pas consommer une
quantité négative de banane ou de carotte). Déterminer l'ensemble des
couples de quantités $(q_B, q_C)$ cohérents avec cette
hypothèse. \textbf{(2)} Comment cet ensemble est-il modifié si le
ménage peut décider de ne pas consommer la totalité de son
revenu. \textbf{(3)} Représenter graphiquement ces deux ensembles.

\bigskip

\exercice{11} (Suite de l'exercice 10) Supposons que l'on puisse
quantifier la satisfaction du ménage à l'aide de la fonction :
\[
u(q_B, q_C) = q_B + q_C
\]
la satisfaction (ou l'utilité) du ménage est d'autant plus élevée que
celui-ci consomme plus. \textbf{(0)} Interpréter cette
fonction. \textbf{(1)} Supposons que le ménage souhaite atteindre un
niveau de satisfaction égal à 300. Déterminer l'ensemble des couples
$(q_B, q_C)$ compatibles avec cet objectif. \textbf{(2)} Compléter le
graphique construit dans l'exercice 9 en représentant cet ensemble.
\textbf{(3)} Étant donnée la contrainte de revenu définie dans
l'exercice 9, déterminer si le ménage peut atteindre ce niveau de
satisfaction. \textbf{(4)} Déterminer le choix du ménage (en termes de
quantités demandées pour les bananes et les carottes) en supposant que
celui-ci cherche à atteindre le niveau de satisfaction le plus élevé
possible. \textbf{(5)} On adopte maintenant une fonction de
satisfaction plus générale de la forme :
\[
u(q_B, q_C) = \alpha q_B + \beta q_C
\]
où $\alpha$ et $\beta$ sont deux réels positifs. Déterminer les choix
du ménage en fonction de $\alpha$ et $\beta$. Commenter le résultat.

\exercice{12} On reprend les exercices 9 et 10 en supposant que $p_B =
p_C = 1$ et que $u(q_B, q_C) = \sqrt{q_Bq_C}$. \textbf{(1)} Exprimer
$q_C$ en fonction du niveau d'utilité, $\bar u$ par exemple, et de
$q_B$. La fonction ainsi construite est la fonction d'iso-utilité :
l'ensemble des points $(q_C,q_B)$ vérifiant la contrainte définie par
cette fonction fournissent un même niveau d'utilité. \textbf{(2)}
Représenter graphiquement la contrainte budgétaire du ménage et les
fonctions d'iso-utilité. \textbf{(3)} Identifier les demandes optimales
du ménage.



\end{document}
