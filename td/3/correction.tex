\synctex=1

\documentclass[10pt,notheorems]{beamer}

\usepackage{etex}
\usepackage{fourier-orns}
\usepackage{ccicons}
\usepackage{amssymb}
\usepackage{amstext}
\usepackage{amsbsy}
\usepackage{amsopn}
\usepackage{amscd}
\usepackage{amsxtra}
\usepackage{amsthm}
\usepackage{float}
\usepackage{color, colortbl}
\usepackage{mathrsfs}
\usepackage{bm}
\usepackage[nice]{nicefrac}
\usepackage{setspace}
\usepackage{ragged2e}
\usepackage{listings}
\usepackage{polynom}
\usepackage{algorithms/algorithm}
\usepackage{algorithms/algorithmic}
\usepackage[frenchb]{babel}
\usepackage{tikz,pgfplots}
\pgfplotsset{compat=newest}
\usetikzlibrary{patterns, arrows, decorations.pathreplacing, decorations.markings, calc}
\pgfplotsset{plot coordinates/math parser=false}
\newlength\figureheight
\newlength\figurewidth
\usepackage{cancel}
\usepackage{tikz-qtree}
\usepackage{dcolumn}
\usepackage{adjustbox}
\usepackage{environ}
\usepackage[cal=boondox]{mathalfa}
\usepackage{manfnt}
\usepackage{hyperref}
\hypersetup{
  colorlinks=true,
  linkcolor=blue,
  filecolor=black,
  urlcolor=black,
}
\usepackage{venndiagram}
\usepackage{caption}
\usepackage{subcaption}


% Git hash
\usepackage{xstring}
\usepackage{catchfile}
\immediate\write18{git rev-parse HEAD > git.hash}
\CatchFileDef{\HEAD}{git.hash}{\endlinechar=-1}
\newcommand{\gitrevision}{\StrLeft{\HEAD}{7}}

\newcommand{\trace}{\mathrm{tr}}
\newcommand{\vect}{\mathrm{vec}}
\newcommand{\tracarg}[1]{\mathrm{tr}\left\{#1\right\}}
\newcommand{\vectarg}[1]{\mathrm{vec}\left(#1\right)}
\newcommand{\vecth}[1]{\mathrm{vech}\left(#1\right)}
\newcommand{\iid}[2]{\mathrm{iid}\left(#1,#2\right)}
\newcommand{\normal}[2]{\mathcal N\left(#1,#2\right)}
\newcommand{\dynare}{\href{http://www.dynare.org}{\color{blue}Dynare}}
\newcommand{\sample}{\mathcal Y_T}
\newcommand{\samplet}[1]{\mathcal Y_{#1}}
\newcommand{\slidetitle}[1]{\fancyhead[L]{\textsc{#1}}}

\newcommand{\R}{{\mathbb R}}
\newcommand{\C}{{\mathbb C}}
\newcommand{\N}{{\mathbb N}}
\newcommand{\Z}{{\mathbb Z}}
\newcommand{\binomial}[2]{\begin{pmatrix} #1 \\ #2 \end{pmatrix}}
\newcommand{\bigO}[1]{\mathcal O \left(#1\right)}
\newcommand{\red}{\color{red}}
\newcommand{\blue}{\color{blue}}

\renewcommand{\qedsymbol}{C.Q.F.D.}

\newcolumntype{d}{D{.}{.}{-1}}
\definecolor{gray}{gray}{0.9}
\newcolumntype{g}{>{\columncolor{gray}}c}

\setbeamertemplate{theorems}[numbered]

\theoremstyle{plain}
\newtheorem{theorem}{Théorème}

\theoremstyle{definition} % insert bellow all blocks you want in normal text
\newtheorem{definition}{Définition}
\newtheorem{properties}{Propriétés}
\newtheorem{lemma}{Lemme}
\newtheorem{property}[properties]{Propriété}
\newtheorem{example}{Exemple}
\newtheorem*{idea}{Éléments de preuve} % no numbered block

\setbeamertemplate{itemize subitem}{--}

\setbeamertemplate{footline}{
  {\hfill\vspace*{1pt}\href{http://creativecommons.org/licenses/by-sa/3.0/legalcode}{\ccbysa}\hspace{.1cm}
    \raisebox{-.1cm}{\href{https://github.com/stepan-a/economic-calculus}{\includegraphics[scale=.015]{../../img/git.png}}}\enspace
    \href{https://github.com/stepan-a/economic-calculus/blob/\HEAD/td/3/correction.tex}{\gitrevision}\enspace--\enspace\today\enspace
  }}

\setbeamertemplate{navigation symbols}{}
\setbeamertemplate{blocks}[rounded][shadow=true]
\setbeamertemplate{caption}[numbered]

\NewEnviron{notes}{\justifying\tiny\begin{spacing}{1.0}\BODY\vfill\pagebreak\end{spacing}}

\newenvironment{exercise}[1]
{\bgroup \small\begin{block}{Ex. #1}}
  {\end{block}\egroup}

\newenvironment{defn}[1]
{\bgroup \small\begin{block}{Définition. #1}}
  {\end{block}\egroup}

\newenvironment{exemple}[1]
{\bgroup \small\begin{block}{Exemple. #1}}
  {\end{block}\egroup}

\newcounter{xnumber}
\setcounter{xnumber}{0}
\newcommand\exonumber{\addtocounter{xnumber}{1}\thexnumber}

\begin{document}

\title{Calcul Économique\\\small{Éléments de correction du TD 3}}
\author[S. Adjemian]{Stéphane Adjemian}
\institute{\texttt{stephane.adjemian@univ-lemans.fr}} \date{Septembre 2024}

\begin{frame}
  \titlepage{}
\end{frame}


\begin{frame}
  \frametitle{Exercice \exonumber}
  \fontsize{8}{10}\selectfont

  \begin{itemize}

  \item  On a $u_1 = \rho$, $u_2 = \rho u_1 = \rho^2$, $u_3 = \rho u_2 = \rho^3$, \ldots\newline

  \item En comparant l'indice de la suite et l'exposant sur $\rho$, on devine que $u_n = \rho^{n}$.\newline

  \item On montre facilement qui cela est vrai alors on doit avoir
    $u_{n+1} = \rho^{n+1}$ (de sorte que le terme général postulé pour $u_n$ est
    vrai pour tout $n$) :
    \[
      u_{n+1} = \rho u_{n} = \rho \rho^{n} = \rho^{n+1}
    \]
    On a donc bien $u_n = \rho^{n}$ pour tout entier naturel $n$.\newline

  \item Pour tout $\varepsilon>0$ on a~:
    \[
      \left| u_n-0 \right| = \left| \rho^n \right| <\varepsilon
    \]
    \[
      \Leftrightarrow \rho^n <\varepsilon \text{, car}\rho\text{ est positif}
    \]
    \[
      \Leftrightarrow n\log\rho <\log\varepsilon \text{, car le logarithme est une fonction croissante}
    \]
    \[
      \Leftrightarrow n > \frac{\log\varepsilon}{\log\rho} \text{, car}\log\rho\text{ est négatif}
    \]
    Ainsi, $\forall \varepsilon>0$, $|u_n-0|<\varepsilon$ dès lors que $n>N(\varepsilon)=\nicefrac{\log\varepsilon}{\log\rho}$. Il faut aller chercher des $n$ d'autant plus grands que $\varepsilon$ est petit ou $\rho$ proche de 1 (le processus est plus persistant, voir la question suivante).

  \end{itemize}

\end{frame}


\addtocounter{xnumber}{-1}
\begin{frame}
  \frametitle{Exercice \exonumber\, (suite)}
  \fontsize{8}{10}\selectfont

  \begin{itemize}

  \item Combien d'itérations faut-il pour réduire de moitié la distance à la limite~? Cela revient à se demander, partant de $u_0=1$, pour quelle valeur de $n$ on a $u_n=\nicefrac{1}{1}$, c'est-à-dire~:
    \[
      \rho^n = \frac{1}{2}
    \]
    ou encore~:
    \[
      n\log \rho = -\log 2
    \]
    et donc~:
    \[
      n = -\frac{\log 2}{\log \rho}>1
    \]
    Il faut plus d'itérations si $\rho$ est plus proche de 1, dans ce cas on dit que le processus est plus persistant.\newline

  \item Le nombre d'itération nécessaires est indépendant de la condition initiale.\newline

  \item[$\rightarrow$] Pour que $u_m$ soit égal à la moitié de $u_n$ il faut et il suffit que $m-n$ soit égal à $-\nicefrac{\log 2}{\log \rho}$.

  \end{itemize}

\end{frame}


\addtocounter{xnumber}{-1}
\begin{frame}
  \frametitle{Exercice \exonumber\, (suite)}
  \fontsize{8}{10}\selectfont

  \begin{itemize}

  \item La suite diverge vers $+\infty$ si $\rho>1$. Pour tout $\mathcal A>0$ on
    peut montrer qu'il existe un rang $N$ tel que pour tout $n>N$ on ait
    $u_n>\mathcal A$.
    \[
      u_n>\mathcal A \Leftrightarrow \rho^{n}>\mathcal A \Leftrightarrow  n \log \rho > \log \mathcal A \Leftrightarrow n> \frac{\log \mathcal A}{\log \rho}\equiv N
    \]
    Conformément à l'intuition, on note que le rang $N$ est d'autant plus petit
    que $\rho$ est grand (c'est-à-dire que $u_n$ croît vite, puisque le taux de
    croissance de $u_n$ est
    $100\left(\frac{u_n}{u_{n-1}}-1\right) = 100(\rho-1)$ en
    pourcentage).\newline

  \end{itemize}

\end{frame}


\begin{frame}
  \frametitle{Exercice \exonumber}
  \fontsize{8}{10}\selectfont

  \begin{table}
    \centering
  \begin{tabular}[H]{r|ccccc}
     \hline
     $n$   & 1 & 2 & 3              &  4             & \ldots\\ \hline
     $u_n$ & 3 & 2 & $\frac{5}{3}$  &  $\frac{2}{2}$ & \ldots \\ \hline\hline
  \end{tabular}
  \end{table}

  \begin{itemize}

  \item Cette suite est monotone décroissance, en effet $u_{n+1}-u_n<0$ pour tout $n$~:
    \[
      u_{n+1}-u_n = \frac{n+3}{n+1}-\frac{n+2}{n} = \frac{n(n+3)-(n+1)(n+2)}{n(n+1)} = -\frac{2}{n(n+1)}<0
    \]

  \item On montre que $\lim_{n\rightarrow\infty}u_n = 1$ en montrant que pour
    tout $\varepsilon>0$ il existe un rang $N$ tel que pour tout $n>N$ on ait
    $|u_n-1|<\varepsilon$ (à partir d'un certain rang la suite se rapproche
    arbitrairement de 1).\newline

  \item Nous avons $|u_n-1| = \left|\frac{n+2}{n}-1\right|=\frac{2}{n}$, et donc~:
    \[
      |u_n-1|<\varepsilon \Leftrightarrow \frac{2}{n}<\varepsilon \Leftrightarrow n > \frac{2}{\varepsilon} \equiv N
    \]

  \end{itemize}

\end{frame}


\begin{frame}
  \frametitle{Exercice \exonumber}
  \fontsize{8}{10}\selectfont

  \begin{itemize}

  \item Cette suite diverge vers $-\infty$.\newline

  \item Pour le montrer, il suffit d'établir que l'on peut rendre arbitrairement petit $u_n$ (vers $-\infty$) dès lors que l'indice $n$ est assez grand.\newline

  \item Il faut montrer que $\forall \mathcal A>0$, $\exists N\in \mathbb N$ tel que $u_n<-\mathcal A$ pour tout $n>N$.\newline

    \[
      u_n<-\mathcal A \Leftrightarrow -n < -\mathcal A \Leftrightarrow n > \mathcal A \equiv N
    \]

  \end{itemize}

\end{frame}


\begin{frame}
  \frametitle{Exercice \exonumber}
  \fontsize{8}{10}\selectfont

  \begin{itemize}

  \item Il s'agit d'une suite alternée (non monotone) à cause de la puissance sur $-1$.\newline

  \item On a $|u_n-0| = \left|\frac{(-1)^{n+1}}{n^2}\right| = \frac{1}{n^2}$.\newline

  \item Soit $\varepsilon>0$ une constante arbitrairement petite.\newline

  \item On a :
    \[
      |u_n-0|<\varepsilon \Leftrightarrow \frac{1}{n^2}<\varepsilon \Leftrightarrow n >\frac{1}{\sqrt{\varepsilon}}\equiv N
    \]

  \item Ainsi, pour tout $\varepsilon>0$, si $n$ est plus grand que le rang $N(\varepsilon)=\frac{1}{\sqrt{\varepsilon}}$ alors $|u_n-0|<\varepsilon$.\newline

  \item On peut rendre $u_n$ arbitrairement proche de 0 à partir du moment où $n$ est assez grand.

  \end{itemize}

\end{frame}


\begin{frame}
  \frametitle{Exercice \exonumber}
  \fontsize{8}{10}\selectfont

    \begin{table}
    \centering
    \begin{tabular}[H]{r|ccccc}
      \hline
     $n$    & 1 & 2             & 3                &  4                 & \ldots\\ \hline
     $u_n$  & 2 & $\frac{3}{2}$ & $\frac{17}{12}$  &  $\frac{577}{408}$ & \ldots \\
     $u_n$  & 2 & 1,5           & 1,4166667        &  1,4142157         & \ldots\\ \hline\hline
    \end{tabular}
  \end{table}

  \bigskip

  \begin{itemize}

  \item On peut écrire $u_{n+1}\geq \sqrt{2}$ de façon équivalente sous la forme~:
    \[
       \frac{u_n}{2}+\frac{1}{u_n} \geq \sqrt{2}
    \]
    \[
      \Leftrightarrow \frac{u_n^2+2}{2u_n} \geq \sqrt{2}
    \]
    \[
      \Leftrightarrow u_n^2- 2u_n\sqrt{2} + 2\geq 0
    \]
    \[
      \Leftrightarrow \left(u_n- \sqrt{2}\right)^2\geq 0
    \]
    Le carré d'un nombre réel ne peut être négatif, cette inégalité est donc nécessairement vérifiée et $u_{n+1}$ est donc forcément supérieur ou égal à $\sqrt{2}$.

  \end{itemize}

\end{frame}


\addtocounter{xnumber}{-1}
\begin{frame}
  \frametitle{Exercice \exonumber\, (suite)}
  \fontsize{8}{10}\selectfont

  \begin{itemize}

  \item Un point fixe de la suite est une valeur réelle $\bar u$ telle que~:
    \[
      \bar u = \frac{\bar u}{2} + \frac{1}{\bar u}
    \]

    \medskip

  \item En toute généralité une suite peut admettre plus d'un point fixe.\newline

  \item Ici il existe un unique point fixe. On a~:
    \[
      \bar u^2 = \frac{\bar u^2}{2} + 1
    \]
    \[
      \frac{\bar u^2}{2} =  1
    \]
    \[
      \bar u^2 = 2
    \]

  \item Comme la suite est positive (puisque $u_n\geq \sqrt{2}$), il existe une
    unique solution pour $\bar u$~:
    \[
      \bar u = \sqrt{2}
    \]

  \end{itemize}

\end{frame}


\addtocounter{xnumber}{-1}
\begin{frame}
  \frametitle{Exercice \exonumber\, (suite)}
  \fontsize{8}{10}\selectfont

  \begin{itemize}

  \item Pour montrer que la suite est décroissante, il faut montrer que les variations sont négatives pour tout $n$.\newline

  \item On a des équivalences suivantes~:
    \[
      u_{n+1}-u_{n} \leq 0
    \]
    \[
      \Leftrightarrow \frac{1}{u_n}-\frac{u_{n}}{2} \leq 0
    \]
    \[
      \Leftrightarrow \frac{2-u_n^2}{2u_n}\leq 0
    \]
    La dernière inégalité est vraie puisque $u_n\geq \sqrt{2}$, on a donc bien $u_{n+1}\leq u_{n}$ pour tout $n$ (l'inégalité est stricte tant que $u_n>\sqrt{2}$).\newline

  \item La suite est donc monotone décroissante.

  \end{itemize}

\end{frame}


\addtocounter{xnumber}{-1}
\begin{frame}
  \frametitle{Exercice \exonumber\, (suite)}
  \fontsize{8}{10}\selectfont

  \begin{itemize}

  \item La suite $u_n$ est décroissante et bornée.\newline

  \item La suite $u_n$ est donc convergente.\newline

  \item On a $\lim_{n\rightarrow\infty}u_n = \sqrt{2}$.\newline

  \item La suite $u_n$ est à valeurs dans $\mathbb Q$, mais sa limite n'appartient pas à $\mathbb Q$ ($\sqrt{2}$ est un nombre irrationnel).

  \end{itemize}

\end{frame}


\addtocounter{xnumber}{-1}
\begin{frame}
  \frametitle{Exercice \exonumber\, (suite)}
  \fontsize{8}{10}\selectfont

  \begin{itemize}

  \item Pour montrer l'inégalité demandée en \textbf{(6)}, on a les équivalences suivantes~:
    \[
      u_{n+1}-\sqrt{2}<\frac{1}{2}\left(u_n-\sqrt{2}\right)
    \]
    \[
      \Leftrightarrow 2\left(\frac{u_n}{2}+\frac{1}{u_n}-\sqrt{2}\right)<u_n-\sqrt{2}
    \]
    \[
      \Leftrightarrow u_n+\frac{2}{u_n}-2\sqrt{2}<u_n-\sqrt{2}
    \]
    \[
      \Leftrightarrow \frac{2}{u_n}<\sqrt{2}
    \]
    \[
      \Leftrightarrow u_n>\sqrt{2}
    \]
    Cette dernière inégalité est vraie, puisque $\sqrt{2}$ est un minorant de la suite, donc la première inégalité est vraie.\newline

  \item En itérant sur l'inégalité, on obtient~:
    \[
      u_n-\sqrt{2} < \frac{1}{2}\left(u_{n-1}-\sqrt{2}\right) < \frac{1}{2^2}\left(u_{n-2}-\sqrt{2}\right) < \ldots\ldots<\frac{1}{2^n}\left(u_{0}-\sqrt{2}\right)
    \]


  \end{itemize}

\end{frame}


\addtocounter{xnumber}{-1}
\begin{frame}
  \frametitle{Exercice \exonumber\, (suite)}
  \fontsize{8}{10}\selectfont

  \begin{itemize}

  \item On a donc bien~:
    \[
      0 < u_n-\sqrt{2} < \frac{1}{2^n}\left(u_{0}-\sqrt{2}\right) = \frac{\sqrt{2}(\sqrt{2}-1)}{2^n}
    \]

    \bigskip

  \item On voit donc qu'il est possible de rendre $|u_n-\sqrt{2}|$ arbitrairement petit à partir du moment où $n$ est assez grand.


  \end{itemize}

\end{frame}


\begin{frame}
  \frametitle{Exercice \exonumber\, (1)}
  \fontsize{8}{10}\selectfont

  \[
    \begin{split}
        \lim_{x\rightarrow\infty} \frac{2x+5}{x^2-3}
         &= \lim_{x\rightarrow\infty} \frac{x(2+\frac{5}{x})}{x(x-\frac{3}{x})}\\
         &= \lim_{x\rightarrow\infty} \frac{2+\frac{5}{x}}{x-\frac{3}{x}}\\
         &= \frac{2+\lim_{x\rightarrow\infty} \frac{5}{x}}{\lim_{x\rightarrow\infty} x-\lim_{x\rightarrow\infty} \frac{3}{x}}\\
         &= \frac{2}{\lim_{x\rightarrow\infty} x} \\
         &= 0\\
      \end{split}
    \]

\end{frame}


\addtocounter{xnumber}{-1}
\begin{frame}
  \frametitle{Exercice \exonumber\, (2)}
  \fontsize{8}{10}\selectfont

  \[
    \begin{split}
      \lim_{x\rightarrow\infty} \frac{x^3-4x^2+8}{x^2+6}
      &= \lim_{x\rightarrow\infty} \frac{x^2(x-4+\frac{8}{x^2})}{x^2(1+\frac{6}{x^2})}\\
      &= \lim_{x\rightarrow\infty} \frac{x-4+\frac{8}{x^2}}{1+\frac{6}{x^2}}\\
      &= \frac{\lim_{x\rightarrow\infty} x-4+\lim_{x\rightarrow\infty} \frac{8}{x^2}}{1+\lim_{x\rightarrow\infty} \frac{6}{x^2}}\\
      &= \frac{\lim_{x\rightarrow\infty} x-4}{1}\\
      &= \infty \\
    \end{split}
  \]

\end{frame}


\addtocounter{xnumber}{-1}
\begin{frame}
  \frametitle{Exercice \exonumber\, (3)}
  \fontsize{8}{10}\selectfont

  \[
    \begin{split}
      \lim_{x\rightarrow\infty} \frac{ax^2+bx+c}{kx^2+lx+m}
      &= \lim_{x\rightarrow\infty} \frac{x^2(a+\frac{b}{x}+\frac{c}{x^2})}{x^2(k+\frac{l}{x}+\frac{m}{x^2})}\\
      &= \lim_{x\rightarrow\infty} \frac{a+\frac{b}{x}+\frac{c}{x^2}}{k+\frac{l}{x}+\frac{m}{x^2}}\\
      &= \lim_{x\rightarrow\infty} \frac{a+\lim_{x\rightarrow\infty}\frac{b}{x}+\lim_{x\rightarrow\infty}\frac{c}{x^2}}{k+\lim_{x\rightarrow\infty}\frac{l}{x}+\lim_{x\rightarrow\infty}\frac{m}{x^2}}\\
      &= \frac{a}{k}
    \end{split}
  \]

  \bigskip

  Il faut bien sûr supposer que $k\ne 0$, sinon la fonction diverge vers $+\infty$.

\end{frame}


\addtocounter{xnumber}{-1}
\begin{frame}
  \frametitle{Exercice \exonumber\, (4)}
  \fontsize{8}{10}\selectfont

  \[
    \begin{split}
      \lim_{x\rightarrow -4} \frac{x^2-16}{x+4}
      &= \lim_{x\rightarrow -4} \frac{(x-4)(x+4)}{x+4}\\
      &= \lim_{x\rightarrow -4} (x-4)\\
      &= \lim_{x\rightarrow -4} x - 4 \\
      &=-8\\
    \end{split}
  \]

\end{frame}


\addtocounter{xnumber}{-1}
\begin{frame}
  \frametitle{Exercice \exonumber\, (5)}
  \fontsize{8}{10}\selectfont

  \begin{itemize}

  \item[$x\rightarrow 0^+$] $\Rightarrow x>0$ et donc $|x| = x$.
    \[
      \lim_{x\rightarrow 0^+} \frac{|x|}{x} = \lim_{x\rightarrow 0^+} \frac{x}{x} = 1
    \]

    \bigskip

  \item[$x\rightarrow 0^-$] $\Rightarrow x<0$ et donc $|x| = -x$.
    \[
      \lim_{x\rightarrow 0^-} \frac{|x|}{x} = \lim_{x\rightarrow 0^+} -\frac{x}{x} = -1
    \]

  \end{itemize}

\end{frame}


\addtocounter{xnumber}{-1}
\begin{frame}
  \frametitle{Exercice \exonumber\, (6)}
  \fontsize{8}{10}\selectfont

  \[
    \begin{split}
      \sqrt{x^2+1}-\sqrt{x^2-1} &= \left( \sqrt{x^2+1}-\sqrt{x^2-1} \right)\frac{\sqrt{x^2+1}+\sqrt{x^2-1}}{\sqrt{x^2+1}+\sqrt{x^2-1}}\\
                                &= \frac{x^2+1-(x^2-1)}{\sqrt{x^2+1}+\sqrt{x^2-1}}\\
                                &= \frac{2}{\sqrt{x^2+1}+\sqrt{x^2-1}}
    \end{split}
  \]

  \bigskip

  Puisqu'au édnominateur nous avons la somme de deux racines carrées qui tendent vers l'infini lorsque $x$ tend vers l'infini, on conclut que
  $\lim_{x\rightarrow \infty} \sqrt{x^2+1}-\sqrt{x^2-1} = 0$.
\end{frame}


\addtocounter{xnumber}{-1}
\begin{frame}
  \frametitle{Exercice \exonumber\, (7)}
  \fontsize{8}{10}\selectfont

  \[
    \begin{split}
      \sqrt{x^2+4x}-x &= \left( \sqrt{x^2+4x}-x \right)\frac{\sqrt{x^2+4x}+x}{\sqrt{x^2+4x}+x}\\
                      &= \frac{4x}{\sqrt{x^2+1}+x}\\
                      &= \frac{4x}{x\sqrt{1+\frac{1}{x}}+x}\\
    \end{split}
  \]

  En effet nous ne nous intéressons qu'aux valeurs positives de $x$, puisque nous considérons la limite quand $x$ tend vers $+\infty$.

  \[
    \sqrt{x^2+4x}-x  = \frac{4}{\sqrt{1+\frac{1}{x}}+1}
  \]
  et donc~:
  \[
    \lim_{x\rightarrow\infty}\sqrt{x^2+4x}-x = 2
  \]

\end{frame}


\addtocounter{xnumber}{-1}
\begin{frame}
  \frametitle{Exercice \exonumber\, (8)}
  \fontsize{8}{10}\selectfont

  \textbf{Remarque:} Pour $x=-2$ le numérateur et le dénominateur sont nuls~!\newline

  $\Rightarrow$ On peut factoriser $x+2$ au dénominateur et au numérateur.\newline

  On a $x^2-4=(x-2)(x+2)$ (identité remarquable).\newline

  Par la méthode des coefficients indéterminés ou division euclidienne (voir le chapitre II) on montre que~:
  \[
    x^3+2x^2-x-2 = (x+2)\left( x^2-1 \right)
  \]

  \bigskip

  On a donc~:

  \[
    \lim_{x\rightarrow -2}\frac{x^3+2x^2-x-2}{x^2-4} = \lim_{x\rightarrow -2} \frac{x^2-1}{x-2} = -\frac{3}{4}
  \]

\end{frame}


\begin{frame}
  \frametitle{Exercice \exonumber}
  \fontsize{8}{10}\selectfont

  Pour que cette fonction soit définie sur $\mathbb R$ il faut et il suffit qu'elle soit continue en -1 et 2, puisque les morceaux sont des droites (c'est-à-dire des fonctions continues).\newline

  Pour que la fonction soit continue en 2, il faut que $\lim_{x\rightarrow 2^+}f(x) = \lim_{x\rightarrow 2^-}f(x) = f(2)$. On a $f(2)=2-1=1$, et~:
  \[
    \lim_{x\rightarrow 2^-}f(x) = -3 \times 2 + 7 = 1 \text{ on considère la deuxième branche avec} x<2
  \]
  puis
  \[
    \lim_{x\rightarrow 2^+}f(x) = 2 - 1 = 1 \text{ on considère la troisième branche avec} x\geq 2
  \]
  Donc la fonction est continue en $x=2$.\newline

  Mais elle n'est pas continue en $x=-1$. En effet, nous avons $f(-1) = 2$ et~:
  \[
    \lim_{x\rightarrow -1^+}f(x) = -3\times (-1) + 7 = 10
  \]
  \[
    \lim_{x\rightarrow -1^-}f(x) = 6\times (-1) + 8 = 2
  \]
  Cette fonction n'admet pas de limite en -1 puisque les limites à droites et à gauche sont différentes. La fonction n'est donc pas continue en -1.

\end{frame}


\begin{frame}
  \frametitle{Exercice \exonumber}
  \fontsize{8}{10}\selectfont

  Pour que cette fonction soit définie sur $\mathbb R$ il faut et il suffit qu'elle soit continue en 2, puisque les morceaux sont des fonctions polynomiales d'ordre 2 (c'est-à-dire continues).\newline

  Pour que la fonction soit continue en 2, il faut que $\lim_{x\rightarrow 2^+}f(x) = \lim_{x\rightarrow 2^-}f(x) = f(2)$. On a $f(2)=4a+2b+1$. Par ailleurs, on a~:
  \[
    \lim_{x\rightarrow 2^-}f(x) = 4a+2b+1
  \]
  puis
  \[
    \lim_{x\rightarrow 2^+}f(x) = 4+2a+b
  \]

  \bigskip

  Pour que la fonction soit continue en $x=2$ et donc sur $\mathbb R$, il faut et il suffit que~:
  \[
    4a+2b+1 = 4 + 2a + b
  \]
  ou de façon équivalente~:
  \[
    b = 3-2a
  \]

\end{frame}


\begin{frame}
  \frametitle{Exercice \exonumber}
  \fontsize{8}{10}\selectfont

  \begin{itemize}

  \item La fonction est continue en tout point différent de -1, 0 ou 1, car il s'agit d'une composition de fonctions continues (la valeur absolue, le logarithme et l'inverse).\newline

  \item En zéro la fonction est continue car~:
    \[
      \lim_{x\rightarrow 0} f(x) = \lim_{u\rightarrow -\infty}\frac{1}{u} = 0 = f(0)
    \]
    puisque $\lim_{x\rightarrow 0}\log |x| = -\infty$.\newline

  \item En $x=1$ la fonction n'est pas continue. En effet, on a~:
    \[
      \begin{cases}
        \lim_{x\rightarrow 1^+}\log |x| &= 0^+\\
        \lim_{x\rightarrow 1^-}\log |x| &= 0^-
      \end{cases}
      \quad
      \Rightarrow
      \quad
      \begin{cases}
        \lim_{x\rightarrow 1^+}f(x) &= \infty\\
        \lim_{x\rightarrow 1^-}f(x) &= -\infty\\
      \end{cases}
    \]
    La fonction n'admet pas de limite en $x=1$ la fonction n'est donc pas continue. De plus, les limites à droite et à gauche sont différentes de $f(1)$.\newline

  \item Même argument pour $x=-1$.

  \end{itemize}

\end{frame}


\begin{frame}
  \frametitle{Exercice \exonumber}
  \fontsize{8}{10}\selectfont

  \begin{itemize}

  \item On a une forme indéterminée $0/0$ en -1.\newline

  \item Calculons, si elle existe, la limite de $f$ quand $x$ tend vers -1.\newline

  \item Puisque -1 est une racine du polynôme au dénominateur, on montre facilement que celui-ci peut s'écrire sous la forme $(x+1)\left( x^2 - x + 1 \right)$ (par la méthode des coefficients indéterminés par exemple).\newline

  \item On a donc~:
    \[
      \lim_{x\rightarrow -1}f(x) = \lim_{x\rightarrow -1} \frac{1}{x^2-x+1} = \frac{1}{3}
    \]

  \item Nous pouvons donc prolonger $f$~:
    \[
      \bar f(x) =
      \begin{cases}
        f(x) &\text{ si }x\neq 1\\
        \frac{1}{3} &\text{ sinon.}
      \end{cases}
    \]

  \end{itemize}

\end{frame}


\begin{frame}
  \frametitle{Exercice \exonumber\, (1)}
  \fontsize{8}{10}\selectfont

  \[
    \begin{split}
      f'(x) &= \lim_{h\rightarrow 0}\frac{f(x+h)-f(x)}{h}\\
            &= \lim_{h\rightarrow 0}\frac{4(x+h)^2+3-4x^2+3}{h}\\
            &= \lim_{h\rightarrow 0}\frac{4(x^2+2xh+h^2)-4x^2}{h}\\
            &= \lim_{h\rightarrow 0}\frac{8xh+h^2}{h}\\
            &= 8x + \lim_{h\rightarrow 0} h\\
            &= 8x
    \end{split}
  \]

\end{frame}


\addtocounter{xnumber}{-1}
\begin{frame}
  \frametitle{Exercice \exonumber\, (2)}
  \fontsize{8}{10}\selectfont

  \[
    \begin{split}
      g'(x) &= \lim_{h\rightarrow 0}\frac{(x+h)^n-x^n}{h}\\
            &= \lim_{h\rightarrow 0}\frac{\sum_{k=0}^n C_n^kx^{n-k}h^k-x^n}{h}\\
            &= \lim_{h\rightarrow 0}\frac{\sum_{k=1}^n C_n^kx^{n-k}h^k}{h}\\
            &= \lim_{h\rightarrow 0}\sum_{k=1}^n C_n^kx^{n-k}h^{k-1}\\
            &= C_n^1x^{n-1}\\
            &= n x^{n-1}
    \end{split}
  \]

\end{frame}


\addtocounter{xnumber}{-1}
\begin{frame}
  \frametitle{Exercice \exonumber\, (3)}
  \fontsize{8}{10}\selectfont

  \[
    \begin{split}
      h'(x) &= \lim_{h\rightarrow 0}\frac{\frac{1}{x+h}-\frac{1}{x}}{h}\\
            &= \lim_{h\rightarrow 0}\frac{\frac{x-x-h}{x(x+h)}}{h}\\
            &= \lim_{h\rightarrow 0}\frac{-1}{x(x+h)}\\
            &= -\frac{1}{x^2}
    \end{split}
  \]

\end{frame}


\addtocounter{xnumber}{-1}
\begin{frame}
  \frametitle{Exercice \exonumber\, (4)}
  \fontsize{8}{10}\selectfont

  \[
    \begin{split}
      j'(x) &= \lim_{h\rightarrow 0}\frac{\sqrt{x+h+1}-\sqrt{x+1}}{h}\\
            &= \lim_{h\rightarrow 0}\frac{\sqrt{x+h+1}-\sqrt{x+1}}{h}\frac{\sqrt{x+h+1}+\sqrt{x+1}}{\sqrt{x+h+1}+\sqrt{x+1}}\\
            &= \lim_{h\rightarrow 0}\frac{x+h+1-x-1}{h}\frac{1}{\sqrt{x+h+1}+\sqrt{x+1}}\\
            &= \lim_{h\rightarrow 0}\frac{1}{\sqrt{x+h+1}+\sqrt{x+1}}\\
            &= \frac{1}{2\sqrt{x+1}}
    \end{split}
  \]

\end{frame}

\begin{frame}
  \frametitle{Exercice \exonumber}
  \fontsize{8}{10}\selectfont

  En notant que $f(0)=0$, on a~:
  \[
    \begin{split}
      f'(0) &= \lim_{x\rightarrow 0}\frac{\frac{x}{1+|x|}-0}{x}\\
            &= \lim_{x\rightarrow 0}\frac{1}{1+|x|}\\
            &= 1
    \end{split}
  \]

\end{frame}



\end{document}


%%% Local Variables:
%%% mode: latex
%%% TeX-master: t
%%% ispell-check-comments: exclusive
%%% ispell-local-dictionary: "francais"
%%% TeX-master: t
%%% TeX-master: t
%%% TeX-master: t
%%% End:
