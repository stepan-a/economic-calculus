\documentclass[10pt,a4paper,notitlepage]{article}
\usepackage{amsmath}
\usepackage{amssymb}
\usepackage{amsbsy}
\usepackage{float}
\usepackage[french]{babel}
\usepackage{graphicx}
\usepackage{enumerate}

\usepackage{palatino}

 \usepackage[active]{srcltx}
\usepackage{scrtime}

\newcommand{\exercice}[1]{\textsc{\textbf{Exercice}} #1}
\newcommand{\question}[1]{\textbf{(#1)}}
\setlength{\parindent}{0cm}

\begin{document}

\title{\textsc{Calcul Économique\\ \small{(Fiche de TD n°3)}}}
\author{Stéphane Adjemian\thanks{Université du Mans. \texttt{stephane DOT adjemian AT univ DASH lemans DOT fr}}}
\date{Le \today\ à \thistime}

\maketitle

\exercice{1} Soit la suite de terme général $u_n = u_{n-1} + 1$ pour
$n \geq 1$, avec la condition initiale $u_1 = 1$. \textbf{(1)} Donner
une expression de $u_n$ en fonction du rang $n$ et de sa condition
initiale. \textbf{(2)} Soit la suite $v_n = \sum_{i=1}^n u_i$ pour
$n\geq 1$. Quelle est la condition initiale de cette suite ?
Déterminer $v_n$.

\bigskip

\exercice{2} Soit la suite de terme général $u_n = \rho u_{n-1}$ pour
$n \geq 1$ avec la condition initiale $u_1 = 1$ et $\rho$ un paramètre
réel non nul. \textbf{(1)} Donner une expression de $u_n$ en fonction
du rang $n$ et de sa condition initiale. \textbf{(2)} Discuter le
comportement asymptotique de $u_n$ en fonction de la valeur de
$\rho$. \textbf{(3)} Dans le cas où la suite admet une limite, combien
de temps faut-il pour réduire de moitié la distance à la limite ?

\bigskip

\exercice{3} Soit la suite de terme général $u_n =
\frac{n+2}{n}$. \textbf{(1)} Donner les premiers termes de cette suite
et montrer que la suite est monotone décroissante. \textbf{(2)}
Montrer que cette suite a pour limite 1.

\bigskip

\exercice{4} Quel est le comportement asymptotique de la suite de
terme général $u_n = -n$.

\bigskip

\exercice{5} La suite de terme général $u_n = \frac{(-1)^{n+1}}{n^2}$
est-elle monotone ? Montrer que cette suite admet 0 pour limite.

\bigskip

\exercice{6} Sur un marché l'offre et la demande sont caractérisées
par :
\[
\begin{split}
  S(p): &\quad q = 1+p\\
  D(p): &\quad q = 2-p\\
\end{split}
\]
\textbf{(1)} Calculer le prix d'équilibre $p^{\star}$ et les quantités échangées à
l'équilibre, $q^{\star}$. \textbf{(2)} Supposons que le marché ne soit pas
équilibré. On admet que dans une situation de déséquilibre le prix
augmente si et seulement si la demande est supérieure à l'offre. Plus
formellement on admet que le prix est mis à jour à l'aide de la
récurrence suivante :
\[
p_{t+1} = p_t + \alpha (D(p_t)-S(p_t))
\]
Déterminer le point fixe de cette récurrence, le prix $\bar p$ tel que
$\bar p = \bar p + \alpha (D(\bar p)-S(\bar p))$. Comparer $\bar p$ et
$p^{\star}$. \textbf{(3)} Supposons que le prix initial $p_1$ soit
différent de $\bar p$. Exprimer $p_t$ en fonction de $p_0$,
et $\alpha$. \textbf{(4)} Montrer que la chronique de prix converge de
façon monotone vers $\bar p$ si $0<\alpha<\frac{1}{2}$. \textbf{(5)}
Quelles sont les prédictions du modèle si $\alpha$ est en dehors de
cet intervalle ?

\bigskip

\exercice{7} Identifier les limites suivantes :
\begin{enumerate}
\item $\lim_{x\rightarrow\infty} \frac{2x+5}{x^2-3}$
\item $\lim_{x\rightarrow\infty} \frac{x^3-4x^2+8}{x^2+6}$
\item $\lim_{x\rightarrow\infty} \frac{ax^2+bx+c}{kx^2+lx+m}$
\item $\lim_{x\rightarrow -4} \frac{x^2-16}{x+4}$
\item $\lim_{x\rightarrow 0^+} \frac{|x|}{x}$ et $\lim_{x\rightarrow 0^-} \frac{|x|}{x}$
\end{enumerate}

\bigskip

\exercice{8} En utilisant la définition de la dérivée, calculer les
dérivées des fonctions suivantes :
\begin{enumerate}
\item $f(x) = 4x^2+3$
\item $g(x) = x^n$, $\forall\quad n\in \mathbb N$ et $\forall\quad x\in \mathbb R$
\item $h(x) = \frac{1}{x}$, $\forall x\in \mathbb R^*$
\end{enumerate}
Pour $g(x)$ vous utiliserez la formule du binôme de Newton (une
généralisation de l'identité remarquable bien connue) :
\[
(a+b)^n = \sum_{k=0}^n C_n^ka^{n-k}b^k
\]
avec
\[
C_n^k = \frac{n!}{k!(n-k)!}
\]
où $m! = m\times(m-1)\times(m-2)\times\dots\times 3\times 2\times 1$
la fonction factorielle.


\end{document}



% Exo. 7 (1)
2q = 3
q = 3/2
p = 1/2

% Exo.7 (2)-(3)-(4)

\[
p_{t+1} = p_t  + \alpha \left(2-p_t-1-p_t\right)
\]

\[
\Leftrightarrow p_{t+1} = (1-2\alpha)p_t  + \alpha
\]
En 1, on a :
\[
p_{1} = (1-2\alpha)p_0  + \alpha
\]
En 2, on a :
\[
p_{2} = (1-2\alpha)p_1  + \alpha
\]
Soit par substitution :
\[
p_{2} = (1-2\alpha)\left((1-2\alpha)p_0  + \alpha\right)  + \alpha
\]
\[
\Leftrightarrow p_{2} = (1-2\alpha)^2 p_0   + \alpha (1 + (1-2\alpha))
\]
Supposons qu'à la date $t$, nous avons:
\[
p_{t} = (1-2\alpha)^t p_0   + \alpha \sum_{i=0}^{t-1}(1-2\alpha)^i
\]
et montrons que l'on a nécessairement
\[
\Leftrightarrow p_{t+1} = (1-2\alpha)^{t+1} p_0   + \alpha \sum_{i=0}^{t}(1-2\alpha)^i
\]
Il s'agit d'une preuve par récurrence. Nous avons:
\[
p_{t+1} = (1-2\alpha)p_t  + \alpha
\]
En substituant l'expression au rang $t$:
\[
p_{t+1} = (1-2\alpha)(1-2\alpha)^t p_0   + (1-2\alpha)\alpha\sum_{i=0}^{t-1}(1-2\alpha)^i + \alpha
\]
\[
\Leftrightarrow p_{t+1} = (1-2\alpha)^{t+1} p_0   + \alpha\sum_{i=0}^{t-1}(1-2\alpha)^{i+1} + \alpha
\]
\[
\Leftrightarrow p_{t+1} = (1-2\alpha)^{t+1} p_0   + \alpha\sum_{i=1}^{t}(1-2\alpha)^{i} + \alpha
\]
\[
\Leftrightarrow p_{t+1} = (1-2\alpha)^{t+1} p_0   + \alpha\sum_{i=0}^{t}(1-2\alpha)^{i}
\]
Ainsi, si l'expression du prix est vraie au rang $t$ elle est
nécessairement vraie au rang $t+1$.

Nous avons donc :
\[
p_{t} = (1-2\alpha)^t p_0   + \alpha \sum_{i=0}^{t-1}(1-2\alpha)^i
\]

Si $\alpha$ est positif et plus petit que $\frac{1}{2}$ le premier
terme converge vers zéro de façon monotone (car le nombre sous la
puissance $t$ est positif et inférieur à un). On reconnaît une série
géométrique dans le second terme (ie les sommes des termes d'une suite
géométrique). Nous avons vu dans la première fiche de TD que:
\[
\sum_{i=0}^{n-1}q^i = \frac{1-q^{n}}{1-q}
\]
Clairement, si $|q|<1$ on a :
\[
\lim_{n\rightarrow\infty}\sum_{i=0}^{n-1}q^i = \frac{1}{1-q}
\]
Dans notre cas $q = 1-2\alpha$ vérifie $0<q<1$, nous avons la limite
du second terme :
\[
\lim_{t\rightarrow\infty} \alpha \sum_{i=0}^{t-1}(1-2\alpha)^i =
\alpha \lim_{t\rightarrow\infty} \sum_{i=0}^{t-1}(1-2\alpha)^i =
\alpha \lim_{t\rightarrow\infty} \frac{1-(1-2\alpha)^t}{1-(1-2\alpha)}
= \frac{\alpha}{2\alpha} = \frac{1}{2}
\]
Si $\alpha\in ]0,.5[$ la dynamique d'ajustement du prix nous ramène de
façon monotone au prix d'équilibre.

Si $\alpha$ est négatif, c'est-à-dire si les prix diminuent lorsque la
demande est supérieure à l'offre, la dynamique des prix diverge
puisque $(1-2\alpha)^t$ tend vers l'infini.

Si $\alpha=\frac{1}{2}$ le prix à la date $t$ est indépendant du prix
à la date $t-1$. Il est toujours égal à $1/2$.

Si $\alpha>\frac{1}{2}$ mais inférieur à 1, on obtient une dynamique
oscillante vers l'équilibre.

Si $\alpha> 1$ la dynamique d'ajustement diverge (avec des
oscillations).
