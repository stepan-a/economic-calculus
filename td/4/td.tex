\documentclass[10pt,a4paper,notitlepage]{article}
\usepackage{amsmath}
\usepackage{amssymb}
\usepackage{amsbsy}
\usepackage{float}
\usepackage[french]{babel}
\usepackage{graphicx}
\usepackage{enumerate}

\usepackage{palatino}

 \usepackage[active]{srcltx}
\usepackage{scrtime}

\newcounter{xnumber}
\setcounter{xnumber}{0}

\newcommand{\exercice}{\textsc{\textbf{Exercice}} \textbf{\addtocounter{xnumber}{1}\thexnumber}\,\,}
\newcommand{\question}[1]{\textbf{(#1)}}
\setlength{\parindent}{0cm}

\begin{document}

\title{\textsc{Calcul Économique\\ \small{(Fiche de TD n°4)}}}
\author{Stéphane Adjemian\thanks{Université du Mans. \texttt{stephane DOT adjemian AT univ DASH lemans DOT fr}}}
\date{Le \today\ à \thistime}

\maketitle

\exercice Calculer les dérivées des fonctions suivantes :
\begin{enumerate}[(i)]
\item $f(x) = \log (x^2+x^4+1)$
\item $g(x) = x^2\log (x^2+x^4+1)$
\item $h(x) = e^{2x}$
\item $j(x) = \log \left(\frac{x^3-2}{x^2+1}\right)$
\item $k(x) = \cos(\theta x)$
\end{enumerate}

\bigskip

\exercice Trouver l'expression générale de la dérivée d'ordre $n$ des
fonctions suivantes :
\begin{enumerate}[(i)]
\item $f(x) = e^{\theta x}$
\item $g(x) = \frac{1}{x}$
\item $h(x) = \log (x)$
\end{enumerate}

\bigskip

\exercice Montrer qu'il est possible d'écrire la fonction
exponentielle sous la forme :
\[
e^{x} = \sum_{i=0}^{\infty} \frac{x^n}{n!}
\]
En déduire une approximation de la constante $e$.

\bigskip

\exercice Montrer l'égalité suivante dans un voisinage de 0 :
\[
\frac{1}{1-x} = \sum_{i=0}^{\infty} x^n
\]
Quel est le radius de convergence ?.

\bigskip

\exercice Faire une étude de la fonction (en identifiant les optima) :
\[
f(x) = -x^3+x^2+2x
\]

\bigskip

\exercice Faire une étude de la fonction :
\[
f(x) = \frac{(\log x)^2}{x}
\]

\bigskip

\exercice Montrer que si la fonction $f(x)=ax^3+bx^2+cx+d$ admet
deux extrema, alors l'un est un maximum et l'autre un minimum.

\bigskip

\exercice Déterminer les limites des fonctions suivantes :
\begin{enumerate}
\item $\lim_{x\rightarrow\infty} \frac{x^2}{e^x}$
\item $\lim_{x\rightarrow\infty} \frac{\log x}{x}$
\item $\lim_{x\rightarrow\infty} \frac{e^x+3x^2}{4e^x+2x^2}$
\item $\lim_{x\rightarrow 1} \frac{3x\log x}{x^2-x}$
\end{enumerate}

\bigskip

\exercice La somme de deux nombres positifs est égale à
100. Trouver les couples de nombres tels que :
\begin{enumerate}
\item Le produit de ces nombres est maximal.
\item La somme des carrés est minimale.
\end{enumerate}


\end{document}