\documentclass[9pt,professionalfonts,handout,hyperref]{beamer}
\mode<presentation>
\usetheme{Frankfurt}
\usecolortheme{seahorse}
\usepackage[french]{babel}
\usepackage{times}
\usepackage{caption}
\usepackage[T1]{fontenc}
\usepackage{multirow}
\usepackage{multicol}
\usepackage{palatino}
\usepackage{eulervm}
\usepackage{epstopdf}

\title{Calcul Economique 1\\Fiche de TD 2}
\author{F.~Karamé}
\date{Année universitaire 2021-2022}
\subject{Talks}

\setbeamertemplate{navigation symbols}{}

\begin{document}

\begin{frame}
  \titlepage
\end{frame}

\section{Exercice 1}

% \exercice{1} La demande de montres SLOUK est de 10 unités si le prix est
% égal à 160 euros et elle est de 20 unités si le prix est 120
% euros. Calculer la fonction de demande supposée linéaire.
%
% \bigskip
%

\begin{frame}

    La demande de montres étant supposée linéaire, elle est de la forme : $$q = a + bp $$ avec $p$ les prix et $q$ les quantités (les prix déterminent les quantités demandées).\newline

    \pause Les points $(q=10,p=160)$ et $(q=20,p=120)$ vérifient l'équation de demande. \pause Donc :
    \[
      \begin{aligned}
        \pause 10 \pause &= \pause a + \pause b\times 160  \\
        \pause 20 \pause &= \pause a + \pause b\times 120
      \end{aligned}
    \]

    \pause On résout en combinant les deux équations, par exemple en les soustrayant : \pause cela fera disparaître $a$ et on aura donc une équation à une inconnue $b$ :

    \[
      \begin{aligned}
        \pause &10-20 = a-a + 160b -120b \\
        \pause \Leftrightarrow&-10 = 40b \pause \Leftrightarrow b = -\frac{10}{40} \pause = -\frac{1}{4} \\
      \end{aligned}
    \]

    \pause On en déduit $a$ :
    \[
      \begin{aligned}
        \pause 10 \pause &= \pause a -\frac{1}{4}\times 160\pause \Leftrightarrow 10 = a - 40 \pause \Leftrightarrow  a = 50\\
        \pause 20 \pause &= \pause a -\frac{1}{4} \times 120\pause \Leftrightarrow 20 = a - 30 \pause \Leftrightarrow  a = 50
      \end{aligned}
    \]

    \pause Conclusion : l'équation de demande est $q = 50 -\frac{1}{4}p $.
  \end{frame}

  \section{Exercice 2}

  % \exercice{2} Quand le prix est de 100 euros la quantité d'appareils
  % photos de marque PISTOL offerte sur le marché est 50 unités. Quand le
  % prix est 50\% plus élevé le nombre d'unités offertes est de
  % 100. Calculer la fonction d'offre supposée linéaire.
  %
  % \bigskip
  \begin{frame}

	L'offre de montres étant supposée linéaire, elle est aussi de la forme : $$q = a + bp $$ avec $p$ les prix et $q$ les quantités (les prix déterminent les quantités offertes).\newline

	\pause D'après l'énoncé, les points $(q=50,p=100)$ et $(q=100,p=150)$ vérifient l'équation d'offre. \pause Donc :
	\[
	\begin{aligned}
\pause 	50 \pause &= a \pause + b\times 100 \\
\pause 	100 \pause &= a \pause + b\times 150
	\end{aligned}
	\]

\pause 	On résout en combinant les deux équations, par exemple en les soustrayant : cela fera disparaître $a$ et on aura donc une équation à une inconnue $b$ :

	\[
	\begin{aligned}
\pause 	&100-50 \pause = a-a \pause + 150b -100b
\pause 	\Leftrightarrow50 = 50b 	\pause \Leftrightarrow b = 1 \\
	\end{aligned}
	\]

\pause 	On en déduit $a$ :
	\[
	\begin{aligned}
\pause 	50 \pause &= a \pause + 100\times 1 \pause \Leftrightarrow  a = -50\\
\pause 	100 \pause &= a \pause + 150 \times 1 \pause \Leftrightarrow  a = -50
	\end{aligned}
	\]

\medskip
\pause Conclusion : l'équation d'offre est $q = -50 +p $.

\end{frame}

\section{Exercice 3}

%\exercice{3} Sur un marché, la demande et l'offre pour un bien sont
%caractérisés par :
%\[
%\begin{split}
%D(p): q &= -2 p  + 6\\
%S(p): q &= \frac{1}{2} p  + 1
%\end{split}
%\]
%où $p$ est le prix du bien et $q$ sa quantité. Calculer la quantité
%d'équilibre et le prix d'équilibre.
%
%\bigskip
%

\begin{frame}

\pause Le prix d'équilibre $p^{\star}$ permet d'égaliser l'offre et la demande sur un marché : $$D(p^{\star}) = S(p^{\star})$$
\pause D'après l'énoncé,

\[\begin{aligned}
\pause D(p^{\star}) = S(p^{\star})
\pause \Leftrightarrow& -2 p^{\star}  + 6 = \frac{1}{2} p^{\star}  + 1\\
\pause \Leftrightarrow& -2 p^{\star} -\frac{1}{2} p^{\star} \pause  = 1 - 6\\
\pause \Leftrightarrow& -\frac{5}{2} p^{\star}  = -5\\
\pause \Leftrightarrow& \frac{5}{2} p^{\star}  = 5\\
\pause \Leftrightarrow& \frac{1}{2} p^{\star}  = 1\\
\pause \Leftrightarrow& p^{\star}  = 2\\
\end{aligned}\]

\pause On en déduit les quantités d'équilibre :

\[\begin{aligned}
\pause D(p^\star): \pause q^\star \pause &= -2 p^\star  + 6 \pause = -2\times 2 +6 \pause = 2\\
\pause S(p^\star): \pause q^\star \pause &= \frac{1}{2} p^\star  + 1 \pause = \frac{1}{2}\times 2  + 1 \pause = 2
\end{aligned}\]

\pause Conclusion : à l'équilibre, le prix est $p^\star = 2$ et les quantités échangées $q^\star = 2$.

\end{frame}


%\section{Exercice 4}

%\exercice{4} Supposons que la consommation agrégée dans une économie,
%notée $C$, soit une fonction linéaire du revenu disponible (hors
%taxes), noté $Y$. Supposons qu'il existe un niveau de consommation
%incompressible, noté $C_0$. Il s'agit du niveau de consommation
%observé même si le revenu disponible est nul. On supposera que lorsque
%le revenu augmente de $x$, la consommation en écart à son niveau
%incompressible, \emph{ie} $C-C_0$, augmente de $0,8x$. Déterminer la
%forme de la fonction de consommation.
%
%\bigskip
%
\section{Exercice 5}

%\exercice{5} Montrer que la fonction $f(x) = x^2+2x+1$ admet un unique
%minimum en $x=-1$.
%
%\bigskip
%
\begin{frame}

\pause On remarque que $f(x) = x^2+2x+1$ est une identité remarquable : \pause $$f(x) = x^2+2x+1= (x+1)^2$$


\pause On a exprimé le polynôme comme un polynôme d'ordre 1 au carré. \pause Donc il n'y a qu'une racine et $f(x)\ge0$. \newline

\pause $f(x)=0$ est donc la valeur minimale prise par la fonction si et seulement si $x=-1$. \newline

\pause Donc $x=-1$ est le minimum unique de la fonction.

\end{frame}


\section{Exercice 6}

%\exercice{6} Sur un marché, la demande et l'offre pour un bien sont
%caractérisés par :
%\[
%\begin{split}
%D(p): q &= -2 p^2  + 3\\
%S(p): q &= p^2 + 5p +2
%\end{split}
%\]
%où $p$ est le prix du bien et $q$ sa quantité (on s'intéresse aux
%valeurs positives de $p$ et $q$). Calculer la quantité
%d'équilibre et le prix d'équilibre.
%
%\bigskip
%

\begin{frame}

Le prix d'équilibre $p^{\star}$ permet d'égaliser l'offre et la demande sur un marché :
	\[\begin{aligned}
\pause 	D(p^{\star}) = S(p^{\star})
\pause 	\Leftrightarrow&-2 p^{\star2}  + 3 \pause = p^{\star2} + 5p^{\star} +2\\
\pause 	\Leftrightarrow& 3 p^{\star2} + 5p^{\star} -1=0\\
	\end{aligned}\]

\pause On a tout regroupé d'un coté pour égaliser à 0. \pause On est en présence d'un polynôme d'ordre 2 en $p^{\star}$ à résoudre.\newline

\pause Calculons le discriminant. \pause Pour cela, on identifie le polynôme général $ax^{2} + bx +c=0$ au polynôme étudié \pause : il vient $a = 3$, $ b=5$ et $c=-1$. \pause Le discriminant s'écrit :

\[
\Delta = b^2 -4ac \pause = 5^2\pause -4\times \pause 3 \pause \times (-1) \pause = 25+12 \pause = 37 \pause >0
\]

\pause Comme $\Delta>0$, il y a deux racines réelles :

\[\begin{aligned}
\pause &p_1^{\star} = \frac{-b + \sqrt{\Delta}}{2a} \pause = \frac{-5 + \sqrt{37}}{2\times3}\pause >0\\
\pause &p_2^{\star} = \frac{-b - \sqrt{\Delta}}{2a} \pause = \frac{-5 - \sqrt{37}}{2\times3}\pause <0\\
\end{aligned}\]

\pause La seconde racine étant négative, la seule solution acceptable est $p^{\star} = \frac{-5 + \sqrt{37}}{2\times3} \pause \approxeq \frac{1}{6}$.

\end{frame}


\section{Exercice 7}

%\exercice{7} Sans calculer le discriminant, montrer que l'équation
%$x^2-2x+2 = 0$ n'admet pas de solution réelle.
%
%\bigskip
%

\begin{frame}

$x^2-2x+2 = 0$ ressemble à l'identité remarquable \pause $x^2-2x+1 = 0$.\newline

En modifiant :

\[\begin{aligned}
\pause x^2-2x+2
& = \underbrace{x^2-2x+1} \pause +1 \\
\pause & = (x-1)^2 +1 \\
\end{aligned}\]

Comme on additionne deux nombres positifs, $x^2-2x+2>0$ pour tout $x$ réel. \newline

\end{frame}

\begin{frame}

Calculons le discriminant. \pause Pour cela, on identifie le polynôme général $ax^{2} + bx +c=0$ au polynôme étudié : \pause il vient $a = 1$, \pause $ b=-2$ \pause et $c=2$.

\[
\Delta = b^2 -4ac \pause = (-2)^2\pause -4\times \pause 1 \pause \times 2 \pause = 4-8 \pause = -4 \pause <0
\]

Comme $\Delta<0$, il n'y a pas de racines réelles. \pause Mais en passant dans l'ensemble des complexes \pause où les nombres sont de la forme $a+ib$ \pause et où il existe le nombre imaginaire $i$ \pause tel que : $i^2=-1$ (!!!) :

\[
\Delta = -4 \pause = 4\times (-1) \pause = 4i^2 \pause = (2i)^2
\]

\pause La formule des racines fonctionne toujours mais les deux racines sont complexes :

\[\begin{aligned}
&x_1 = \frac{-b + \sqrt{\Delta}}{2a} \pause = \frac{2 + \sqrt{4i^2}}{2}\pause = \frac{2 + 2i}{2} \pause = 1+i\\
&x_2 = \frac{-b - \sqrt{\Delta}}{2a} \pause = \frac{2 - \sqrt{4i^2}}{2} \pause = \frac{2 - 2i}{2} \pause = 1-i\\
\end{aligned}\]

\pause Ce polynôme n'a pas de racine réelle \pause mais deux racines complexes $(1+i)$ et $(1-i)$.

\end{frame}

\section{Exercice 8}

%\exercice{8} Chercher les solutions des équations suivantes :
%\begin{enumerate}[(i)]
%	\item $x^3-2x^2+2x = 0$
%	\item $x^3+2x^2-x-2 = 0$
%	\item $x^4-5x^2+4 = 0$
%	\item $x^2 - 2\sqrt{2}x + 2 = 0$
%\end{enumerate}
%
%\bigskip


\begin{frame}

$$x^3-2x^2+2x = 0$$

\pause Pas de règle pour les polynômes d'ordre 3.\newline

\pause Ici, pas de terme constant donc factorisons par $x$ :

\[\begin{aligned}
&x^3-2x^2+2x =\pause x\pause (x^2\pause -2x\pause +2) \\
\end{aligned}\]

\pause A l'exercice précédent, on a vu que le polynôme $x^2-2x+2 = 0$ n'a pas de racine réelle.\newline

\pause La seule solution pour $x(x^2-2x+2) = 0$ \pause est donc $x=0$.

\end{frame}


\begin{frame}

$$x^3+2x^2-x-2 = 0$$

\pause Pas de règle pour les polynômes d'ordre 3. \pause Il faut trouver des racines évidentes. \pause Ici, il est clair que $x=1$ vérifie l'égalité :

\[\begin{aligned}
&(1)^3\pause +2(1)^2\pause -(1)\pause -2 = 0
\end{aligned}\]

On peut donc factoriser par $(x-1)$. \pause Partant d'un polynôme d'ordre 3, il reste donc un polynôme d'ordre 2 :

\[\begin{aligned}
&x^3+2x^2-x-2 = (x-1)(ax^2+bx+c) \\
\end{aligned}\]

\pause Pour trouver $a$, $b$, $c$, on développe le terme de droite, on regroupe par puissance et on identifie au terme de gauche :

\[\begin{aligned}
\pause (x-1)(ax^2+bx+c)
&= \pause ax^3\pause +bx^2\pause +cx \pause - ax^2\pause -bx\pause -c\\
&= ax^3\pause +(b-a)x^2\pause +(c-b)x \pause -c\\
\pause &= x^3+2x^2-x-2\end{aligned}\]

\end{frame}

\begin{frame}

\[\begin{aligned}
ax^3+(b-a)x^2+(c-b)x -c = x^3+2x^2-x-2
\end{aligned}\]

\bigskip

Il vient :

\[\begin{aligned}
\begin{cases}
	a   & = 1 \\
	b-a & = 2 \\
	c-b & =-1 \\
	-c  & =-2
\end{cases}
\pause \Leftrightarrow
\begin{cases}
	a           & = 1 \\
	b-\alert{1} & = 2 \\
	c           & =2  \\
	\alert{2}-b & =-1
\end{cases}
\pause \Leftrightarrow
\begin{cases}
	a= 1         \\
	\alert{b= 3} \\
	c =2
\end{cases}
\end{aligned}\]

\bigskip
\pause Donc :
\[\begin{aligned}
\pause x^3+2x^2-x-2 = (x-1)(x^2+3x+2)
\end{aligned}\]

\bigskip

\pause Reste à factoriser $x^2+3x+2 = 0$, soit en repérant les racines évidentes, soit avec le discriminant.
\end{frame}

\begin{frame}

Avec le discriminant : $x^2+3x+2 = 0$ implique \pause $a=1$, \pause $b=3$ et \pause $c=2$.

\[
\Delta = b^2 -4ac \pause = 3^2 \pause - 4 \pause \times \pause 1 \pause \times 2 \pause = 9-8 \pause = 1 \pause >0
\]


\pause Comme $\Delta>0$, il y a deux racines réelles :

\[\begin{aligned}
&x_1 = \frac{-b + \sqrt{\Delta}}{2a} \pause = \frac{-3 + \sqrt{1}}{2\times1} \pause = -1\\
&x_2 = \frac{-b - \sqrt{\Delta}}{2a} \pause = \frac{-3 - \sqrt{1}}{2\times1} \pause = -2\\
\end{aligned}\]

\pause On peut donc factoriser par \pause $(x-(-1)) \pause = (x+1)$ \pause et $(x-(-2)) \pause = (x+2)$.

\[\begin{aligned}
\pause x^3+2x^2-x-2 = (x-1)(x+1)(x+2)
\end{aligned}\]

\bigskip
\pause Les racines du polynôme sont donc : $x=1$, $x=-1$ et $x=-2$.

\end{frame}

\begin{frame}

$$x^4-5x^2+4 = 0$$

\pause Ici il n'y a que des puissances paires. \pause Posons $X = x^2$.

\[\begin{aligned}
\pause X^2-5X+4 = 0
\end{aligned}\]

\pause Il devient facile de résoudre en $X$, avec le discriminant ou avec des racines évidentes : \pause $X=1$ et $X=4$.\newline
\pause Factorisons et repassons à $x$ :

\[\begin{aligned}
X^2-5X+4
\pause &= (X-1)(X-4) \\
\pause &= (x^2-1)(x^2-4) \\
\end{aligned}\]

On reconnaît l'identité remarquable : \pause $a^2-b^2 \pause = (a-b)(a+b)$.\\

\[\begin{aligned}
X^2-5X+4
\pause &= (x^2-1^2)(x^2-2^2) \\
\pause &= (x-1)(x+1)(x-2)(x+2) \\
\end{aligned}\]

\pause Donc les racines sont $x = -2$, $x = -1$, $x = 1$ et $x = 2$.

\end{frame}

\begin{frame}

$$x^2 - 2\sqrt{2}x + 2 = 0$$

\bigskip

\pause Ici on reconnaît l'identité remarquable : \pause $a^2 -2ab +b^2 \pause = (a-b)^2$.

\[\begin{aligned}
&x^2 - 2\sqrt{2}x + 2  = 0\\
\pause \Leftrightarrow&(x -\sqrt{2})^2 = 0
\end{aligned}\]

\pause Donc $x = \sqrt{2}$ est racine double.
\end{frame}

%
%\exercice{9} Représenter graphiquement à l'aide d'un tableur les
%fonctions suivantes :
%\begin{enumerate}[(i)]
%	\item $f(x) = x^5- 5x^3 + 4x$
%	\item $f(x) = \frac{x-\frac{1}{2}}{x^3-x}$
%\end{enumerate}
%Et déterminer graphiquement les valeurs de $x$ telles que $f(x)=0$
%dans chaque cas.
%
%\bigskip

\section{Exercice 10}

%\exercice{10} Soit un ménage disposant d'un revenu de 100. On suppose
%qu'il ne peut acheter que des bananes et des carottes et que les prix
%de ces deux biens sont respectivement $p_B= 1$ et $p_C = \frac{1}{2}$
%(l'unité dans les deux cas est le kilogramme). \textbf{(1)} Supposons
%que le ménage décide de consommer la totalité de son revenu en
%achetant ces deux biens (on admet qu'il ne peut pas consommer une
%quantité négative de banane ou de carotte). Déterminer l'ensemble des
%couples de quantités $(q_B, q_C)$ cohérents avec cette
%hypothèse. \textbf{(2)} Comment cet ensemble est-il modifié si le
%ménage peut décider de ne pas consommer la totalité de son
%revenu. \textbf{(3)} Représenter graphiquement ces deux ensembles.
%
%\bigskip

\begin{frame}

Supposons que le ménage décide de consommer la totalité de son revenu en
achetant ces deux biens (on admet qu'il ne peut pas consommer une
quantité négative de banane ou de carotte). \newline

\pause L'ensemble des couples de quantités $(q_B, q_C)$ cohérents avec cette hypothèse :

$$ \left\{(q_B, q_C) \pause \in \mathbb{R}^{+}\times\mathbb{R}^{+} \pause ~~|~~\pause p_B\pause q_B \pause + p_C\pause q_C\pause =R\right\}$$

\pause C'est l'équation de la droite de budget dans le plan $(0,q_B,q_C)$.

\bigskip

\pause Si le ménage décide de ne pas consommer la totalité de son revenu :

$$ \left\{(q_B, q_C) \pause \in \mathbb{R}^{+}\times\mathbb{R}^{+} \pause ~~|~~p_Bq_B + p_Cq_C\pause <R\right\}$$


\pause Cet ensemble est le demi-plan de consommation sous la droite de budget dans le plan $(0,q_B,q_C)$.

\end{frame}


\section{Exercice 11}

%\exercice{11} (Suite de l'exercice 9) Supposons que l'on puisse
%quantifier la satisfaction du ménage à l'aide de la fonction :
%\[
%u(q_B, q_C) = q_B + q_C
%\]
%la satisfaction (ou l'utilité) du ménage est d'autant plus élevée que
%celui-ci consomme plus. \textbf{(0)} Interpréter cette
%fonction. \textbf{(1)} Supposons que le ménage souhaite atteindre un
%niveau de satisfaction égal à 300. Déterminer l'ensemble des couples
%$(q_B, q_C)$ compatibles avec cet objectif. \textbf{(2)} Compléter le
%graphique construit dans l'exercice 9 en représentant cet ensemble.
%\textbf{(3)} Étant donnée la contrainte de revenu définie dans
%l'exercice 9, déterminer si le ménage peut atteindre ce niveau de
%satisfaction. \textbf{(4)} Déterminer le choix du ménage (en termes de
%quantités demandées pour les bananes et les carottes) en supposant que
%celui-ci cherche à atteindre le niveau de satisfaction le plus élevé
%possible. \textbf{(5)} On adopte maintenant une fonction de
%satisfaction plus générale de la forme :
%\[
%u(q_B, q_C) = \alpha q_B + \beta q_C
%\]
%où $\alpha$ et $\beta$ sont deux réels positifs. Déterminer les choix
%du ménage en fonction de $\alpha$ et $\beta$. Commenter le résultat.
%

\begin{frame}

$$u(q_B, q_C) = q_B + q_C$$

\bigskip

\begin{itemize}
\item[-] traduction numérique des préférences d'un consommateur, \newline

\item[-] mesure la satisfaction du consommateur, \newline

\item[-] fonction croissante en $q_B$ et $q_C$.\newline

\item[-] donc plus on consomme, plus on est content.

\end{itemize}
\end{frame}

\begin{frame}

Déterminer l'ensemble des couples $(q_B, q_C)$ permettant d'atteindre un niveau de satisfaction égal à 300 :

$$ \left\{\pause (q_B, q_C) \pause \in \mathbb{R}^{+}\times\mathbb{R}^{+} \pause ~~|~~u(q_B, q_C) \pause = q_B + q_C \pause =300 \right\} $$

\bigskip

\pause C'est une droite dans le plan $(0,q_B,q_C)$.

\end{frame}

\begin{frame}

\[\begin{aligned}
\pause \text{ Satisfaction :   }&\left\{(q_B, q_C) \in \mathbb{R}^{+}\times\mathbb{R}^{+} ~~|~~u(q_B, q_C) = q_B + q_C = 300 \right\} \\
\pause \text{ Budget :   }&\left\{(q_B, q_C) \in \mathbb{R}^{+}\times\mathbb{R}^{+} ~~|~~p_Bq_B + p_Cq_C=R\right\}
\end{aligned}\]

\bigskip

\pause On est en présence de deux droites dans le plan $(0,q_B,q_C)$.

\[\begin{aligned}
\pause \text{ Satisfaction :   }&q_B + q_C = 300 \pause \Leftrightarrow q_C = 300-q_B \\
\pause \text{ Budget :   }&p_Bq_B + p_Cq_C=R \pause \Leftrightarrow q_C = \frac{R}{p_C} -\frac{p_B}{p_C}q_B
\end{aligned}\]

\pause Le ménage peut atteindre son objectif d'utilité de 300 \pause si l'intersection entre les deux droites $(q_B^\star,q_C^\star)$ existe \pause ET est acceptable.

\end{frame}


\begin{frame}


Avec l'application numérique $R =100$, $p_B= 1$ et $p_C = \frac{1}{2}$ :

\[\begin{aligned}
\pause \text{ Satisfaction :   }&q_C = 300-q_B \\
\pause \text{ Budget :   }q_C = \frac{100}{1/2} -\frac{1}{1/2}q_B\pause \Leftrightarrow &q_C = 200-2q_B
\end{aligned}\]

\pause L'intersection vérifie :

\[\begin{aligned}
\pause &300-q_B^\star \pause = 200-2q_B^\star \pause \Leftrightarrow \pause 2q_B^\star-q_B^\star \pause = 200-300 \pause \Leftrightarrow q_B^\star = -100
\end{aligned}\]

\bigskip

\pause Impossible pour une quantité !

\end{frame}


\begin{frame}

Réécrivons le problème avec un niveau d'utilité inconnu $U$ :

	\[\begin{aligned}
\pause 	\text{ Satisfaction :   }&\left\{(q_B, q_C) \in \mathbb{R}^{+}\times\mathbb{R}^{+} ~~|~~u(q_B, q_C) = q_B + q_C = \alert{U} \right\} \\
\pause 	\text{ Budget :   }&\left\{(q_B, q_C) \in \mathbb{R}^{+}\times\mathbb{R}^{+} ~~|~~p_Bq_B + p_Cq_C=R\right\}
	\end{aligned}\]

	\bigskip

\pause 	On est en présence de deux droites dans le plan $(0,q_B,q_C)$. Avec l'application numérique $R =100$, $p_B= 1$ et $p_C = \frac{1}{2}$ :

	\[\begin{aligned}
\pause 	\text{ Satisfaction :   }&q_B + q_C = U \pause \Leftrightarrow q_C = U-q_B \\
\pause 	\text{ Budget :   }&q_C = 200-2q_B
	\end{aligned}\]

\pause  En substituant :

	\[\begin{aligned}
\pause 	&U-q_B^\star = 200-2q_B^\star \pause \Leftrightarrow U = 200-q_B^\star
	\end{aligned}\]

\pause Si on veut $U$ maximum, \pause il faut que $q_B^\star = 0$ \pause (et donc $U=200$). \pause Donc l'agent ne consomme pas de bananes \pause mais que des carottes : \pause $$q_C^\star = 200-2q_B^\star \pause = 200$$

\end{frame}


\begin{frame}

	Soit la fonction d'utilité $u(q_B, q_C) = \alpha q_B + \beta q_C$ avec $(\alpha>0, \beta>0)$ :

	\[\begin{aligned}
\pause 	\text{ Satisfaction :   }&\left\{(q_B, q_C) \in \mathbb{R}^{+}\times\mathbb{R}^{+} ~~|~~u(q_B, q_C) = \alert{\alpha q_B + \beta q_C = U} \right\} \\
\pause 	\text{ Budget :   }&\left\{(q_B, q_C) \in \mathbb{R}^{+}\times\mathbb{R}^{+} ~~|~~p_Bq_B + p_Cq_C=R\right\}
	\end{aligned}\]

	\bigskip

\pause 	On est en présence de deux droites dans le plan $(0,q_B,q_C)$. Avec l'application numérique $R =100$, $p_B= 1$ et $p_C = \frac{1}{2}$ :

	\[\begin{aligned}
\pause 	\text{ Satisfaction :   }\alpha q_B + \beta q_C = U \Leftrightarrow &q_C = \frac{U}{\beta}-\frac{\alpha}{\beta}q_B \\
\pause 	\text{ Contrainte de budget : }&q_C = 200-2q_B
	\end{aligned}\]

\bigskip

\pause On va résoudre graphiquement en fonction des pentes relatives de la droite d'utilité $(-\frac{\alpha}{\beta})$ et de la droite budgétaire $(-2)$.

\end{frame}

\begin{frame}

\[\begin{aligned}
\text{ Satisfaction :   }\alpha q_B + \beta q_C = U \Leftrightarrow &q_C = \frac{U}{\beta}-\frac{\alpha}{\beta}q_B \\
\text{ Contrainte de budget :   }&q_C = 200-2q_B
\end{aligned}\]

\bigskip

\begin{itemize}
\pause 	\item si $-\frac{\alpha}{\beta}>-2$ (donc $\frac{\alpha}{\beta}<2$) : \pause la droite d'utilité est moins pentue que la droite de budget \pause : donc $q_C^\star=200$ et $q_B^\star=0$ pour une utilité maximum...\newline

\pause 	\item si $-\frac{\alpha}{\beta}<-2$ (donc $\frac{\alpha}{\beta}>2$) : \pause la droite d'utilité est plus pentue que la droite de budget \pause : donc $q_C^\star=0$ et $q_B^\star=100$ pour une utilité maximum...\newline

\pause 	\item si $-\frac{\alpha}{\beta}=-2$ (donc $\frac{\alpha}{\beta}=2$) : \pause même pente, même droite et donc infinité de solutions possibles.
\end{itemize}

\end{frame}


\section{Exercice 12}

%\exercice{12} On reprend les exercices 10 et 11 en supposant que $p_B =
%p_C = 1$ et que $u(q_B, q_C) = \sqrt{q_Bq_C}$. \textbf{(1)} Exprimer
%$q_C$ en fonction du niveau d'utilité, $\bar u$ par exemple, et de
%$q_B$. La fonction ainsi construite est la fonction d'iso-utilité :
%l'ensemble des points $(q_C,q_B)$ vérifiant la contrainte définie par
%cette fonction fournissent un même niveau d'utilité. \textbf{(2)}
%Représenter graphiquement la contrainte budgétaire du ménage et les
%fonctions d'iso-utilité. \textbf{(3)} Identifier les demandes optimales
%du ménage.
%

\begin{frame}

Soit la fonction d'utilité $u(q_B, q_C) = \sqrt{q_Bq_C} = q_B^{1/2}q_C^{1/2}$, $p_B = p_C = 1$ et $R=100$.

\[\begin{aligned}
\pause \text{ Satisfaction :   }q_B^{1/2}q_C^{1/2} = U \Leftrightarrow q_C^{1/2} = \frac{U}{q_B^{1/2}} \Leftrightarrow &q_C = \frac{U^2}{q_B} \\
\pause \text{ Contrainte budgétaire :   }q_B+q_C = 100 \Leftrightarrow &q_C = 100-q_B
\end{aligned}\]

\bigskip

\pause Soit la solution $(q_B^\star,q_C^\star)$ le couple de quantités permettant de maximiser l'utilité sous la contrainte de budget. \pause C'est aussi le point d'intersection entre la droite de budget et la courbe d'utilité la plus haute. \pause Par substitution :

\[\begin{aligned}
\pause &q_C^\star = \frac{U^2}{q_B^\star} \pause = 100-q_B^\star\\
\pause \Leftrightarrow&  U^2 = q_B^\star(100-q_B^\star) \\
\pause \Leftrightarrow&  U^2 = 100q_B^\star -q_B^{\star 2} \\
\pause \Leftrightarrow&  q_B^{\star 2} -100q_B^\star + U^2 = 0\\
\end{aligned}\]

\pause Il faut trouver les racines du polynôme en $q_B^\star$.

\end{frame}

\begin{frame}

\[\begin{aligned}
q_B^{\star 2} -100q_B^\star + U^2 = 0\\
\end{aligned}\]

\pause Identifions les paramètres du polynôme \pause ($a=1$, \pause $b=-100$ \pause et $c=U^2$) \pause et calculons le discriminant $\Delta =0$ ici car on sait que la solution est unique :
\[
\Delta = b^2 -4ac \pause = 100^2 \pause - 4 \times \pause 1 \pause \times U^2 \pause = 100^2 - (2U)^2 \pause = (100- 2U)(100+ 2U) \pause = 0
\]

\pause Dans ce cas, $100- 2U = 0$ est la seule solution possible car $100+ 2U = 0$ implique $U<0$. \pause On en déduit donc $U = 50$. \newline

\pause Comme il n'y a qu'une racine, on a facilement (et sans $U$) :

\[\begin{aligned}
\pause &q_B^\star = \frac{-b}{2a} \pause = \frac{100}{2\times1} \pause = 50\\
\end{aligned}\]

\pause Et on en déduit facilement $q_C^\star$ \pause par la fonction d'utilité : \pause $$q_C^\star= \frac{U^2}{q_B^\star} \pause = \frac{50^2}{50} \pause = 50$$ \pause ou la contrainte de budget :
$$q_C^\star = 100-q_B^\star \pause = 100-50 \pause = 50$$

\end{frame}

\end{document}
