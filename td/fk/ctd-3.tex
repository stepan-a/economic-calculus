\documentclass[9pt,handout,hyperref]{beamer}
\mode<presentation>
\usetheme{Frankfurt}
\usecolortheme{seahorse}
\usepackage[french]{babel}
%\usepackage{times}
\usepackage{caption}
%\usepackage[T1]{fontenc}
\usepackage{multirow}
\usepackage{multicol}
\usepackage{palatino}
\usepackage{eulervm}
\usepackage{epstopdf}
\usepackage{nicefrac}

\title{Calcul Economique 1\\Fiche de TD 3}
\author{F.~Karamé}
\date{Année universitaire 2021-2022}
\subject{Talks}

\setbeamertemplate{navigation symbols}{}

\begin{document}

\begin{frame}
  \titlepage
\end{frame}

\section{Exercice 1}

% \exercice{1} Soit la suite de terme général $u_n = u_{n-1} + 1$ pour
% $n \geq 1$, avec la condition initiale $u_1 = 1$. \textbf{(1)} Donner
% une expression de $u_n$ en fonction du rang $n$ et de sa condition
% initiale. \textbf{(2)} Soit la suite $v_n = \sum_{i=1}^n u_i$ pour
% $n\geq 1$. Quelle est la condition initiale de cette suite ?
% Déterminer $v_n$.

\begin{frame}

    \[
      \begin{cases}
        u_n = u_{n-1} + 1~~\alert{\forall n \ge 2}\\
        u_1 = 1
      \end{cases}
    \]

    \pause Exprimons $u_n$ en fonction de $n$ et de sa condition initiale.\newline

    \pause On passe d'une expression du type "chaîne de nombres" $u_n = f(u_{n-1})$ \pause à une expression du type $u_n = g(n,u_1)$ uniquement en fonction du rang et de la condition initiale.\newline

    \pause La première expression signifie :

    \[\begin{aligned}
        u_n &= {\color{blue}{{u_{n-1}}}} + 1\\
        \pause {\color{blue}{u_{n-1}}} &= {\color{green}{u_{n-2}}} + 1\\
        \pause {\color{green}{u_{n-2}}} &= u_{n-3} + 1\\
        \pause \dots\\
        \pause u_{3} &= {\color{red}{u_{2}}} + 1\\
        \pause {\color{red}{u_{2}}} &= u_{1} + 1
      \end{aligned}\]

    \pause On réinjecte les expressions inférieures en cascade dans la première expression.

  \end{frame}

  \begin{frame}

    \[\begin{aligned}
        u_n &= u_{n-1} + 1\\
        u_{n-1} &= u_{n-2} + 1\\
        \pause \Rightarrow u_n &= (u_{n-2} + 1) + 1 \pause = u_{n-2} + 2
      \end{aligned}\]

    \pause Comme $u_{n-2} = u_{n-3} + 1$ :

    \[\begin{aligned}
        \pause u_n
        & = u_{n-2} + 2\\
        \pause & = (u_{n-3} + 1) + 2\\
        \pause & = u_{n-3} + 3
      \end{aligned}\]

    \pause Comme $u_{n-3} = u_{n-4} + 1$ :

    \[\begin{aligned}
        \pause u_n &= u_{n-3} + 3\\
        \pause & = (u_{n-4} + 1) + 3\\
        \pause & = u_{n-4} + 4
      \end{aligned}\]

    \pause Et ainsi de suite.

    \pause Il faut remarquer les relations entre les nombres et les indices...

  \end{frame}

  \begin{frame}

	\[\begin{aligned}
	u_{\alert{n}} &= u_{\alert{n-1}} + \alert{1}\\
	u_{\alert{n}} &= u_{\alert{n-2}} + \alert{2}\\
	u_{\alert{n}} &= u_{\alert{n-3}} + \alert{3}\\
	u_{\alert{n}} &= u_{\alert{n-4}} + \alert{4}\\
	\dots \\
 	u_n &= u_{\alert{1}} + \pause \alert{(n-1)}
	\end{aligned}\]

\pause de sorte que $1+n-1 = n$ aussi...

\bigskip
\pause Il vient donc :

\[\begin{aligned}
u_n
&= u_{1} + (n-1)\\
\pause & = 1+n-1 \\
\pause & = n
\end{aligned}\]

\bigskip
\pause Conclusion : $u_n = n$.

\end{frame}

\begin{frame}

On définit $v_n = \sum_{i=1}^n u_i$ pour $n\geq 1$.\newline

\pause Rappel : $v_{\alert{n}} = \sum_{i=1}^{\alert{n}} u_i = \pause u_1 \pause + u_2 \pause + u_3 \pause+ \dots + u_{n-1}  \pause + u_{\alert{n}}$. \newline

La condition initiale est $v_{\alert{1}} = \pause \sum_{i=1}^{\alert{1}} u_i \pause = u_{\alert{1}} \pause = 1$. \newline

\end{frame}

\begin{frame}

Déterminons $v_n$ : \pause si $u_n=n$, alors $u_i\pause =i$ \pause donc \newline

\[\begin{aligned}
v_n
\pause &= \sum_{i=1}^n u_i\\
\pause &= \sum_{i=1}^n i\\
\pause &= 1+2+3+\dots+n\\
\pause &= \frac{n(n+1)}{2}\\
\end{aligned}\]

(revoir fiche TD1).
\end{frame}

\section{Exercice 2}

%\bigskip
%
%\exercice{2} Soit la suite de terme général $u_n = \rho u_{n-1}$ pour
%$n \geq 1$ avec la condition initiale $u_1 = 1$ et $\rho$ un paramètre
%réel non nul. \textbf{(1)} Donner une expression de $u_n$ en fonction
%du rang $n$ et de sa condition initiale. \textbf{(2)} Discuter le
%comportement asymptotique de $u_n$ en fonction de la valeur de
%$\rho$. \textbf{(3)} Dans le cas où la suite admet une limite, combien
%de temps faut-il pour réduire de moitié la distance à la limite ?
%
%\bigskip
%

\begin{frame}

	\[
	\begin{cases}
	u_n = \rho u_{n-1} ~~\alert{\forall n \ge 2}\\
	u_1 = 1
	\end{cases}
	\]

\pause 	$u_n$ en fonction du rang $n$ et de sa condition initiale.\newline

\pause 	On passe d'une expression du type $u_n = f(u_{n-1})$ à une expression du type $u_n = g(n,u_1)$.\newline

\pause 	La première expression signifie :

	\[\begin{aligned}
	u_n &= {\color{blue}{{\rho u_{n-1}}}} \\
\pause 	{\color{blue}{u_{n-1}}} &= {\color{green}{\rho u_{n-2}}} \\
\pause 	{\color{green}{u_{n-2}}} &= \rho u_{n-3} \\
\pause 	\dots\\
\pause 	u_{3} &= {\color{red}{\rho u_{2}}}\\
\pause 	{\color{red}{u_{2}}} &= \rho u_{1}
	\end{aligned}\]

\pause On réinjecte les expressions inférieures en cascade dans la première expression.

\end{frame}

\begin{frame}

	\[\begin{aligned}
	u_n &= \rho u_{n-1}\\
	u_{n-1} &= \rho u_{n-2} \\
	\Rightarrow u_n &= \pause \rho (\rho u_{n-2}) \pause = \rho^2 u_{n-2}
	\end{aligned}\]

\pause 	Comme $u_{n-2} = \rho u_{n-3}$ :

	\[\begin{aligned}
\pause 	u_n &= \rho^2 u_{n-2}\\
\pause 	& = \rho^2 \pause (\rho u_{n-3})\\
\pause 	& = \rho^3 u_{n-3}
	\end{aligned}\]

\pause 	Comme $u_{n-3} = \rho  u_{n-4}$ :

	\[\begin{aligned}
\pause 	u_n &= \rho^3 u_{n-3}\\
\pause 	& = \rho^3 (\rho u_{n-4})\\
\pause 	& = \rho^4 u_{n-4}
	\end{aligned}\]

\pause 	Et ainsi de suite.

\pause 	Il faut remarquer les relations entre les nombres et les indices...

\end{frame}

\begin{frame}

	\[\begin{aligned}
	u_{\alert{n}} &= \rho^{\alert{1}} u_{\alert{n-1}}\\
	u_{\alert{n}} &= \rho^{\alert{2}} u_{\alert{n-2}}\\
	u_{\alert{n}} &= \rho^{\alert{3}} u_{\alert{n-3}}\\
	u_{\alert{n}} &= \rho^{\alert{4}} u_{\alert{n-4}}\\
	\dots \\
	u_{\alert{n}} &= \pause \rho^{\alert{n-1}} \pause u_{\alert{1}}
	\end{aligned}\]

	\pause de sorte que $1+n-1 = n$... \pause Il vient donc :

	\[\begin{aligned}
	u_n
	&= \rho^{n-1} u_{1} \\
	\pause & = \pause \rho^{n-1} \times 1  \\
	\pause & = \rho^{n-1}
	\end{aligned}\]

	\bigskip \pause Conclusion : $u_n =\rho^{n-1}$.

\end{frame}

\begin{frame}

Comportement \alert{asymptotique} de $u_n =\rho^{n-1}$ en fonction de la valeur de $\rho$.\newline

\[
\begin{cases}
\text{si } |\rho|<1, \pause \underset{n \rightarrow +\infty}\lim \rho^{n-1}=0
\begin{cases}
\pause\text{ si } \rho>0 \text{ (convergence monotone)}\\
\pause\text{ si } \rho<0 \text{ (convergence avec oscillations)}
\end{cases}\\
\pause \text{si } |\rho|>1, \pause \underset{n \rightarrow +\infty}\lim \rho^{n-1} \pause =
\begin{cases}
\pause +\infty \text{ si } \rho>0 \text{ (divergence monotone)}\\
\pause \pm\infty \text{ si } \rho<0 \text{ (divergence avec oscillations)}
\end{cases}
\\
\pause \text{si } |\rho|=1, \pause \underset{n \rightarrow \infty}\lim \rho^{n-1} \pause =
\begin{cases}
\pause 1 \text{ si } \rho=1 \\
\pause \pm1 \text{ si } \rho=-1
\end{cases}
\end{cases}
\]

\end{frame}

\begin{frame}

Par exemple : \newline

\[\begin{aligned}
\begin{cases}
\text{si } |\rho|<1:~~
\begin{cases}
\pause\text{ si } \rho=\frac{1}{2} : &u_1 = 1, u_2 = \frac{1}{2}, u_3 = \frac{1}{4}, \dots, 0 \\
\pause\text{ si } \rho=-\frac{1}{2} : &u_1 = 1, u_2 = -\frac{1}{2}, u_3 =\frac{1}{4}, u_4=-\frac{1}{8}, \dots, 0
\end{cases}\\
\pause \text{si } |\rho|>1:~~
\begin{cases}
\pause \text{ si } \rho=2 : & u_1 = 1, u_2 =2, u_3=4, u_4=8, \dots, +\infty \\
\pause \text{ si } \rho=-2 : &u_1=1, u_2=-2, u_3=4, u_4=-8, \dots, \pm\infty
\end{cases}\\
\pause \text{si } |\rho|=1:~~
\begin{cases}
	\pause \text{ si } \rho=1 :  & u_1=1, u_2=1,u_3=1, \dots, 1      \\
	\pause \text{ si } \rho=-1 : & u_1=1, u_2=-1, u_3=1, \dots, \pm1
\end{cases}
\end{cases}
\end{aligned}\]

\end{frame}

\begin{frame}

%\textbf{(3)} Dans le cas où la suite admet une limite, combien de temps faut-il pour réduire de moitié la distance à la limite ?

	Posons $0<\rho<1$ pour que $u_n$ ait une limite unique. \newline

	\pause La condition initiale est $u_1=1$ et la limite 0.\newline

	\pause La moitié de la distance à la limite est donc $\frac{1}{2}$. \newline

	\pause Il faut donc déterminer l'indice $n^\star$ tel que $u_{n^\star} \le \frac{1}{2}$.\pause

\[\begin{aligned}
u_{n^\star} \le \frac{1}{2}
\pause &\Leftrightarrow \rho^{n^\star-1} \le \frac{1}{2} \\
\pause &\Leftrightarrow \ln(\rho^{n^\star-1}) \alert{\le} \ln(\frac{1}{2}) ~~~~~~~~~~~~~~\pause\text{ $\ln()$ est une fonction croissante}\\
\pause &\Leftrightarrow (n^\star-1)\ln(\rho) \le \ln(\frac{1}{2}) ~~~~~~~\pause\ln(A^\alpha)=\alpha\ln(A)  \\
\pause &\Leftrightarrow (n^\star-1) \alert{\ge} \frac{\ln(\frac{1}{2})}{\ln(\rho)} ~~~~~~~~~~~~~~~~\pause\text{ $\ln(\rho)$<0 }\\
\pause &\Leftrightarrow n^\star \ge \frac{\ln(\frac{1}{2})}{\ln(\rho)} +1
\end{aligned}\]

\end{frame}

\section{Exercice 3}

%\exercice{3} Soit la suite de terme général $u_n =
%\frac{n+2}{n}$. \textbf{(1)} Donner les premiers termes de cette suite
%et montrer que la suite est monotone décroissante. \textbf{(2)}
%Montrer que cette suite a pour limite 1.
%
%\bigskip
%

\begin{frame}

$$\alert{\forall n\ge 1},~~u_n = \frac{n+2}{\alert{n}}$$

Premiers termes de la suite :
\[\begin{aligned}
\pause &u_1 = \frac{1+2}{1} \pause= 3\\
\pause &u_2 = \frac{2+2}{2} \pause= 2 \\
\pause &u_3 = \frac{3+2}{3} \pause= \frac{5}{3} \\
\pause &u_4 = \frac{4+2}{4} \pause= \frac{6}{4} \pause= \frac{3}{2} \\
&\dots
\end{aligned}\]

Décroissance monotone...

\end{frame}

\begin{frame}

Il suffit d'étudier le signe de $u_{n+1} - u_n, \forall n$.

\[\begin{aligned}
\begin{cases}
u_{n+1} - u_n >0 \text{ : $u_n$ est croissante}\\
u_{n+1} - u_n <0 \text{ : $u_n$ est décroissante}
\end{cases}
\end{aligned}\]

\end{frame}

\begin{frame}

\[\begin{aligned}
u_{n+1} - u_n
\pause &=  \frac{\alert{(n+1)}+2}{\alert{(n+1)}} \pause - \frac{\alert{n}+2}{\alert{n}}\\
\pause &= \frac{n+3}{n+1} - \frac{n+2}{n}\\
\pause &= \frac{\alert{n}(n+3)}{\alert{n}(n+1)} \pause - \frac{\alert{(n+1)}(n+2)}{n\alert{(n+1)}}\\
\pause &= \frac{n(n+3)- (n+1)(n+2)}{n(n+1)}\\
\pause &= \frac{n^2+3n - (n^2+3n+2)}{n(n+1)}\\
\pause &= \frac{n^2+3n - n^2-3n-2}{n(n+1)}\\
\pause &= \frac{-2}{n(n+1)}\pause <0 \\
\end{aligned}\]

Donc $u_n$ est décroissante.

\end{frame}

\begin{frame}

	\[\begin{aligned}
	\underset{n \rightarrow +\infty}\lim u_n
\pause	&= \underset{n \rightarrow +\infty}\lim \frac{n+2}{n}\\
\pause	&= \underset{n \rightarrow +\infty}\lim (1+\frac{2}{n})\\
\pause	&= 1+ \underset{n \rightarrow +\infty}\lim \frac{2}{n}\\
\pause	&= 1
	\end{aligned}\]

\end{frame}

\section{Exercice 5}

%\exercice{5} Quel est le comportement asymptotique de la suite de
%terme général $u_n = -n$.
%
%\bigskip
%
\begin{frame}

$$\forall n\ge 0, u_n = -n$$

Premiers termes de la suite :
\[\begin{aligned}
\pause &u_0 = 0\\
\pause &u_1 = -1 \\
\pause &u_2 = -2 \\
\pause &u_3 = -3 \\
\pause &\dots
\end{aligned}\]

Décroissance monotone...

\end{frame}

\begin{frame}

\[\begin{aligned}
u_{n+1} - u_n
\pause &=  (\alert{-(n+1)}) \pause - (\alert{-n})\\
\pause &= -n-1 + n\\
\pause &= -1<0 \\
\end{aligned}\]

\end{frame}

\begin{frame}

\[\begin{aligned}
\underset{n \rightarrow +\infty}\lim u_n
\pause &= \underset{n \rightarrow +\infty}\lim -n\\
\pause &= -\infty
\end{aligned}\]

La suite est divergente.

\end{frame}

\section{Exercice 6}

%\exercice{6} La suite de terme général $u_n = \frac{(-1)^{n+1}}{n^2}$
%est-elle monotone ? Montrer que cette suite admet 0 pour limite.
%
%\bigskip

\begin{frame}

	$$\alert{\forall n\ge 1}, ~~u_n = \frac{(-1)^{n+1}}{\alert{n}^2}$$

\pause $(-1)^{n+1}$ va induire une alternance de signes selon que la puissance est paire ou impaire :

\[
(-1)^{n+1} =
\begin{cases}
\pause -1 \text{ si $n$ pair et donc $n+1$ impair} \\
\pause +1 \text{ si $n$ impair et donc $n+1$ pair}\\
\end{cases}
\]

\pause 	Premiers termes de la suite :
\[\begin{aligned}
\pause 	&u_1 = \frac{(-1)^{2}}{1^2} = +1 \\
\pause 	&u_2 = \frac{(-1)^{2+1}}{2^2} = \frac{-1}{4} \\
\pause 	&u_3 = \frac{(-1)^{3+1}}{3^2} = \frac{+1}{9} \\
\pause 	&u_4 = \frac{(-1)^{4+1}}{4^2} = \frac{-1}{16} \\
&\dots
\end{aligned}\]

\pause Alternance de signes donc non monotone...

\end{frame}

\begin{frame}

  \begin{itemize}
  \item Montrons que la limite de la suite $u_n$ est 0.\newline

  \item Pour que cela soit bien le cas il faut et il suffit que la distance entre $u_n$ et 0, mesurée par la valeur absolue de la différence entre $u_n$ et sa limite 0, puisse être rendue arbitrairement petite ($<\varepsilon$) à partir du moment où $n$ est plus grand qu'un certain rang ($n>N(\varepsilon)$, a priori le rang $N(\varepsilon)$ est une fonction décroissante de $\varepsilon$).\newline

  \item Pour tout $\varepsilon>0$, nous avons~:
    \[
      \begin{split}
        |u_n - 0| < \varepsilon \Leftrightarrow &  |u_n| < \varepsilon \\
        \Leftrightarrow & \frac{1}{n^2} < \varepsilon \\
        \Leftrightarrow & n > \frac{1}{\sqrt{\varepsilon}} \quad\small{\text{puisque les valeurs négatives de $n$ ne sont pas pertinentes ici.}}
      \end{split}
    \]

  \item Ainsi, si $n>N(\varepsilon)=\nicefrac{1}{\sqrt{\varepsilon}}$ on a forcément $|u_n-0|<\varepsilon$ et ce pour tout $\epsilon>0$ (arbitrairement petit).\newline

  \item Et donc~:\newline
    \[
      \forall \varepsilon>0, \exists N(\varepsilon) \text{ tel que } n>N(\varepsilon) \Rightarrow |u_n-0|<\varepsilon
    \]

    \bigskip

    \noindent on a donc bien $\lim_{n\rightarrow\infty}u_n=0$.

  \end{itemize}

\end{frame}

\section{Exercice 7}

%\exercice{7} Sur un marché l'offre et la demande sont caractérisées
%par :
%\[
%\begin{split}
%S(p): &\quad q = 1+p\\
%D(p): &\quad q = 2-p\\
%\end{split}
%\]
%\textbf{(1)} Calculer le prix d'équilibre $p^{\star}$ et les quantités échangées à
%l'équilibre, $q^{\star}$. \textbf{(2)} Supposons que le marché ne soit pas
%équilibré. On admet que dans une situation de déséquilibre le prix
%augmente si et seulement si la demande est supérieure à l'offre. Plus
%formellement on admet que le prix est mis à jour à l'aide de la
%récurrence suivante :
%\[
%p_{t+1} = p_t + \alpha (D(p_t)-S(p_t))
%\]
%Déterminer le point fixe de cette récurrence, le prix $\bar p$ tel que
%$\bar p = \bar p + \alpha (D(\bar p)-S(\bar p))$. Comparer $\bar p$ et
%$p^{\star}$. \textbf{(3)} Supposons que le prix initial $p_1$ soit
%différent de $\bar p$. Exprimer $p_t$ en fonction de $p_0$,
%et $\alpha$. \textbf{(4)} Montrer que la chronique de prix converge de
%façon monotone vers $\bar p$ si $0<\alpha<\frac{1}{2}$. \textbf{(5)}
%Quelles sont les prédictions du modèle si $\alpha$ est en dehors de
%cet intervalle ?
%
%\bigskip
%
\begin{frame}
Le prix d'équilibre $p^{\star}$ permet d'égaliser l'offre et la demande sur un marché : $$D(p^{\star}) = S(p^{\star})$$
\pause D'après l'énoncé,
\[
\begin{split}
S(p): &\quad q = 1+p\\
D(p): &\quad q = 2-p\\
\end{split}
\]

\[\begin{aligned}
\pause S(p^{\star}) = D(p^{\star})
\pause \Leftrightarrow & 1+p^\star = 2-p^\star\\
\pause \Leftrightarrow & 2p^\star = 1\\
\pause \Leftrightarrow & p^\star = \frac{1}{2}\\
\end{aligned}\]

Il vient :

\[\begin{aligned}
q^{\star}=
\begin{cases}
\pause &S(p^{\star}) \pause = 1+p^\star \pause = 1+\frac{1}{2} \pause = \frac{3}{2}\\
\pause &D(p^{\star}) \pause = 2-p^\star \pause = 2-\frac{1}{2} \pause = \frac{3}{2}
\end{cases}
\end{aligned}\]

\end{frame}

\begin{frame}
Supposons que le marché ne soit pas	équilibré. \pause Le prix augmente si et seulement si la demande est supérieure à l'offre. \newline

\pause On admet que le prix est mis à jour à l'aide de la récurrence suivante :
\[
p_{t+1} = p_t + \alpha [D(p_t)-S(p_t)]
\]

Posons $\alpha>0$. \\
\begin{itemize}
\pause 	\item[-] Si $D(p_t)-S(p_t)>0$, le prix augmente.\\
\pause 	\item[-] Si $D(p_t)-S(p_t)<0$, le prix diminue.\newline
\end{itemize}

\end{frame}

\begin{frame}

  Déterminons le point fixe de cette récurrence, le prix $\bar{p} = p_t = p_{t+1}$ :

  \[\begin{aligned}
        &p_{t+1} = p_t + \alpha [D(p_t)-S(p_t)]\\
        \Rightarrow\pause & \bar p = \bar p + \alpha [D(\bar p)-S(\bar p)]\\
        \pause \Leftrightarrow & \alpha \left[D(\bar p)-S(\bar p)\right] = 0\\
        \pause \Leftrightarrow & D(\bar p)-S(\bar p) = 0\\
        \pause \Leftrightarrow & D(\bar p)=S(\bar p)\\
      \end{aligned}\]

    \pause \bigskip C'est la condition de détermination du prix d'équilibre $p^{\star}$. \pause Donc $$\bar p = p^{\star} = \frac{1}{2}$$

  \end{frame}

  \begin{frame}

    \begin{itemize}
    \item[-]Si $p_1=p^\star$, alors on part de l'équilibre et comme c'est le point fixe, on reste à l'équilibre.\newline

      \pause \item[-]Dans le cas contraire, que se passe-t-il ? Rejoint-on l'équilibre (convergence) ou non (divergence) ?
    \end{itemize}

  \end{frame}

  \begin{frame}

    \[\begin{aligned}
        p_{t+1}
        \pause &= p_t  + \alpha \left[D(p_t)-S(p_t)\right]\\
        \pause &= p_t  + \alpha \left(2-p_t-1-p_t\right)\\
        \pause &= p_t  + \alpha \left(1-2p_t\right)\\
        \pause &= (1-2\alpha)p_t  + \alpha
      \end{aligned}\]

    \pause C'est une suite arithmético-géométrique !

  \end{frame}

  \begin{frame}

    \[\begin{aligned}
        &p_{t+1} = (1-2\alpha)p_t  + \alpha\\
        \pause &\color{blue}{p_{t} = (1-2\alpha)p_{t-1}  + \alpha}\\
        \pause &\color{orange}{p_{t-1} = (1-2\alpha)p_{t-2}  + \alpha}\\
        &\dots
      \end{aligned}\]

    \pause Réinjectons les lignes inférieures dans l'expression de la première ligne.

    \[\begin{aligned}
        \pause &p_{t+1} = (1-2\alpha){\color{blue}{p_t}}  + \alpha\\
        \pause \Leftrightarrow &p_{t+1} = (1-2\alpha){\color{blue}{[(1-2\alpha)p_{t-1}  + \alpha]}} + \alpha\\
        \pause \Leftrightarrow &p_{t+1} = (1-2\alpha)^2p_{t-1}  + (1-2\alpha) \alpha + \alpha\\
        \pause \Leftrightarrow &p_{t+1} = (1-2\alpha)^2{\color{orange}{p_{t-1}}}  + \alpha[1+(1-2\alpha)] 	\\
        \pause \Leftrightarrow &p_{t+1} = (1-2\alpha)^2{\color{orange}{[(1-2\alpha)p_{t-2}  + \alpha]}}  + \alpha[1+(1-2\alpha)]\\
        \pause \Leftrightarrow &p_{t+1} = (1-2\alpha)^3p_{t-2}  + \alpha[1+(1-2\alpha)+(1-2\alpha)^2]\\
      \end{aligned}\]

  \end{frame}

  \begin{frame}

    \[\begin{aligned}
        p_{t+1} &= (1-2\alpha)^{\alert{3}}p_{\alert{t-2}}  + \alpha[1+(1-2\alpha)+(1-2\alpha)^{\alert{2}}]\\
        \pause &= (1-2\alpha)^{\alert{4}}p_{\alert{t-3}}  + \alpha[1+(1-2\alpha)+(1-2\alpha)^2+(1-2\alpha)^{\alert{3}}]\\
        \pause &= (1-2\alpha)^{\alert{5}}p_{\alert{t-4}}  + \alpha[1+(1-2\alpha)+(1-2\alpha)^2+(1-2\alpha)^3+(1-2\alpha)^{\alert{4}}] \\
        \pause &\dots\\
        \pause &= (1-2\alpha)^{\alert{t}}p_{\alert{1}}  \pause + \alpha[1+(1-2\alpha)+(1-2\alpha)^2+(1-2\alpha)^3+\dots+(1-2\alpha)^{\alert{t-1}}]
      \end{aligned}\]

    \bigskip
    \pause ou en l'écrivant pour $p_t$ :

    \[\begin{aligned}
        p_{t}
        \pause &= (1-2\alpha)^{\alert{t-1}}p_{\alert{1}}  + \alpha[{1+(1-2\alpha)+(1-2\alpha)^2+(1-2\alpha)^3+\dots+(1-2\alpha)^{\alert{t-2}}}]
      \end{aligned}\]

  \end{frame}

  \begin{frame}

    \[\begin{aligned}
        p_{t}
        &= (1-2\alpha)^{{t-1}}p_{{1}}  + \alpha\alert{[\underbrace{1+(1-2\alpha)+(1-2\alpha)^2+(1-2\alpha)^3+\dots+(1-2\alpha)^{{t-2}}}_{\pause \text{somme de $(t-1)$ termes d'une suite géométrique de premier terme 1 et de raison $(1-2\alpha)$}}]}\\
        \pause &= (1-2\alpha)^{{t-1}}p_{{1}}  + \alpha \alert{\frac{1-(1-2\alpha)^{t-1}}{1-(1-2\alpha)}}\\
        \pause &= (1-2\alpha)^{{t-1}}p_{{1}}  + \alpha \frac{1-(1-2\alpha)^{t-1}}{2\alpha}\\
        \pause &= (1-2\alpha)^{{t-1}}p_{{1}}  + \frac{1-(1-2\alpha)^{t-1}}{2}\\
        \pause &= {\color{blue}{(1-2\alpha)^{{t-1}}}\alert{p_{{1}}}}  + \frac{1}{2} \alert{-}\frac{\color{blue}{(1-2\alpha)^{t-1}}}{\alert{2}}\\
        \pause &= {\color{blue}{(1-2\alpha)^{t-1}}}\alert{(p_{{1}}-\frac{1}{2})}  + \frac{1}{2}\\
      \end{aligned}\]

    \pause Remarque : si $p_{{1}}=\frac{1}{2}$, alors $ \forall t, p_t=\frac{1}{2}$. \pause Donc si on part du point fixe qui est aussi l'équilibre, on reste toujours à l'équilibre.

  \end{frame}

  \begin{frame}

    % \textbf{(4)} Montrer que la chronique de prix converge de
    % façon monotone vers $\bar p$ si $0<\alpha<\frac{1}{2}$. \textbf{(5)}
    % Quelles sont les prédictions du modèle si $\alpha$ est en dehors de
    % cet intervalle ?

    Question : si on ne part pas de l'équilibre (ici $p_1 \ne \bar p = \frac{1}{2}$), est-ce qu'on va y arriver (convergence) ?\newline

    \pause On étudie donc $\underset{t \rightarrow +\infty}\lim p_t$.

    \[\begin{aligned}
        \underset{t \rightarrow +\infty}\lim p_{t}
        \pause &= \underset{t \rightarrow +\infty}\lim [(1-2\alpha)^{t-1}(p_{{1}}-\frac{1}{2})  + \frac{1}{2}]\\
        \pause &= \frac{1}{2} + (p_{{1}}-\frac{1}{2})\underset{t \rightarrow +\infty}\lim (1-2\alpha)^{t-1}\\
      \end{aligned}\]

    \bigskip

    \pause Dépend si $\underset{t \rightarrow +\infty}\lim (1-2\alpha)^{t-1} = 0 $ \pause et donc si $|1-2\alpha|<1$. \newline

    \pause Si c'est le cas, alors $\underset{t \rightarrow +\infty}\lim p_{t} = \frac{1}{2} = \bar p$.

  \end{frame}

  \begin{frame}

	% \textbf{(4)} Montrer que la chronique de prix converge de
	% façon monotone vers $\bar p$ si $0<\alpha<\frac{1}{2}$. \textbf{(5)}
	% Quelles sont les prédictions du modèle si $\alpha$ est en dehors de
	% cet intervalle ?
    \[\begin{aligned}
        \underset{t \rightarrow +\infty}\lim p_{t}
        \pause &= \frac{1}{2} + (p_{{1}}-\frac{1}{2})\underset{t \rightarrow +\infty}\lim (1-2\alpha)^{t-1}\\
      \end{aligned}\]

    \bigskip

    \pause Etudions les différents cas de figure de $\underset{t \rightarrow +\infty}\lim (1-2\alpha)^{t-1}$.\newline

    \bigskip

    \begin{itemize}
    \item[-] si $|1-2\alpha|<1$ : \pause convergence :\newline

      \begin{itemize}
        \pause \item[-] 	si $0<1-2\alpha<1$ : \pause convergence vers $\bar p$ de façon monotone.\newline

        \pause \item[-] 	si $-1<1-2\alpha<0$ : \pause convergence avec oscillations vers $\bar p$.\newline
      \end{itemize}

      \medskip

      \pause \item[-] si $|1-2\alpha|>1$ : \pause divergence.\newline

      \begin{itemize}
        \pause 	\item[-] 	si $1-2\alpha>1$ : \pause divergence vers $+\infty$ de façon monotone.\newline

        \pause 	\item[-] 	si $1-2\alpha<-1$ : \pause divergence avec oscillations de signe.\newline
      \end{itemize}

    \end{itemize}

  \end{frame}

  \section{Exercice 8}

  % \exercice{8} Identifier les limites suivantes :
  % \begin{enumerate}
  %	\item $\lim_{x\rightarrow\infty} \frac{2x+5}{x^2-3}$
  %	\item $\lim_{x\rightarrow\infty} \frac{x^3-4x^2+8}{x^2+6}$
  %	\item $\lim_{x\rightarrow\infty} \frac{ax^2+bx+c}{kx^2+lx+m}$
  %	\item $\lim_{x\rightarrow -4} \frac{x^2-16}{x+4}$
  %	\item $\lim_{x\rightarrow 0^+} \frac{|x|}{x}$ et $\lim_{x\rightarrow 0^-} \frac{|x|}{x}$
  % \end{enumerate}
  %
  % \bigskip
  %

  \begin{frame}
    $$\lim_{x\rightarrow\infty} \frac{2x+5}{x^2-3}\pause =\frac{\infty}{\infty}$$

    \bigskip

    \[\begin{aligned}
        \lim_{x\rightarrow\infty} \frac{2x+5}{x^2-3}
        \pause &= \lim_{x\rightarrow\infty} \frac{x(2+\frac{5}{x})}{x(x-\frac{3}{x})}\\
        \pause &= \lim_{x\rightarrow\infty} \frac{2+\frac{5}{x}}{x-\frac{3}{x}}\\
        \pause &= \frac{2+\lim_{x\rightarrow\infty} \frac{5}{x}}{\lim_{x\rightarrow\infty} x-\lim_{x\rightarrow\infty} \frac{3}{x}}\\
        \pause &= \frac{2}{\lim_{x\rightarrow\infty} x} \\
        \pause &= 0\\
      \end{aligned}\]

  \end{frame}


  \begin{frame}
	$$\lim_{x\rightarrow\infty} \frac{x^3-4x^2+8}{x^2+6}\pause =\frac{\infty}{\infty}$$

	\bigskip

	\[\begin{aligned}
	\lim_{x\rightarrow\infty} \frac{x^3-4x^2+8}{x^2+6}
\pause 	&= \lim_{x\rightarrow\infty} \frac{x^2(x-4+\frac{8}{x^2})}{x^2(1+\frac{6}{x^2})}\\
\pause 	&= \lim_{x\rightarrow\infty} \frac{x-4+\frac{8}{x^2}}{1+\frac{6}{x^2}}\\
\pause 	&= \frac{\lim_{x\rightarrow\infty} x-4+\lim_{x\rightarrow\infty} \frac{8}{x^2}}{1+\lim_{x\rightarrow\infty} \frac{6}{x^2}}\\
\pause 	&= \frac{\lim_{x\rightarrow\infty} x-4}{1}\\
\pause 	&= \infty \\
	\end{aligned}\]

\end{frame}

\begin{frame}
	$$\lim_{x\rightarrow\infty} \frac{ax^2+bx+c}{kx^2+lx+m}\pause =\frac{\infty}{\infty}$$

\bigskip

	\[\begin{aligned}
	\lim_{x\rightarrow\infty} \frac{ax^2+bx+c}{kx^2+lx+m}
\pause 	&= \lim_{x\rightarrow\infty} \frac{x^2(a+\frac{b}{x}+\frac{c}{x^2})}{x^2(k+\frac{l}{x}+\frac{m}{x^2})}\\
\pause 	&= \lim_{x\rightarrow\infty} \frac{a+\frac{b}{x}+\frac{c}{x^2}}{k+\frac{l}{x}+\frac{m}{x^2}}\\
\pause 	&= \lim_{x\rightarrow\infty} \frac{a+\lim_{x\rightarrow\infty}\frac{b}{x}+\lim_{x\rightarrow\infty}\frac{c}{x^2}}{k+\lim_{x\rightarrow\infty}\frac{l}{x}+\lim_{x\rightarrow\infty}\frac{m}{x^2}}\\
\pause 	&= \frac{a}{k} \pause ~~\text{si $k \ne 0$ !}
	\end{aligned}\]

\pause Dans le cas contraire, $	\lim_{x\rightarrow\infty} \frac{ax^2+bx+c}{kx^2+lx+m}
=\infty$.

\end{frame}

\begin{frame}
	$$\lim_{x\rightarrow -4} \frac{x^2-16}{x+4}=\frac{0}{0}$$

	\bigskip

	\[\begin{aligned}
	\lim_{x\rightarrow -4} \frac{x^2-16}{x+4}
\pause &= \lim_{x\rightarrow -4} \frac{(x-4)(x+4)}{x+4}\\
\pause &= \lim_{x\rightarrow -4} (x-4)\\
\pause &= \lim_{x\rightarrow -4} x - 4 \\
\pause &=-8\\
	\end{aligned}\]

\end{frame}


\begin{frame}

Rappel :
\[
|x| =
\begin{cases}
\pause  x & \pause \text{ si } x>0 \\
\pause -x & \pause \text{ si } x<0
\end{cases}
\]

Exemple :
\[\begin{aligned}
|2| &= 2 \text{ puisque } 2>0 \\
|-2| &= -(-2) = 2 \text{ puisque } -2<0
\end{aligned}\]

\end{frame}

\begin{frame}

	$$\lim_{x\rightarrow 0^+} \frac{|x|}{x}$$

\pause 	$x\rightarrow 0^+$ \pause donc $x>0$ \pause donc $|x| = x$ ici.


	\[\begin{aligned}
\pause 	\lim_{x\rightarrow 0^+} \frac{|x|}{x} \pause = \lim_{x\rightarrow 0^+} \frac{x}{x} \pause = 1
	\end{aligned}\]

	\bigskip
	\bigskip

\pause	$$\lim_{x\rightarrow 0^-} \frac{|x|}{x}$$

\pause $x\rightarrow 0^-$ \pause donc $x<0$ \pause donc $|x| = -x$ ici.

\[\begin{aligned}
\pause \lim_{x\rightarrow 0^-} \frac{|x|}{x} \pause = \lim_{x\rightarrow 0^-} \frac{-x}{x} \pause = -1
\end{aligned}\]

\end{frame}


\section{Exercice 9}

%\exercice{9} En utilisant la définition de la dérivée, calculer les
%dérivées des fonctions suivantes :
%\begin{enumerate}
%	\item $f(x) = 4x^2+3$
%	\item $g(x) = x^n$, $\forall\quad n\in \mathbb N$ et $\forall\quad x\in \mathbb R$
%	\item $h(x) = \frac{1}{x}$, $\forall x\in \mathbb R^*$
%\end{enumerate}
%Pour $g(x)$ vous utiliserez la formule du binôme de Newton (une
%généralisation de l'identité remarquable bien connue) :
%\[
%(a+b)^n = \sum_{k=0}^n C_n^ka^{n-k}b^k
%\]
%avec
%\[
%C_n^k = \frac{n!}{k!(n-k)!}
%\]
%où $m! = m\times(m-1)\times(m-2)\times\dots\times 3\times 2\times 1$
%la fonction factorielle.
%

\begin{frame}
	$$\lim_{h\rightarrow 0} \frac{f(x+h)-f(x)}{h} = f^{'}(x) = \frac{d}{d x}f$$

$$f(x) = 4x^2+3$$
\pause Avec les formules automatiques, on s'attend à ce que $f^{'}(x) = 8x$.

\[\begin{aligned}
\pause f(\alert{x+h})-f(x)
\pause &= [4(\alert{x+h})^2+3] - (4x^2+3)\\
\pause &= 4[({x+h})^2 - x^2]\\
\pause &= 4(x+h-x)(x+h+x)\\
\pause &= 4h(2x+h)\\
\pause \frac{f({x+h})-f(x)}{h}
\pause &= \frac{4h(2x+h)}{h}\\
\pause &= 4(2x+h)\\
\pause \lim_{h\rightarrow 0} \frac{f({x+h})-f(x)}{h}
\pause &= \lim_{h\rightarrow 0} 4(2x+h)\\
\pause &= 4(2x+\lim_{h\rightarrow 0}h)\\
\pause &= 8x
\end{aligned}\]

\end{frame}

\begin{frame}
	$$\lim_{h\rightarrow 0} \frac{f(x+h)-f(x)}{h} = f^{'}(x)= \frac{d}{d x}f$$

	$$f(x) = \frac{1}{x} ~~\forall x\in \mathbb R^*$$
    \pause 	Avec les formules automatiques, on s'attend à ce que $f^{'}(x) = -\frac{1}{x^2}$.

    \[\begin{aligned}
        \pause 	f(\alert{x+h})-f(x)
        \pause 	&= \frac{1}{\alert{x+h}} - \frac{1}{{x}}\\
        \pause 	&= \frac{{\color{blue}{x}}}{{\color{blue}{x}}({x+h})} - \frac{{\color{orange}{(x+h)}}}{{x}\color{orange}{(x+h)}}\\
        \pause 	&= \frac{{x}- (x+h)}{{x}({x+h})}\\
        \pause 	&= \frac{-h}{{x}({x+h})}\\
        \pause 	\frac{f({x+h})-f(x)}{h} &=  \frac{-1}{{x}({x+h})} \\
        \pause 	\lim_{h\rightarrow 0} \frac{f({x+h})-f(x)}{h} \pause &= \lim_{h\rightarrow 0} \frac{-1}{{x}({x+h})} \pause =  \frac{-1}{{x}({x+\lim_{h\rightarrow 0} h})}\pause =  -\frac{1}{{x^2}}
      \end{aligned}\]

  \end{frame}

\end{document}
%%% Local Variables:
%%% mode: latex
%%% TeX-master: t
%%% End:
