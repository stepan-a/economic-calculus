\documentclass[9pt,handout,professionalfonts,hyperref]{beamer}
\mode<presentation>
\usetheme{Frankfurt}
\usecolortheme{seahorse}
\usepackage[french]{babel}
\usepackage{amssymb}
\usepackage{amsmath}
\usepackage{float}
\usepackage{setspace}
\usepackage{mathrsfs}
\usepackage{graphicx}
\usepackage{enumitem}
\usepackage{epstopdf}
\usepackage{psfrag}
\usepackage{dcolumn}
\usepackage{multirow}
\usepackage{hyperref}
\usepackage{authblk}
\usepackage[french]{babel}
\usepackage[round]{natbib}
\usepackage{fancybox}
\usepackage{multicol}
\usepackage[utf8]{inputenc}
\usepackage[T1]{fontenc}
\usepackage{pdflscape}
\usepackage{array}
\usepackage[flushleft]{threeparttable}
\usepackage{tcolorbox}
\usepackage{placeins}
\usepackage{tkz-tab}
\usepackage{adjustbox}

\title{Calcul Economique 1\\Fiche de TD 4\\F.~Karamé}
\date{Année universitaire 2021-2022}
\subject{Talks}

\setbeamertemplate{navigation symbols}{}

\begin{document}

\begin{frame}
\titlepage
\end{frame}

\section{Exercice 1}

\begin{frame}
$$f(x)=\ln(x^2+x^4+1)$$

\pause de la forme $$[\ln(U)]'=\frac{U'}{U}$$

\pause
avec $$U=x^2+x^4+1$$
et donc
\pause $$U'=2x+4x^3+0$$
\medskip
\pause donc :
\pause
$$f'(x)=\dfrac{2x+4x^3}{x^2+x^4+1}$$

\end{frame}

\begin{frame}

$$
g(x)=x^2\ln(x^2+x^4+1)
$$

\pause de la forme $$(UV)'=U'V+UV'$$

\medskip

\pause avec $U=x^2$ et $V=\ln(x^2+x^4+1)$

\pause
\medskip

donc $U'=2x$ \pause et $V'=\dfrac{2x+4x^3}{x^2+x^4+1}$ (question 1)
 \pause
\medskip

Donc :
$$
g'(x)=(2x)\pause\ln(x^2+x^4+1)\pause+(x^2)\pause\left(\dfrac{2x+4x^3}{x^2+x^4+1}\right)
$$

\end{frame}


\begin{frame}

$$
h(x)=e^{2x}
$$

 \pause de la forme $$(e^U)'=U'e^U$$

 \medskip
\pause avec $U=2x$ et $U'=2$
 \pause
\medskip

Donc :
$$
h'(x)=2e^{2x}
$$

\end{frame}


\begin{frame}

$$
j(x)=\ln\left(\dfrac{x^3-2}{x^2+1}\right)
$$

 \pause de la forme $$[\ln(Z)]'=\frac{Z'}{Z}$$
 \medskip

\pause  avec $Z=\frac{x^3-2}{x^2+1}$ \pause et donc $$Z'=\left(\frac{U}{V}\right)'=\frac{U'V-UV'}{V^2}$$
\pause

avec $U = x^3-2$ et $V=x^2+1$ \pause et donc $U' = 3x^2$ \pause et $V'=2x$ \\
\medskip
\pause
$$j'(x)=\frac{\dfrac{(3x^2)(x^2+1)-(x^3-2)(2x)}{(x^2+1)^2}}{\dfrac{x^3-2}{x^2+1}}=\dots$$

\end{frame}


\begin{frame}
Plus simple : on voit que $$j(x)=\ln\left(\dfrac{x^3-2}{x^2+1}\right)\pause=\ln(x^3-2)-\ln(x^2+1)$$
 \pause
\medskip

Comme la dérivée d'une somme est la somme des dérivées ($(U+V)' = U'+V'$), en utilisant deux fois $$[\ln(Z)]'=\frac{Z'}{Z}$$ il vient très facilement :

$$
j'(x)=\frac{3x^2}{x^3-2}\pause -\frac{2x}{x^2+1}
$$
\end{frame}

\section{Exercice 2}

\begin{frame}

$$f(x)=e^{\theta x}$$
\pause

$f'(x)=f^{(1)}(x)=\theta e^{\theta x}$ \pause car $(e^U)'=U'e^U$ \pause avec $U=\theta x$ \pause et $U'=\theta$
\pause
\medskip

$f''(x)=f^{(2)}(x)=(\theta e^{\theta x})^{'}\pause =\theta(e^{\theta x})^{'} \pause =\theta (\theta e^{\theta x}) \pause = \theta^2 e^{\theta x}$
 \pause
\medskip

$f^{(3)}(x)=(\theta^2 e^{\theta x})^{'}\pause = \theta^2(e^{\theta x})^{'} \pause =\theta^2 (\theta e^{\theta x}) \pause =\theta^3 e^{\theta x}$
 \pause
\medskip

On peut deviner que : $f^{(n)}(x)=\theta^n e^{\theta x}$
 \pause
\medskip

Par récurrence ? \pause au rang $n+1$ : \pause \alert{$f^{(n+1)}(x)\pause =\theta^{n+1} e^{\theta x}$} ?
 \pause
\medskip

$f^{(n+1)}(x)=[f^{(n)}(x)]^{'}\pause =(\theta^n  e^{\theta x})'\pause =\theta^n  (e^{\theta x})'\pause =\theta^n (\theta e^{\theta x})\pause =\theta^{n+1} e^{\theta x} $\newline

Si la proposition est vraie au rang $n$, alors elle est vraie au rang $n+1$.

Comme elle est vraie au rang 1, elle est vraie au rang 2, $\dots$, et de proche en proche, elle est vraie pour tout $n$.

\end{frame}

\begin{frame}
$$
g(x)=\dfrac{1}{x}\pause = \alert{x^{-1}}
$$
 \pause
de la forme $(x^\alpha)^{'} = \alpha x^{\alpha-1}$.
\pause
\medskip

$g^{(1)}(x)= (x^{-1})^{'} \pause = (-1)x^{-1-1} \pause = -x^{-2} \pause = -\dfrac{1}{x^2}$
 \pause
\medskip

$g^{(2)}(x)=(-x^{-2})^{'} \pause =-(-2)x^{-2-1} \pause =2x^{-3}\pause = \dfrac{2}{x^3}$
 \pause
\medskip

$g^{(3)}(x)= (2x^{-3})^{'}\pause = 2 (-3)x^{-3-1}\pause = -6x^{-4}\pause =-\dfrac{6}{x^4}$
 \pause
\medskip

$g^{(4)}(x)=(-6x^{-4})^{'} \pause = -6(-4)x^{-4-1} \pause = 24x^{-5} \pause = \dfrac{24}{x^5}$
 \pause
\medskip

Donc on devine : $$g^{(n)}(x)=(-1)^n\dfrac{n!}{x^{n+1}}$$

\end{frame}

\begin{frame}
$$g^{(n)}(x)=(-1)^n\dfrac{n!}{x^{n+1}}$$
 \pause
\medskip

Par récurrence, vraie au rang $n+1$ ? \pause $\alert{g^{(n+1)}(x) \pause =(-1)^{n+1}\dfrac{(n+1)!}{x^{n+2}}}$ ?
\pause
\medskip

\[\begin{aligned}
g^{(n+1)}(x)
& =[g^{(n)}(x)]' \\
\pause & = \left((-1)^n\dfrac{n!}{x^{n+1}}\right)'\\
\pause & = (-1)^n n! \left(x^{-(n+1)}\right)'\\
\pause & =(-1)^n n! [-(n+1)] x^{-(n+1)-1} \\
\pause & = {\color{orange}{(-1)^n}} n! {\color{orange}{(-1)}}(n+1) x^{-(n+2)} \\
\pause & ={\color{orange}{(-1)^{n+1}}} {\color{cyan}{n!(n+1)}} x^{-(n+2)} \\
\pause & =(-1)^{n+1} {\color{cyan}{(n+1)!}} \dfrac{1}{x^{n+2}} \\
\end{aligned}\]\newline

Si la proposition est vraie au rang $n$, alors elle est vraie au rang $n+1$.

Comme elle est vraie au rang 1, elle est vraie au rang 2, $\dots$, et de proche en proche, elle est vraie pour tout $n$.

\end{frame}

\begin{frame}
$$
h(x)=\ln(x)
$$
\pause
\medskip

$$h^{(1)}(x)\pause =\dfrac{1}{x}\pause =g(x)$$

\pause donc :

$$h^{(2)}(x)=g^{(1)}(x)$$

\pause et donc :

$$h^{(n)}(x)=g^{(n-1)}(x)$$

\end{frame}

\section{Exercice 3}
\begin{frame}
\textbf{Théorème de Taylor Young (développement limité en 0)} :
\pause
\medskip
\[\begin{aligned}
f(x)
&=\sum_{i=0}^{\infty}\dfrac{f^{(i)}(0)[x-0]^i}{i!} \\
\pause &=\dfrac{f^{(0)}(0)[x-0]^0}{0!}\pause +\dfrac{f^{(1)}(0)[x-0]^1}{1!}\pause +\dfrac{f^{(2)}(0)[x-0]^2}{2!}\pause +\dfrac{f^{(3)}(0)[x-0]^3}{3!}\pause +\dots\\
&=f(0)+\dfrac{f^{(1)}(0)[x-0]}{1!}+\dfrac{f^{(2)}(0)[x-0]^2}{2!}+\dfrac{f^{(3)}(0)[x-0]^3}{3!}+\dots
\end{aligned}\]

\medskip
\pause Rappels : $n! = n\times (n-1)\times (n-2)\times \dots\times 1$ \pause et $0! = 1$.

\medskip
\pause Attention : $f^{(i)}(0)$ est la dérivée $ieme$ de $f$ qu'on évalue ensuite en $0$.

\end{frame}

\begin{frame}

\[\begin{aligned}
f(x)=f(0)+\dfrac{f^{(1)}(0)[x-0]}{1!}+\dfrac{f^{(2)}(0)[x-0]^2}{2!}+\dfrac{f^{(3)}(0)[x-0]^3}{3!}+\dots
\end{aligned}\]

$$
e^x=\sum_{i=0}^{\infty} \dfrac{x^i}{i!}
$$
\pause
\medskip

$\forall i\ge0,~f^{(i)}(x) = e^x$ \pause et $f^{(i)}(0) = e^0 =1 $

\pause
\medskip
\[\begin{aligned}
e^x
&=\pause 1\pause +x\pause +\frac{x^2}{2!}\pause +\dfrac{x^3}{3!}\pause +\dots\\
\pause &=\frac{x^0}{0!}+\frac{x^1}{1!}+\frac{x^2}{2!}+\dfrac{x^3}{3!}+\dots\\
\pause &=\sum_{i=0}^{\infty}\dfrac{x^i}{i!}
\end{aligned}\]

\end{frame}

\begin{frame}
si $x=1$, on trouve la constante d'Euler notée $e$ :

$$e^1= \pause e\pause =\sum_{i=0}^{\infty}\dfrac{1}{i!} \approxeq 2.718281828459045$$

La précision de l'approximation augmente avec l'ordre du DL.
\[\begin{aligned}
e
&\approxeq \sum_{i=0}^{0}\dfrac{1}{i!} = 1\\
&\approxeq \sum_{i=0}^{1}\dfrac{1}{i!} = 1 + \frac{1}{1!} = 2\\
&\approxeq \sum_{i=0}^{2}\dfrac{1}{i!} = 1 + \frac{1}{1!} + \frac{1}{2!} = 2.5\\
&\approxeq \sum_{i=0}^{3}\dfrac{1}{i!} = 1 + \frac{1}{1!} + \frac{1}{2!} + \frac{1}{3!} = 2.6666\\
&\approxeq \sum_{i=0}^{4}\dfrac{1}{i!} = 1 + \frac{1}{1!} + \frac{1}{2!} + \frac{1}{3!} + \frac{1}{4!}= 2.708333\\
&\approxeq \sum_{i=0}^{5}\dfrac{1}{i!} = 1 + \frac{1}{1!} + \frac{1}{2!} + \frac{1}{3!} + \frac{1}{4!}+ \frac{1}{5!}= 2.716666\\
\dots
\end{aligned}\]

\end{frame}

\section{Exercice 4}
\begin{frame}
$$
\dfrac{1}{1-x}=\sum_{i=0}^{\infty}x^i
$$
\pause
\medskip

Posons $\dfrac{1}{1-x} = (1-x)^{-1} $ \pause et $(U^\alpha)^{'} = \alpha U^{'} U^{\alpha-1}$ \pause avec $U = 1-x$ \pause et $U^{'} = -1$.

\pause
\medskip
$f^{(1)}(x) \pause = \left[{(1-x)^{-1}}\right]^{'} \pause = (-1)\pause (-1)\pause (1-x)^{-1-1} \pause = (1-x)^{-2} \pause = \dfrac{1}{(1-x)^2}$
\pause
\medskip

$f^{(2)}(x) = \left[ (1-x)^{-2} \right]^{'} \pause =(-2)(-1)(1-x)^{-2-1} \pause =\dfrac{2}{(1-x)^3}$
\pause
\medskip

$f^{(3)}(x) = \left[ 2(1-x)^{-3} \right]^{'} \pause =(2)(-3)(-1)(1-x)^{-3-1} \pause =\dfrac{2\times3}{(1-x)^4}\pause =\dfrac{6}{(1-x)^4}$
\pause
\medskip

$f^{(4)}(x) = \left[ (2\times 3)(1-x)^{-4} \right]^{'} \pause =(2\times3)(-4)(-1)(1-x)^{-4-1} \pause =\dfrac{2\times3\times4}{(1-x)^5}\pause =\dfrac{24}{(1-x)^5}$
\pause

\medskip
On peut deviner que :
$$f^{(n)}(x) =\dfrac{n!}{(1-x)^{n+1}}$$
\end{frame}

\begin{frame}

	\textbf{Théorème de Taylor Young (développement limité en 0)}
	\[\begin{aligned}
	f(x)=\sum_{i=0}^{\infty}\dfrac{f^{(i)}(0)[x-0]^i}{i!}
	\end{aligned}\]
	\pause


\[\begin{aligned}
f^{(n)}(x) =\dfrac{n!}{(1-x)^{n+1}} \Leftrightarrow
&f^{(i)}(x) =\dfrac{i!}{(1-x)^{i+1}}\\
\pause &f^{(i)}(0) =\dfrac{i!}{(1-0)^{i+1}} \pause = i!
\end{aligned}\]

Donc :

\[\begin{aligned}
\dfrac{1}{1-x}
&=\sum_{i=0}^{\infty}\dfrac{f^{(i)}(0)[x-0]^i}{i!}\\
\pause &=\sum_{i=0}^{\infty}\dfrac{i!}{i!}x^i \\
\pause &=\sum_{i=0}^{\infty}x^i
\end{aligned}\]\\

Cela ressemble bcp au résultat sur la somme des termes d'une suite géométrique de raison $x$ (si $|x|<1$).
\end{frame}

\section{Exercice 5}

\begin{frame}
$$
\forall x \in \mathbb{R}, ~f(x)=-x^3+x^2+2x
$$

\pause On veut résoudre :

$$f(x)=0 \Leftrightarrow -x^3+x^2+2x=0$$
\pause

$x=0$ est racine évidente puisqu'il n'y a pas de constante. \pause En factorisant, il reste un polynôme du second degré :
$$-x(x^2-x-2)=0$$

\pause En utilisant la méthode du discriminant ou en trouvant les racines évidentes $x=2$ et $x=-1$, il reste :

$$-x(x-2)(x+1)=0$$

\end{frame}

\begin{frame}

\[\begin{aligned}
&f(x)=-x^3+x^2+2x\\
&\pause f'(x)=-3x^2+2x+2
\end{aligned}\]\\

\pause \medskip
$f'(x)=0$ : polynôme du second degré, \pause $\Delta=28>0$ \newline

\pause Les deux racines sont : $\dfrac{1-\sqrt{7}}{3}$ et  $\dfrac{1+\sqrt{7}}{3}$.\newline

\pause Le signe devant $x^2$ étant négatif, \pause la fonction est positive entre les racines et négatives en dehors.\newline

\pause En ces deux points, la dérivée s'annule et change de signes : ce sont donc des points de retournement.\newline

\pause Enfin on calcule les limites de $f$ au bord de l'ensemble de définition :

\[\begin{aligned}
&\lim_{x \to -\infty} f(x)=\lim_{x \to -\infty} (-x^3+x^2+2x) =\lim_{x \to -\infty} -x^3 = +\infty \\
&\lim_{x \to +\infty} f(x)=\lim_{x \to +\infty} (-x^3+x^2+2x) =\lim_{x \to +\infty} -x^3 = -\infty
\end{aligned}\]\\


\end{frame}

\begin{frame}
\begin{adjustbox}{width=\textwidth,center}
\begin{tikzpicture}
   \tkzTabInit{$x$ / 1 , $f'(x)$ / 1,$f(x)$ / 2}{$-\infty$, $\dfrac{1-\sqrt{7}}{3}$, $\dfrac{1+\sqrt{7}}{3}$, $+\infty$}
\pause   \tkzTabLine{, -, z,+,z, -, }
\pause   \tkzTabVar{+/ $+\infty$, -/  , +/  , -/ $-\infty$}
   \tkzTabVal{1}{2}{0.5}{$-1$}{0}
   \tkzTabVal{2}{3}{0.5}{$0$}{0}
   \tkzTabVal{3}{4}{0.5}{$2$}{0}
\end{tikzpicture}
\end{adjustbox}


\end{frame}

\section{Exercice 6}

\begin{frame}
$$
\forall x \in \mathbb{R}^{+*}, ~f(x)=\dfrac{[\ln(x)]^2}{x}
$$

\pause La fonction est toujours positive. \newline

\pause Dérivée de la forme $(\frac{U}{V})^{'}=\frac{U'V-UV'}{V^2}$ \pause avec $U=[\ln(x)]^2$ et $V=x$.\newline

\pause D'où $V^{'} = 1$ \pause et $U^{'} = \left\{[\ln(x)]^2\right\}^{'}$ \pause de la forme $(W^2)^{'} \pause = 2W^{'}W$ \pause en posant $W=\ln(x)$ \pause donc $W^{'}=\frac{1}{x}$ :

\pause $$U^{'} = \left\{[\ln(x)]^2\right\}^{'} \pause = 2 \frac{\ln(x)}{x}$$

\pause
Il vient :
\[\begin{aligned}
f'(x)
&=\dfrac{\pause \frac{2\ln(x)}{x}\pause x \pause - [\ln(x)]^2 \pause \times 1}{x^2}\\
\pause &=\dfrac{2 \ln(x) -[\ln(x)]^2}{x^2} \\
\pause &=\dfrac{\ln(x) [2-\ln(x)]}{x^2}
\end{aligned}\]

\end{frame}

\begin{frame}

\[\begin{aligned}
f'(x)&=\dfrac{\ln(x) [2-\ln(x)]}{x^2}
\end{aligned}\]

\pause Le signe de la dérivée dépend du signe de $\ln(x) [2-\ln(x)]$. \pause Elle s'annule ssi
$$\begin{cases}
\pause \ln(x) =0 \pause \Leftrightarrow e^{\ln(x)} = e^0 \pause \Leftrightarrow  x=1 \\
\pause 2-\ln(x)=0 \pause \Leftrightarrow \ln(x)=2 \pause \Leftrightarrow e^{\ln(x)} = e^2 \pause \Leftrightarrow x=e^2
\end{cases}$$

\medskip

Reste à calculer les limites de $f$ au bord de l'ensemble de définition :

\[\begin{aligned}
&\lim_{x \to 0^+} f(x)= +\infty \\
&\lim_{x \to +\infty} f(x)=0^+
\end{aligned}\]\\

\end{frame}

\begin{frame}
\begin{adjustbox}{width=\textwidth,center}
\begin{tikzpicture}
\tkzTabInit{$x$ / 1 ,$\ln(x)$ / 1, $2-\ln(x)$ / 1, $f'(x)$ / 1,$f(x)$ / 2}{$0$, $1$, $e^2$, $+\infty$}
\pause \tkzTabLine{d, -, z,+,, +, }
\pause \tkzTabLine{d, +, ,+,z, -, }
\pause \tkzTabLine{d, -, z,+,z, -, }
\pause \tkzTabVar{D+/ $+\infty$, -/ $0$ , +/  $\frac{4}{e^2}$, -/ $0$}
\end{tikzpicture}
\end{adjustbox}
\end{frame}

\section{Exercice 7}

\begin{frame}
\[\begin{aligned}
&f(x)=ax^3+bx^2+cx+d \\
\pause &f'(x)= \pause 3ax^2+2bx+c
\end{aligned}\]\newline
un polynôme du second degré.\newline

\pause Avec la méthode du discriminant : \alert{attention à la confusion possible avec les formules automatiques du discriminant du fait de la présence de $a,b,c$.}

$$\Delta = (\alert{2b})^2\pause -4\times\pause (\alert{3a})\times\pause  c \pause = 4b^2 -12ac$$
\pause

\pause Si deux extrema : $\Delta>0 \Rightarrow 2$ racines %\pause donc $4b^2>12ac$
\pause
\[\begin{cases}
x_1 = \dfrac{-\alert{2b}-\sqrt\Delta}{2\times \alert{3a}}\pause=\dfrac{-2b-\sqrt\Delta}{6a} \\
x_2 = \dfrac{-\alert{2b}+\sqrt\Delta}{2\times \alert{3a}}\pause =\dfrac{-2b+\sqrt\Delta}{6a}
\end{cases}\]\newline

Deux cas possibles selon le signe de $a$ (les deux extrema évoqués dans l'énoncé excluent $a=0$).
\end{frame}

\begin{frame}
Si $\alert{a>0}$ :
\[\begin{aligned}
\pause &\lim_{x \to -\infty} f(x)=\lim_{x \to -\infty} ax^3 =a\times -\infty = -\infty \\
\pause &\lim_{x \to +\infty} f(x)=\lim_{x \to +\infty} ax^3 =a\times +\infty = +\infty
\end{aligned}\]

\pause
\medskip
\begin{adjustbox}{width=\textwidth,center}
\begin{tikzpicture}
   \tkzTabInit{$x$ / 1 , $f'(x)$ / 1,$f(x)$ / 2}{$-\infty$, $\dfrac{-2b-\sqrt\Delta}{6a}$, $\dfrac{-2b+\sqrt\Delta}{6a}$, $+\infty$}
\pause    \tkzTabLine{, \alert{+}, z,\alert{-},z, \alert{+}, }
 \pause   \tkzTabVar{ -/ $-\infty$, +/  , -/   , +/$+\infty$}
\end{tikzpicture}
\end{adjustbox}


\end{frame}

\begin{frame}
Si $\alert{a<0}$ :
\[\begin{aligned}
&\lim_{x \to -\infty} f(x)=\lim_{x \to -\infty} ax^3 = a\times -\infty = +\infty \\
&\lim_{x \to +\infty} f(x)=\lim_{x \to +\infty} ax^3 = a\times +\infty =-\infty
\end{aligned}\]

\pause
\medskip
\begin{adjustbox}{width=\textwidth,center}
\begin{tikzpicture}
   \tkzTabInit{$x$ / 1 , $f'(x)$ / 1,$f(x)$ / 2}{$-\infty$,  $\dfrac{-2b-\sqrt\Delta}{6a}$, $\dfrac{-2b+\sqrt\Delta}{6a}$, $+\infty$}
\pause    \tkzTabLine{, \alert{-}, z,\alert{+},z, \alert{-}, }
 \pause   \tkzTabVar{ +/$+\infty$ , -/  , +/   , -/ $-\infty$ }
\end{tikzpicture}
\end{adjustbox}

\end{frame}

\section{Exercice 8}

\begin{frame}
$$\lim_{x \to +\infty} \dfrac{x^2}{e^x}\pause =\dfrac{+\infty}{+\infty} \pause = \text{forme indéterminée}$$

\medskip
\begin{itemize}
\pause 	\item[-] soit par le théorème des croissances comparées : \pause l'exponentielle l'emporte sur la puissance : $$\lim_{x \to +\infty} \dfrac{x^2}{e^x}=0^+$$ \newline
\pause 	\item[-] soit par le théorème de l'Hospital (TH) : \pause on prend les dérivées du numérateur et du dénominateur jusqu'à trouver une forme non indéterminée.
\end{itemize}

\[\begin{aligned}
\lim_{x \to +\infty} \dfrac{x^2}{e^x}
\pause &\overset{TH}{=}\lim_{x \to +\infty} \dfrac{2x}{e^x} \pause = \dfrac{+\infty}{+\infty} \pause = \text{forme indéterminée}\\
\pause &\overset{TH}{=} \lim_{x \to +\infty} \dfrac{2}{e^x} \pause = \dfrac{2}{+\infty} \pause =0^+
\end{aligned}\]
\end{frame}

\begin{frame}
$$\lim_{x \to +\infty} \dfrac{\ln(x)}{x}\pause =\dfrac{+\infty}{+\infty}\pause = \text{forme indéterminée}$$

\medskip
\begin{itemize}
\pause 	\item[-] soit par le théorème des croissances comparées : \pause la puissance l'emporte sur le logarithme : $$\lim_{x \to +\infty} \dfrac{\ln(x)}{x}=0^+$$ \newline

\pause 	\item[-] soit par le théorème de l'Hospital (TH) :
\end{itemize}

\[\begin{aligned}
\lim_{x \to +\infty} \dfrac{\ln(x)}{x} \pause \overset{TH}{=} \lim_{x \to +\infty}\dfrac{1/x}{1}\pause =\lim_{x \to +\infty}\dfrac{1}{x}\pause =0^+
\end{aligned}\]

\end{frame}

\begin{frame}
$$\lim_{x \to +\infty} \dfrac{e^x+3x^2}{4e^x+2x^2}\pause =\dfrac{+\infty}{+\infty}= \text{forme indéterminée}$$
\pause
\medskip
\begin{itemize}
	\item[-] soit par le théorème des croissances comparées : $$\dfrac{e^x+3x^2}{4e^x+2x^2} \pause= \dfrac{e^x(1+\frac{3x^2}{e^x})}{e^x(4+\frac{2x^2}{e^x})} \pause= \dfrac{1+\overbrace{\frac{3x^2}{e^x}}^{\to 0}}{4+\underbrace{\frac{2x^2}{e^x}}_{\to 0}} \pause \underset{x \to +\infty}{\to} \frac{1}{4} $$ \newline

 \pause	\item[-] soit par le théorème de l'Hospital (TH) :
\end{itemize}

\[\begin{aligned}
\lim_{x \to +\infty} \dfrac{e^x+3x^2}{4e^x+2x^2}
\pause &\overset{TH}{=}\lim_{x \to +\infty} \dfrac{e^x+6x}{4e^x+4x}\pause =\dfrac{+\infty}{+\infty}= \text{forme indéterminée}\\
\pause &\overset{TH}{=}\lim_{x \to +\infty} \dfrac{e^x+6}{4e^x+4}\pause =\dfrac{+\infty}{+\infty}= \text{forme indéterminée}\\
\pause &\overset{TH}{=}\lim_{x \to +\infty} \dfrac{e^x}{4e^x} \pause = \dfrac{1}{4}
\end{aligned}\]
\end{frame}

\begin{frame}
\[\begin{aligned}
\lim_{x \to +\infty} \dfrac{3 x \ln(x)}{x^2-x}\pause =\lim_{x \to +\infty} \dfrac{3 \ln(x)}{x-1}\pause =\dfrac{+\infty}{+\infty} \pause = \text{forme indéterminée}
\end{aligned}\]

\pause
\medskip
\begin{itemize}
	\item[-] soit par le théorème des croissances comparées : $$\lim_{x \to +\infty} \dfrac{3 \ln(x)}{x-1} \pause= \dfrac{x(\frac{3 \ln(x)}{x})}{x(1-\frac{1}{x})} \pause= \dfrac{\overbrace{\frac{3\ln(x)}{x}}^{\to 0}}{1-\underbrace{\frac{1}{x}}_{\to 0}} \pause \underset{x \to +\infty}{\to} 0 $$ \newline

	\pause	\item[-] soit par le théorème de l'Hospital (TH) :
\end{itemize}

\[\begin{aligned}
\lim_{x \to +\infty} \dfrac{3 \ln(x)}{x-1} \overset{TH}{=} \lim_{x \to +\infty} \dfrac{3/x}{1}\pause =\lim_{x \to +\infty} \dfrac{3}{x}\pause  = 0^+
\end{aligned}\]

\end{frame}


\section{Exercice 9}

\begin{frame}

Soit ici : \newline

\[\begin{aligned}
\begin{cases}
\underset{x,y}{\max}~xy \\
\pause \pause \text{sc : } x + y =100
\end{cases}
\end{aligned}\]\newline

\pause C'est un programme type programme du consommateur.

\[\begin{aligned}
\begin{cases}
\underset{x,y}{\max}~U(x,y) \\
\pause \text{sc budget : } p_xx + p_yy \le R
\end{cases}
\end{aligned}\]\\

qui est un programme de maximisation d'une fonction à 2 variables sous contrainte linéaire.
\end{frame}

\begin{frame}

Comme
$$
x+y=100  \Leftrightarrow y=100-x
$$

\pause on remplace $y$ par son expression :

\[\begin{aligned}
\underset{x}{\max}~\pause x(100-x)
\end{aligned}\]

\pause qui devient donc un programme de maximisation d'une fonction à 1 variable $x$ sans contrainte.

\end{frame}

\begin{frame}
\[\begin{aligned}
\underset{x}{\max}~x(100-x) \pause = 100x-x^2
\end{aligned}\]

\pause En la solution $x^\star$, la dérivée de la fonction s'annule. On commence donc par calculer la dérivée :

\[\begin{aligned}
\frac{d(100x-x^2)}{dx} \pause = 100-2x
\end{aligned}\]

\pause puis on l'écrit en la solution donc :

\[\begin{aligned}
&100-2x^\star=0\\
\pause \Leftrightarrow&2x^\star=100\\
\pause \Leftrightarrow&x^\star=50\\
\end{aligned}\]

\pause On déduit $y^{\star}$ grâce à la contrainte :
\[\begin{aligned}
&x+y=100  \\
\pause \Leftrightarrow &x^\star+y^\star=100 \\
\pause \Leftrightarrow &y^\star=100-x^\star \pause = 50
\end{aligned}\]

\end{frame}

\begin{frame}

Soit ici : \newline

	\[\begin{aligned}
	\begin{cases}
	\underset{x,y}{\min}~x^2 + y^2 \\
	\pause \text{sc : } x + y =100
	\end{cases}
	\end{aligned}\]\newline


C'est un programme de minimisation

\[\begin{aligned}
\begin{cases}
\underset{x,y}{\min}~C(x,y) \\
\pause \text{sc équation linéaire}
\end{cases}
\end{aligned}\]\\

d'une fonction à 2 variables sous contrainte linéaire. \pause

\end{frame}

\begin{frame}

	Comme
	$x+y=100  \Leftrightarrow y=100-x
	$, on remplace $y$ :

	\[\begin{aligned}
	\underset{x}{\min}~\pause x^2 + (100-x)^2
	\end{aligned}\]

	qui devient donc un programme de minimisation d'une fonction à 1 variable $x$ sans contrainte.

\end{frame}

\begin{frame}
	\[\begin{aligned}
	\underset{x}{\min}~x^2 + (100-x)^2  \pause &= x^2 + 100^2 + x^2 - 200x \\\pause &= 2x^2 -200x + 100^2
	\end{aligned}\]

\pause 	En la solution $x^\star$, la dérivée de la fonction s'annule. On commence donc par calculer la dérivée :

	\[\begin{aligned}
	\frac{d(2x^2 -200x + 100^2)}{dx} \pause = 4x -200
	\end{aligned}\]

\pause puis on l'écrit en la solution donc :

	\[\begin{aligned}
	&4x^\star -200=0\\
	\pause \Leftrightarrow&4x^\star=200\\
	\pause \Leftrightarrow&x^\star=50\\
	\end{aligned}\]

	On déduit $y^\star$ grâce à la contrainte :
	\[\begin{aligned}
	\pause y^\star=100-x^\star \pause = 50
	\end{aligned}\]

\end{frame}

\begin{frame}

Remarques :
\begin{itemize}
\pause \item[-] La démarche est la même pour un programme de maximisation ou de minimisation. Donc comment savoir si $x^\star$ est un maximum ou un minimum ?\newline

\pause \item[-] Il faut caractériser le signe de la dérivée seconde de la fonction (éventuellement en la solution si elle dépend de l'inconnue) :

\begin{itemize}
\pause \item[o] si la dérivée seconde >0 : on est à un minimum
\pause \item[o] si la dérivée seconde <0 : on est à un maximum. \newline
\end{itemize}

\pause \item[-] Q1 : $\frac{d^2(100x-x^2)}{dx^2} \pause = \frac{d\left(\frac{d(100x-x^2)}{dx}\right)}{dx}\pause = \frac{d(100-2x)}{dx} \pause = -2\pause <0   $ \pause donc maximum. \newline

\pause \item[-] Q2 : $\frac{d^2(2x^2 -200x + 100^2)}{dx^2} \pause = \frac{d\left(\frac{d(2x^2 -200x + 100^2)}{dx}\right)}{dx}\pause = \frac{d(4x -200)}{dx} \pause = 4 \pause >0 $ \pause donc minimum.

\end{itemize}

\end{frame}

\section{Suppléments}

\begin{frame}{Dérivées}

Soit $U$ la fonction d'utilité d'un agent. On définit l'aversion absolue pour le risque par :

\[A_U(x) = -\frac{U^{''}(x)}{U^{'}(x)}\]

avec $x$ le niveau de richesse de l'agent, $U^{'}(.)$ et $U^{''}(.)$ respectivement les dérivées première et seconde de la fonction $U(.)$ si elles existent. \newline

Calculer l'aversion absolue pour le risque pour les fonctions suivantes~:
\begin{itemize}
	\item $U(x) = ax+b$
	\item $U(x) = \ln(x) $
	\item $U(x) = \frac{1}{1-r}x^{1-r}$
	\item $U(x) = -e^{-ax}$
\end{itemize}

\end{frame}

\begin{frame}{Dérivées partielles}

Soient les fonctions de demande de biens 1 et 2 à l'optimum :

$$x_1^d(p_1,p_2,R) = \frac{R}{5~p_1} \text{ et } x_2^d(p_1,p_2,R) = \frac{4~R}{5~p_2}$$

Calculer

\begin{itemize}
	\item[-] les élasticités revenu des demandes en biens 1 et 2 : $$\epsilon_{R,1}
	= \frac{\frac{\partial x_1^d(p_1,p_2,R)}{\partial R}}{\frac{x_1^d(p_1,p_2,R) }{R}}\text{ et } \epsilon_{R,2}
	= \frac{\frac{\partial x_2^d(p_1,p_2,R)}{\partial R}}{\frac{x_2^d(p_1,p_2,R) }{R}}$$
	\item[-] leurs élasticités prix directes : $$\epsilon_{1,1}
	= \frac{\frac{\partial x_1^d(p_1,p_2,R)}{\partial p_1}}{\frac{x_1^d(p_1,p_2,R) }{p_1}}\text{ et } \epsilon_{2,2}
	= \frac{\frac{\partial x_2^d(p_1,p_2,R)}{\partial p_2}}{\frac{x_2^d(p_1,p_2,R) }{p_2}}$$
	\item[-] leurs élasticités prix croisées : $$\epsilon_{2,1}
	= \frac{\frac{\partial x_2^d(p_1,p_2,R)}{\partial p_1}}{\frac{x_2^d(p_1,p_2,R) }{p_1}}\text{ et }\epsilon_{1,2}
	= \frac{\frac{\partial x_1^d(p_1,p_2,R)}{\partial p_2}}{\frac{x_1^d(p_1,p_2,R) }{p_2}}$$
\end{itemize}

\end{frame}

\begin{frame}
	Commençons par le numérateur :

	\begin{columns}
		\begin{column}{0.5\textwidth}
			$$x_1^d(p_1,p_2,R) = \frac{R}{5p_1}$$

			\[\begin{aligned}
			\frac{\partial x_1^d(p_1,p_2,R) }{\partial R} \pause= \frac{\partial (\frac{R}{5p_1}) }{\partial R} \pause= \frac{1}{5p_1}\frac{\partial R}{\partial R} = \frac{1}{5p_1}\\
			\end{aligned}\]
		\end{column}
		\begin{column}{0.5\textwidth}
			\pause	$$x_2^d(p_1,p_2,R) = \frac{4R}{5p_2}$$
			\[\begin{aligned}
			\frac{\partial x_2^d(p_1,p_2,R) }{\partial R} \pause= \frac{\partial (\frac{4R}{5p_2}) }{\partial R} \pause= \frac{4}{5p_2}\frac{\partial R}{\partial R}\pause =\frac{4}{5p_2}\\
			\end{aligned}\]
		\end{column}
	\end{columns}

	\bigskip

	Et l'élasticité revenu :\newline

	\begin{columns}
		\begin{column}{0.5\textwidth}
			\Large
			\[\begin{aligned}
			\epsilon_{R,1}
			\pause&= \frac{\frac{\partial x_1^d(p_1,p_2,R)}{\partial R}}{\frac{x_1^d(p_1,p_2,R) }{R}} \pause= \frac{\frac{1}{5p_1}}{\frac{\frac{R}{5p_1}}{R}} \\
			\pause&= \frac{\frac{1}{5p_1}}{\frac{R}{5Rp_1}} \pause=1
			\end{aligned}\]
		\end{column}
		\begin{column}{0.5\textwidth}
			\Large
			\[\begin{aligned}
			\pause\epsilon_{R,2}
			\pause &= \frac{\frac{\partial x_2^d(p_1,p_2,R)}{\partial R}}{\frac{x_2^d(p_1,p_2,R) }{R}} \pause = \frac{\frac{4}{5p_2}}{\frac{\frac{4R}{5p_2}}{R}} \\
			\pause&= \frac{\frac{4}{5p_2}}{\frac{4R}{5Rp_2}} \pause=1
			\end{aligned}\]
		\end{column}
	\end{columns}

\end{frame}

\begin{frame}

	Les élasticités prix directes :\newline

	\begin{columns}
		\begin{column}{0.5\textwidth}
			\large
			\[\begin{aligned}
			\epsilon_{1,1}
			&= \frac{\frac{\partial x_1^d(p_1,p_2,R)}{\partial p_1}}{\frac{x_1^d(p_1,p_2,R) }{p_1}} \pause= \frac{\frac{-R}{5p_1^2}}{\frac{\frac{R}{5p_1}}{p_1}} \\
			\pause&= \frac{\frac{-R}{5p_1^2}}{\frac{R}{5p_1^2}}\pause= -1
			\end{aligned}\]
		\end{column}
		\begin{column}{0.5\textwidth}
			\large
			\[\begin{aligned}
			\pause 	\epsilon_{2,2}
			\pause 	&= \frac{\frac{\partial x_2^d(p_1,p_2,R)}{\partial p_2}}{\frac{x_2^d(p_1,p_2,R) }{p_2}} \pause= \frac{\frac{-4R}{5p_2^2}}{\frac{\frac{4R}{5p_2}}{p_2}} \\
			\pause&= \frac{\frac{-4R}{5p_2^2}}{\frac{4R}{5p_2^2}}\pause= -1
			\end{aligned}\]
		\end{column}
	\end{columns}

	\bigskip

	Les élasticités prix croisées  :\newline

	\begin{columns}
		\begin{column}{0.5\textwidth}
			\large
			\[\begin{aligned}
			\epsilon_{1,2}
			\pause 	&= \frac{\frac{\partial x_1^d(p_1,p_2,R)}{\partial p_2}}{\frac{x_1^d(p_1,p_2,R) }{p_2}} \pause= \frac{0}{\frac{\frac{R}{5p_1}}{p_2}} = 0
          \end{aligned}\]
		\end{column}
		\begin{column}{0.5\textwidth}
			\large
			\[\begin{aligned}
			\pause 		\epsilon_{2,1}
			\pause 		&= \frac{\frac{\partial x_2^d(p_1,p_2,R)}{\partial p_1}}{\frac{x_2^d(p_1,p_2,R) }{p_1}} \pause= \frac{0}{\frac{\frac{4R}{5p_2}}{p_2}} = 0
			\end{aligned}\]
		\end{column}
	\end{columns}

  \end{frame}

\end{document}

%%% Local Variables:
%%% mode: latex
%%% TeX-master: t
%%% End:
